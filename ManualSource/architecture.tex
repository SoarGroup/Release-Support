% ----------------------------------------------------------------------------
\typeout{--------------- The Soar ARCHitecture ------------------------------}
\chapter{The Soar Architecture}
\label{ARCH}

This chapter describes the Soar architecture.  It covers all aspects of Soar
except for the specific syntax of Soar's memories and descriptions of the
Soar user-interface commands.

This chapter gives an abstract description of Soar.  It starts by giving
an overview of Soar and then goes into more detail for each of Soar's
main memories (working memory, production memory, and preference memory)
and processes (the decision procedure, learning, and input and output).

% ----------------------------------------------------------------------------
\section{An Overview of Soar}
\label{ARCH-overview}
\index{state}
\index{operator}
\index{goal}

The design of Soar is based on the hypothesis that all deliberate
\textit{goal}-oriented behavior can be cast as the selection and application
of \textit{operators} to a \textit{state}. A \soarb{state} is a representation of the
current problem-solving situation; an \soarb{operator} transforms a state (makes
changes to the representation); and a \soarb{goal} is a desired outcome of the
problem-solving activity.

As Soar runs, it is continually trying to apply the current operator and
select the next operator (a state can have only one operator at a time),
until the goal has been achieved. The selection and application of
operators is illustrated in Figure \ref{fig:select-apply}. 

\begin{figure}
\insertfigure{select-apply}{1.5in}
\insertcaption{Soar is continually trying to select and apply operators.}
\label{fig:select-apply}
\end{figure}

Soar has separate memories (and different representations) for
descriptions of its current situation and its long-term procedural knowledge.  In
Soar, the current situation, including data from sensors, results of
intermediate inferences, active goals, and active operators is held in
\soarb{working memory}.  Working memory is organized as
\emph{objects}. Objects are described in terms of their
\emph{attributes}; the values of the attributes may correspond to
sub-objects, so the description of the state can have a hierarchical
organization. (This need not be a strict hierarchy; for example, there's
nothing to prevent two objects from being ``substructure'' of each
other.)
\index{working memory}
\index{object}
\index{attribute}

\index{production memory}
Long-term procedural knowledge is held in \soarb{production memory}.
Procedural knowledge specifies how to respond to different
situations in working memory, can be thought of as the program for Soar.
The Soar architecture cannot solve any problems without the addition of
long-term procedural knowledge.  (Note the distinction between the ``Soar
architecture'' and the ``Soar program'': The former refers to the system
described in this manual, common to all users, and the latter refers to
knowledge added to the architecture.)

A Soar program contains the knowledge to be used for solving a specific
task (or set of tasks), including information about how to select and
apply operators to transform the states of the problem, and a means of
recognizing that the goal has been achieved.  

\subsection{Types of Procedural Knowledge in Soar}
\label{LIST:4KnowledgeTypes}

Soar's procedural knowledge can be categorized into  four distinct types of
knowledge:\vspace{-10pt} 
\begin{enumerate}
  \item \textit{Inference Rules} \newline 
In Soar, we call these state elaborations.  This knowledge provides monotonic inferences
that can be made about the state in a given situation. The knowledge created by such rules
are not persistent and exist only as long as the conditions of the rules are met.
  \item \textit{Operator Proposal Knowledge} \newline
Knowledge about when a particular operator is appropriate for a situation.
Note that multiple operators may be appropriate in a given context.
So, Soar also needs knowledge to determine which of the candidates to choose:  
  \item	\textit{Operator Selection Knowledge:} \newline
Knowledge about the desirability of an operator in a particular situation.
Such knowledge can be either in terms of a single operator (e.g. never choose this 
operator in this situation) or relational (e.g. prefer this operator over another
in this situation).
  \item \textit{Operator Application Rules} \newline
Knowledge of how a specific selected operator modifies the state.
This knowledge creates persistent changes to the state that remain even 
after the rule no longer matches or the operator is no longer selected.
\end{enumerate}

Note that state elaborations can indirectly affect operator selection
and application by creating knowledge that the proposal and application
rules match on.

\subsection{Problem-Solving Functions in Soar}
\label{ARCH-functions}
\index{problem solving}
\index{production}
\index{production!match}
\index{match|see{production!match}}
\index{fire|see{production!firing}}
\index{retract|see{production!retraction}}
\index{elaboration}

These problem-solving functions are the primitives for generating 
behavior that is relevant to the current situation: elaborating the 
state, proposing candidate operators, comparing the candidates, 
and applying the operator by modifying the state. 
These functions are driven by the knowledge encoded in a Soar program.
 
\index{production}
Soar represents that knowledge as \soarb{production rules}.  
Production rules are similar to ``if-then'' statements in conventional 
programming languages. (For example, a
production might say something like ``if there are two blocks on the
table, then suggest an operator to move one block on top of the other
block'').  The ``if'' part of the production is called its
\textit{conditions} and the ``then'' part of the production is called
its \textit{actions}. When the conditions are met in the current
situation as defined by working memory, the production is \emph{matched}
and it will \emph{fire}, which means that its actions are executed,
making changes to working memory.

\index{decision procedure}
\index{decision cycle}
Selecting the current operator, involves making a
\soarb{decision} once sufficient knowledge has been retrieved.  This is
performed by Soar's \emph{decision procedure}, which is a fixed
procedure that interprets \emph{preferences} that have been created by
the knowledge retrieval functions. The knowledge-retrieval and decision-making
functions combine to form Soar's \emph{decision cycle}.

\index{impasse}
When the knowledge to perform the problem-solving functions is not
directly available in productions, Soar is unable to make progress and
reaches an \soarb{impasse}.  There are three types of possible impasses
in Soar:
\begin{enumerate}
\item An operator cannot be selected because no new operators are proposed.\vspace{-4pt}
\item An operator cannot be selected because multiple operators are
        proposed and the comparisons are insufficient to determine which
        one should be selected.\vspace{-4pt}
\item An operator has been selected, but there is insufficient knowledge
        to apply it.\vspace{-4pt}
\end{enumerate}
In response to an impasse, the Soar architecture creates a
\soarb{substate} in which operators can be selected and applied to
generate or deliberately retrieve the knowledge that was not directly
available; the goal in the substate is to resolve the impasse. For
example, in a substate, a Soar program may do a lookahead search to
compare candidate operators if comparison knowledge is not directly
available.  Impasses and substates are described in more detail in Section \ref{ARCH-impasses}.
\index{substate|see{subgoal}}


% ----------------------------------------------------------------------------
\subsection{An Example Task: The Blocks-World}

We will use a task called the blocks-world as an example throughout this
manual. In the blocks-world task, the initial state has three blocks named
\soar{A}, \soar{B}, and \soar{C} on a table; the operators move one block at a
time to another location (on top of another block or onto the table); and the
goal is to build a tower with \soar{A} on top, \soar{B} in the middle, and
\soar{C} on the bottom. The initial state and the goal are illustrated in
Figure \ref{fig:blocks}.

The Soar code for this task is available online at \\
\url{https://web.eecs.umich.edu/~soar/blocksworld.soar}. \\
You do not need to look at the code at this point.

\begin{figure}
\insertfigure{blocks}{2in}
\insertcaption{The initial state and goal of the ``blocks-world'' task.}
\label{fig:blocks}
\end{figure}

The operators in this task move a single block from its current location to a
new location; each operator is represented with the following information: 
\vspace{-12pt}
\begin{itemize}
\item the name of the block being moved \vspace{-9pt}
\item the current location of the block (the ``thing'' it is on top of) \vspace{-9pt}
\item the destination of the block (the ``thing'' it will be on top of) 
\vspace{-9pt}
\end{itemize}

The goal in this task is to stack the blocks so that \soar{C} is on the
table, with block \soar{B} on top of block \soar{C}, and block \soar{A} on
top of block \soar{B}.

% ----------------------------------------------------------------------------
\subsection{Representation of States, Operators, and Goals}
\label{OVERVIEW-ps-representation}
\index{state!representation}
\index{operator!representation}
\index{goal!representation}
\index{problem space}
\index{attribute}

The initial state in our blocks-world task --- before any operators have been
proposed or selected --- is illustrated in Figure \ref{fig:ab-wmem}.

\begin{figure}
\insertfigure{ab-wmem}{3.5in}
\insertcaption{An abstract illustration of the initial state of the blocks
	world as working memory objects. At this stage of problem solving, no
	operators have been proposed or selected.}
\label{fig:ab-wmem}
\end{figure}

A state can have only one selected operator at a time
but it may also have a number of \emph{potential} operators that are in consideration.
These proposed operators should not be confused with the active, selected operator.

Figure \ref{fig:ab-wmem2} illustrates working memory after the first operator
has been selected. There are six operators proposed, and only one of
these is actually selected.

\begin{figure}
\insertfigure{ab-wmem2}{4.25in}
\insertcaption{An abstract illustration of working memory in the blocks world
	after the first operator has been selected.}
\label{fig:ab-wmem2}
\end{figure}

Goals are either represented explicitly as substructures of the working memory state
with general rules that recognize when the goal is achieved, or are
implicitly represented in the Soar program by goal-specific rules that
test the state for specific features and recognize when the goal is
achieved.  The point is that sometimes a description of the goal will be
available in the state for focusing the problem solving, whereas other
times it may not.  Although representing a goal explicitly has many advantages,
some goals are difficult to explicitly represent on the state.

For example, the goal in our blocks-world task is represented implicitly in the provided Soar program. This is because a single production rule monitors the state for completion of the goal and halts Soar when the goal is achieved. (Syntax of Soar programs will be explained in Chapter \ref{SYNTAX}.) If the goal was an explicit working memory structure, a rule could compare the configuration of blocks to that structure instead of having the goal embedded within the rule's programming.

% ----------------------------------------------------------------------------
\subsection{Proposing candidate operators}
\index{operator!proposal}

As a first step in selecting an operator, one or more candidate
operators are \soarb{proposed}.  Operators are proposed by rules that test
features of the current state.  When the blocks-world task is run, the
Soar program will propose six distinct (but similar) operators for the
initial state as illustrated in Figure \ref{fig:proposal}. These
operators correspond to the six different actions that are possible
given the initial state.

\begin{figure}
\insertfigure{blocks-proposal}{2.5in}
\insertcaption{The six operators proposed for the initial state of the blocks
	world each move one block to a new location.}
\label{fig:proposal}
\end{figure}


% ----------------------------------------------------------------------------
\subsection{Comparing candidate operators: Preferences}
\index{operator!comparison|see{preferences}}
\index{preference}
\index{preference memory}

The second step Soar takes in selecting an operator is to evaluate or
compare the candidate operators. In Soar, this is done via rules that
test the proposed operators and the current state, and then create
\soarb{preferences} (stored in \emph{preference memory}).  Preferences assert the relative or absolute merits of the
candidate operators. For example, a preference may say that operator A
is a ``better'' choice than operator B at this particular time, or a
preference may say that operator A is the ``best'' thing to do at this
particular time. Preferences are discussed in detail in section \ref{PREFERENCES}. 

% ----------------------------------------------------------------------------
\subsection{Selecting a single operator: Decision}
\index{operator!selection}
\index{decision procedure}

Soar attempts to select a single operator as a \soarb{decision}, based on the preferences available
for the candidate operators. There are four different situations that may
arise:\vspace{-14pt}

\begin{enumerate}
\item The available preferences unambiguously prefer a single operator.\vspace{-
6pt}
\item The available preferences suggest multiple operators, and 
       prefer a subset that can be selected from randomly.\vspace{-6pt}
\item The available preferences suggest multiple operators,but neither case
       1 or 2 above hold.\vspace{-6pt}
\item The available preferences do not suggest any operators.
\end{enumerate}

In the first case, the preferred operator is selected.  In the second
case, one of the subset is selected randomly. In the third and fourth
cases, Soar has reached an \emph{impasse} in problem solving, and a new
substate is created.  Impasses are discussed in Section
\ref{ARCH-impasses}.

In our blocks-world example, the second case holds, and Soar can select one of
the operators randomly.

% ----------------------------------------------------------------------------
\subsection{Applying the operator}
\index{operator!application}

An operator \soarb{applies} by making changes to the state; the specific changes
that are appropriate depend on the operator and the current state.

\index{I/O}
\index{problem solving}
There are two primary approaches to modifying the state: indirect and direct.
\emph{Indirect} changes are used in Soar programs that interact with
an external environment: The Soar program sends motor commands to the
external environment and monitors the external environment for
changes. The changes are reflected in an updated state description,
garnered from sensors. Soar may also make \emph{direct} changes to the
state; these correspond to Soar doing problem solving ``in its
head''. Soar programs that do not interact with an external environment
can make only direct changes to the state.

Internal and external problem solving should not be viewed as mutually
exclusive activities in Soar. Soar programs that interact with an
external environment will generally have operators that make direct and
indirect changes to the state: The motor command is represented as
substructure of the state \emph{and} it is a command to the environment. Also, a Soar program may maintain an internal
model of how it expects an external operator will modify the world; if
so, the operator must update the internal model (which is substructure
of the state).

When Soar is doing internal problem solving, it must know how to modify
the state descriptions appropriately when an operator is being
applied. If it is solving the problem in an external environment, it
must know what possible motor commands it can issue in order to affect
its environment.

The example blocks-world task described here does not interact with an external
environment. Therefore, the Soar program directly makes changes to the state
when operators are applied. There are four changes that may need to be made
when a block is moved in our task: \vspace{-14pt}

\begin{enumerate}\label{LIST:blocks-app}
\item The block that is being moved is no longer where it was (it is no longer
   	``on top'' of the same thing).\vspace{-6pt}
\item The block that is being moved is now in a new location (it is ``on top''
	of a new thing).\vspace{-6pt}
\item The place that the block used to be in is now clear.\vspace{-6pt}
\item The place that the block is moving to is no longer clear --- unless it
	is the table, which is always considered ``clear''\footnote{In this
	blocks-world task, the table always has room for another block, so it
	is represented as always being ``clear''.}.
\end{enumerate}

The blocks-world task could also be implemented using an external simulator. In this case,
the Soar program does not update all the ``on top'' and ``clear'' relations;
the updated state description comes from the simulator.

% ----------------------------------------------------------------------------
\subsection{Making inferences about the state}
\index{elaboration}

Making monotonic inferences about the state is the other role that Soar
long-term procedural knowledge may fulfill. Such \soarb{elaboration} knowledge can simplify
the encoding of operators because entailments of a set of core features
of a state do not have to be explicitly included in application of the
operator.  In Soar, these inferences will be automatically retracted
when the situation changes such that the inference no longer holds.

For instance, our example blocks-world task uses an elaboration to keep track
of whether or not a block is ``clear''. The elaboration tests for the absence
of a block that is ``on top'' of a particular block; if there is no such ``on top'',
the block is ``clear''. When an operator application creates a new ``on top'', the
corresponding elaboration retracts, and the block is no longer ``clear''.


% ----------------------------------------------------------------------------
\subsection{Problem Spaces}
\label{ARCH-functions-ps}
\index{problem space}

If we were to construct a Soar system that worked on a large number of
different types of problems, we would need to include large numbers of
operators in our Soar program. For a specific problem and a
particular stage in problem solving, only a subset of all possible operators
are actually relevant. For example, if our goal is to \textit{count} the
blocks on the table, operators having to do with moving blocks are probably
not important, although they may still be ``legal''. The operators that are
relevant to current problem-solving activity define the space of possible
states that might be considered in solving a problem, that is, they define the
\emph{problem space}.

Soar programs are implicitly organized in terms of problem spaces
because the conditions for proposing operators will restrict an operator
to be considered only when it is relevant.  The complete problem space
for the blocks world is shown in Figure \ref{fig:blocks-ps}.  Typically,
when Soar solves a problem in this problem space, it does not explicitly
generate all of the states, examine them, and then create a path.
Instead, Soar is \emph{in} a specific state at a given time (represented
in working memory), attempting to select an operator that will move it
to a new state.  It uses whatever knowledge it has about selecting
operators given the current situation, and if its knowledge is
sufficient, it will move toward its goal.

\begin{figure}
\insertfigure{blocks-ps}{4.9in}
\insertcaption{The problem space in the blocks-world includes all operators
	that move blocks from one location to another and all possible
	configurations of the three blocks.}
\label{fig:blocks-ps}
\end{figure}

The same problem could be recast in Soar as a planning problem, where
the goal is to develop a plan to solve the problem, instead of just
solving the problem.  In that case, a state in Soar would consist of a
plan, which in turn would have representations of blocks-world states
and operators from the original space.  The operators would perform
editing operations on the plan, such as adding new blocks-world
operators, simulating those operators, etc.  In both formulations of the
problem, Soar is still applying operators to generate new states, it is
just that the states and operators have different content.

The remaining sections in this chapter describe the memories and processes of Soar:
working memory, production memory, preference memory, Soar's execution cycle (the decision
procedure), learning, and how  input and output fit in.
% ----------------------------------------------------------------------------
\section{Working memory: The Current Situation} 
\label{ARCH-wm}
\index{working memory}

Soar represents the current problem-solving situation in its \emph{working
memory}. Thus, working memory holds the current state and operator and is Soar's
``short-term'' knowledge, reflecting the current knowledge of the world and
the status in problem solving.

\index{working memory element}
\index{WME|see{working memory element}}
\index{identifier}
\index{attribute}
\index{value}
Working memory contains elements called working memory elements, or WMEs for
short. Each WME contains a very specific piece of information; for example, a WME
might say that ``B1 is a block''. 
Several WMEs collectively may provide more information about the same
\textit{object}, for example, ``B1 is a block'', ``B1 is named A'', ``B1 is on
the table'', etc. These WMEs are related because they are all contributing to
the description of something that is internally known to Soar as ``B1''. B1 is
called an \soarbit{identifier}; the group of WMEs that share this identifier
are referred to as an \textit{object} in working memory. 
Each WME describes a different \soarbit{attribute} of the object, for example,
its name or type or location; each \textit{attribute} has a \soarbit{value} associated
with it, for example, the name is A, the type is block, and the position is on
the table. Therefore, each WME is an identifier-attribute-value triple, and
all WMEs with the same identifier are part of the same object.

\index{object}
\index{working memory!object|see{object}}
\index{value}
\index{link}
Objects in working memory are \emph{linked} to other objects: The value of one
WME may be an identifier of another object. For example, a WME might say that
``B1 is ontop of T1'', and another collection of WMEs might describe the
object T1: ``T1 is a table'', ``T1 is brown'', and ``T1 is ontop of F1''. And
still another collection of WMEs might describe the object F1: ``F1 is a
floor'', etc. All objects in working memory must be linked to a state, either
directly or indirectly (through other objects). Objects that are not linked to
a state will be automatically removed from working memory by the Soar
architecture. 

\index{augmentation|see{working memory element}}
WMEs are also often called \textit{augmentations} because they
``augment'' the object, providing more detail about it. While these two
terms are somewhat redundant, WME is a term that is used more often to
refer to the contents of working memory (as a single \textit{identifier-attribute-value} triple), 
while augmentation is a term that is used more often to refer to the description of an object.
Working memory is illustrated at an abstract level in Figure
\ref{fig:ab-wmem} on page \pageref{fig:ab-wmem}. 

The attribute of an augmentation is usually a constant, such as ``\soar{name}'' or
``\soar{type}'', because in a sense, the attribute is just a label used to
distinguish one link in working memory from another.\footnote{In order to
allow these links to have some substructure, the attribute name may be an
identifier, which means that the attribute may itself have attributes and
values, as specified by additional working memory elements.}

The value of an augmentation may be either a constant, such as ``\soar{red}'', or
an identifier, such as \soar{06}. When the value is an identifier, it refers
to an object in working memory that may have additional substructure. In
semantic net terms, if a value is a constant, then it is a terminal node with
no links; if it is an identifier it is a nonterminal node.

\index{multi-valued attribute}
\index{multi-attribute|see{multi-valued attribute}}
One key concept of Soar is that working memory is a set, which means that there can never be two elements in
working memory at the same time that have the same identifier-attribute-value
triple (this is prevented by the architecture). However, it is possible to have
multiple working memory elements that have the same identifier and attribute,
but that each have different values.  When this happens, we say the attribute
is a \emph{multi-valued attribute}, which is often shortened to be
\emph{multi-attribute}.

An object is defined by its augmentations and
\emph{not} by its identifier. An identifier is simply a label or pointer to the object. On subsequent runs of the same Soar program,
there may be an object with exactly the same augmentations, but a different
identifier, and the program will still reason about the object
appropriately. Identifiers are internal markers for Soar; they can appear
in working memory, but they never appear in a production.

There is no predefined relationship between objects in working memory and
``real objects'' in the outside world.  Objects in working memory may refer to
real objects, such as \soar{block A}; features of an object, such as the
color \soar{red} or shape \soar{cube}; a relation between objects, such as \soar{ontop}; classes of
objects, such as \soar{blocks}; etc. The actual names of attributes and
values have no meaning to the Soar architecture (aside from a few WMEs
created by the architecture itself). For example, Soar doesn't care whether
the things in the blocks world are called ``blocks'' or ``cubes'' or
``chandeliers''. It is up to the Soar programmer to pick suitable labels and to
use them consistently.

The elements in working memory arise from one of four sources:

\index{Spatial Visual System}
\vspace{-12pt}
\begin{enumerate}
	\item \textbf{\textit{Productions:}} The actions on the RHS of productions create most working memory elements. \vspace{-8pt}
	\item \textbf{\textit{Architecture:}} \vspace{-8pt}
	\begin{enumerate}
		\item \textit{State augmentations:} The decision procedure automatically creates some special state augmentations (type, superstate, impasse, ...) whenever a state is created.  States are created during initialization (the first state) or because of an impasse (a substate).  
		\vspace{-4pt}
	\item \textit{Operator augmentations:}  The decision procedure creates the operator augmentation of the state 
	based on preferences. This records the selection of the current operator.
	\vspace{-8pt}
	\end{enumerate}
	\item \textbf{\textit{Memory Systems}} \vspace{-8pt}
	\item \textbf{\textit{SVS}} \vspace{-8pt}
	\item \textbf{\textit{The Environment:}} External I/O systems create working memory elements on the input-link for sensory data.
\end{enumerate}

The elements in working memory are removed in six different ways:
\index{reject preference}
\index{i-support}
\index{decision procedure}
\index{I/O}
\vspace{-12pt}
\begin{enumerate}
\item The decision procedure automatically removes all state
augmentations it creates when the impasse that led to their creation is 
resolved.\vspace{-8pt}
\item The decision procedure removes the operator augmentation of the
state when that operator is no longer selected as the current operator.\vspace{-
8pt}
\item Production actions that use \soar{reject} preferences remove
      working memory elements that were created by other productions.\vspace{-8pt}
\item The architecture automatically removes i-supported WMEs when the productions that created them no longer match.\vspace{-8pt}
\item The I/O system removes sensory data from the input-link when it
is no longer valid. \vspace{-8pt}
\item The architecture automatically removes WMEs that are no longer linked to 
a state (because some other WME has been removed).
\end{enumerate}


\index{state}
For the most part, the user is free to use any attributes and values
that are appropriate for the task. However, states have special
augmentations that cannot be directly created, removed, or modified by
rules.  These include the augmentations created when a state is created,
and the state's operator augmentation that signifies the current
operator (and is created based on preferences).  The specific
attributes that the Soar architecture automatically creates are listed in Section
\ref{SYNTAX-impasses}. Productions may create any other attributes for
states.

Preferences are held in a separate \emph{preference memory} where they cannot be tested by productions.  There is one notable exception.  Since a soar program may need to reason about candidate operators, \soar{acceptable} preferences are made available in working memory as well. The acceptable preferences can then be tested by productions, which allows a Soar program to reason about candidates operators to determine which one should be selected. Preference memory and the different types of preferences will be discussed in Section \ref{ARCH-prefmem}.

% ----------------------------------------------------------------------------
\section{\texorpdfstring{Production Memory:\\ Long-term Procedural Knowledge}{Production Memory: Long-term Procedural Knowledge}} 
\label{ARCH-pm}
\index{production memory}
\index{production}

\begin{figure}
\insertfigure{ab-prodmem}{3.5in}
\insertcaption{An abstract view of production memory. The productions are not
	related to one another.}
\label{fig:ab-prodmem}
\end{figure}

\index{production!firing}
Soar represents long-term procedural knowledge as \soarb{productions} that are stored in
\emph{production memory}, illustrated in Figure \ref{fig:ab-prodmem}. Each
production has a set of conditions and a set of actions.  If the
conditions of a production match working memory, the production
\emph{fires}, and the actions are performed.

\subsection{The structure of a production}
\label{ARCH-pm-structure}
\index{conditions|see{production}}
\index{actions|see{production}}
\index{production!condition side (LHS)}
\index{production!action side (RHS)}

In the simplest form of a production, conditions and actions refer directly to
the presence (or absence) of objects in working memory. For example, a
production might say:
\begin{verbatim}
  CONDITIONS: block A is clear 
              block B is clear 
  ACTIONS:    suggest an operator to move block A ontop of block B
\end{verbatim}
This is not the literal syntax of productions, but a simplification.
The actual syntax is presented in Chapter \ref{SYNTAX}.

The conditions of a production may also specify the \emph{absence} of patterns
in working memory. For example, the conditions could also specify that ``block
A is not red'' or ``there are no red blocks on the table''. But since these are
not needed for our example production, there are no examples of negated
conditions for now.

The order of the conditions of a production do not matter to Soar except
that the first condition must directly test the state. Internally, Soar
will reorder the conditions so that the matching process can be more
efficient. This is a mechanical detail that need not concern most
users. However, you may print your productions to the screen or save
them in a file; if they are not in the order that you expected them to
be, it is likely that the conditions have been reordered by Soar.

\subsubsection{Variables in productions and multiple instantiations}
\index{variables}
\index{production!instantiation}

In the example production above, the names of the blocks are ``hardcoded'',
that is, they are named specifically. In Soar productions, variables are used
so that a production can apply to a wider range of situations.

 When variables are bound to specific symbols in working memory elements by Soar’s matching process, Soar creates an \emph{instantiation} of the production. This instantiation consists of the matched production along with a specific and consistent set of symbols that matched the variables. A production instantiation is consistent only if every occurrence of a variable is bound to the same value. Multiple instantiations of the same production can be created since the same production may match multiple times, each with different variable bindings. If blocks \soar{A} and \soar{B} are clear, the first production (without variables) will suggest one operator. However, consider a new proposal production that used variables to test the names of the block.  Such a production will be instantiated twice and therefore suggest \textit{two} operators: one operator to move block \soar{A} on top of block \soar{B} and a second operator to move block \soar{B} on top of block \soar{A}.

Because the identifiers of objects are determined at runtime, literal
identifiers cannot appear in productions. Since identifiers occur in
every working memory element, variables must be used to test for
identifiers, and using the same variables across multiple occurrences is what links conditions together.

\index{production!condition side (LHS)}
Just as the elements of working memory must be linked to a state 
in working memory, so must the objects referred to in a production's
conditions. That is, one condition must test a state object 
\emph{and} all other conditions must test that same state or objects that
are linked to that state.

\subsection{Architectural roles of productions}
\label{ARCH-pm-roles}
\index{production}

Soar productions can fulfill the following four roles, by retrieving
different types of procedural knowledge, all described on page \pageref{LIST:4KnowledgeTypes}:\vspace{-10pt}
\begin{enumerate}
\item Operator proposal\vspace{-10pt}
\item Operator comparison\vspace{-10pt}
\item Operator application\vspace{-10pt}
\item State elaboration
\end{enumerate}

A single production should not fulfill more than one of these roles
(except for proposing an operator and creating an absolute preference
for it). Although productions are not declared to be of one type or the
other, Soar examines the structure of each production and classifies the
rules automatically based on whether they propose and compare operators,
apply operators, or elaborate the state. 

\subsection{Production Actions and Persistence}
\index{i-support}
\index{production!action side (RHS)}
\label{ARCH-prefmem-persistence}
\index{persistence}
\index{o-support}
\index{operator!support}
\label{PAGE:O-support}

Generally, actions of a production either create preferences for
operator selection, or create/remove working memory elements.  For
operator proposal and comparison, a production creates preferences for
operator selection.  These preferences should persist only as long as
the production instantiation that created them continues to match.  When
the production instantiation no longer matches, the situation has
changed, making the preference no longer relevant.  Soar automatically
removes the preferences in such cases.  These preferences are said to
have \emph{i-support} (for ``instantiation support'').  Similarly, state
elaborations are simple inferences that are valid only so long as the
production matches.  Working memory elements created as state
elaborations also have i-support and remain in working memory only as
long as the production instantiation that created them continues to
match working memory.  For example, the set of relevant operators changes
as the state changes, thus the proposal of operators is done with
i-supported preferences. This way, the operator proposals will be
retracted when they no longer apply to the current situation.

However, the actions of productions that \emph{apply} an operator, either
by adding or removing elements from working memory, persist regardless of
whether the operator is still selected or the operator application 
production instantiation still matches. For example, in placing a
block on another block, a condition is that the second block be
clear. However, the action of placing the first block removes the fact
that the second block is clear, so the condition will no longer be
satisfied.

Thus, operator application productions do not retract their actions, even
if they no longer match working memory.  This is called \emph{o-support} 
(for ``operator support''). Working memory elements that participate in
the application of operators are maintained throughout the existence of
the state in which the operator is applied, unless explicitly removed (or
if they become unlinked).  Working memory elements are removed by a
\emph{reject} action of a operator-application rule.  
\index{o-support!reject}

Whether a working memory element receives o-support or i-support is
determined by the structure of the production instantiation that creates
the working memory element.  O-support is given only to working memory
elements created by operator-application productions in the state where
the operator was selected.

An operator-application production tests the current operator of a state
and modifies the state. Thus, a working memory element receives
o-support if it is for an augmentation of the current state or
substructure of the state, and the conditions of the instantiation that
created it test augmentations of the current operator.  

During productions matching, all productions that have their conditions
met fire, creating preferences which may add or remove working memory elements. 
Also, working memory elements and preferences that lose i-support are removed 
from working memory. Thus, several new working memory elements and preferences
may be created, and several existing working memory elements and preferences 
may be removed at the same time. (Of course, all this doesn’t happen literally 
at the same time, but the order of firings and retractions is unimportant, 
and happens in parallel from a functional perspective.)

\subsection{The calculation of o-support}
\label{SUPPORT}
\index{support}
\index{i-support}
\index{o-support}
\index{persistence}
\index{production!instantiation}

This section provides a more detailed description of when an action is given o-support by an instantiation.\footnote{In the past, Soar had various experimental support mode settings. Since version 9.6, the support mode used is what was previously called \soar{mode 4}.} The content here is somewhat more advanced, and the reader unfamiliar with rule syntax (explained in Chapter \ref{SYNTAX}) may wish to skip this section and return at a later point.

Support is given by production; that is, all working memory changes generated by the actions of a single instantiated production will have the same support (an action that is not given o-support will have i-support). The conditions and actions of a production rule will here be referred to using the shorthand of LHS and RHS (for Left-Hand Side and Right-Hand Side), respectively.

A production must meet the following two requirements to have o-supported actions:
\begin{enumerate}
	\item The RHS has no operator proposals, i.e. nothing of the form \begin{verbatim}(<s> ^operator <o> +) \end{verbatim}
	\item The LHS has a condition that tests the current operator, i.e. something of the form
	\begin{verbatim}(<s> ^operator <o>)\end{verbatim}
	\comment{this is only true if mode 3's checks are improved}
\end{enumerate}

In condition 1, the variable \soar{<s>} must be bound to a state identifier.
In condition 2, the variable \soar{<s>} must be bound to the lowest state identifier. That is to say, each (positive) condition on the LHS takes the form \soar{(id \carat attr value)}, some of these id's match state identifiers, and the system looks for the deepest matched state identifier. The tested current operator must be on this state. For example, in this production,

\begin{verbatim}
sp {elaborate*state*operator*name
    (state <s> ^superstate <s1>)
    (<s1> ^operator <o>)
    (<o> ^name <name>)
    -->
    (<s> ^name something)}
\end{verbatim}


the RHS action gets i-support. Of course, the state bound to \soar{<s>} is destroyed when \soar{(<s1> \carat operator <o>)} retracts, so o-support would make little difference. On the other hand, this production,

\begin{verbatim}
sp {operator*superstate*application
    (state <s> ^superstate <s1>)
               ^operator <o>)
    (<o> ^name <name>)
    -->
    (<s1> ^sub-operator-name <name>)}
\end{verbatim}

gives o-support to its RHS action, which remains after the substate bound to \soar{<s>} is destroyed. 

An extension of condition 1 is that operator augmentations should always receive i-support. Soar has been written to recognize augmentations directly off the operator \\
(ie, \soar{(<o> \carat augmentation value)}), and to attempt to give them i-support. However, what should be done about a production that simultaneously tests an operator, doesn't propose an operator, adds an operator augmentation, and adds a non-operator augmentation? For example:

\begin{verbatim}
sp {operator*augmentation*application
    (state <s> ^task test-support
               ^operator <o>)
    -->
    (<o> ^new augmentation)
    (<s> ^new augmentation)}
\end{verbatim}


In such cases, both receive i-support. Soar will print a warning on firing this production, because this is considered bad coding style.

\nocomment{Support calculations are done at run time, as each production is fired. Could these decisions be done at compile time? Much of the decision is based on the structure of the production, which could be analyzed once as the production was loaded or chunked. However, it may be impossible to guarantee that a variable will be bound to a state id just by examining production syntax. Another issue is whether the state tested in condition 2 is the lowest state - this potentially could differ from instantiation to instantiation. For instance the operator*augmentation*application production above could match against multiple states in the state stack. 
	
	
	%-----------------------------------------------------------
	\section{Possible problems with implementation of modes 3 \& 4}
	
	\begin{enumerate}
		\item Default mode is actually o-support mode 3. Do we not want 4 to be default?
		\item There is still the bug Andy pointed out. In condition 1, the variable \soar{<s>} is \textit{supposed} to be bound to a state variable, but the code does not actually check for this.
		\item There is one additional, strange difference between modes 3 \& 4. In condition 3, the \soar{id} of each RHS action is tested to see if it is the id of the operator. This id is represented either as a symbol or as a rete location. Mode 4 tests the id both as a symbol and as a rete location, while mode 3 does only the symbol test. The rete test should be added to mode 3.
	\end{enumerate}
	
	
	\section{O-support modes 1 \& 2}
	
	In o-support modes 1 \& 2, there are some of the same calculations as in 3 \& 4 when a production is matched (which occurs when a wme is added to the rete). In particular, if it is an operator proposal, it is set as IE\_PRODS. Otherwise, if it tests the current operator, it is set as PE\_PRODS, without testing for operator  elaborations. The match is placed on the appropriate dll, according to IE\_PRODS or PE\_PRODS.
	
	Later, when the production is instantiated and the new preferences are built, there are no support calculations for 3 \& 4. But 1 \& 2 have support calculations. I suppose that the purpose of the earlier support calculations is that it places the production on the proper list to be fired during apply or propose,that is, whether it is an IE\_PROD or a PE\_PROD.
	
	During this instantiation process, the function calculate\_support\_for\_instantiation\_preferences() is called to redo support IF the variable need\_to\_do\_support\_calculations is set to TRUE. This variable can be true only when-
	
	\begin{enumerate}
		\item  called from chunk\_instantiation OR
		\item  \#ifndef SOAR\_8\_ONLY
		SOAR\_8\_ONLY is a compile option, which is not defined by default. I think that its purpose is that, when defined, there is no run-time option to switch out of Soar 8. This allows a significant portion of code to be left out. Check out function Soar\_Operand2. 
	\end{enumerate}
	
	
	Mode 2 computes support in what is called 'Doug Pearson's way', which is described as-
	
	\begin{verbatim}
	For a particular preference p=(id ^attr ...) on the RHS of an
	instantiation [LHS,RHS]:
	
	RULE #1 (Context pref's): If id is the match state and attr="operator", 
	then p does NOT get o-support.  This rule overrides all other rules.
	
	RULE #2 (O-A support):  If LHS includes (match-state ^operator ...),
	then p gets o-support.
	
	RULE #3 (O-M support):  If LHS includes (match-state ^operator ... +),
	then p gets o-support.
	
	RULE #4 (O-C support): If RHS creates (match-state ^operator ... +/!),
	and p is in TC(RHS-operators, RHS), then p gets o-support.
	
	Here "TC" means transitive closure; the starting points for the TC are 
	all operators the RHS creates an acceptable/require preference for (i.e., 
	if the RHS includes (match-state ^operator such-and-such +/!), then 
	"such-and-such" is one of the starting points for the TC).  The TC
	is computed only through the preferences created by the RHS, not
	through any other existing preferences or WMEs.
	
	If none of rules 1-4 apply, then p does NOT get o-support.
	
	Note that rules 1 through 3 can be handled in linear time (linear in 
	the size of the LHS and RHS); rule 4 can be handled in time quadratic 
	in the size of the RHS (and typical behavior will probably be linear).
	
	
	What is 'match state'? The match goal for the instantiation.
	Match goal - (a match goal is associated with an instantiation).
	Look through instantiated LHS conditions.
	Find the lowest goal state matched to one of the condition's ids.
	\end{verbatim}  
	
	O-support mode 1 computes Doug's support and compares it to the poor cousin of mode 3 \& 4 support calculations, ie calculation without checking for operator elaboration. It prints any differences it finds.
	
}


\nocomment{
	
	3. the RHS has no direct elaborations of the current operator, ie no actions of the form 
	(<o> ^augmentation value).
	However, an indirect elaboration such as
	(<o> ^name <d>)
	-->
	(<d> ^augmentation value)
	will not prevent o-support.
	
	
	In mode 3, an instantiation will generate o-supported preferences iff
	1. the RHS has no operator proposals (nothing of the form (<s> ^operator <o> +))
	2. the LHS has a condition that tests the current operator (something of the form 
	(<s> ^operator <o>))
	3. 
	
	
	Operator proposal - a production whose RHS has action (<s> ^operator <o> +))
	Operator test - 
	LHS has condition of the form (<s> ^operator <o>)
	Operator elaboration -
	o_support_mode 3:
	
	o_support_mode 4:
	
	
	o_support_mode 4: 
	1. if an operator proposal - i-support
	2. if not an operator test - i-support
	3. if an operator test with no elaborations - o-support
	4. if an operator test with some elaborations and some non-elaboration, non-function RHS action - i-support (warns)
	5. if an operator test with only elaborations - i-support
	
	o_support_mode 3:
	1. if an operator proposal - i-support
	2. if not an operator test - i-support
	3. if an operator test with no elaborations - o-support
	4. if an operator test with some elaborations and some non-elaboration, non-function RHS action - o-support (warns)
	5. if an operator test with only elaborations - i-support
	
	
	o_support_mode 0:
	1. if an operator proposal - i-support
	2. if test operator - o-support
	3. else - i-support
	
}

% ----------------------------------------------------------------------------
\section{Preference Memory: Selection Knowledge} 
\label{ARCH-prefmem}
\index{preference}
\index{preference memory}

\nocomment{need to find the right word there. None of selection, evaluation, or
	comparison seems quite right.}

The selection of the current operator is determined by the \soarb{preferences} in
\emph{preference memory}. Preferences are suggestions or imperatives about the
current operator, or information about how suggested operators compare
to other operators.  Preferences refer to operators by using the
identifier of a working memory element that stands for the operator.
After preferences have been created for a state, the decision procedure
evaluates them to select the current operator for that state.

For an operator to be selected, there will be at least one preference
for it, specifically, a preference to say that the value is a candidate
for the operator attribute of a state (this is done with either an
``\soar{acceptable}'' or ``\soar{require}'' preference). There may also
be others, for example to say that the value is ``best''.

%\index{persistence}
%\index{preference!persistence|see{persistence}}
Preferences remain in preference memory until removed for one of the reasons previously discussed in
Section \ref{ARCH-prefmem-persistence}.

% ----------------------------------------------------------------------------
\subsection{Preference Semantics}
\label{ARCH-prefmem-semantics}
\index{preference}

This section describes the semantics of each type of preference.  More
details on the preference resolution process are provided in
section \ref{PREFERENCES}.

\nocomment{preference resolution or preference evaluation? resolution.}

\index{decision procedure}
Only a single value can be selected as the current operator, that is,
all values are mutually exclusive.  In addition, there is no implicit
transitivity in the semantics of preferences.  If A is indifferent to B,
and B is indifferent to C, A and C will not be indifferent to one
another unless there is a preference that A is indifferent to C (or C
and A are both indifferent to all competing values).

\begin{description}
\index{preference!acceptable("+)}
\index{"+|see{preference}}
\index{acceptable preference|see{preference}}
\item [Acceptable (+)] 
	An \soar{acceptable} preference states that a value is a candidate for selection. All values, except those with \soar{require} preferences, must have an \soar{acceptable} preference in order to be selected. If there is only one value with an \soar{acceptable} preference (and none with a \soar{require} preference), that value will be selected as long as it does not also have a \soar{reject} or a \soar{prohibit} preference.
\vspace{-8pt}

\index{preference!reject(-)}
\index{"-|see{preference}}
\index{reject preference|see{preference}}
\item [Reject ($-$)] 
	A \soar{reject} preference states that the value is not a candidate for selection.
\vspace{-8pt}

\index{preference!better(\textgreater \textit{val})}
\index{">|see{preference}}
\index{better preference|see{preference}}
\index{preference!worse(\textless \textit{val})}
\index{"<|see{preference}}
\index{worse preference|see{preference}}
\item [Better ($>$ \emph{value}), Worse ($<$ \emph{value})] 
	A \soar{better} or \soar{worse} preference states, for the two values involved, that one value should not be selected if the other value is a candidate. \soar{Better} an         \soar{worse} allow for the creation of a partial ordering between candidate values. \soar{Better} and \soar{worse} are simple inverses of each other, so that \soar{A} better than \soar{B} is equivalent to \soar{B} worse than \soar{A}.
\vspace{-8pt}

\index{preference!best(\textgreater)}
\index{">|see{preference}}
\index{best preference|see{preference}}
\item [Best ($>$)] 
	A \soar{best} preference states that the value may be better than any competing value (unless there are other competing values that are also ``best''). If a value is \soar{best} (and not \soar{reject}ed, \soar{prohibit}ed, or \soar{worse} than another), it will be selected over any other value that is not also \soar{best} (or \soar{require}d). If two such values are \soar{best}, then any remaining preferences for those candidates (\soar{worst}, \soar{indifferent}) will be examined to determine the selection. Note that if a value (that is not \soar{reject}ed or \soar{prohibit}ed) is \soar{better} than a \soar{best} value, the \soar{better} value will be selected.  (This result is counter-intuitive, but allows explicit knowledge about the relative worth of two values to dominate knowledge of only a single value. A \soar{require} preference should be used when a value \emph{must} be selected for the goal to be achieved.)
\vspace{-8pt}

\index{preference!worst(\textless)}
\index{"<|see{preference!worst}}
\index{worst preference|see{preference!worst}}
\item [Worst ($<$)] 
	A \soar{worst} preference states that the value should be selected only if there are no alternatives.  It allows for a simple type of default specification. The semantics of the \soar{worst} preference are similar to those for the \soar{best} preference.
\vspace{-8pt}

\index{preference!unary indifferent(=)}
\index{"=|see{preference}}
\index{indifferent preference|see{preference}}
\index{indifferent-selection}
\item [Unary Indifferent (=)] 
	A \soar{unary indifferent} preference states that there is positive knowledge that a single value is as good or as bad a choice as other expected alternatives. 
	
	When two or more competing values both have indifferent preferences, by default, Soar chooses randomly from among the alternatives. (The \soar{decide indifferent-selection} function can be used to change this behavior as described on page \pageref{decide-indifferent-selection} in Chapter \ref{INTERFACE}.)
\vspace{-8pt}

\index{preference!binary indifferent(=\textit{val})}
\item [Binary Indifferent (= \emph{value})] 
	A \soar{binary indifferent} preference states that two values are mutually indifferent and it does not matter which of these values are selected. It behaves like a \soar{unary indifferent} preference, except that the operator value given this preference is only made indifferent to the operator value given as the argument.
\vspace{-8pt}

\index{preference!numeric-indifferent("= \textit{num})}
\index{numeric-indifferent preference|see{preference}}
\item [Numeric-Indifferent (= \emph{number})]
	A \soar{numeric-indifferent} preference is used to bias the random selection from mutually indifferent values. This preference includes a \soar{unary indifferent} preference, and behaves in that manner when competing with another value having a unary indifferent preference. 
	%When a set of operators are determined to be indifferent based on all other asserted preference types and at least one operator has a numeric-indifferent preference, 
	But when a set of competing operator values have \soar{numeric-indifferent} preferences, the decision mechanism will choose an operator based on their numeric-indifferent values and the exploration policy. The available exploration policies and how they calculate selection probability are detailed in the documentation for the \soar{indifferent-selection} command on page \pageref{decide-indifferent-selection}. When a single operator is given multiple numeric-indifferent preferences, they are either averaged or summed into a single value based on the setting of the \soar{numeric-indifferent-mode} command (see page \pageref{decide-numeric-indifferent-mode}).

	Numeric-indifferent preferences that are created by RL rules can be adjusted by the reinforcement learning mechanism. In this way, it's possible for an agent to begin a task with only arbitrarily initialized numeric indifferent preferences and with experience learn to make the optimal decisions. See chapter \ref{RL} for more information.
	
\index{preference!require("!)}
\index{"!|see{preference}}
\index{require preference|see{preference}}
\item [Require (!)] 
	A \soar{require} preference states that the value \emph{must} be selected if the goal is to be achieved. A \soar{require}d value is preferred over all others. Only a single operator value should be given a \soar{require} preference at a time.
\vspace{-8pt}

\index{preference!prohibit}
\index{"~|see{preference}}
\index{prohibit preference|see{preference}}
\item [Prohibit ($\tild$)] 
	A \soar{prohibit} preference states that the value cannot be selected if the goal is to be achieved.  If a value has a \soar{prohibit} preference, it will not be selected for a value of an augmentation, independent of the other preferences.
\vspace{-8pt}
\end{description}


\index{preference!acceptable("+)}
\index{preference!require("!)}
If there is an \soar{acceptable} preference for a value of an operator, and there are no other competing values, that operator will be selected. If there are multiple \soar{acceptable} preferences for the same state but with different values, the preferences must be evaluated to determine which candidate is selected.

If the preferences can be evaluated without conflict, the appropriate operator augmentation of the state will be added to working memory. This can happen when they all suggest the same operator or when one operator is preferable to the others that have been suggested. When the preferences conflict, Soar reaches an impasse, as described in Section \ref{ARCH-impasses}.

Preferences can be confusing; for example, there can be two suggested values that are both ``best'' (which again will lead to an impasse unless additional preferences resolve this conflict); or there may be one preference to say that value \soar{A} is better than value \soar{B} and a second preference to say that value \soar{B} is better than value \soar{A}.

\subsection{How preferences are evaluated to decide an operator}
\label{PREFERENCES}
\index{preference}
% This is a technical discussion of the filtering done to evaluate preferences;
% it might belong in a different version of the manual, but not 492

During the decision phase, operator preferences are evaluated in a sequence 
of eight steps, in an effort to select a single operator. 
Each step handles a specific type of preference, as illustrated in Figure 
\ref{fig:prefsem}. (The figure should be read starting at the top
where all the operator preferences are collected and passed into the procedure. At
each step, the procedure either exits through a arrow to the right, or passes to 
the next step through an arrow to the left.)

Input to the procedure is the set of current operator preferences, and the output
consists of:
\begin{enumerate}
	\item a subset of the candidate operators, which is either the empty set, a set consisting of a single, 
	winning candidate, or a larger set of candidates that may be conflicting,
	tied, or indifferent.
	\item an impasse-type. %, possibly NONE\_IMPASSE\_TYPE
\end{enumerate}
The procedure has several potential exit points. Some occur when the procedure
has detected a particular type of impasse. The others occur when the number of
candidates has been reduced to 
one (necessarily the winner) or zero (a no-change impasse).

\nocomment{
	There are nine filter-like operations involved in evaluating the preferences
	available for a particular identifier and attribute. These filters are
	executed in a specific order to determine the correct values for the working
	memory augmentation, as illustrated in Figure \ref{fig:prefsem}. (The figure
	should be read starting at the top left where all the values for an attribute
	are collected and passed to the first filter.) Each filter reduces the number
	of preferences that need to be considered. If a conflict is found, then an
	impasse is generated and the filtering process is halted. The impasse
	generation is handled as a special exit from a filter and is indicated with a
	grey line in Figure \ref{fig:prefsem}.
	
	The preference semantics module takes as input one or more preferences for a
	given identifier and attribute; its output includes: \vspace{-10pt}
	\begin{enumerate}
		\item a possibly empty set of candidate augmentations that may be conflicting,
		indifferent, or parallel\vspace{-10pt}
		\item possibly, an impasse type (if the
		candidates are conflicting)
	\end{enumerate}
}

\index{decision procedure}

\begin{figure}
	\insertfigure{newprefsem}{\textwidth}
	\insertcaption{An illustration of the preference resolution process. There are eight
		steps; only five of these provide exits from the  resolution process.}
	\label{fig:prefsem}
\end{figure}

Each step in Figure \ref{fig:prefsem} is described below:

\index{preference!require("!)}
\index{require preference|see{preference}}
\index{"!}
\index{impasse!constraint-failure}
\index{impasse!no-change}
\begin{description}
	\item[RequireTest (!)]
	This test checks for required candidates in preference memory and
	also constraint-failure impasses involving require preferences (see
	Section \ref{ARCH-impasses} on page \pageref{ARCH-impasses}).
	
	\begin{itemize}
		\item If there is exactly one candidate operator with a require preference and
		that candidate does not have a prohibit preference, then that candidate
		is the winner and preference semantics terminates.
		\item Otherwise ---
		If there is more than one required candidate, then a constraint-
		failure impasse is recognized and preference semantics terminates 
		by returning the set of required candidates.
		\item Otherwise ---
		If there is a required candidate that is also prohibited, a
		constraint-failure impasse with the required/prohibited value is
		recognized and preference semantics terminates.
		\item Otherwise ---
		There is no required candidate; candidates are passed to AcceptableCollect.
	\end{itemize}
	
	\item[AcceptableCollect (+) ] This operation builds a list of operators
	for which there is an acceptable preference in preference memory.
	This list of candidate operators is passed to the ProhibitFilter.\index{+}
	\nocomment{
		\begin{itemize}
			\item If there are no acceptable preferences in memory for the value of an
			attribute then exit preference semantics with no items picked. 
			(This is an efficiency termination, and does not apply to other filters.)
			\item Otherwise ---
			The candidates are passed to the ProhibitFilter.
		\end{itemize}
	}
	\index{preference!acceptable("+)}
	
	
	\item[ProhibitFilter ($\sim$) ] This filter removes the candidates that
	have prohibit preferences in memory. The rest of the candidates are passed to
	the RejectFilter.
	\index{preference!prohibit}
	
	\item[RejectFilter ($-$) ] This filter removes the candidates that have
	reject preferences in memory. 
	\index{preference!reject(-)}
	\index{reject preference(-)}
	
	\item[Exit Point 1]:
	\begin{itemize}
		\item At this point, if the set of remaining candidates is empty, a no-change impasse
		is created with no operators being selected.
		\item If the set has one member, preference semantics terminates and this set is returned.
		\item Otherwise, the remaining candidates are passed to the
		BetterWorseFilter.
	\end{itemize}
	\index{-}
	
	\item[BetterWorseFilter ($>$), ($<$) ] This filter removes any candidates that are worse
	than another candidate.
	\index{preference!worse(\textless \textit{val})}
	\index{worse preference|see{preference}}
	\index{preference!better(\textgreater \textit{val})}
	\index{better preference|see{preference}}
	
	\item[Exit Point 2]:
	\begin{itemize}
		\item If the set of remaining candidates is empty, a conflict impasse is created
		returning the set of all candidates passed into this filter, i.e. all of the
		conflicted operators.
		\item If the set of remaining candidates has one
		member, preference semantics terminates and this set is returned.
		\item Otherwise, the remaining candidates are passed to the
		BestFilter.
	\end{itemize}
	\index{-}
	
	\item[BestFilter ($>$) ] If some remaining candidate has a best preference,
	this filter removes any candidates that do not have
	a best preference. If there are no best preferences for any of the current
	candidates, the filter has no effect. The remaining candidates are passed
	to the WorstFilter.
	\index{preference!best(\textgreater)}
	
	\item[Exit Point 3]:
	\begin{itemize}
		\item At this point, if the set of remaining candidates is empty,
		a no-change impasse is created with no operators being selected.
		\item If  the set has one member, preference semantics terminates 
		and this set is returned.
		\item Otherwise, the remaining candidates are passed to the
		WorstFilter.
	\end{itemize}
	\index{-}
	
	\item[WorstFilter ($<$) ] This filter removes any candidates that have
	a worst preference. If all remaining candidates have worst preferences or there
	are no worst preferences, this filter has no effect.
	\index{preference!worst(\textless)}
	
	\item[Exit Point 4]:
	\begin{itemize}
		\item At this point, if the set of remaining candidates is empty,
		a no-change impasse is created with no operators being selected. 
		\item If the set has one member, preference semantics terminates 
		and this set is returned.
		\item Otherwise, the remaining candidates are passed to the
		IndifferentFilter.
	\end{itemize}
	
	\item[IndifferentFilter (=) ] This operation traverses the remaining candidates and marks 
	each candidate for which one of the following is true:
	\begin{itemize}
		\item the candidate has a unary indifferent preference
		\item the candidate has a numeric indifferent preference
	\end{itemize}
	This filter then checks every candidate that is not one of the above two types
	to see if it has a binary indifferent preference with every other candidate.
	If one of the candidates fails this test, then the procedure signals a tie impasse
	and returns the complete set of candidates that were passed into the 
	IndifferentFilter. Otherwise, the candidates are mutually indifferent, in which case 
	an operator is chosen according to the method set by the \soar{decide indifferent-selection} 
	command, described on page \pageref{decide-indifferent-selection}.
	\index{preference!unary indifferent(=)}
\end{description}

% ----------------------------------------------------------------------------
% ----------------------------------------------------------------------------
\section{Soar's Execution Cycle: Without Substates}
\label{ARCH-decision}

\index{decision cycle}
\index{quiescence}
\index{elaboration cycle}

The execution of a Soar program proceeds through a number of \soarb{decision cycles}. Each cycle has five phases:

\begin{enumerate} 
\item \textbf{Input}: 
	New sensory data comes into working memory.
\item \textbf{Proposal}: 
	Productions fire (and retract) to interpret new data (state elaboration), propose operators for the current situation (operator proposal), and compare proposed operators (operator comparison).  All of the actions of these productions are i-supported.  All matched productions fire in parallel (and all retractions occur in parallel), and matching and firing continues until there are no more additional complete matches or retractions of productions (\emph{quiescence}). 
\item \textbf{Decision}:
	A new operator is selected, or an impasse is detected and a new state is created.
\item \textbf{Application}: 
	Productions fire to apply the operator (operator application).  The actions of these productions will be o-supported. Because of changes from operator application productions, other productions with i-supported actions may also match or retract. Just as during proposal, productions fire and retract in parallel until quiescence.
\item \textbf{Output}: 
	Output commands are sent to the external environment.
\end{enumerate}

The cycles continue until the halt action is issued from the Soar program (as the action of a production) or until Soar is interrupted by the user.

An important aspect of productions in Soar to keep in mind is that all productions will always fire whenever their conditions are met, and retract whenever their conditions are no longer met. The exact details of this process are shown in Figure \ref{fig:decisioncycle}. The \emph{Proposal} and \emph{Application} phases described above are both composed of as many \soarb{elaboration cycles} as are necessary to reach quiescence. In each elaboration cycle, all matching productions fire and the working memory changes or operator preferences described through their actions are made. After each elaboration cycle, if the working memory changes just made change the set of matching productions, another cycle ensues. This repeats until the set of matching rules remains unchanged, a situation called \soarb{quiescence}.


\begin{figure}
\insertfigure{decisioncycle}{\textwidth}
\insertcaption{A detailed illustration of Soar's decision cycle: out of date}
\label{fig:decisioncycle}
\end{figure}

\begin{figure}
\index{decision cycle}
\begin{verbatim}
Soar
  while (HALT not true) Cycle;
  
Cycle
  InputPhase;
  ProposalPhase;
  DecisionPhase;
  ApplicationPhase;
  OutputPhase;


ProposalPhase
  while (some i-supported productions are waiting to fire or retract)
    FireNewlyMatchedProductions;
    RetractNewlyUnmatchedProductions;

DecisionPhase
  for (each state in the stack, 
       starting with the top-level state)
  until (a new decision is reached)
    EvaluateOperatorPreferences; /* for the state being considered */
    if (one operator preferred after preference evaluation)
      SelectNewOperator;
    else                  /* could be no operator available or */
      CreateNewSubstate;  /* unable to decide between more than one */

ApplicationPhase
  while (some productions are waiting to fire or retract)
    FireNewlyMatchedProductions;
    RetractNewlyUnmatchedProductions;
\end{verbatim}

\insertcaption{A simplified version of the Soar algorithm.}
\label{fig:pseudocode}
\end{figure}

After quiescence is reached in the \emph{Proposal} phase, the \emph{Decision} phase ensues, which is the architectural selection of a single operator, if possible. Once an operator is selected, the \emph{Apply} phase ensues, which is practically the same as the \emph{Proposal} phase, except that any productions that apply the chosen operator (they test for the selection of that operator in their conditions) will now match and fire.

During the processing of these phases, it is possible that the preferences that resulted in the selection of the current operator could change.  Whenever operator preferences change, the preferences are re-evaluated and if a different operator selection would be made, then the current operator augmentation of the state is immediately removed. However, a new operator is not selected until the next decision phase, when all knowledge has had a chance to be retrieved. In other words, if, during the \emph{Apply} phase, the production(s) that proposed the selected operator retract, that \emph{Apply} phase will immediately end.

% ----------------------------------------------------------------------------
\section{Impasses and Substates}
\label{ARCH-impasses}
\index{decision procedure}
\index{impasse}
\index{goal!subgoal|see{subgoal}}
\index{result}

When the decision procedure is applied to evaluate preferences and determine the operator augmentation of the state, it is possible that the preferences are either incomplete or inconsistent. 
The preferences can be incomplete in that no \soar{acceptable} operators are suggested, or that there are insufficient preferences to distinguish among \soar{acceptable} operators. 
The preferences can be inconsistent if, for instance, operator \soar{A} is preferred to operator \soar{B}, and operator \soar{B} is preferred to operator \soar{A}. Since preferences are generated independently across different production instantiations, there is no guarantee that they will be consistent.

% ----------------------------------------------------------------------------
\subsection{Impasse Types}
\label{ARCH-impasses-types}

\index{impasse!tie}
\index{impasse!conflict}
\index{impasse!constraint-failure}
\index{impasse!no-change}
\index{tie impasse|see{impasse}}
\index{conflict impasse|see{impasse}}
\index{constraint-failure impasse|see{impasse}}
\index{no-change impasse|see{impasse}}

There are four types of impasses that can arise from the preference scheme.
\vspace{-12pt}

\begin{description}
\item[Tie impasse ---] 
	A \emph{tie} impasse arises if the preferences do not distinguish between two or more operators that have \soar{acceptable} preferences. If two operators both have \soar{best} or \soar{worst} preferences, they will tie unless additional preferences distinguish between them.
	\vspace{-8pt}
\item[Conflict impasse ---]
	A \emph{conflict} impasse arises if at least two values have conflicting better or worse preferences (such as \soar{A} is better than \soar{B} and \soar{B} is better than \soar{A}) for an operator, and neither one is rejected, prohibited, or \soar{require}d.
	\vspace{-8pt}
\item[Constraint-failure impasse ---]
	A \emph{constraint-failure} impasse arises if there is more than one \soar{require}d value for an operator, or if a value has both a \soar{require} and a \soar{prohibit} preference. These preferences represent constraints on the legal selections that can be made for a decision and if they conflict, no progress can be made from the current situation and the impasse cannot be resolved by additional preferences.
	\vspace{-8pt}
\item[No-change impasse ---]
	A \emph{no-change} impasse arises if a new operator is not selected during the decision procedure. There are two types of no-change impasses: state no-change and operator no-change:
	\vspace{-8pt} 
	\begin{description}
		\item[State no-change impasse ---] 
			A state no-change impasse occurs when there are no \soar{acceptable} (or \soar{require}) preferences to suggest operators for the current state (or all the \soar{acceptable} values have also been \soar{reject}ed). The decision procedure cannot select a new operator.\vspace{-8pt}
        \item[Operator no-change impasse ---] 
	        An operator no-change impasse occurs when either a new operator is selected for the current state but no additional productions match during the application phase, or a new operator is not selected during the next decision phase.
	\end{description}
	\index{state!no-change impasse|see{impasse}}
	\index{operator!no-change impasse|see{impasse}}
	\index{impasse!state no-change}
	\index{impasse!operator no-change}
\end{description}

There can be only one type of impasse at a given level of subgoaling at a time. Given the semantics of the preferences, it is possible to have a tie or conflict impasse and a constraint-failure impasse at the same time.  In these cases, Soar detects only the constraint-failure impasse.

The impasse is detected \textit{during} the selection of the operator, but happens \textit{because} one of the four problem-solving functions (described in section \ref{ARCH-functions}) was incomplete.

% ----------------------------------------------------------------------------
\subsection{Creating New States}

Soar handles these inconsistencies by creating a new state, called a \soarb{substate} in which the
goal of the problem solving is to resolve the impasse.  Thus, in the
substate, operators will be selected and applied in an attempt either to
discover which of the tied operators should be selected, or to apply the
selected operator piece by piece.  The substate is often called a
\emph{subgoal} because it exists to resolve the impasse, but is
sometimes called a substate because the representation of the subgoal in
Soar is as a state.
\index{subgoal}
\index{subgoal|see{goal}}
\index{impasse}

The initial state in the subgoal contains a complete description of the
cause of the impasse, such as the operators that could not be decided
among (or that there were no operators proposed) and the state that the
impasse arose in. From the perspective of the new state, the latter is
called the \soarb{superstate}. Thus, the superstate is part of the
substructure of each state, represented by the Soar architecture using
the \soar{superstate} attribute. (The initial state, created in the 0th
decision cycle, contains a \soar{superstate} attribute with the value of
\soar{nil} --- the top-level state has no superstate.)
\index{superstate}

The knowledge to resolve the impasse may be retrieved by any type of
problem solving, from searching to discover the implications of different
decisions, to asking an outside agent for advice. There is no \emph{a priori}
restriction on the processing, except that it involves applying operators to
states.
\index{subgoal}

\begin{figure}
\insertfigure{stack1}{7.75in}
\insertcaption{A simplified illustration of a subgoal stack.}
\label{fig:stack1}
\end{figure}

\index{goal!stack}
\index{stack|see{goal}}
In the substate, operators can be selected and applied as Soar attempts to
solve the subgoal. (The operators proposed for solving the subgoal may be
similar to the operators in the superstate, or they may be entirely
different.) While problem solving in the subgoal, additional impasses may be
encountered, leading to new subgoals.  Thus, it is possible for Soar to have a
\emph{stack} of subgoals, represented as states: Each state has 
a single superstate (except the initial state) and each state may have at most 
one substate. Newly created
subgoals are considered to be added to the bottom of the stack; the first
state is therefore called the \emph{top-level state}.\footnote{The
original state is the ``top'' of the stack because as Soar
runs, this state (created first), will be at the top of the computer screen,
and substates will appear on the screen below the top-level state.}  See
Figure \ref{fig:stack1} for a simplified illustrations of a subgoal stack.

Soar continually attempts to retrieve knowledge relevant to all goals in the
subgoal stack, although problem-solving activity will tend to focus on the
most recently created state. However, problem solving is active at
all levels, and productions that match at any level will fire.

% ----------------------------------------------------------------------------
\subsection{Results}
\label{ARCH-impasses-results}
\index{goal!result|see{result}}
\index{result}

In order to resolve impasses, subgoals must generate results that allow
the problem solving at higher levels to proceed.  The {\em results} of a
subgoal are the working memory elements and preferences that were
created in the substate, and that are also linked directly or indirectly
to a superstate (\emph{any} superstate in the stack). A preference or
working memory element is said to be created in a state if the
production that created it tested that state and this is the most recent
state that the production tested. Thus, if a production tests multiple
states, the preferences and working memory elements in its actions are
considered to be created in the most recent of those states (and is not
considered to have been created in the other states). The architecture
automatically detects if a preference or working memory element created
in a substate is also linked to a superstate.

These working memory elements and preferences will not be removed when
the impasse is resolved because they are still linked to a superstate,
and therefore, they are called the \textit{results of the subgoal}.  A
result has either i-support or o-support; the determination of support is
described below.

%A production that creates a result is illustrated in Figure \ref{fig:result}.
%The figure illustrates the result as a working memory element: 
%``\soar{new-attribute X1}''.

%\begin{figure}
%\insertfigure{result}{3.7in}
%\insertcaption{An abstract illustration of a production that creates a
%	result. In the figure, S2 is the lowest state in the subgoal stack
%	that is tested by the production, and the working memory element
%	is said to have been created in state S2.  }
%\label{fig:result}
%\end{figure}

%\begin{figure}
%\insertfigure{result-indirect}{3in}
%\insertcaption{An abstract illustration of a production that creates a
%	working memory element that indirectly becomes a result. S2 is the
%	lowest state in the subgoal stack that is tested by the production,
%	and the working memory element is said to be created in state S2. Some other
%	production instantiation creates the working memory element that links X2 to the
%	superstate.
%	}
%\label{fig:result-indirect}
%\end{figure}

A working memory element or preference will be a result if
its identifier is already linked to a superstate.
%(as illustrated inFigure \ref{fig:result})
A working memory element or preference can also become a result
indirectly if, after it is created and it is still in working memory or
preference memory, its identifier becomes linked to a superstate through
the creation of another result. For example, if the problem solving in a
state constructs an operator for a superstate, it may wait until
the operator structure is complete before creating an
\soar{acceptable} preference for the operator in the superstate. The
\soar{acceptable} preference is a result because it was created in the
state and is linked to the superstate (and, through the superstate, is
linked to the top-level state). The substructures of the operator then
become results because the operator's identifier is now linked to the
superstate. 
% An indirect result is illustrated in Figure \ref{fig:result-indirect}). 

\subsubsection*{Justifications: Determination of support for results}

\nocomment{Define justification as substate result up front.}

\index{result!support}
\index{i-support}
\index{o-support}
Some results receive i-support, while others receive o-support.  The
type of support received by a result is determined by the function it
plays in the superstate, and not the function it played in the state in
which it was created. For example, a result might be created through
operator application in the state that created it; however, it might
only be a state elaboration in the superstate. The first function would
lead to o-support, but the second would lead to i-support.

\index{justification}
\index{justification!creation}
In order for the architecture to determine whether a result receives i-support
or o-support, Soar must first determine the function that the working
memory element or preference plays
(that is, whether the result should be considered an operator application or
not). To do this, Soar creates a temporary production, called a
\textit{justification}. The justification summarizes the processing in the
substate that led to the result:

\vspace{-10pt}
\begin{description}
	\item[The conditions] of a justification are those working memory elements that exist in the superstate (and above) that were necessary for producing the result.  This is determined by collecting all of the working memory elements tested by the productions that fired in the subgoal that led to the creation of the result, and then removing those conditions that test working memory elements created in the subgoal.
	\vspace{-6pt}
	\item[The action] of the justification is the result of the subgoal.
\end{description} 

Soar determines i-support or o-support for the justification just as it
would for any other production, as described in Section
\ref{ARCH-prefmem-persistence}.  If the justification is an operator
application, the result will receive o-support.  Otherwise, the result
gets i-support from the justification. If such a result loses
i-support from the justification, it will be retracted if there is no
other support.

\index{justification!conditions}
\index{chunk!overgeneral}
\index{justification!overgeneral}
Justifications include any negated conditions that were in the original
productions that participated in producing the results, and that test for the
absence of superstate working memory elements. Negated conditions that test for
the absence of working memory elements that are local to the substate are not
included, which can lead to overgeneralization in the justification (see Section
\ref{CHUNKING-problems} on page \pageref{CHUNKING-problems} for details).

Justifications can also include operator evaluation knowledge that led to the
selection of the operator that produced the result.  For example, the conditions
of any production that creates a prohibit preference for an operator in the substate that was not selected will be backtraced through
and may produce additional conditions in the justification.  Moreover, if the
add-desirability-prefs learn setting is on, conditions from other preference
types (better, best, worse, worst indifferent) can be included as well.  For a more
complete description of how Soar chooses which desirability preferences to
include, see Section \ref{CDPS} on page \pageref{CDPS}.
\index{desirability preference} 
\index{preference!desirability}
\index{Context-Dependent Preference Set}

% ----------------------------------------------------------------------------
\subsection{Removal of Substates: Impasse Resolution}
%\label{elim-impa}
\label{ARCH-impasses-elimination}
\index{impasse!resolution}
\index{impasse!elimination}
\index{goal!termination}

Problem solving in substates is an important part of what Soar
\textit{does}, and an operator impasse does not necessarily indicate a
problem in the Soar program.  They are a way to decompose a complex
problem into smaller parts and they provide a context for a program to
deliberate about which operator to select.  Operator impasses are necessary, for
example, for Soar to do any learning about problem solving (as will be
discussed in Chapter \ref{CHUNKING}). This section describes how
impasses may be resolved during the execution of a Soar program, how
they may be eliminated during execution without being resolved, and some
tips on how to modify a Soar program to prevent a specific impasse from
occurring in the first place.  

\subsubsection*{Resolving Impasses}

An impasse is \textit{resolved} when processing in a subgoal creates
results that lead to the selection of a new operator for the state
where the impasse arose. When an operator impasse is resolved, Soar has
an opportunity to learn, and the substate (and all its substructure) is
removed from working memory.

Here are possible approaches for resolving specific types
of impasses are listed below:\vspace{-12pt}
\begin{description}
\item[Tie impasse ---]
	A tie impasse can be resolved by productions that create preferences
	that prefer one option (\soar{better}, \soar{best}, \soar{require}),
	eliminate alternatives (\soar{worse}, \soar{worst}, \soar{reject},
	\soar{prohibit}), or make all of the objects indifferent
	(\soar{indifferent}).\vspace{-8pt}
\item[Conflict impasse ---]
	A conflict impasse can be resolved by productions that create
	preferences to \soar{require} one option (\soar{require}), or
	eliminate the alternatives (reject, prohibit).\vspace{-8pt}
\item[Constraint-failure impasse ---]
	A constraint-failure impasse cannot be resolved by additional
	preferences, but may be prevented by changing productions so that they
	create fewer \soar{require} or \soar{prohibit} preferences.\vspace{-8pt}
\item[State no-change impasse ---]
	A state no-change impasse can be resolved by productions that create 
	\soar{acceptable} or \soar{require} preferences for operators.\vspace{-
8pt}
\item[Operator no-change impasse ---]
	An operator no-change impasse can be resolved by productions that
	apply the operator, changing the state so the operator proposal
	no longer matches or other operators are proposed and preferred.
\end{description}

\subsubsection*{Eliminating Impasses}

An impasse is resolved when results are created that allow progress to
be made in the state where the impasse arose.  In Soar, an impasse can be
\textit{eliminated} (but not resolved) when a higher level impasse is
resolved, eliminated, or regenerated.  In these cases, the impasse
becomes irrelevant because higher-level processing can proceed.  An
impasse can also become irrelevant if input from the outside world
changes working memory which in turn causes productions to fire that
make it possible to select an operator.  In all these cases, the impasse
is eliminated, but not ``resolved'', and Soar does not learn in this
situation.

\subsubsection*{Regenerating Impasses}

An impasse is \textit{regenerated} when the problem solving in the
subgoal becomes {\em inconsistent} with the current situation.  During
problem solving in a subgoal, Soar monitors which aspect of the
surrounding situation (the working memory elements that exist in
superstates) the problem solving in the subgoal has depended upon.  If
those aspects of the surrounding situation change, either because of
changes in input or because of results, the problem solving in the
subgoal is inconsistent, and the state created in response to the
original impasse is removed and a new state is created. Problem solving
will now continue from this new state.  The impasse is not ``resolved'',
and Soar does not learn in this situation.

The reason for regeneration is to guarantee that the working memory
elements and preferences created in a substate are consistent with
higher level states.  As stated above, inconsistency can arise when a
higher level state changes either as a result of changes in what is
sensed in the external environment, or from results produced in the
subgoal.  The problem with inconsistency is that once inconsistency
arises, the problem being solved in the subgoal may no longer be the
problem that actually needs to be solved.  Luckily, not all changes to a
superstate lead to inconsistency.

In order to detect inconsistencies, Soar maintains a 
\emph{dependency set} for every \\
subgoal/substate. 
The dependency set consists of all working
memory elements that were tested in the conditions of productions that
created o-supported working memory elements that are directly or
indirectly linked to the substate.  Thus, whenever such an o-supported
working memory element is created, Soar records which working memory
elements that exist in a superstate were tested, directly or indirectly
in creating that working memory element. \index{dependency-set} Whenever
any of the working memory elements in the dependency set of a substate
change, the substate is regenerated.

Note that the creation of i-supported structures in a subgoal does not
increase the dependency set, nor do o-supported results.  Thus, only
subgoals that involve the creation of internal o-support working memory
elements risk regeneration, and then only when the basis for the
creation of those elements changes.

\subsubsection*{Substate Removal}

Whenever a substate is removed, all working memory elements and
preferences that were created in the substate that are not
results are removed from working memory. In Figure \ref{fig:stack1},
state \soar{S3} will be removed from working memory when the impasse
that created it is resolved, that is, when sufficient preferences have
been generated so that one of the operators for state \soar{S2} can be
selected. When state \soar{S3} is removed, operator \soar{O9} will also be removed,
as will the acceptable
preferences for \soar{O7}, \soar{O8}, and \soar{O9}, and the
\soar{impasse}, \soar{attribute}, and \soar{choices} augmentations of state
\soar{S3}. These working memory elements are removed because they are no
longer linked to the subgoal stack. The acceptable preferences for
operators \soar{O4}, \soar{O5}, and \soar{O6} remain in working memory. They
were linked to state \soar{S3}, but since they are also linked to state
\soar{S2}, they will stay in working memory until \soar{S2} is removed (or
until they are retracted or rejected).

%-----------------------------------------------------
\subsection{Removal of Substates:  The Goal Dependency Set}
This subsection describes the Goal Dependency Set (GDS) with discussions 
on the motivation for the GDS and consequences of the GDS from a behavior 
developer/modeler's point of view. 

\subsubsection{Why the GDS was needed}

As a symbol system, Soar attempts to approximate the knowledge level
but will necessarily always fall short\cite{Newell90:UTC}.  We can
informally think of the way in which Soar falls short of the knowledge
level as its peculiar ``psychology.''  Those interested in using Soar
to model human psychology would like Soar's ``psychology'' to
approximate human psychology. Those using Soar to create agent
systems would like to make Soar's processing approximate the knowledge
level as closely as possible. However, Soar~7 had a number of
symbol-level ``quirks'' that appeared inconsistent with human
psychology and that made building large-scale, knowledge-based systems
in Soar more difficult than necessary.  Bob Wray's thesis 
\footnote{Robert E. Wray. \textit{Ensuring Reasoning Consistency in Hierarchical Architectures}. PhD thesis, University of Michigan, 1998.}
addressed many of these symbol-level problems
in Soar, among them logical inconsistency in symbol manipulations,
non-contemporaneous constraints in chunks \cite{Wray96:Compilation},
race conditions in rule firings and in the decision process, and
contention between original task knowledge and learned knowledge
\cite{Wray01:Resolving}.

The Goal Dependency Set implements a solution to logical
inconsistencies between persistent (o-supported) working memory
elements (WMEs) in a substate and its ``context''. The context
consists of all the WMEs in any superstates above the local
goal/state\footnote{This subsection will use ``state,'' not ``goal.''  At
	the kernel level, states are still called ``goals'' and ``goal'' is often
	still used to refer to states. As a result, a confusion in terminology results, 
	with ``\textbf{Goal} Dependency Set'' a specific example, even though ``goals'' 
	have not been an explicit, behavior-level Soar construct since Soar~6.}. In Soar, any
action (application) of an operator receives an o-support preference.
This preference makes the resulting WME persistent: it will remain in
memory until explicitly removed or until its local state is removed,
regardless of whether it continues to be justified.

Persistent WMEs are pervasive in Soar, because operators are the main
unit of problem solving. Persistence is necessary for taking any
non-monotonic step in a problem space. However, persistent WMEs also
are dependent on WMEs in the superstate context. The problem in
Soar prior to GDS, especially when trying to create a large-scale system\cite{Jones99:Automated}, is that the knowledge developer
must always think about which dependencies can be ``ignored'' and
which may affect the persistent WME. For
example, imagine an exploration robot that makes a persistent decision
to travel to some distant destination based, in part, on its power
reserves.  Now suppose that the agent notices that its power reserves
have failed.  If this change is not communicated to the state where
the travel decision was made, the agent will continue to act as if its
full power reserves were still available.

Of course, for this specific example, the knowledge designer can
encode some knowledge to react to this inconsistency. The fundamental
problem is that the knowledge designer has to consider \emph{all}
possible interactions between all o-supported WMEs and all contexts.
Soar systems often use the architecture's impasse mechanism to realize
a form of decomposition. These potential interactions mean that the
knowledge developer cannot focus on individual problem spaces in isolation when
creating knowledge, which makes knowledge development more difficult.
Further, in all but the simplest systems, the knowledge designer will
miss some potential interactions. The result is agents are that were
unnecessarily brittle, failing in difficult-to-understand,
difficult-to-duplicate ways.  

The GDS also solves the the problem of non-contemporaneous constraints
in chunks. A non-contemporaneous constraint refers to two or more
conditions that never co-occur simultaneously. An example might be a
driving robot that learned a rule that attempted to match ``red
light'' and ``green light'' simultaneously. Obviously, for functioning
traffic lights, this rule would never fire. By ensuring that local
persistent elements are always consistent with the higher-level
context, non-contemporaneous constraints in chunks are
\emph{guaranteed} not to happen.


The GDS captures context dependencies during processing, meaning the
architecture will identify and respond to inconsistencies
automatically.  The knowledge designer then does not have to consider
potential inconsistencies between local, o-supported WMEs and the
context.


\subsubsection{Behavior-level view of the Goal Dependency Set}

The following discussion covers what the GDS does, and how that impacts
production knowledge design and implementation.

\paragraph{Operation of the Goal Dependency Set:}


\begin{figure}
	\insertfigure{simple-ncc}{3in}
	\caption{Simplified Representation of the context dependencies (above the line), local os-upported WMEs (below the line), and the generation of a result.  Prior to GDS, this situation led to non-contemporaneous constraints in the chunk that generates {\bf 3}.}
	\label{'ncc'}
\end{figure}

Whenever a feature is created (added to working memory) in the Soar
architecture, that feature will persist for some time.  The
persistence of features may differ with respect to how long the
features remain in memory, and more importantly, what circumstances
cause the feature to be removed.  The Soar architecture utilizes
two primary types of persistence: i-support and o-support.

\index{production!instantiation}
The weakest persistence is instantiation support. An i-supported
feature exists in memory only as long as the production which lead to
the feature's creation remains instantiated. Thus, the WME depends
upon this production instantiation (and, more specifically, the
features the instantiation tests). When one of the conditions in the
production instantiation no longer matches, the instantiation is
retracted, resulting in the loss of the acceptable preference for the
WME.\footnote{Importantly, in a technical sense, the WME is only
	retracted when it loses instantiation support, not when the creating
	production is retracting.  For example, a WME could receive i-support
	from several different instantiations and the retraction of one would
	not lead to the retraction of the WME.  However, the the following
	generally discusses direct dependency unmediated by preferences,
	ignoring this complication for clarity.}  I-support is illustrated in
Figure~\ref{'ncc'}. A copy of {\bf A} in the subgoal, {\bf A$_s$}, is
retracted automatically when {\bf A} changes to {\bf A'}.  The
substate WME persists only as long as it remains justified by {\bf A}.
This is called ``instantiation support'' (i-support) in
Soar.

In the broadest sense, we can say that some feature $<$b$>$ is
``dependent'' upon another element $<$a$>$ if $<$a$>$ was used in the
creation of $<$b$>$, i.e., if $<$a$>$ was tested in the production
instantiation that created $<$b$>$. Further, a dependent change with
respect to feature $<$b$>$ is a change to any of its instantiating
features.  In Figure~\ref{'ncc'}, the change from {\bf A} to {\bf A'}
is a dependent change for feature {\bf 1} because {\bf A} was used to
create {\bf 1}.

In Soar, some features are insensitive to dependent changes. These
features are often referred to as ``persistent WMEs'' because, unlike
i-supported WMEs, they remain in memory until explicitly removed. Any feature created by the action of an operator
receives ``operator support.'' An o-supported feature remains in
memory until explicitly rejected (or until the superstructure to which
it is attached is removed). Removal is architecturally
independent of the WME's instantiating conditions.

The GDS provides a solution to the first problem. When {\bf A}
changes, the persistent WME {\bf 1} may be no longer consistent with
its context (e.g., {\bf A'}).  The specific solution is inspired by
the dependency analysis portion of the chunking algorithm. Whenever an o-supported WME is
created in the local state, the superstate dependencies of that new
feature are determined and added to the {\em goal dependency set}
(GDS) of that state. Conceptually speaking, whenever a working memory
change occurs, the dependency sets for every state in the context
hierarchy are compared to working memory changes. \textit{If a removed element 
is found in a GDS, the state is removed from memory (along with all existing
substructure).} The dependency set includes only dependencies for
o-supported features.  For example, in Figure~\ref{'gds'}, at time
$t_0$, because only i-supported features have been created in the
subgoal, the dependency set is empty.

\begin{figure}
	\insertfigure{gomor-o-support}{3in}
	\caption{The Dependency Set in Soar.}
	\label{'gds'}
\end{figure}


Three types of features can be tested in the creation of an
o-supported feature.  Each requires a slightly different type of
update to the dependency set.\vspace{-10pt}
\begin{enumerate}
	\item \textbf{Elements in the superstate:} WMEs in the superstate are added
	directly to the goal's dependency set. In Figure~\ref{'gds'}, the
	persistent subgoal item {\bf 3} is dependent upon {\bf A} and {\bf
		D}. These superstate WMEs are added to the subgoal's dependency set when
	{\bf 3} is added to working memory at time $t_1$. It does not matter
	that {\bf A} is i-supported and {\bf D} o-supported.
	\item \textbf{Local i-supported features:} Local i-supported features are not
	added to the goal dependency set.  Instead, the superstate WMEs that
	led to the creation of the i-supported feature are determined and
	added to the GDS. In the example, when {\bf 4} is created, {\bf A},
	{\bf B} and {\bf C} must be added to the dependency set because they
	are the superstate features that led to {\bf 1}, which in turn led to
	{\bf 2} and finally {\bf 4}. However, because item {\bf A} was
	previously added to the dependency set at $t_1$, it is unnecessary to
	add it again.
	\item \textbf{Local o-supported features:} The dependencies of a local
	o-supported feature have already been added to the state's GDS. Thus,
	tests of local o-supported WMEs do not require additions to the
	dependency set. In Figure~\ref{'gds'}, the creation of element {\bf
		5} does not change the dependency set because it is dependent only
	upon persistent items {\bf 3} and {\bf 4}, whose features had been
	previously added to the GDS.
\end{enumerate}

Any change to the current dependency set will cause
the retraction of all subgoal structure. Thus, any time after time
$t_1$, either the {\bf D} to {\bf D'} or {\bf A} to {\bf A'}
transition would cause the removal of the entire subgoal. The {\bf E}
to {\bf E'} transition causes no retraction because {\bf E} is not in
the goal's dependency set.

\paragraph{The role of the GDS in agent design:}


The GDS places some design time constraints on operator implementation.
These constraints are:
\begin{itemize} \vspace{-10pt}
	\item Operator actions that are used to remember a previous state/situation should be asserted in the top state \vspace{-8pt}
	\item All operator elaborations should be i-supported \vspace{-8pt}
	\item Any operator with local actions should be designed to be re-entrant
\end{itemize}

Soar says any operator effect is o-supported, regardless of whether
that assertion is entailed by the current situation, or whether it
reflects an assumption about it. The GDS adds additional (needed)
constraint.  Because any context dependencies for subgoal, o-supported
assertions will be added to the GDS, the developer must decide if an
o-supported element should be represented in a substate or the top
state.

This decision is straightforward if the functional role of the
persistent element is considered. Four important capabilities that
require persistence are: \vspace{-8pt}
\begin{enumerate}
	
	\item \textbf{Reasoning hypothetically:} Some assertions may need to
	reflect hypothetical states.  Such assertions are ``assumptions''
	because a hypothetical inference cannot always be grounded in the
	current context.  In other problem solvers with truth maintenance,
	only assumptions are persistent.
	\vspace{-8pt}
	\item \textbf{Reasoning non-monotonically:} 
	Sometimes the result of an inference changes one of the assertions on
	which the inference is dependent.  As an example, consider the task of
	counting.  Each newly counted item replaces the old value of the
	count. 
	\vspace{-8pt}
	\item \textbf{Remembering:} 
	Agents oftentimes need to remember an external situation or stimulus,
	even when that perception is no longer available.  
	\vspace{-8pt}
	\item \textbf{Avoiding Expensive Computations:}  In some situations,
	an agent may have the information needed to assert some belief in a
	new world state but the expense of performing the computation
	necessary for the assertion, given what is already known, makes the
	computation avoidable.  For example, in dynamic, complex domains,
	determining when to make an expensive calculation is often formulated
	as an explicit agent task\cite{Jones99:Automated}.
\end{enumerate}

When remembering or avoiding an expensive computation, the
agent/designer is making a commitment to retain something even though
it might not be supported in the current context. \textbf{These
	WMEs should be asserted in the top state. \emph{For many Soar systems,
		especially those focused on execution in a dynamic environment, 
		most o-supported elements will need to be stored on the top state.}} 

For any kind of local, non-monotonic reasoning about the context
(counting, projection planning), features should be stored locally.
When a dependent context change occurs, the GDS interrupts the
processing by removing the state. While this may seem like a severe
over-reaction, formal and empirical analysis have suggested that this
solution is less computationally expensive than attempting to identify
the specific dependent assumption \cite{Wray03:Ensuring}.

%-----------------------------------------------------
\subsection{Soar's Cycle: With Substates}
\label{ARCH-decision-substates}

When there are multiple substates, Soar's cycle remains basically the
same but has a few minor changes.  


The first change is that during the decision procedure, Soar will detect
impasses and create new substates.  For example, following the proposal
phase, the decision phase will detect if a decision cannot be made given
the current preferences.  If an impasse arises, a new substate is
created and added to working memory.  

%The decision procedure will detect an operator no-change impasse as soon
%as an operator is selected and added to working memory by checking to
%see whether or not productions will create o-supported actions during
%the next application phae.  If no o-supported actions will be created,
%the decision procedure will immediately create an operator no-change
%impasse, and then proceed to output, input, and so on.  In these cases,
%the operator no-change is made in the same decision as the operator
%selection.  There will be cases where the operator no-change happens on
%the following decisions, such as when there are o-supported productions
%that will fire, but do not lead to a change in the selected operator.

The second change when there are multiple substates is that at each
phase, Soar goes through the substates, from oldest (highest) to newest
(lowest), completing any necessary processing at that level for that
phase before doing any processing in the next substate.  When firing
productions for the proposal or application phases, Soar processes the
firing (and retraction) of rules, starting from those matching the
oldest substate to the newest.  Whenever a production fires or retracts,
changes are made to working memory and preference memory, possibly
changing which productions will match at the lower levels (productions
firing within a given level are fired in parallel -- simulated).
Productions firings at higher levels can resolve impasses and thus
eliminate lower states before the productions at the lower level ever
fire.  Thus, whenever a level in the state stack is reached, all
production activity is guaranteed to be consistent with any processing
that has occurred at higher levels.


% ----------------------------------------------------------------------------
\section{Learning Procedural Knowledge}
\label{ARCH-learning} 
\index{chunking}
\index{subgoal}

When an operator impasse is resolved, it means that Soar has, through problem 
solving, gained access to knowledge that was not readily available before. Therefore,
when an impasse is resolved, Soar has an opportunity to learn, by summarizing
and generalizing the processing in the substate.

\index{chunk}
One of Soar's learning mechanisms is called \textit{chunking} 
(See chapter \ref{CHUNKING} for more information); it attempts to
create a new production, called a chunk. The conditions of 
the chunk are the elements of the state that (through some chain of 
production firings) allowed the impasse to be resolved; the action of the 
production is the working memory element or preference that resolved the impasse
(the result of the impasse). The conditions and action are variablized so that this
new production may match in a similar situation in the future and
prevent an impasse from arising. 

\index{justification}
Chunks are very similar to justifications in that they are both
formed via the backtracing process and both create a result in their
actions. However, there are some important distinctions:
\vspace{-12pt}

\begin{enumerate}
\item Justifications disappear as soon as its conditions no longer match. 
\vspace{-8pt}
\item Chunks contain variables so that they may match working memory in other situations; justifications are similar to an instantiated chunk.
\end{enumerate}




% ----------------------------------------------------------------------------
\section{Input and Output}
\label{ARCH-io}	%\label{ch-abst-symb-inpu}
\index{I/O}

Many Soar users will want their programs to interact with a real or simulated
environment. For example, Soar programs may control a robot, receiving sensory
inputs and sending command outputs. Soar programs may also interact with
simulated environments, such as a flight simulator. Input is viewed as
Soar's perception and output is viewed as Soar's motor abilities.

When Soar interacts with an external environment, it must make use of
mechanisms that allow it to receive input from that environment and to effect
changes in that environment; the mechanisms provided in Soar are called
\textit{input functions} and \textit{output functions}.

\begin{description}
\item[Input functions] add and delete elements from working memory in response
	to changes in the external environment.
\item[Output functions] attempt to effect changes in the external
	environment. 
\end{description}

Input is processed at the beginning of each execution cycle and output
occurs at the end of each execution cycle.

For instructions on how to use input and output functions with Soar, refer to the
\textit{SML Quick Start Guide}.




