%---------------------------------------------------------------------------
% Description       : LaTeX source for Soar 9 User's Manual
% Author(s)         : J. Laird, C. Congdon
% Organization      : University of Michigan
%---------------------------------------------------------------------------
%---------------------------------------------------------------------------
%
% TO PRINT A `FINAL' VERSION OF THIS MANUAL, here's the drill:
%   (not necessarily `final', but to be handed out to folks)
%
% 1. comment out any `includeonly's in this file.
% 2. change 'include' to 'input' for appendices -- may fix TOC problem (see
% 	NOTES below).
% 3. run latex repeatedly and fix source files until you don't get any more
% 	errors. (a few font errors seem to be unavoidable).
%       watch for ``overfull'' messages, which mean that your lines are too
%       long. ``underfull'' can often be ignored.
% 3b. run makeindex
% 3c. edit manual.ind for special characters, such as & 
%	I think the best approach here is probably to put \verb+ + around the
%	special chars.
%	Note that if we put the '\verb+ +' in the index command, the chars
%       might be indexed under \verb...which would maybe be okay, now that I
%       think about it. (should still be alphabetically at the front of the
%       manual.) 
% 3d. run latex yet again
% 4. run dvips and use ghostview to have a peek.
% 5. fix errors.
%     (note that the easiest way to remove all the comments, if that's really
%     what you want to do, is to redefine `comment' and 'betacomment' in this
%     file to be the same as 'nocomment')
% 6. pay particular attention to the toc (table of contents) and the function
% 	summary and index.
%    * the appendices often get screwed up, requiring a hand-edit of the toc
%       file 
%    * the commands in the toc look better if you remove the args to
%       user-interface commands (again, by hand)
%    * if any of the commands in the user-interface has changed, chances are
%    	good that the function summary is no longer current. You can edit
%       manual.glo, as described functions.tex, or just make individual
%       changes.
%    NOTE: there is no point in editing the TOC unless you've got everything
%    else perfect. You have to edit it one run before the last one because
%    latex will overwrite it. (It's sometimes a good idea to save an edited
%    toc if you have to make minor changes that will not change page
%    numbering.)
% 7. If everything looks fine, and you've just re-edited the manual.toc file
%    for the last time, run latex and dvips for the last time.
%
% TIPS ON DEBUGGING LATEX (random problems I'm having as I try to wrap this up)
% * DO NOT edit and save a file (including this one) while you are running
%   latex on said file. Latex seems to have a pointer into the file, and
%   saving can cause huge hunks of text to get skipped or duplicated. And
%   you'll get bizarre errors.
% * If you get an ``undefined reference'' error for a label you know you've
%   defined, the most likely cause is that you've forgotten an \end{verbatim}
%   line in the previous section. The next possibility is that you opened a
%   comment environment and closed it way later than you thought.
% * If the page count seems to suddenly shoot up, its a good bet that you
%   forgot an \end{verbatim} somewhere. If it suddenly drops, you probably
%   forgot to close a comment environment.
%
% OTHER CHANGES MADE IN THIS VERSION OF THE MANUAL:
%  * set odd and even margins differently for twosided printing and binding
%  * created \soar mode, which is just \texttt, but will allow us to change
%    this at whim by editing this file only, rather than search and replaces
%    all over creation.
%  * other playing around with typestyles, trying to get bold typewriter font.
%    I think I've finally found one; sansserif font is redefined to the new
%    typewriter font (probably available only at UofM?), which has italic and
%    bold styles.
%  * created \carat, which puts the ^ character into a non-verbose environment
%    (e.g. soar mode). Might as well note here that you can't easily use
%    verbose mode in the middle of text because it won't add line endings in
%    the appropriate places (you'll just get overfulls). That's why soar mode
%    is needed.
%  * also created \tild for adding tildes in
%  * the function summary and index is now tied to the user-interface chapter.
%    It doesn't have to be in the future, but while the command set is in
%    flux, this seems the best way to keep things consistent. I'm using
%    glossary commands (in the user interface chapter) to write to the
%    manual.glo file, which I then edit into the function summary. See the
%    function summary file for instructions.
%
% NOTES:
%  * to print manual without the .ps files for figures, edit this file only.
%    comment out the definition of \insertfigures and uncomment the other
%    version of \insertfigures, which will just leave whitespace
%  * likewise, all ``comments'' in the text can be removed by redefining the
%    \comment environment to be the same as the \nocomment environment below,
%    rather than searching through all files
%  * to print only some of the chapters, use the \includeonly command; see
%    below 
%  * for final version of manual, have to edit the TOC by hand:
%    1. word ``APPENDIX'' appears the line *after* the listing of the first
%       appendix 
%    2. looks a heck of a lot better if we remove the command arguments from
%       the TOC listing for chapter 6. (This can be done in emacs with a
%       replace-rexexp.) (Replace-regexp ``.\texttt.{.*}}'' with ``}'')
%---------------------------------------------------------------------------
\documentclass[12pt,twoside,named]{book}
%\documentclass[12pt,twoside,final]{book}
% final restricts line overruns, but might be default
\usepackage{named}		% for bibliography (named.sty is in this dir)
\usepackage{makeidx}		% for index
\usepackage{epsfig}		% for figures (was 'psfig')
\usepackage[colorlinks]{hyperref}     % for hyperlinks
\usepackage[figure,figure*]{hypcap}   % to correct anchor placement for figures
%\usepackage{amstext}
%\usepackage{verbatim}
%\usepackage{subfigure}
%\usepackage{rotating}
%\usepackage{epic}
%\usepackage{lscape}
\usepackage{tabularx}
\usepackage{longtable}
\usepackage{amsmath}
\usepackage{booktabs}

\providecommand{\tightlist}{%
   \setlength{\itemsep}{0pt}\setlength{\parskip}{0pt}}

%---------------------------------------------------------------------------
% setup the environment
%---------------------------------------------------------------------------
\setlength{\topmargin}{-0.5in}		
%\setlength{\oddsidemargin}{.5in}	% for binding, odd and even have
%\setlength{\evensidemargin}{0in}	% different margins
%\setlength{\textwidth}{6in}             % Gives 1 in. side margins; 1.5in on inside edge 
\setlength{\oddsidemargin}{0in}
\setlength{\evensidemargin}{0in}
\setlength{\textwidth}{6.5in}
\setlength{\textheight}{9.2in}		% Gives 1 inch on top and bottom.

\setlength{\parskip}{8pt}
\setlength{\parindent}{0pt}

% try these for fixing spacing in itemize
%  --> none of these seem to work
\setlength{\itemsep}{0pt}	% space between items
\setlength{\parsep}{0pt}	% space between paragraphs w/in an item
\setlength{\topsep}{0pt}	% between prec. text and first item
\setlength{\partopsep}{0pt}	% extra space, if list env is preceeded by blank line

%changing those last two had bad side effects; also changed spacing after
%verbatim environments (but didn't always fix list environments; maybe because
%it is a rubber space)

%\setlength{\topskip}{-20pt}	% have no idea what this will change -->
				%didn't work

% write an .idx file; this may need to be edited to be incorporated
\makeindex
\renewcommand{\seealso}[1]{\textit{see also} #1}

% write an .glo file; this will need to be edited to be incorporated
% --> used for the function summary, NOT for the glossary
\makeglossary

% to play with the numbering depth (defaults for both are 2 in book.cls)
\setcounter{secnumdepth}{3}
\setcounter{tocdepth}{2}

%---------------------------------------------------------------------------
% specify an alternate directory for the ps figures
%---------------------------------------------------------------------------
%\psfigsearchpath{Figures}   
\graphicspath{{Figures/}}   % at Bowdoin

%---------------------------------------------------------------------------
% define new commands
%---------------------------------------------------------------------------

% soar version information
\newcommand{\SoarVersionMajor}{9}
\newcommand{\SoarVersionMinor}{6}
\newcommand{\SoarVersionRevision}{0}


% to add functions to .glo file:
\newcommand{\funsum}[2]{\glossary{#1 & #2 & \textit{#1}}}

% to change spacing 
\newcommand{\dou}{\renewcommand{\baselinestretch}{1.6}\small\normalsize}
\newcommand{\singl}{\renewcommand{\baselinestretch}{1.0}\small\normalsize}
\newcommand{\halfs}{\renewcommand{\baselinestretch}{0.5}\small\normalsize}


% to insert figures (first option allows blank space in lieu of actual .ps file)
% insertcaption inserts a rule after the caption, which is handy in separating
% caption from text

%\newcommand{\insertfigure}[2]{\vspace{2in}}
\newcommand{\insertfigure}[2]{		%\protect\rule[0.0in]{6in}{.01in}
			   %   \capstart   % works with hypcap package, figures must have captions
			      \begin{center}
                              \ \epsfig{file=#1,height=#2}
                              \end{center}}
\newcommand{\inserttwofigs}[3]{\begin{center}
                              \ \psfig{figure=#1.ps,height=#3} \psfig{figure=#2.ps,height=#3}
                              \end{center}}
\newcommand{\insertthreefigs}[4]{\begin{center}
                              \ \psfig{figure=#1.ps,height=#4} \psfig{figure=#2.ps,height=#4}  \psfig{figure=#3.ps,height=#4}
                              \end{center}}

\newcommand{\insertcaption}[1]{\singl\caption{\small#1}
                                        \protect\rule[0.0in]{6in}{.01in}   }

% to include in-text comments (for drafts)
%\newcommand{\comment}[1]{\begin{quote}{\small {\em comment:} \ #1} \end{quote}}
\newcommand{\nocomment}[1]{}
\newcommand{\betacomment}[1]{\begin{quote}{\small {\em Version 7.0.3 comment:} \ #1} \end{quote}}

% To take all comments out, just redefine `comment' and `betacomment' like 'nocomment':
 \newcommand{\comment}[1]{}
% \newcommand{\betacomment}[1]{}

%\newcommand{\nocomment}[1]{\begin{quote}{\small {\em Old comment:} \ #1} \end{quote}}
%\newcommand{\cbc}[1]{\begin{quote}{\small {\em CBC comment:} \ #1} \end{quote}}
%\newcommand{\umcomment}[1]{\begin{quote}{\small {\em UMich comment: \\} \ #1} \end{quote}}

\renewcommand{\cite}[1]{}
%---------------------------------------------------------------------------
% aesthetics: typefaces and special characters
%---------------------------------------------------------------------------

% Change sanserif font to a typewriter font (this is different from the
% default typewriter font, which is a bit narrower, but doesn't have a bold
% weight -- with this command, \textsf will now be a typewriter font.
\renewcommand{\sfdefault}{pcr}    % new typewriter font
%\renewcommand{\sfdefault}{phv}   % helvetica
%\renewcommand{\sfdefault}{ppl}   % palatino, which I rather like

%\renewcommand{\sfdefault}{pcr}  % new typewriter font
%\renewcommand{\ttdefault}{pcr}  % new typewriter font

% to write soar code into text -- soar bold is now a bold tt font.
\newcommand{\soar}[1]{\texttt{#1}}
%\newcommand{\soarb}[1]{\texttt{\textbf{#1}}}   % can't get bold typewriter?
\newcommand{\soarb}[1]{\textsf{\textbf{#1}}}
\newcommand{\soarit}[1]{\textsf{\textit{#1}}}	% italic typewriter
\newcommand{\soarbit}[1]{\textsf{\textit{\textbf{#1}}}}	% bold italic typewriter

%try to get just bold to be the new typewriter font, while leaving regular tt alone
%....not sure how to do this, since it involves two commands.

\def\btt{\fontfamily{cmttss}\fontseries{b}\fontshape{n}\fontsize{12}{13.6}\selectfont}

%carat and tilde symbols
\newcommand{\carat}{\ensuremath{^\wedge}}  %This is how to get a carat symbol
					   %  (^) in the text
\newcommand{\tild}{\ensuremath{\sim}}      %This is how to get a tilde (~) in
					   %  the text 
\newcommand{\cmark}{\ensuremath{\surd}}    %This is how to get a checkmark


%section dividers
\newcommand{\divider}{ \hfil\rule[-2.0ex]{0.95\linewidth}{1.25pt}\hfil }
\newcommand{\subdivider}{ \hfil\rule{0.7\linewidth}{1pt}\hfil }

%---------------------------------------------------------------------------
% aesthetics: fill the page better with illustrations
% I believe the top \def's are redundant with the \renewcommands
%---------------------------------------------------------------------------
% to fill pages better when there's figures or tables (aka 'floats')
%\def\topfraction{1.0}        %maximum fraction of floats at the top of the page
%\def\bottomfraction{1.0}     %ditto, for the bottom of the page
%\def\textfraction{0}         %minimum fraction of text (--> 100% floats is okay)
%\def\floatpagefraction{0.8}  % if a page is full of floats, it'd better be FULL


% I don't like it when a figure floats on a page with no text, so fiddle with
% these parameters to change this. A trial and error sort of thing; the first
% command didn't seem to do anything. The percentages may be excessive; I
% don't know what the defaults are
\renewcommand{\topfraction}{1.0}          %up to 1.0 of a page can be a figure
\renewcommand{\bottomfraction}{1.0}       %up to 1.0 of a page can be a figure
\renewcommand{\textfraction}{0.0}         %up to 1.0 of a page can be text
\renewcommand{\floatpagefraction}{0.9}    %minimum of .9 of a page for floats only


\setcounter{topnumber}{4}		% up to 4 floats on a page
\setcounter{bottomnumber}{4}		% up to 4 floats on a page


%the distance between a figure and the text on the page (probably in addition
%  to parsep)
\renewcommand{\textfloatsep}{10pt}


%---------------------------------------------------------------------------
% aesthetics: don't use white space to fill to bottom of page
%---------------------------------------------------------------------------
\raggedbottom

%---------------------------------------------------------------------------
% aesthetics: set second level of itemize to be something other than dashes
%  (just playing around for now)
%---------------------------------------------------------------------------
%\renewcommand{\labelitemii}{$\triangleright$}
%\renewcommand{\labelitemii}{$\diamond$}
\renewcommand{\labelitemii}{$\star$}

%---------------------------------------------------------------------------
% I may need to use this to solve the headers problem
%   e.g. preface has header ``list of figures'' and function summary has
%   header ``bibliography''. I suspect that what needs to be redefined in
%   those situations is 'leftmark' and 'rightmark' and not the head itself.
%   Try to change these with renewcommands...
%     \def\@evenhead{\thepage\hfil\slshape\leftmark}%
%     \def\@oddhead{{\slshape\rightmark}\hfil\thepage}%
% Neither of the following (three) approaches work -- first two led to
% 'preface' being the header throughout the manual; third did nothing
%     \renewcommand\leftmark{\textit{PREFACE}}
%     \renewcommand\rightmark{\textit{PREFACE}}
%     \def\leftmark{\textit{PREFACE}}
%     \def\rightmark{\textit{PREFACE}}
%     \def\@evenhead{\thepage\hfil\slshape{PREFACE}}
%     \def\@oddhead{{\slshape{PREFACE}}\hfil\thepage}
%	[renewcommand didn't work with evenhead and oddhead]
%---------------------------------------------------------------------------


%---------------------------------------------------------------------------
% includeonly's, for drafts
%---------------------------------------------------------------------------

%\includeonly{m-preface8,m-intro8,m-architecture8}

%\includeonly{m-multiple,m-advanced}
%\includeonly{m-intro8}

%\includeonly{interface}
%\includeonly{m-preface8}

%\includeonly{m-preface8,m-architecture8}
%\includeonly{m-functions}

%\includeonly{a-grammars}

%---------------------------------------------------------------------------
% uncomment the next line if you manually edit the toc and index files.
%\nofiles          % don't overwrite the toc and ind files
%---------------------------------------------------------------------------
% BEGIN
%---------------------------------------------------------------------------
\begin{document}

\bibliographystyle{named}
%\pagestyle{empty}			% looks to be overwritten below

%----------------------------------------------------------------------------
% Title page
%----------------------------------------------------------------------------


\begin{titlepage}
\vspace{1.5in}

\begin{center}
\begin{huge}
	The Soar User's Manual \vspace{10pt} \\
	Version \SoarVersionMajor.\SoarVersionMinor.\SoarVersionRevision \vspace{20pt} \\
\end{huge}
\begin{large}
	\nocomment{Edition 1}
\end{large} \vspace{36pt}


\begin{large}
	John E. Laird and Clare Bates Congdon  \\
         User interface sections by Karen J. Coulter \\
         Soar 9 Modules by Nate Derbinsky and Joseph Xu \\
         Additional contributions by \\ Mazin Assanie, Preeti Ramaraj, Bryan Stearns, and Steven Jones \\
    \vspace{10pt}
	Computer Science and Engineering Department\\
	University of Michigan \vspace{.3in} \\

\end{large}
	Draft of:
	\today

\end{center}


\vspace*{0pt plus 1filll}
	{\em
	Errors may be reported to John E. Laird (laird@umich.edu)\\
        \\
	}
	Copyright \copyright\ 1998 - \the\year, The Regents of the University of Michigan
\vspace{.1in}

Development of earlier versions of this manual were supported under
contract N00014-92-K-2015 from the Advanced Systems Technology Office of
the Advanced Research Projects Agency and the Naval Research Laboratory,
and contract N66001-95-C-6013 from the Advanced Systems Technology
Office of the Advanced Research Projects Agency and the Naval Command
and Ocean Surveillance Center, RDT\&E division.

	%\vspace*{0pt plus 1filll}
\comment{
\newpage
\setlength{\parskip}{3pt}

\nocomment
	{\em This is a draft version of the manual; there are lots of little
	changes and a few big changes still to be completed....Some of the
	undone things:
	\begin{enumerate}
	\item all indexing needs to be redone
	\item compare index against glossary to see if I left anything out.
	\item change format for user-interface syntax -- use italics to
		specify args that are variable.
	\item have to regenerate a lot of the examples for the user-interface
		chapter. 
	\item the commands in the user-interface chapter may or may not
		correspond to the final release
	\item appendices are still on the sketchy side... getting better \\
		grammars and o-support definitely still need attention
	\item Needs a few more figures
	\item remember to check for consistency, such as upper/lowercase in
		section headings. (and i-support vs. I-support.) Looking over
		TOC and List of Figures would be a good idea.
	\item impasses are resolved; preferences are evaluated -- check for
		consistency 
	\item watch the difference between productions/operators making
		changes to the state and suggesting changes
	\item it would be nice to add a brief troubleshooting guide \\
		for example, to note what happens when someone is running
		``soar'' (rather than ``soartk''), but tries to load Tk code.
	\end{enumerate}


	Fri Jun 20 16:22:27 1997, Major undone things:

	\begin{enumerate}
	\item Haven't updated learning chapter
	\item Not sure how much ``advanced'' chapter has changed in the source
		since the last printing (leaving this chapter for next pass)

	\item Haven't incorporated comments from Frank and Aladin
	\item should, perhaps, include them in acknowledgements?

	\item indexing

	\item appendices

	\end{enumerate}

\nocomment{
	Things that I think are done:
	\begin{enumerate}
	\item Too much of the Tcl stuff got removed (moved into the ``advanced
		manual''). Added ``advanced'' chapter, but haven't written Tcl
		section yet.
	\item blocks-world task has become a big mess -- too many versions
		have been used for different things: \begin{enumerate}
		\item create THE version of the blocks-world to be used with
			this manual (or possibly two versions, one with
			subgoaling?) 
		\item redo figures
		\item include as an appendix
		\item add to distribution
		\end{enumerate}
	\item subgoal stack illustration is from old blocks-world task and
		should be updated
	\item need a brief description of SDE and pointer to docs
	\item also need pointers to tutorial and advanced manual

	\item check double and single quotes -- not always in latex format.
	\item ``evaluate preferences'', not ``resolve'' (impasses are
		resolved; preferences are evaluated). Exception: I left
		``resolved'' for preferences in just a couple of places, to
		mean that the preferences were evaluated and were not
		contradictory. Similarly, they cannot be resolved if they are
		contradictory.
 	\item change 'object' attribute to 'block' in all examples and
		illustrations -- no, change it to `thing'
	\item make sure that command names shown in examples in early chapters
		are consistent with actual command names -- go/run,
		load/source, etc.
	\item operators are not distinguished objects -- the operator
		augmentation of the state is distinguished (not entirely true
		-- operators ARE distinguished, for example, because their
		preferences are not evaluated until the decision phase)
	\item ``everything retracts'', is the impression we give early on and
		then later, we amend this. Not a great idea. From the
		beginning, make it clear that some retract and some don't
	\item use ``substate'' more often, as in ``soar creates a new
		substate''
	\item don't call them databases; call them memories
	\item we've changed the language for talking about preferences, and
	this has not been consistently incorporated. For example, we're trying
	to say that a preference has i-support or o-support, not that a
	production has i-support or o-support. Also a preference may be an
	operator proposal, operator application, operator termination, or
	elaboration, but productions aren't those four types because
	productions can create multiple preferences that fulfill different
	roles.
	
	\end{enumerate}
	}	 
	}
        }
\normalsize
\setlength{\parskip}{8pt}

\end{titlepage}

% ----------------------------------------------------------------------------
% Table of contents and preface
% ----------------------------------------------------------------------------
\cleardoublepage
\pagestyle{headings}
\pagenumbering{roman}
\setlength{\parskip}{0pt}	% try this for condensing TOC
\tableofcontents

\cleardoublepage

\addcontentsline{toc}{chapter}{Contents}

\listoffigures            	
\cleardoublepage
\setlength{\parskip}{8pt}

%\cleardoublepage
%\addcontentsline{toc}{chapter}{Preface}

%\markboth{PREFACE}{PREFACE}
%\chaptermark{WORKING?}
%\include{m-preface8}

% ----------------------------------------------------------------------------
% Body of document
%
% This version of the manual is primarily the intro and chapters 4, 5, and 8
%   of the old manual (SLCM syntax, chunking, encoding a task)
% leave out appendices for now
% ----------------------------------------------------------------------------
\cleardoublepage
\pagenumbering{arabic}

% ----------------------------------------------------------------------------
\typeout{--------------- INTROduction ---------------------------------------}
\chapter{Introduction}
\label{INTRO}

Soar has been developed to be an architecture for constructing general
intelligent systems. It has been in use since 1983, and has evolved through
many different versions. This manual documents the most current of these:
Soar, version \SoarVersionMajor.\SoarVersionMinor.\SoarVersionRevision.

Our goals for Soar include that it is to be an architecture that can: \vspace{-12pt}

\begin{itemize} 
\item be used to build systems that work on the full range of tasks expected
	of an \linebreak intelligent agent, from highly routine to extremely difficult,
	open-ended problems;\vspace{-6pt}
\item represent and use appropriate forms of knowledge, such as procedural,
	declarative, episodic, and possibly iconic;\vspace{-6pt}
\item employ the full range of problem solving methods;\vspace{-6pt}
\item interact with the outside world; and\vspace{-6pt}
\item learn about all aspects of the tasks and its performance on those tasks.
\end{itemize} 

In other words, our intention is for Soar to support all the capabilities
required of a general intelligent agent. Below are the major principles that
are the cornerstones of Soar's design:  \vspace{-12pt}

\begin{enumerate} 
\item The number of distinct architectural mechanisms should be minimized.
        Classically Soar had a single representation of permanent knowledge
        (productions), a single representation of temporary knowledge (objects
        with attributes and values), a single mechanism for generating goals
        (automatic subgoaling), and a single learning mechanism (chunking).
        It was only as Soar was applied to diverse tasks in complex environments that 
        we found these mechanisms to be insufficient and have recently added new 
        long-term memories (semantic and episodic) and learning mechanisms
        (semantic, episodic, and reinforcement learning) to extend Soar agents
        with crucial new functionalities.
\vspace{-6pt}

\item All decisions are made through the combination of relevant knowledge at
        run-time.  In Soar, every decision is based on the current
        interpretation of sensory data and any relevant knowledge retrieved
        from permanent memory.  Decisions are never precompiled into
        uninterruptible sequences.
\end{enumerate}


% ----------------------------------------------------------------------------
% ----------------------------------------------------------------------------
\section{Using this Manual}

\nocomment{check that this describes the final form of the manual}

We expect that novice Soar users will read the manual in the order it is
presented: 

\begin{description}
\item[Chapter \ref{ARCH} and Chapter \ref{SYNTAX}] describe Soar from
different perspectives: \textbf{Chapter \ref{ARCH}} describes the Soar
architecture, but avoids issues of syntax, while \textbf{Chapter \ref{SYNTAX}}
describes the syntax of Soar, including the specific conditions and actions
allowed in Soar productions.

\item[Chapter \ref{CHUNKING}] describes chunking, Soar's
mechanism to learn new procedural knowledge.  Not all users will make use of 
chunking, but it is important to know that this capability exists.

\item[Chapter \ref{RL}] describes reinforcement learning (RL), a mechanism
by which Soar's procedural knowledge is tuned given task experience.
Not all users will make use of RL, but it is important to know that this capability exists.

\item[Chapter \ref{SMEM} and Chapter \ref{EPMEM}] describe Soar's long-term declarative
memory systems, semantic and episodic. Not all users will make use of these mechanisms, 
but it is important to know that they exist.

\item[Chapter \ref{INTERFACE}] describes the Soar user interface --- how the
user interacts with Soar. The chapter is a catalog of user-interface commands,
grouped by functionality.  The most accurate and up-to-date information on the syntax of the 
Soar User Interface is found online, at the Soar web site, at

\hspace{2em}\soar{\htmladdnormallink{http://soar.eecs.umich.edu/articles/articles/documentation/73-command-line-help}{http://soar.eecs.umich.edu/articles/articles/documentation/73-command-line-help}}.

\end{description}

Advanced users will refer most often to Chapter \ref{INTERFACE}, flipping back
to Chapters \ref{ARCH} and \ref{SYNTAX} to answer specific questions.

There are several appendices included with this manual: 
\begin{description}

%\item[Appendix \ref{GLOSSARY}] is a glossary of terminology used in this manual.

\item[Appendix \ref{BLOCKSCODE}] contains an example Soar program for a simple
version of the blocks world. This blocks-world program is used as an example
throughout the manual.

%\item[Appendix \ref{USING}] is an overview of example programs currently available
%(provided with the Soar distribution) with explanations of how to run them,
%and pointers to other help sources available for novices.

%\item[Appendix \ref{DEFAULT}] describes Soar's default knowledge, which can be used
%(or not) with any Soar task.

\item[Appendix \ref{GRAMMARS}] provides a grammar for Soar productions.

\item[Appendix \ref{SUPPORT}] describes the determination of o-support.

\item[Appendix \ref{PREFERENCES}] provides a detailed explanation of the preference
resolution process.

%\item[Appendix \ref{Tcl-I/O}] gives an example of Soar I/O functions, written in Tcl.

\item[Appendix \ref{GDS}] provides an explanation of the Goal Dependency Set. 
\end{description}

\subsubsection*{Additional Back Matter}

The appendices are followed by an index; the last
pages of this manual contain a summary and index of the user-interface
functions for quick reference.


\subsubsection*{Not Described in This Manual}

Some of the more advanced features of Soar are not described in this
manual, such as how to interface with a simulator, or how to create Soar
applications using multiple interacting agents.  A discussion of
these topics is provided in a separate document, the \textit{SML Quick Start Guide}, 
which is available at the Soar project website (see link below).

For novice Soar users, try \textit{The Soar} \textit{\SoarVersionMajor} \textit{Tutorial}, which guides the reader 
through several example tasks and exercises.

See Section \ref{CONTACT} for information about obtaining Soar documentation.

% ----------------------------------------------------------------------------
%\section{Other Soar Documentation}
%\label{DOCUMENTATION}
%
%In addition to this manual, there are three other documents that you may want
%to obtain for more information about different aspects of Soar:
%
%\begin{description}
%\item[The Soar 8 Tutorial] is written for novice Soar users, and guides the
%	reader through several example tasks and exercises.
%\item[The Soar Advanced Applications Manual] is written for advanced Soar
%	users. This guide describes how to add input and output routines to
%	Soar programs, how to run multiple Soar ``agents'' from a single Soar
%	image, and how to extend Soar by adding your own user-interface
%	functions, simulators, or graphical user interfaces.
%\item[Soar Design Dogma] gives advice and examples about good Soar programming style. 
%        It may be helpful to both the novice and mid-level Soar user. 
%\end{description}
% ----------------------------------------------------------------------------
\section{Contacting the Soar Group}
\label{CONTACT}

\subsection*{Resources on the Internet}

The primary website for Soar is:

\hspace{2em}\soar{\htmladdnormallink{http://soar.eecs.umich.edu/}{http://soar.eecs.umich.edu/}}

Look here for the latest downloads, documentation, and Soar-related announcements, as well
as links to information about specific Soar research projects and researchers and a FAQ
(list of frequently asked questions) about Soar.

Soar kernel development is hosted on GitHub at

\hspace{2em}\soar{\htmladdnormallink{https://github.com/SoarGroup}{https://github.com/SoarGroup}}

This site contains the public subversion repository, active
documentation wiki, and is also where bugs should be reported.

To contact the Soar group or get help, simply sign up on the web site.  
This will give you access to the forums, where you can ask questions and 
participate in Soar-related discussions.  It will also register you
for our announcement mailing list.  You can also sign up to be notified
of new topics in the forums related to your interests.

Also, please do not hesitate to file bugs on our issue tracker:

\hspace{2em}\soar{\htmladdnormallink{https://github.com/SoarGroup/Soar/issues}{https://github.com/SoarGroup/Soar/issues}}

To avoid redundant entries, please search for duplicate issues first.
\newpage
\subsection*{For Those Without Internet Access}

Mailing Address:

\begin{flushleft}
\hspace{2em}The Soar Group \\
\hspace{2em}Artificial Intelligence Laboratory \\
\hspace{2em}University of Michigan\\
\hspace{2em}2260 Hayward Street\\
\hspace{2em}Ann Arbor, MI 48109-2121 \\
\hspace{2em}USA \\
\end{flushleft}

% ----------------------------------------------------------------------------
% ----------------------------------------------------------------------------
\section{A Note on Different Platforms and Operating Systems}
\label{INTRO-platforms}
\index{Unix}
\index{Linux}
\index{Macintosh}
\index{Personal Computer}
\index{Windows}
\index{Operating System}

Soar runs on a wide variety of platforms, including Linux, Unix
(although not heavily tested), Mac OS X, and Windows 10, 7, Vista, and XP
(and probably 2000 and NT).

This manual documents Soar generally, although all references to files
and directories use Unix format conventions rather than Windows-style folders.


%\include{m-overview}
% ----------------------------------------------------------------------------
\typeout{--------------- The Soar ARCHitecture ------------------------------}
\chapter{The Soar Architecture}
\label{ARCH}

This chapter describes the Soar architecture.  It covers all aspects of Soar
except for the specific syntax of Soar's memories and descriptions of the
Soar user-interface commands.

This chapter gives an abstract description of Soar.  It starts by giving
an overview of Soar and then goes into more detail for each of Soar's
main memories (working memory, production memory, and preference memory)
and processes (the decision procedure, learning, and input and output).

% ----------------------------------------------------------------------------
\section{An Overview of Soar}
\label{ARCH-overview}
\index{state}
\index{operator}
\index{goal}

The design of Soar is based on the hypothesis that all deliberate
\textit{goal}-oriented behavior can be cast as the selection and application
of \textit{operators} to a \textit{state}. A \soarb{state} is a representation of the
current problem-solving situation; an \soarb{operator} transforms a state (makes
changes to the representation); and a \soarb{goal} is a desired outcome of the
problem-solving activity.

As Soar runs, it is continually trying to apply the current operator and
select the next operator (a state can have only one operator at a time),
until the goal has been achieved. The selection and application of
operators is illustrated in Figure \ref{fig:select-apply}. 

\begin{figure}
\insertfigure{select-apply}{1.5in}
\insertcaption{Soar is continually trying to select and apply operators.}
\label{fig:select-apply}
\end{figure}

Soar has separate memories (and different representations) for
descriptions of its current situation and its long-term procedural knowledge.  In
Soar, the current situation, including data from sensors, results of
intermediate inferences, active goals, and active operators is held in
\soarb{working memory}.  Working memory is organized as
\emph{objects}. Objects are described in terms of their
\emph{attributes}; the values of the attributes may correspond to
sub-objects, so the description of the state can have a hierarchical
organization. (This need not be a strict hierarchy; for example, there's
nothing to prevent two objects from being ``substructure'' of each
other.)
\index{working memory}
\index{object}
\index{attribute}

\index{production memory}
Long-term procedural knowledge is held in \soarb{production memory}.
Procedural knowledge specifies how to respond to different
situations in working memory, can be thought of as the program for Soar.
The Soar architecture cannot solve any problems without the addition of
long-term procedural knowledge.  (Note the distinction between the ``Soar
architecture'' and the ``Soar program'': The former refers to the system
described in this manual, common to all users, and the latter refers to
knowledge added to the architecture.)

A Soar program contains the knowledge to be used for solving a specific
task (or set of tasks), including information about how to select and
apply operators to transform the states of the problem, and a means of
recognizing that the goal has been achieved.  

\subsection{Types of Procedural Knowledge in Soar}
\label{LIST:4KnowledgeTypes}

Soar's procedural knowledge can be categorized into  four distinct types of
knowledge:\vspace{-10pt} 
\begin{enumerate}
  \item \textit{Inference Rules} \newline 
In Soar, we call these state elaborations.  This knowledge provides monotonic inferences
that can be made about the state in a given situation. The knowledge created by such rules
are not persistent and exist only as long as the conditions of the rules are met.
  \item \textit{Operator Proposal Knowledge} \newline
Knowledge about when a particular operator is appropriate for a situation.
Note that multiple operators may be appropriate in a given context.
So, Soar also needs knowledge to determine which of the candidates to choose:  
  \item	\textit{Operator Selection Knowledge:} \newline
Knowledge about the desirability of an operator in a particular situation.
Such knowledge can be either in terms of a single operator (e.g. never choose this 
operator in this situation) or relational (e.g. prefer this operator over another
in this situation).
  \item \textit{Operator Application Rules} \newline
Knowledge of how a specific selected operator modifies the state.
This knowledge creates persistent changes to the state that remain even 
after the rule no longer matches or the operator is no longer selected.
\end{enumerate}

Note that state elaborations can indirectly affect operator selection
and application by creating knowledge that the proposal and application
rules match on.

\subsection{Problem-Solving Functions in Soar}
\label{ARCH-functions}
\index{problem solving}
\index{production}
\index{production!match}
\index{match|see{production!match}}
\index{fire|see{production!firing}}
\index{retract|see{production!retraction}}
\index{elaboration}

These problem-solving functions are the primitives for generating 
behavior that is relevant to the current situation: elaborating the 
state, proposing candidate operators, comparing the candidates, 
and applying the operator by modifying the state. 
These functions are driven by the knowledge encoded in a Soar program.
 
\index{production}
Soar represents that knowledge as \soarb{production rules}.  
Production rules are similar to ``if-then'' statements in conventional 
programming languages. (For example, a
production might say something like ``if there are two blocks on the
table, then suggest an operator to move one block on top of the other
block'').  The ``if'' part of the production is called its
\textit{conditions} and the ``then'' part of the production is called
its \textit{actions}. When the conditions are met in the current
situation as defined by working memory, the production is \emph{matched}
and it will \emph{fire}, which means that its actions are executed,
making changes to working memory.

\index{decision procedure}
\index{decision cycle}
Selecting the current operator, involves making a
\soarb{decision} once sufficient knowledge has been retrieved.  This is
performed by Soar's \emph{decision procedure}, which is a fixed
procedure that interprets \emph{preferences} that have been created by
the knowledge retrieval functions. The knowledge-retrieval and decision-making
functions combine to form Soar's \emph{decision cycle}.

\index{impasse}
When the knowledge to perform the problem-solving functions is not
directly available in productions, Soar is unable to make progress and
reaches an \soarb{impasse}.  There are three types of possible impasses
in Soar:
\begin{enumerate}
\item An operator cannot be selected because no new operators are proposed.\vspace{-4pt}
\item An operator cannot be selected because multiple operators are
        proposed and the comparisons are insufficient to determine which
        one should be selected.\vspace{-4pt}
\item An operator has been selected, but there is insufficient knowledge
        to apply it.\vspace{-4pt}
\end{enumerate}
In response to an impasse, the Soar architecture creates a
\soarb{substate} in which operators can be selected and applied to
generate or deliberately retrieve the knowledge that was not directly
available; the goal in the substate is to resolve the impasse. For
example, in a substate, a Soar program may do a lookahead search to
compare candidate operators if comparison knowledge is not directly
available.  Impasses and substates are described in more detail in Section \ref{ARCH-impasses}.
\index{substate|see{subgoal}}


% ----------------------------------------------------------------------------
\subsection{An Example Task: The Blocks-World}

We will use a task called the blocks-world as an example throughout this
manual. In the blocks-world task, the initial state has three blocks named
\soar{A}, \soar{B}, and \soar{C} on a table; the operators move one block at a
time to another location (on top of another block or onto the table); and the
goal is to build a tower with \soar{A} on top, \soar{B} in the middle, and
\soar{C} on the bottom. The initial state and the goal are illustrated in
Figure \ref{fig:blocks}.

The Soar code for this task is available online at \\
\url{https://web.eecs.umich.edu/~soar/blocksworld.soar}. \\
You do not need to look at the code at this point.

\begin{figure}
\insertfigure{blocks}{2in}
\insertcaption{The initial state and goal of the ``blocks-world'' task.}
\label{fig:blocks}
\end{figure}

The operators in this task move a single block from its current location to a
new location; each operator is represented with the following information: 
\vspace{-12pt}
\begin{itemize}
\item the name of the block being moved \vspace{-9pt}
\item the current location of the block (the ``thing'' it is on top of) \vspace{-9pt}
\item the destination of the block (the ``thing'' it will be on top of) 
\vspace{-9pt}
\end{itemize}

The goal in this task is to stack the blocks so that \soar{C} is on the
table, with block \soar{B} on top of block \soar{C}, and block \soar{A} on
top of block \soar{B}.

% ----------------------------------------------------------------------------
\subsection{Representation of States, Operators, and Goals}
\label{OVERVIEW-ps-representation}
\index{state!representation}
\index{operator!representation}
\index{goal!representation}
\index{problem space}
\index{attribute}

The initial state in our blocks-world task --- before any operators have been
proposed or selected --- is illustrated in Figure \ref{fig:ab-wmem}.

\begin{figure}
\insertfigure{ab-wmem}{3.5in}
\insertcaption{An abstract illustration of the initial state of the blocks
	world as working memory objects. At this stage of problem solving, no
	operators have been proposed or selected.}
\label{fig:ab-wmem}
\end{figure}

A state can have only one selected operator at a time
but it may also have a number of \emph{potential} operators that are in consideration.
These proposed operators should not be confused with the active, selected operator.

Figure \ref{fig:ab-wmem2} illustrates working memory after the first operator
has been selected. There are six operators proposed, and only one of
these is actually selected.

\begin{figure}
\insertfigure{ab-wmem2}{4.25in}
\insertcaption{An abstract illustration of working memory in the blocks world
	after the first operator has been selected.}
\label{fig:ab-wmem2}
\end{figure}

Goals are either represented explicitly as substructures of the working memory state
with general rules that recognize when the goal is achieved, or are
implicitly represented in the Soar program by goal-specific rules that
test the state for specific features and recognize when the goal is
achieved.  The point is that sometimes a description of the goal will be
available in the state for focusing the problem solving, whereas other
times it may not.  Although representing a goal explicitly has many advantages,
some goals are difficult to explicitly represent on the state.

For example, the goal in our blocks-world task is represented implicitly in the provided Soar program. This is because a single production rule monitors the state for completion of the goal and halts Soar when the goal is achieved. (Syntax of Soar programs will be explained in Chapter \ref{SYNTAX}.) If the goal was an explicit working memory structure, a rule could compare the configuration of blocks to that structure instead of having the goal embedded within the rule's programming.

% ----------------------------------------------------------------------------
\subsection{Proposing candidate operators}
\index{operator!proposal}

As a first step in selecting an operator, one or more candidate
operators are \soarb{proposed}.  Operators are proposed by rules that test
features of the current state.  When the blocks-world task is run, the
Soar program will propose six distinct (but similar) operators for the
initial state as illustrated in Figure \ref{fig:proposal}. These
operators correspond to the six different actions that are possible
given the initial state.

\begin{figure}
\insertfigure{blocks-proposal}{2.5in}
\insertcaption{The six operators proposed for the initial state of the blocks
	world each move one block to a new location.}
\label{fig:proposal}
\end{figure}


% ----------------------------------------------------------------------------
\subsection{Comparing candidate operators: Preferences}
\index{operator!comparison|see{preferences}}
\index{preference}
\index{preference memory}

The second step Soar takes in selecting an operator is to evaluate or
compare the candidate operators. In Soar, this is done via rules that
test the proposed operators and the current state, and then create
\soarb{preferences} (stored in \emph{preference memory}).  Preferences assert the relative or absolute merits of the
candidate operators. For example, a preference may say that operator A
is a ``better'' choice than operator B at this particular time, or a
preference may say that operator A is the ``best'' thing to do at this
particular time. Preferences are discussed in detail in section \ref{PREFERENCES}. 

% ----------------------------------------------------------------------------
\subsection{Selecting a single operator: Decision}
\index{operator!selection}
\index{decision procedure}

Soar attempts to select a single operator as a \soarb{decision}, based on the preferences available
for the candidate operators. There are four different situations that may
arise:\vspace{-14pt}

\begin{enumerate}
\item The available preferences unambiguously prefer a single operator.\vspace{-
6pt}
\item The available preferences suggest multiple operators, and 
       prefer a subset that can be selected from randomly.\vspace{-6pt}
\item The available preferences suggest multiple operators,but neither case
       1 or 2 above hold.\vspace{-6pt}
\item The available preferences do not suggest any operators.
\end{enumerate}

In the first case, the preferred operator is selected.  In the second
case, one of the subset is selected randomly. In the third and fourth
cases, Soar has reached an \emph{impasse} in problem solving, and a new
substate is created.  Impasses are discussed in Section
\ref{ARCH-impasses}.

In our blocks-world example, the second case holds, and Soar can select one of
the operators randomly.

% ----------------------------------------------------------------------------
\subsection{Applying the operator}
\index{operator!application}

An operator \soarb{applies} by making changes to the state; the specific changes
that are appropriate depend on the operator and the current state.

\index{I/O}
\index{problem solving}
There are two primary approaches to modifying the state: indirect and direct.
\emph{Indirect} changes are used in Soar programs that interact with
an external environment: The Soar program sends motor commands to the
external environment and monitors the external environment for
changes. The changes are reflected in an updated state description,
garnered from sensors. Soar may also make \emph{direct} changes to the
state; these correspond to Soar doing problem solving ``in its
head''. Soar programs that do not interact with an external environment
can make only direct changes to the state.

Internal and external problem solving should not be viewed as mutually
exclusive activities in Soar. Soar programs that interact with an
external environment will generally have operators that make direct and
indirect changes to the state: The motor command is represented as
substructure of the state \emph{and} it is a command to the environment. Also, a Soar program may maintain an internal
model of how it expects an external operator will modify the world; if
so, the operator must update the internal model (which is substructure
of the state).

When Soar is doing internal problem solving, it must know how to modify
the state descriptions appropriately when an operator is being
applied. If it is solving the problem in an external environment, it
must know what possible motor commands it can issue in order to affect
its environment.

The example blocks-world task described here does not interact with an external
environment. Therefore, the Soar program directly makes changes to the state
when operators are applied. There are four changes that may need to be made
when a block is moved in our task: \vspace{-14pt}

\begin{enumerate}\label{LIST:blocks-app}
\item The block that is being moved is no longer where it was (it is no longer
   	``on top'' of the same thing).\vspace{-6pt}
\item The block that is being moved is now in a new location (it is ``on top''
	of a new thing).\vspace{-6pt}
\item The place that the block used to be in is now clear.\vspace{-6pt}
\item The place that the block is moving to is no longer clear --- unless it
	is the table, which is always considered ``clear''\footnote{In this
	blocks-world task, the table always has room for another block, so it
	is represented as always being ``clear''.}.
\end{enumerate}

The blocks-world task could also be implemented using an external simulator. In this case,
the Soar program does not update all the ``on top'' and ``clear'' relations;
the updated state description comes from the simulator.

% ----------------------------------------------------------------------------
\subsection{Making inferences about the state}
\index{elaboration}

Making monotonic inferences about the state is the other role that Soar
long-term procedural knowledge may fulfill. Such \soarb{elaboration} knowledge can simplify
the encoding of operators because entailments of a set of core features
of a state do not have to be explicitly included in application of the
operator.  In Soar, these inferences will be automatically retracted
when the situation changes such that the inference no longer holds.

For instance, our example blocks-world task uses an elaboration to keep track
of whether or not a block is ``clear''. The elaboration tests for the absence
of a block that is ``on top'' of a particular block; if there is no such ``on top'',
the block is ``clear''. When an operator application creates a new ``on top'', the
corresponding elaboration retracts, and the block is no longer ``clear''.


% ----------------------------------------------------------------------------
\subsection{Problem Spaces}
\label{ARCH-functions-ps}
\index{problem space}

If we were to construct a Soar system that worked on a large number of
different types of problems, we would need to include large numbers of
operators in our Soar program. For a specific problem and a
particular stage in problem solving, only a subset of all possible operators
are actually relevant. For example, if our goal is to \textit{count} the
blocks on the table, operators having to do with moving blocks are probably
not important, although they may still be ``legal''. The operators that are
relevant to current problem-solving activity define the space of possible
states that might be considered in solving a problem, that is, they define the
\emph{problem space}.

Soar programs are implicitly organized in terms of problem spaces
because the conditions for proposing operators will restrict an operator
to be considered only when it is relevant.  The complete problem space
for the blocks world is shown in Figure \ref{fig:blocks-ps}.  Typically,
when Soar solves a problem in this problem space, it does not explicitly
generate all of the states, examine them, and then create a path.
Instead, Soar is \emph{in} a specific state at a given time (represented
in working memory), attempting to select an operator that will move it
to a new state.  It uses whatever knowledge it has about selecting
operators given the current situation, and if its knowledge is
sufficient, it will move toward its goal.

\begin{figure}
\insertfigure{blocks-ps}{4.9in}
\insertcaption{The problem space in the blocks-world includes all operators
	that move blocks from one location to another and all possible
	configurations of the three blocks.}
\label{fig:blocks-ps}
\end{figure}

The same problem could be recast in Soar as a planning problem, where
the goal is to develop a plan to solve the problem, instead of just
solving the problem.  In that case, a state in Soar would consist of a
plan, which in turn would have representations of blocks-world states
and operators from the original space.  The operators would perform
editing operations on the plan, such as adding new blocks-world
operators, simulating those operators, etc.  In both formulations of the
problem, Soar is still applying operators to generate new states, it is
just that the states and operators have different content.

The remaining sections in this chapter describe the memories and processes of Soar:
working memory, production memory, preference memory, Soar's execution cycle (the decision
procedure), learning, and how  input and output fit in.
% ----------------------------------------------------------------------------
\section{Working memory: The Current Situation} 
\label{ARCH-wm}
\index{working memory}

Soar represents the current problem-solving situation in its \emph{working
memory}. Thus, working memory holds the current state and operator and is Soar's
``short-term'' knowledge, reflecting the current knowledge of the world and
the status in problem solving.

\index{working memory element}
\index{WME|see{working memory element}}
\index{identifier}
\index{attribute}
\index{value}
Working memory contains elements called working memory elements, or WMEs for
short. Each WME contains a very specific piece of information; for example, a WME
might say that ``B1 is a block''. 
Several WMEs collectively may provide more information about the same
\textit{object}, for example, ``B1 is a block'', ``B1 is named A'', ``B1 is on
the table'', etc. These WMEs are related because they are all contributing to
the description of something that is internally known to Soar as ``B1''. B1 is
called an \soarbit{identifier}; the group of WMEs that share this identifier
are referred to as an \textit{object} in working memory. 
Each WME describes a different \soarbit{attribute} of the object, for example,
its name or type or location; each \textit{attribute} has a \soarbit{value} associated
with it, for example, the name is A, the type is block, and the position is on
the table. Therefore, each WME is an identifier-attribute-value triple, and
all WMEs with the same identifier are part of the same object.

\index{object}
\index{working memory!object|see{object}}
\index{value}
\index{link}
Objects in working memory are \emph{linked} to other objects: The value of one
WME may be an identifier of another object. For example, a WME might say that
``B1 is ontop of T1'', and another collection of WMEs might describe the
object T1: ``T1 is a table'', ``T1 is brown'', and ``T1 is ontop of F1''. And
still another collection of WMEs might describe the object F1: ``F1 is a
floor'', etc. All objects in working memory must be linked to a state, either
directly or indirectly (through other objects). Objects that are not linked to
a state will be automatically removed from working memory by the Soar
architecture. 

\index{augmentation|see{working memory element}}
WMEs are also often called \textit{augmentations} because they
``augment'' the object, providing more detail about it. While these two
terms are somewhat redundant, WME is a term that is used more often to
refer to the contents of working memory (as a single \textit{identifier-attribute-value} triple), 
while augmentation is a term that is used more often to refer to the description of an object.
Working memory is illustrated at an abstract level in Figure
\ref{fig:ab-wmem} on page \pageref{fig:ab-wmem}. 

The attribute of an augmentation is usually a constant, such as ``\soar{name}'' or
``\soar{type}'', because in a sense, the attribute is just a label used to
distinguish one link in working memory from another.\footnote{In order to
allow these links to have some substructure, the attribute name may be an
identifier, which means that the attribute may itself have attributes and
values, as specified by additional working memory elements.}

The value of an augmentation may be either a constant, such as ``\soar{red}'', or
an identifier, such as \soar{06}. When the value is an identifier, it refers
to an object in working memory that may have additional substructure. In
semantic net terms, if a value is a constant, then it is a terminal node with
no links; if it is an identifier it is a nonterminal node.

\index{multi-valued attribute}
\index{multi-attribute|see{multi-valued attribute}}
One key concept of Soar is that working memory is a set, which means that there can never be two elements in
working memory at the same time that have the same identifier-attribute-value
triple (this is prevented by the architecture). However, it is possible to have
multiple working memory elements that have the same identifier and attribute,
but that each have different values.  When this happens, we say the attribute
is a \emph{multi-valued attribute}, which is often shortened to be
\emph{multi-attribute}.

An object is defined by its augmentations and
\emph{not} by its identifier. An identifier is simply a label or pointer to the object. On subsequent runs of the same Soar program,
there may be an object with exactly the same augmentations, but a different
identifier, and the program will still reason about the object
appropriately. Identifiers are internal markers for Soar; they can appear
in working memory, but they never appear in a production.

There is no predefined relationship between objects in working memory and
``real objects'' in the outside world.  Objects in working memory may refer to
real objects, such as \soar{block A}; features of an object, such as the
color \soar{red} or shape \soar{cube}; a relation between objects, such as \soar{ontop}; classes of
objects, such as \soar{blocks}; etc. The actual names of attributes and
values have no meaning to the Soar architecture (aside from a few WMEs
created by the architecture itself). For example, Soar doesn't care whether
the things in the blocks world are called ``blocks'' or ``cubes'' or
``chandeliers''. It is up to the Soar programmer to pick suitable labels and to
use them consistently.

The elements in working memory arise from one of four sources:

\index{Spatial Visual System}
\vspace{-12pt}
\begin{enumerate}
	\item \textbf{\textit{Productions:}} The actions on the RHS of productions create most working memory elements. \vspace{-8pt}
	\item \textbf{\textit{Architecture:}} \vspace{-8pt}
	\begin{enumerate}
		\item \textit{State augmentations:} The decision procedure automatically creates some special state augmentations (type, superstate, impasse, ...) whenever a state is created.  States are created during initialization (the first state) or because of an impasse (a substate).  
		\vspace{-4pt}
	\item \textit{Operator augmentations:}  The decision procedure creates the operator augmentation of the state 
	based on preferences. This records the selection of the current operator.
	\vspace{-8pt}
	\end{enumerate}
	\item \textbf{\textit{Memory Systems}} \vspace{-8pt}
	\item \textbf{\textit{SVS}} \vspace{-8pt}
	\item \textbf{\textit{The Environment:}} External I/O systems create working memory elements on the input-link for sensory data.
\end{enumerate}

The elements in working memory are removed in six different ways:
\index{reject preference}
\index{i-support}
\index{decision procedure}
\index{I/O}
\vspace{-12pt}
\begin{enumerate}
\item The decision procedure automatically removes all state
augmentations it creates when the impasse that led to their creation is 
resolved.\vspace{-8pt}
\item The decision procedure removes the operator augmentation of the
state when that operator is no longer selected as the current operator.\vspace{-
8pt}
\item Production actions that use \soar{reject} preferences remove
      working memory elements that were created by other productions.\vspace{-8pt}
\item The architecture automatically removes i-supported WMEs when the productions that created them no longer match.\vspace{-8pt}
\item The I/O system removes sensory data from the input-link when it
is no longer valid. \vspace{-8pt}
\item The architecture automatically removes WMEs that are no longer linked to 
a state (because some other WME has been removed).
\end{enumerate}


\index{state}
For the most part, the user is free to use any attributes and values
that are appropriate for the task. However, states have special
augmentations that cannot be directly created, removed, or modified by
rules.  These include the augmentations created when a state is created,
and the state's operator augmentation that signifies the current
operator (and is created based on preferences).  The specific
attributes that the Soar architecture automatically creates are listed in Section
\ref{SYNTAX-impasses}. Productions may create any other attributes for
states.

Preferences are held in a separate \emph{preference memory} where they cannot be tested by productions.  There is one notable exception.  Since a soar program may need to reason about candidate operators, \soar{acceptable} preferences are made available in working memory as well. The acceptable preferences can then be tested by productions, which allows a Soar program to reason about candidates operators to determine which one should be selected. Preference memory and the different types of preferences will be discussed in Section \ref{ARCH-prefmem}.

% ----------------------------------------------------------------------------
\section{\texorpdfstring{Production Memory:\\ Long-term Procedural Knowledge}{Production Memory: Long-term Procedural Knowledge}} 
\label{ARCH-pm}
\index{production memory}
\index{production}

\begin{figure}
\insertfigure{ab-prodmem}{3.5in}
\insertcaption{An abstract view of production memory. The productions are not
	related to one another.}
\label{fig:ab-prodmem}
\end{figure}

\index{production!firing}
Soar represents long-term procedural knowledge as \soarb{productions} that are stored in
\emph{production memory}, illustrated in Figure \ref{fig:ab-prodmem}. Each
production has a set of conditions and a set of actions.  If the
conditions of a production match working memory, the production
\emph{fires}, and the actions are performed.

\subsection{The structure of a production}
\label{ARCH-pm-structure}
\index{conditions|see{production}}
\index{actions|see{production}}
\index{production!condition side (LHS)}
\index{production!action side (RHS)}

In the simplest form of a production, conditions and actions refer directly to
the presence (or absence) of objects in working memory. For example, a
production might say:
\begin{verbatim}
  CONDITIONS: block A is clear 
              block B is clear 
  ACTIONS:    suggest an operator to move block A ontop of block B
\end{verbatim}
This is not the literal syntax of productions, but a simplification.
The actual syntax is presented in Chapter \ref{SYNTAX}.

The conditions of a production may also specify the \emph{absence} of patterns
in working memory. For example, the conditions could also specify that ``block
A is not red'' or ``there are no red blocks on the table''. But since these are
not needed for our example production, there are no examples of negated
conditions for now.

The order of the conditions of a production do not matter to Soar except
that the first condition must directly test the state. Internally, Soar
will reorder the conditions so that the matching process can be more
efficient. This is a mechanical detail that need not concern most
users. However, you may print your productions to the screen or save
them in a file; if they are not in the order that you expected them to
be, it is likely that the conditions have been reordered by Soar.

\subsubsection{Variables in productions and multiple instantiations}
\index{variables}
\index{production!instantiation}

In the example production above, the names of the blocks are ``hardcoded'',
that is, they are named specifically. In Soar productions, variables are used
so that a production can apply to a wider range of situations.

 When variables are bound to specific symbols in working memory elements by Soar’s matching process, Soar creates an \emph{instantiation} of the production. This instantiation consists of the matched production along with a specific and consistent set of symbols that matched the variables. A production instantiation is consistent only if every occurrence of a variable is bound to the same value. Multiple instantiations of the same production can be created since the same production may match multiple times, each with different variable bindings. If blocks \soar{A} and \soar{B} are clear, the first production (without variables) will suggest one operator. However, consider a new proposal production that used variables to test the names of the block.  Such a production will be instantiated twice and therefore suggest \textit{two} operators: one operator to move block \soar{A} on top of block \soar{B} and a second operator to move block \soar{B} on top of block \soar{A}.

Because the identifiers of objects are determined at runtime, literal
identifiers cannot appear in productions. Since identifiers occur in
every working memory element, variables must be used to test for
identifiers, and using the same variables across multiple occurrences is what links conditions together.

\index{production!condition side (LHS)}
Just as the elements of working memory must be linked to a state 
in working memory, so must the objects referred to in a production's
conditions. That is, one condition must test a state object 
\emph{and} all other conditions must test that same state or objects that
are linked to that state.

\subsection{Architectural roles of productions}
\label{ARCH-pm-roles}
\index{production}

Soar productions can fulfill the following four roles, by retrieving
different types of procedural knowledge, all described on page \pageref{LIST:4KnowledgeTypes}:\vspace{-10pt}
\begin{enumerate}
\item Operator proposal\vspace{-10pt}
\item Operator comparison\vspace{-10pt}
\item Operator application\vspace{-10pt}
\item State elaboration
\end{enumerate}

A single production should not fulfill more than one of these roles
(except for proposing an operator and creating an absolute preference
for it). Although productions are not declared to be of one type or the
other, Soar examines the structure of each production and classifies the
rules automatically based on whether they propose and compare operators,
apply operators, or elaborate the state. 

\subsection{Production Actions and Persistence}
\index{i-support}
\index{production!action side (RHS)}
\label{ARCH-prefmem-persistence}
\index{persistence}
\index{o-support}
\index{operator!support}
\label{PAGE:O-support}

Generally, actions of a production either create preferences for
operator selection, or create/remove working memory elements.  For
operator proposal and comparison, a production creates preferences for
operator selection.  These preferences should persist only as long as
the production instantiation that created them continues to match.  When
the production instantiation no longer matches, the situation has
changed, making the preference no longer relevant.  Soar automatically
removes the preferences in such cases.  These preferences are said to
have \emph{i-support} (for ``instantiation support'').  Similarly, state
elaborations are simple inferences that are valid only so long as the
production matches.  Working memory elements created as state
elaborations also have i-support and remain in working memory only as
long as the production instantiation that created them continues to
match working memory.  For example, the set of relevant operators changes
as the state changes, thus the proposal of operators is done with
i-supported preferences. This way, the operator proposals will be
retracted when they no longer apply to the current situation.

However, the actions of productions that \emph{apply} an operator, either
by adding or removing elements from working memory, persist regardless of
whether the operator is still selected or the operator application 
production instantiation still matches. For example, in placing a
block on another block, a condition is that the second block be
clear. However, the action of placing the first block removes the fact
that the second block is clear, so the condition will no longer be
satisfied.

Thus, operator application productions do not retract their actions, even
if they no longer match working memory.  This is called \emph{o-support} 
(for ``operator support''). Working memory elements that participate in
the application of operators are maintained throughout the existence of
the state in which the operator is applied, unless explicitly removed (or
if they become unlinked).  Working memory elements are removed by a
\emph{reject} action of a operator-application rule.  
\index{o-support!reject}

Whether a working memory element receives o-support or i-support is
determined by the structure of the production instantiation that creates
the working memory element.  O-support is given only to working memory
elements created by operator-application productions in the state where
the operator was selected.

An operator-application production tests the current operator of a state
and modifies the state. Thus, a working memory element receives
o-support if it is for an augmentation of the current state or
substructure of the state, and the conditions of the instantiation that
created it test augmentations of the current operator.  

During productions matching, all productions that have their conditions
met fire, creating preferences which may add or remove working memory elements. 
Also, working memory elements and preferences that lose i-support are removed 
from working memory. Thus, several new working memory elements and preferences
may be created, and several existing working memory elements and preferences 
may be removed at the same time. (Of course, all this doesn’t happen literally 
at the same time, but the order of firings and retractions is unimportant, 
and happens in parallel from a functional perspective.)

\subsection{The calculation of o-support}
\label{SUPPORT}
\index{support}
\index{i-support}
\index{o-support}
\index{persistence}
\index{production!instantiation}

This section provides a more detailed description of when an action is given o-support by an instantiation.\footnote{In the past, Soar had various experimental support mode settings. Since version 9.6, the support mode used is what was previously called \soar{mode 4}.} The content here is somewhat more advanced, and the reader unfamiliar with rule syntax (explained in Chapter \ref{SYNTAX}) may wish to skip this section and return at a later point.

Support is given by production; that is, all working memory changes generated by the actions of a single instantiated production will have the same support (an action that is not given o-support will have i-support). The conditions and actions of a production rule will here be referred to using the shorthand of LHS and RHS (for Left-Hand Side and Right-Hand Side), respectively.

A production must meet the following two requirements to have o-supported actions:
\begin{enumerate}
	\item The RHS has no operator proposals, i.e. nothing of the form \begin{verbatim}(<s> ^operator <o> +) \end{verbatim}
	\item The LHS has a condition that tests the current operator, i.e. something of the form
	\begin{verbatim}(<s> ^operator <o>)\end{verbatim}
	\comment{this is only true if mode 3's checks are improved}
\end{enumerate}

In condition 1, the variable \soar{<s>} must be bound to a state identifier.
In condition 2, the variable \soar{<s>} must be bound to the lowest state identifier. That is to say, each (positive) condition on the LHS takes the form \soar{(id \carat attr value)}, some of these id's match state identifiers, and the system looks for the deepest matched state identifier. The tested current operator must be on this state. For example, in this production,

\begin{verbatim}
sp {elaborate*state*operator*name
    (state <s> ^superstate <s1>)
    (<s1> ^operator <o>)
    (<o> ^name <name>)
    -->
    (<s> ^name something)}
\end{verbatim}


the RHS action gets i-support. Of course, the state bound to \soar{<s>} is destroyed when \soar{(<s1> \carat operator <o>)} retracts, so o-support would make little difference. On the other hand, this production,

\begin{verbatim}
sp {operator*superstate*application
    (state <s> ^superstate <s1>)
               ^operator <o>)
    (<o> ^name <name>)
    -->
    (<s1> ^sub-operator-name <name>)}
\end{verbatim}

gives o-support to its RHS action, which remains after the substate bound to \soar{<s>} is destroyed. 

An extension of condition 1 is that operator augmentations should always receive i-support. Soar has been written to recognize augmentations directly off the operator \\
(ie, \soar{(<o> \carat augmentation value)}), and to attempt to give them i-support. However, what should be done about a production that simultaneously tests an operator, doesn't propose an operator, adds an operator augmentation, and adds a non-operator augmentation? For example:

\begin{verbatim}
sp {operator*augmentation*application
    (state <s> ^task test-support
               ^operator <o>)
    -->
    (<o> ^new augmentation)
    (<s> ^new augmentation)}
\end{verbatim}


In such cases, both receive i-support. Soar will print a warning on firing this production, because this is considered bad coding style.

\nocomment{Support calculations are done at run time, as each production is fired. Could these decisions be done at compile time? Much of the decision is based on the structure of the production, which could be analyzed once as the production was loaded or chunked. However, it may be impossible to guarantee that a variable will be bound to a state id just by examining production syntax. Another issue is whether the state tested in condition 2 is the lowest state - this potentially could differ from instantiation to instantiation. For instance the operator*augmentation*application production above could match against multiple states in the state stack. 
	
	
	%-----------------------------------------------------------
	\section{Possible problems with implementation of modes 3 \& 4}
	
	\begin{enumerate}
		\item Default mode is actually o-support mode 3. Do we not want 4 to be default?
		\item There is still the bug Andy pointed out. In condition 1, the variable \soar{<s>} is \textit{supposed} to be bound to a state variable, but the code does not actually check for this.
		\item There is one additional, strange difference between modes 3 \& 4. In condition 3, the \soar{id} of each RHS action is tested to see if it is the id of the operator. This id is represented either as a symbol or as a rete location. Mode 4 tests the id both as a symbol and as a rete location, while mode 3 does only the symbol test. The rete test should be added to mode 3.
	\end{enumerate}
	
	
	\section{O-support modes 1 \& 2}
	
	In o-support modes 1 \& 2, there are some of the same calculations as in 3 \& 4 when a production is matched (which occurs when a wme is added to the rete). In particular, if it is an operator proposal, it is set as IE\_PRODS. Otherwise, if it tests the current operator, it is set as PE\_PRODS, without testing for operator  elaborations. The match is placed on the appropriate dll, according to IE\_PRODS or PE\_PRODS.
	
	Later, when the production is instantiated and the new preferences are built, there are no support calculations for 3 \& 4. But 1 \& 2 have support calculations. I suppose that the purpose of the earlier support calculations is that it places the production on the proper list to be fired during apply or propose,that is, whether it is an IE\_PROD or a PE\_PROD.
	
	During this instantiation process, the function calculate\_support\_for\_instantiation\_preferences() is called to redo support IF the variable need\_to\_do\_support\_calculations is set to TRUE. This variable can be true only when-
	
	\begin{enumerate}
		\item  called from chunk\_instantiation OR
		\item  \#ifndef SOAR\_8\_ONLY
		SOAR\_8\_ONLY is a compile option, which is not defined by default. I think that its purpose is that, when defined, there is no run-time option to switch out of Soar 8. This allows a significant portion of code to be left out. Check out function Soar\_Operand2. 
	\end{enumerate}
	
	
	Mode 2 computes support in what is called 'Doug Pearson's way', which is described as-
	
	\begin{verbatim}
	For a particular preference p=(id ^attr ...) on the RHS of an
	instantiation [LHS,RHS]:
	
	RULE #1 (Context pref's): If id is the match state and attr="operator", 
	then p does NOT get o-support.  This rule overrides all other rules.
	
	RULE #2 (O-A support):  If LHS includes (match-state ^operator ...),
	then p gets o-support.
	
	RULE #3 (O-M support):  If LHS includes (match-state ^operator ... +),
	then p gets o-support.
	
	RULE #4 (O-C support): If RHS creates (match-state ^operator ... +/!),
	and p is in TC(RHS-operators, RHS), then p gets o-support.
	
	Here "TC" means transitive closure; the starting points for the TC are 
	all operators the RHS creates an acceptable/require preference for (i.e., 
	if the RHS includes (match-state ^operator such-and-such +/!), then 
	"such-and-such" is one of the starting points for the TC).  The TC
	is computed only through the preferences created by the RHS, not
	through any other existing preferences or WMEs.
	
	If none of rules 1-4 apply, then p does NOT get o-support.
	
	Note that rules 1 through 3 can be handled in linear time (linear in 
	the size of the LHS and RHS); rule 4 can be handled in time quadratic 
	in the size of the RHS (and typical behavior will probably be linear).
	
	
	What is 'match state'? The match goal for the instantiation.
	Match goal - (a match goal is associated with an instantiation).
	Look through instantiated LHS conditions.
	Find the lowest goal state matched to one of the condition's ids.
	\end{verbatim}  
	
	O-support mode 1 computes Doug's support and compares it to the poor cousin of mode 3 \& 4 support calculations, ie calculation without checking for operator elaboration. It prints any differences it finds.
	
}


\nocomment{
	
	3. the RHS has no direct elaborations of the current operator, ie no actions of the form 
	(<o> ^augmentation value).
	However, an indirect elaboration such as
	(<o> ^name <d>)
	-->
	(<d> ^augmentation value)
	will not prevent o-support.
	
	
	In mode 3, an instantiation will generate o-supported preferences iff
	1. the RHS has no operator proposals (nothing of the form (<s> ^operator <o> +))
	2. the LHS has a condition that tests the current operator (something of the form 
	(<s> ^operator <o>))
	3. 
	
	
	Operator proposal - a production whose RHS has action (<s> ^operator <o> +))
	Operator test - 
	LHS has condition of the form (<s> ^operator <o>)
	Operator elaboration -
	o_support_mode 3:
	
	o_support_mode 4:
	
	
	o_support_mode 4: 
	1. if an operator proposal - i-support
	2. if not an operator test - i-support
	3. if an operator test with no elaborations - o-support
	4. if an operator test with some elaborations and some non-elaboration, non-function RHS action - i-support (warns)
	5. if an operator test with only elaborations - i-support
	
	o_support_mode 3:
	1. if an operator proposal - i-support
	2. if not an operator test - i-support
	3. if an operator test with no elaborations - o-support
	4. if an operator test with some elaborations and some non-elaboration, non-function RHS action - o-support (warns)
	5. if an operator test with only elaborations - i-support
	
	
	o_support_mode 0:
	1. if an operator proposal - i-support
	2. if test operator - o-support
	3. else - i-support
	
}

% ----------------------------------------------------------------------------
\section{Preference Memory: Selection Knowledge} 
\label{ARCH-prefmem}
\index{preference}
\index{preference memory}

\nocomment{need to find the right word there. None of selection, evaluation, or
	comparison seems quite right.}

The selection of the current operator is determined by the \soarb{preferences} in
\emph{preference memory}. Preferences are suggestions or imperatives about the
current operator, or information about how suggested operators compare
to other operators.  Preferences refer to operators by using the
identifier of a working memory element that stands for the operator.
After preferences have been created for a state, the decision procedure
evaluates them to select the current operator for that state.

For an operator to be selected, there will be at least one preference
for it, specifically, a preference to say that the value is a candidate
for the operator attribute of a state (this is done with either an
``\soar{acceptable}'' or ``\soar{require}'' preference). There may also
be others, for example to say that the value is ``best''.

%\index{persistence}
%\index{preference!persistence|see{persistence}}
Preferences remain in preference memory until removed for one of the reasons previously discussed in
Section \ref{ARCH-prefmem-persistence}.

% ----------------------------------------------------------------------------
\subsection{Preference Semantics}
\label{ARCH-prefmem-semantics}
\index{preference}

This section describes the semantics of each type of preference.  More
details on the preference resolution process are provided in
section \ref{PREFERENCES}.

\nocomment{preference resolution or preference evaluation? resolution.}

\index{decision procedure}
Only a single value can be selected as the current operator, that is,
all values are mutually exclusive.  In addition, there is no implicit
transitivity in the semantics of preferences.  If A is indifferent to B,
and B is indifferent to C, A and C will not be indifferent to one
another unless there is a preference that A is indifferent to C (or C
and A are both indifferent to all competing values).

\begin{description}
\index{preference!acceptable("+)}
\index{"+|see{preference}}
\index{acceptable preference|see{preference}}
\item [Acceptable (+)] 
	An \soar{acceptable} preference states that a value is a candidate for selection. All values, except those with \soar{require} preferences, must have an \soar{acceptable} preference in order to be selected. If there is only one value with an \soar{acceptable} preference (and none with a \soar{require} preference), that value will be selected as long as it does not also have a \soar{reject} or a \soar{prohibit} preference.
\vspace{-8pt}

\index{preference!reject(-)}
\index{"-|see{preference}}
\index{reject preference|see{preference}}
\item [Reject ($-$)] 
	A \soar{reject} preference states that the value is not a candidate for selection.
\vspace{-8pt}

\index{preference!better(\textgreater \textit{val})}
\index{">|see{preference}}
\index{better preference|see{preference}}
\index{preference!worse(\textless \textit{val})}
\index{"<|see{preference}}
\index{worse preference|see{preference}}
\item [Better ($>$ \emph{value}), Worse ($<$ \emph{value})] 
	A \soar{better} or \soar{worse} preference states, for the two values involved, that one value should not be selected if the other value is a candidate. \soar{Better} an         \soar{worse} allow for the creation of a partial ordering between candidate values. \soar{Better} and \soar{worse} are simple inverses of each other, so that \soar{A} better than \soar{B} is equivalent to \soar{B} worse than \soar{A}.
\vspace{-8pt}

\index{preference!best(\textgreater)}
\index{">|see{preference}}
\index{best preference|see{preference}}
\item [Best ($>$)] 
	A \soar{best} preference states that the value may be better than any competing value (unless there are other competing values that are also ``best''). If a value is \soar{best} (and not \soar{reject}ed, \soar{prohibit}ed, or \soar{worse} than another), it will be selected over any other value that is not also \soar{best} (or \soar{require}d). If two such values are \soar{best}, then any remaining preferences for those candidates (\soar{worst}, \soar{indifferent}) will be examined to determine the selection. Note that if a value (that is not \soar{reject}ed or \soar{prohibit}ed) is \soar{better} than a \soar{best} value, the \soar{better} value will be selected.  (This result is counter-intuitive, but allows explicit knowledge about the relative worth of two values to dominate knowledge of only a single value. A \soar{require} preference should be used when a value \emph{must} be selected for the goal to be achieved.)
\vspace{-8pt}

\index{preference!worst(\textless)}
\index{"<|see{preference!worst}}
\index{worst preference|see{preference!worst}}
\item [Worst ($<$)] 
	A \soar{worst} preference states that the value should be selected only if there are no alternatives.  It allows for a simple type of default specification. The semantics of the \soar{worst} preference are similar to those for the \soar{best} preference.
\vspace{-8pt}

\index{preference!unary indifferent(=)}
\index{"=|see{preference}}
\index{indifferent preference|see{preference}}
\index{indifferent-selection}
\item [Unary Indifferent (=)] 
	A \soar{unary indifferent} preference states that there is positive knowledge that a single value is as good or as bad a choice as other expected alternatives. 
	
	When two or more competing values both have indifferent preferences, by default, Soar chooses randomly from among the alternatives. (The \soar{decide indifferent-selection} function can be used to change this behavior as described on page \pageref{decide-indifferent-selection} in Chapter \ref{INTERFACE}.)
\vspace{-8pt}

\index{preference!binary indifferent(=\textit{val})}
\item [Binary Indifferent (= \emph{value})] 
	A \soar{binary indifferent} preference states that two values are mutually indifferent and it does not matter which of these values are selected. It behaves like a \soar{unary indifferent} preference, except that the operator value given this preference is only made indifferent to the operator value given as the argument.
\vspace{-8pt}

\index{preference!numeric-indifferent("= \textit{num})}
\index{numeric-indifferent preference|see{preference}}
\item [Numeric-Indifferent (= \emph{number})]
	A \soar{numeric-indifferent} preference is used to bias the random selection from mutually indifferent values. This preference includes a \soar{unary indifferent} preference, and behaves in that manner when competing with another value having a unary indifferent preference. 
	%When a set of operators are determined to be indifferent based on all other asserted preference types and at least one operator has a numeric-indifferent preference, 
	But when a set of competing operator values have \soar{numeric-indifferent} preferences, the decision mechanism will choose an operator based on their numeric-indifferent values and the exploration policy. The available exploration policies and how they calculate selection probability are detailed in the documentation for the \soar{indifferent-selection} command on page \pageref{decide-indifferent-selection}. When a single operator is given multiple numeric-indifferent preferences, they are either averaged or summed into a single value based on the setting of the \soar{numeric-indifferent-mode} command (see page \pageref{decide-numeric-indifferent-mode}).

	Numeric-indifferent preferences that are created by RL rules can be adjusted by the reinforcement learning mechanism. In this way, it's possible for an agent to begin a task with only arbitrarily initialized numeric indifferent preferences and with experience learn to make the optimal decisions. See chapter \ref{RL} for more information.
	
\index{preference!require("!)}
\index{"!|see{preference}}
\index{require preference|see{preference}}
\item [Require (!)] 
	A \soar{require} preference states that the value \emph{must} be selected if the goal is to be achieved. A \soar{require}d value is preferred over all others. Only a single operator value should be given a \soar{require} preference at a time.
\vspace{-8pt}

\index{preference!prohibit}
\index{"~|see{preference}}
\index{prohibit preference|see{preference}}
\item [Prohibit ($\tild$)] 
	A \soar{prohibit} preference states that the value cannot be selected if the goal is to be achieved.  If a value has a \soar{prohibit} preference, it will not be selected for a value of an augmentation, independent of the other preferences.
\vspace{-8pt}
\end{description}


\index{preference!acceptable("+)}
\index{preference!require("!)}
If there is an \soar{acceptable} preference for a value of an operator, and there are no other competing values, that operator will be selected. If there are multiple \soar{acceptable} preferences for the same state but with different values, the preferences must be evaluated to determine which candidate is selected.

If the preferences can be evaluated without conflict, the appropriate operator augmentation of the state will be added to working memory. This can happen when they all suggest the same operator or when one operator is preferable to the others that have been suggested. When the preferences conflict, Soar reaches an impasse, as described in Section \ref{ARCH-impasses}.

Preferences can be confusing; for example, there can be two suggested values that are both ``best'' (which again will lead to an impasse unless additional preferences resolve this conflict); or there may be one preference to say that value \soar{A} is better than value \soar{B} and a second preference to say that value \soar{B} is better than value \soar{A}.

\subsection{How preferences are evaluated to decide an operator}
\label{PREFERENCES}
\index{preference}
% This is a technical discussion of the filtering done to evaluate preferences;
% it might belong in a different version of the manual, but not 492

During the decision phase, operator preferences are evaluated in a sequence 
of eight steps, in an effort to select a single operator. 
Each step handles a specific type of preference, as illustrated in Figure 
\ref{fig:prefsem}. (The figure should be read starting at the top
where all the operator preferences are collected and passed into the procedure. At
each step, the procedure either exits through a arrow to the right, or passes to 
the next step through an arrow to the left.)

Input to the procedure is the set of current operator preferences, and the output
consists of:
\begin{enumerate}
	\item a subset of the candidate operators, which is either the empty set, a set consisting of a single, 
	winning candidate, or a larger set of candidates that may be conflicting,
	tied, or indifferent.
	\item an impasse-type. %, possibly NONE\_IMPASSE\_TYPE
\end{enumerate}
The procedure has several potential exit points. Some occur when the procedure
has detected a particular type of impasse. The others occur when the number of
candidates has been reduced to 
one (necessarily the winner) or zero (a no-change impasse).

\nocomment{
	There are nine filter-like operations involved in evaluating the preferences
	available for a particular identifier and attribute. These filters are
	executed in a specific order to determine the correct values for the working
	memory augmentation, as illustrated in Figure \ref{fig:prefsem}. (The figure
	should be read starting at the top left where all the values for an attribute
	are collected and passed to the first filter.) Each filter reduces the number
	of preferences that need to be considered. If a conflict is found, then an
	impasse is generated and the filtering process is halted. The impasse
	generation is handled as a special exit from a filter and is indicated with a
	grey line in Figure \ref{fig:prefsem}.
	
	The preference semantics module takes as input one or more preferences for a
	given identifier and attribute; its output includes: \vspace{-10pt}
	\begin{enumerate}
		\item a possibly empty set of candidate augmentations that may be conflicting,
		indifferent, or parallel\vspace{-10pt}
		\item possibly, an impasse type (if the
		candidates are conflicting)
	\end{enumerate}
}

\index{decision procedure}

\begin{figure}
	\insertfigure{newprefsem}{\textwidth}
	\insertcaption{An illustration of the preference resolution process. There are eight
		steps; only five of these provide exits from the  resolution process.}
	\label{fig:prefsem}
\end{figure}

Each step in Figure \ref{fig:prefsem} is described below:

\index{preference!require("!)}
\index{require preference|see{preference}}
\index{"!}
\index{impasse!constraint-failure}
\index{impasse!no-change}
\begin{description}
	\item[RequireTest (!)]
	This test checks for required candidates in preference memory and
	also constraint-failure impasses involving require preferences (see
	Section \ref{ARCH-impasses} on page \pageref{ARCH-impasses}).
	
	\begin{itemize}
		\item If there is exactly one candidate operator with a require preference and
		that candidate does not have a prohibit preference, then that candidate
		is the winner and preference semantics terminates.
		\item Otherwise ---
		If there is more than one required candidate, then a constraint-
		failure impasse is recognized and preference semantics terminates 
		by returning the set of required candidates.
		\item Otherwise ---
		If there is a required candidate that is also prohibited, a
		constraint-failure impasse with the required/prohibited value is
		recognized and preference semantics terminates.
		\item Otherwise ---
		There is no required candidate; candidates are passed to AcceptableCollect.
	\end{itemize}
	
	\item[AcceptableCollect (+) ] This operation builds a list of operators
	for which there is an acceptable preference in preference memory.
	This list of candidate operators is passed to the ProhibitFilter.\index{+}
	\nocomment{
		\begin{itemize}
			\item If there are no acceptable preferences in memory for the value of an
			attribute then exit preference semantics with no items picked. 
			(This is an efficiency termination, and does not apply to other filters.)
			\item Otherwise ---
			The candidates are passed to the ProhibitFilter.
		\end{itemize}
	}
	\index{preference!acceptable("+)}
	
	
	\item[ProhibitFilter ($\sim$) ] This filter removes the candidates that
	have prohibit preferences in memory. The rest of the candidates are passed to
	the RejectFilter.
	\index{preference!prohibit}
	
	\item[RejectFilter ($-$) ] This filter removes the candidates that have
	reject preferences in memory. 
	\index{preference!reject(-)}
	\index{reject preference(-)}
	
	\item[Exit Point 1]:
	\begin{itemize}
		\item At this point, if the set of remaining candidates is empty, a no-change impasse
		is created with no operators being selected.
		\item If the set has one member, preference semantics terminates and this set is returned.
		\item Otherwise, the remaining candidates are passed to the
		BetterWorseFilter.
	\end{itemize}
	\index{-}
	
	\item[BetterWorseFilter ($>$), ($<$) ] This filter removes any candidates that are worse
	than another candidate.
	\index{preference!worse(\textless \textit{val})}
	\index{worse preference|see{preference}}
	\index{preference!better(\textgreater \textit{val})}
	\index{better preference|see{preference}}
	
	\item[Exit Point 2]:
	\begin{itemize}
		\item If the set of remaining candidates is empty, a conflict impasse is created
		returning the set of all candidates passed into this filter, i.e. all of the
		conflicted operators.
		\item If the set of remaining candidates has one
		member, preference semantics terminates and this set is returned.
		\item Otherwise, the remaining candidates are passed to the
		BestFilter.
	\end{itemize}
	\index{-}
	
	\item[BestFilter ($>$) ] If some remaining candidate has a best preference,
	this filter removes any candidates that do not have
	a best preference. If there are no best preferences for any of the current
	candidates, the filter has no effect. The remaining candidates are passed
	to the WorstFilter.
	\index{preference!best(\textgreater)}
	
	\item[Exit Point 3]:
	\begin{itemize}
		\item At this point, if the set of remaining candidates is empty,
		a no-change impasse is created with no operators being selected.
		\item If  the set has one member, preference semantics terminates 
		and this set is returned.
		\item Otherwise, the remaining candidates are passed to the
		WorstFilter.
	\end{itemize}
	\index{-}
	
	\item[WorstFilter ($<$) ] This filter removes any candidates that have
	a worst preference. If all remaining candidates have worst preferences or there
	are no worst preferences, this filter has no effect.
	\index{preference!worst(\textless)}
	
	\item[Exit Point 4]:
	\begin{itemize}
		\item At this point, if the set of remaining candidates is empty,
		a no-change impasse is created with no operators being selected. 
		\item If the set has one member, preference semantics terminates 
		and this set is returned.
		\item Otherwise, the remaining candidates are passed to the
		IndifferentFilter.
	\end{itemize}
	
	\item[IndifferentFilter (=) ] This operation traverses the remaining candidates and marks 
	each candidate for which one of the following is true:
	\begin{itemize}
		\item the candidate has a unary indifferent preference
		\item the candidate has a numeric indifferent preference
	\end{itemize}
	This filter then checks every candidate that is not one of the above two types
	to see if it has a binary indifferent preference with every other candidate.
	If one of the candidates fails this test, then the procedure signals a tie impasse
	and returns the complete set of candidates that were passed into the 
	IndifferentFilter. Otherwise, the candidates are mutually indifferent, in which case 
	an operator is chosen according to the method set by the \soar{decide indifferent-selection} 
	command, described on page \pageref{decide-indifferent-selection}.
	\index{preference!unary indifferent(=)}
\end{description}

% ----------------------------------------------------------------------------
% ----------------------------------------------------------------------------
\section{Soar's Execution Cycle: Without Substates}
\label{ARCH-decision}

\index{decision cycle}
\index{quiescence}
\index{elaboration cycle}

The execution of a Soar program proceeds through a number of \soarb{decision cycles}. Each cycle has five phases:

\begin{enumerate} 
\item \textbf{Input}: 
	New sensory data comes into working memory.
\item \textbf{Proposal}: 
	Productions fire (and retract) to interpret new data (state elaboration), propose operators for the current situation (operator proposal), and compare proposed operators (operator comparison).  All of the actions of these productions are i-supported.  All matched productions fire in parallel (and all retractions occur in parallel), and matching and firing continues until there are no more additional complete matches or retractions of productions (\emph{quiescence}). 
\item \textbf{Decision}:
	A new operator is selected, or an impasse is detected and a new state is created.
\item \textbf{Application}: 
	Productions fire to apply the operator (operator application).  The actions of these productions will be o-supported. Because of changes from operator application productions, other productions with i-supported actions may also match or retract. Just as during proposal, productions fire and retract in parallel until quiescence.
\item \textbf{Output}: 
	Output commands are sent to the external environment.
\end{enumerate}

The cycles continue until the halt action is issued from the Soar program (as the action of a production) or until Soar is interrupted by the user.

An important aspect of productions in Soar to keep in mind is that all productions will always fire whenever their conditions are met, and retract whenever their conditions are no longer met. The exact details of this process are shown in Figure \ref{fig:decisioncycle}. The \emph{Proposal} and \emph{Application} phases described above are both composed of as many \soarb{elaboration cycles} as are necessary to reach quiescence. In each elaboration cycle, all matching productions fire and the working memory changes or operator preferences described through their actions are made. After each elaboration cycle, if the working memory changes just made change the set of matching productions, another cycle ensues. This repeats until the set of matching rules remains unchanged, a situation called \soarb{quiescence}.


\begin{figure}
\insertfigure{decisioncycle}{\textwidth}
\insertcaption{A detailed illustration of Soar's decision cycle: out of date}
\label{fig:decisioncycle}
\end{figure}

\begin{figure}
\index{decision cycle}
\begin{verbatim}
Soar
  while (HALT not true) Cycle;
  
Cycle
  InputPhase;
  ProposalPhase;
  DecisionPhase;
  ApplicationPhase;
  OutputPhase;


ProposalPhase
  while (some i-supported productions are waiting to fire or retract)
    FireNewlyMatchedProductions;
    RetractNewlyUnmatchedProductions;

DecisionPhase
  for (each state in the stack, 
       starting with the top-level state)
  until (a new decision is reached)
    EvaluateOperatorPreferences; /* for the state being considered */
    if (one operator preferred after preference evaluation)
      SelectNewOperator;
    else                  /* could be no operator available or */
      CreateNewSubstate;  /* unable to decide between more than one */

ApplicationPhase
  while (some productions are waiting to fire or retract)
    FireNewlyMatchedProductions;
    RetractNewlyUnmatchedProductions;
\end{verbatim}

\insertcaption{A simplified version of the Soar algorithm.}
\label{fig:pseudocode}
\end{figure}

After quiescence is reached in the \emph{Proposal} phase, the \emph{Decision} phase ensues, which is the architectural selection of a single operator, if possible. Once an operator is selected, the \emph{Apply} phase ensues, which is practically the same as the \emph{Proposal} phase, except that any productions that apply the chosen operator (they test for the selection of that operator in their conditions) will now match and fire.

During the processing of these phases, it is possible that the preferences that resulted in the selection of the current operator could change.  Whenever operator preferences change, the preferences are re-evaluated and if a different operator selection would be made, then the current operator augmentation of the state is immediately removed. However, a new operator is not selected until the next decision phase, when all knowledge has had a chance to be retrieved. In other words, if, during the \emph{Apply} phase, the production(s) that proposed the selected operator retract, that \emph{Apply} phase will immediately end.

% ----------------------------------------------------------------------------
\section{Impasses and Substates}
\label{ARCH-impasses}
\index{decision procedure}
\index{impasse}
\index{goal!subgoal|see{subgoal}}
\index{result}

When the decision procedure is applied to evaluate preferences and determine the operator augmentation of the state, it is possible that the preferences are either incomplete or inconsistent. 
The preferences can be incomplete in that no \soar{acceptable} operators are suggested, or that there are insufficient preferences to distinguish among \soar{acceptable} operators. 
The preferences can be inconsistent if, for instance, operator \soar{A} is preferred to operator \soar{B}, and operator \soar{B} is preferred to operator \soar{A}. Since preferences are generated independently across different production instantiations, there is no guarantee that they will be consistent.

% ----------------------------------------------------------------------------
\subsection{Impasse Types}
\label{ARCH-impasses-types}

\index{impasse!tie}
\index{impasse!conflict}
\index{impasse!constraint-failure}
\index{impasse!no-change}
\index{tie impasse|see{impasse}}
\index{conflict impasse|see{impasse}}
\index{constraint-failure impasse|see{impasse}}
\index{no-change impasse|see{impasse}}

There are four types of impasses that can arise from the preference scheme.
\vspace{-12pt}

\begin{description}
\item[Tie impasse ---] 
	A \emph{tie} impasse arises if the preferences do not distinguish between two or more operators that have \soar{acceptable} preferences. If two operators both have \soar{best} or \soar{worst} preferences, they will tie unless additional preferences distinguish between them.
	\vspace{-8pt}
\item[Conflict impasse ---]
	A \emph{conflict} impasse arises if at least two values have conflicting better or worse preferences (such as \soar{A} is better than \soar{B} and \soar{B} is better than \soar{A}) for an operator, and neither one is rejected, prohibited, or \soar{require}d.
	\vspace{-8pt}
\item[Constraint-failure impasse ---]
	A \emph{constraint-failure} impasse arises if there is more than one \soar{require}d value for an operator, or if a value has both a \soar{require} and a \soar{prohibit} preference. These preferences represent constraints on the legal selections that can be made for a decision and if they conflict, no progress can be made from the current situation and the impasse cannot be resolved by additional preferences.
	\vspace{-8pt}
\item[No-change impasse ---]
	A \emph{no-change} impasse arises if a new operator is not selected during the decision procedure. There are two types of no-change impasses: state no-change and operator no-change:
	\vspace{-8pt} 
	\begin{description}
		\item[State no-change impasse ---] 
			A state no-change impasse occurs when there are no \soar{acceptable} (or \soar{require}) preferences to suggest operators for the current state (or all the \soar{acceptable} values have also been \soar{reject}ed). The decision procedure cannot select a new operator.\vspace{-8pt}
        \item[Operator no-change impasse ---] 
	        An operator no-change impasse occurs when either a new operator is selected for the current state but no additional productions match during the application phase, or a new operator is not selected during the next decision phase.
	\end{description}
	\index{state!no-change impasse|see{impasse}}
	\index{operator!no-change impasse|see{impasse}}
	\index{impasse!state no-change}
	\index{impasse!operator no-change}
\end{description}

There can be only one type of impasse at a given level of subgoaling at a time. Given the semantics of the preferences, it is possible to have a tie or conflict impasse and a constraint-failure impasse at the same time.  In these cases, Soar detects only the constraint-failure impasse.

The impasse is detected \textit{during} the selection of the operator, but happens \textit{because} one of the four problem-solving functions (described in section \ref{ARCH-functions}) was incomplete.

% ----------------------------------------------------------------------------
\subsection{Creating New States}

Soar handles these inconsistencies by creating a new state, called a \soarb{substate} in which the
goal of the problem solving is to resolve the impasse.  Thus, in the
substate, operators will be selected and applied in an attempt either to
discover which of the tied operators should be selected, or to apply the
selected operator piece by piece.  The substate is often called a
\emph{subgoal} because it exists to resolve the impasse, but is
sometimes called a substate because the representation of the subgoal in
Soar is as a state.
\index{subgoal}
\index{subgoal|see{goal}}
\index{impasse}

The initial state in the subgoal contains a complete description of the
cause of the impasse, such as the operators that could not be decided
among (or that there were no operators proposed) and the state that the
impasse arose in. From the perspective of the new state, the latter is
called the \soarb{superstate}. Thus, the superstate is part of the
substructure of each state, represented by the Soar architecture using
the \soar{superstate} attribute. (The initial state, created in the 0th
decision cycle, contains a \soar{superstate} attribute with the value of
\soar{nil} --- the top-level state has no superstate.)
\index{superstate}

The knowledge to resolve the impasse may be retrieved by any type of
problem solving, from searching to discover the implications of different
decisions, to asking an outside agent for advice. There is no \emph{a priori}
restriction on the processing, except that it involves applying operators to
states.
\index{subgoal}

\begin{figure}
\insertfigure{stack1}{7.75in}
\insertcaption{A simplified illustration of a subgoal stack.}
\label{fig:stack1}
\end{figure}

\index{goal!stack}
\index{stack|see{goal}}
In the substate, operators can be selected and applied as Soar attempts to
solve the subgoal. (The operators proposed for solving the subgoal may be
similar to the operators in the superstate, or they may be entirely
different.) While problem solving in the subgoal, additional impasses may be
encountered, leading to new subgoals.  Thus, it is possible for Soar to have a
\emph{stack} of subgoals, represented as states: Each state has 
a single superstate (except the initial state) and each state may have at most 
one substate. Newly created
subgoals are considered to be added to the bottom of the stack; the first
state is therefore called the \emph{top-level state}.\footnote{The
original state is the ``top'' of the stack because as Soar
runs, this state (created first), will be at the top of the computer screen,
and substates will appear on the screen below the top-level state.}  See
Figure \ref{fig:stack1} for a simplified illustrations of a subgoal stack.

Soar continually attempts to retrieve knowledge relevant to all goals in the
subgoal stack, although problem-solving activity will tend to focus on the
most recently created state. However, problem solving is active at
all levels, and productions that match at any level will fire.

% ----------------------------------------------------------------------------
\subsection{Results}
\label{ARCH-impasses-results}
\index{goal!result|see{result}}
\index{result}

In order to resolve impasses, subgoals must generate results that allow
the problem solving at higher levels to proceed.  The {\em results} of a
subgoal are the working memory elements and preferences that were
created in the substate, and that are also linked directly or indirectly
to a superstate (\emph{any} superstate in the stack). A preference or
working memory element is said to be created in a state if the
production that created it tested that state and this is the most recent
state that the production tested. Thus, if a production tests multiple
states, the preferences and working memory elements in its actions are
considered to be created in the most recent of those states (and is not
considered to have been created in the other states). The architecture
automatically detects if a preference or working memory element created
in a substate is also linked to a superstate.

These working memory elements and preferences will not be removed when
the impasse is resolved because they are still linked to a superstate,
and therefore, they are called the \textit{results of the subgoal}.  A
result has either i-support or o-support; the determination of support is
described below.

%A production that creates a result is illustrated in Figure \ref{fig:result}.
%The figure illustrates the result as a working memory element: 
%``\soar{new-attribute X1}''.

%\begin{figure}
%\insertfigure{result}{3.7in}
%\insertcaption{An abstract illustration of a production that creates a
%	result. In the figure, S2 is the lowest state in the subgoal stack
%	that is tested by the production, and the working memory element
%	is said to have been created in state S2.  }
%\label{fig:result}
%\end{figure}

%\begin{figure}
%\insertfigure{result-indirect}{3in}
%\insertcaption{An abstract illustration of a production that creates a
%	working memory element that indirectly becomes a result. S2 is the
%	lowest state in the subgoal stack that is tested by the production,
%	and the working memory element is said to be created in state S2. Some other
%	production instantiation creates the working memory element that links X2 to the
%	superstate.
%	}
%\label{fig:result-indirect}
%\end{figure}

A working memory element or preference will be a result if
its identifier is already linked to a superstate.
%(as illustrated inFigure \ref{fig:result})
A working memory element or preference can also become a result
indirectly if, after it is created and it is still in working memory or
preference memory, its identifier becomes linked to a superstate through
the creation of another result. For example, if the problem solving in a
state constructs an operator for a superstate, it may wait until
the operator structure is complete before creating an
\soar{acceptable} preference for the operator in the superstate. The
\soar{acceptable} preference is a result because it was created in the
state and is linked to the superstate (and, through the superstate, is
linked to the top-level state). The substructures of the operator then
become results because the operator's identifier is now linked to the
superstate. 
% An indirect result is illustrated in Figure \ref{fig:result-indirect}). 

\subsubsection*{Justifications: Determination of support for results}

\nocomment{Define justification as substate result up front.}

\index{result!support}
\index{i-support}
\index{o-support}
Some results receive i-support, while others receive o-support.  The
type of support received by a result is determined by the function it
plays in the superstate, and not the function it played in the state in
which it was created. For example, a result might be created through
operator application in the state that created it; however, it might
only be a state elaboration in the superstate. The first function would
lead to o-support, but the second would lead to i-support.

\index{justification}
\index{justification!creation}
In order for the architecture to determine whether a result receives i-support
or o-support, Soar must first determine the function that the working
memory element or preference plays
(that is, whether the result should be considered an operator application or
not). To do this, Soar creates a temporary production, called a
\textit{justification}. The justification summarizes the processing in the
substate that led to the result:

\vspace{-10pt}
\begin{description}
	\item[The conditions] of a justification are those working memory elements that exist in the superstate (and above) that were necessary for producing the result.  This is determined by collecting all of the working memory elements tested by the productions that fired in the subgoal that led to the creation of the result, and then removing those conditions that test working memory elements created in the subgoal.
	\vspace{-6pt}
	\item[The action] of the justification is the result of the subgoal.
\end{description} 

Soar determines i-support or o-support for the justification just as it
would for any other production, as described in Section
\ref{ARCH-prefmem-persistence}.  If the justification is an operator
application, the result will receive o-support.  Otherwise, the result
gets i-support from the justification. If such a result loses
i-support from the justification, it will be retracted if there is no
other support.

\index{justification!conditions}
\index{chunk!overgeneral}
\index{justification!overgeneral}
Justifications include any negated conditions that were in the original
productions that participated in producing the results, and that test for the
absence of superstate working memory elements. Negated conditions that test for
the absence of working memory elements that are local to the substate are not
included, which can lead to overgeneralization in the justification (see Section
\ref{CHUNKING-problems} on page \pageref{CHUNKING-problems} for details).

Justifications can also include operator evaluation knowledge that led to the
selection of the operator that produced the result.  For example, the conditions
of any production that creates a prohibit preference for an operator in the substate that was not selected will be backtraced through
and may produce additional conditions in the justification.  Moreover, if the
add-desirability-prefs learn setting is on, conditions from other preference
types (better, best, worse, worst indifferent) can be included as well.  For a more
complete description of how Soar chooses which desirability preferences to
include, see Section \ref{CDPS} on page \pageref{CDPS}.
\index{desirability preference} 
\index{preference!desirability}
\index{Context-Dependent Preference Set}

% ----------------------------------------------------------------------------
\subsection{Removal of Substates: Impasse Resolution}
%\label{elim-impa}
\label{ARCH-impasses-elimination}
\index{impasse!resolution}
\index{impasse!elimination}
\index{goal!termination}

Problem solving in substates is an important part of what Soar
\textit{does}, and an operator impasse does not necessarily indicate a
problem in the Soar program.  They are a way to decompose a complex
problem into smaller parts and they provide a context for a program to
deliberate about which operator to select.  Operator impasses are necessary, for
example, for Soar to do any learning about problem solving (as will be
discussed in Chapter \ref{CHUNKING}). This section describes how
impasses may be resolved during the execution of a Soar program, how
they may be eliminated during execution without being resolved, and some
tips on how to modify a Soar program to prevent a specific impasse from
occurring in the first place.  

\subsubsection*{Resolving Impasses}

An impasse is \textit{resolved} when processing in a subgoal creates
results that lead to the selection of a new operator for the state
where the impasse arose. When an operator impasse is resolved, Soar has
an opportunity to learn, and the substate (and all its substructure) is
removed from working memory.

Here are possible approaches for resolving specific types
of impasses are listed below:\vspace{-12pt}
\begin{description}
\item[Tie impasse ---]
	A tie impasse can be resolved by productions that create preferences
	that prefer one option (\soar{better}, \soar{best}, \soar{require}),
	eliminate alternatives (\soar{worse}, \soar{worst}, \soar{reject},
	\soar{prohibit}), or make all of the objects indifferent
	(\soar{indifferent}).\vspace{-8pt}
\item[Conflict impasse ---]
	A conflict impasse can be resolved by productions that create
	preferences to \soar{require} one option (\soar{require}), or
	eliminate the alternatives (reject, prohibit).\vspace{-8pt}
\item[Constraint-failure impasse ---]
	A constraint-failure impasse cannot be resolved by additional
	preferences, but may be prevented by changing productions so that they
	create fewer \soar{require} or \soar{prohibit} preferences.\vspace{-8pt}
\item[State no-change impasse ---]
	A state no-change impasse can be resolved by productions that create 
	\soar{acceptable} or \soar{require} preferences for operators.\vspace{-
8pt}
\item[Operator no-change impasse ---]
	An operator no-change impasse can be resolved by productions that
	apply the operator, changing the state so the operator proposal
	no longer matches or other operators are proposed and preferred.
\end{description}

\subsubsection*{Eliminating Impasses}

An impasse is resolved when results are created that allow progress to
be made in the state where the impasse arose.  In Soar, an impasse can be
\textit{eliminated} (but not resolved) when a higher level impasse is
resolved, eliminated, or regenerated.  In these cases, the impasse
becomes irrelevant because higher-level processing can proceed.  An
impasse can also become irrelevant if input from the outside world
changes working memory which in turn causes productions to fire that
make it possible to select an operator.  In all these cases, the impasse
is eliminated, but not ``resolved'', and Soar does not learn in this
situation.

\subsubsection*{Regenerating Impasses}

An impasse is \textit{regenerated} when the problem solving in the
subgoal becomes {\em inconsistent} with the current situation.  During
problem solving in a subgoal, Soar monitors which aspect of the
surrounding situation (the working memory elements that exist in
superstates) the problem solving in the subgoal has depended upon.  If
those aspects of the surrounding situation change, either because of
changes in input or because of results, the problem solving in the
subgoal is inconsistent, and the state created in response to the
original impasse is removed and a new state is created. Problem solving
will now continue from this new state.  The impasse is not ``resolved'',
and Soar does not learn in this situation.

The reason for regeneration is to guarantee that the working memory
elements and preferences created in a substate are consistent with
higher level states.  As stated above, inconsistency can arise when a
higher level state changes either as a result of changes in what is
sensed in the external environment, or from results produced in the
subgoal.  The problem with inconsistency is that once inconsistency
arises, the problem being solved in the subgoal may no longer be the
problem that actually needs to be solved.  Luckily, not all changes to a
superstate lead to inconsistency.

In order to detect inconsistencies, Soar maintains a 
\emph{dependency set} for every \\
subgoal/substate. 
The dependency set consists of all working
memory elements that were tested in the conditions of productions that
created o-supported working memory elements that are directly or
indirectly linked to the substate.  Thus, whenever such an o-supported
working memory element is created, Soar records which working memory
elements that exist in a superstate were tested, directly or indirectly
in creating that working memory element. \index{dependency-set} Whenever
any of the working memory elements in the dependency set of a substate
change, the substate is regenerated.

Note that the creation of i-supported structures in a subgoal does not
increase the dependency set, nor do o-supported results.  Thus, only
subgoals that involve the creation of internal o-support working memory
elements risk regeneration, and then only when the basis for the
creation of those elements changes.

\subsubsection*{Substate Removal}

Whenever a substate is removed, all working memory elements and
preferences that were created in the substate that are not
results are removed from working memory. In Figure \ref{fig:stack1},
state \soar{S3} will be removed from working memory when the impasse
that created it is resolved, that is, when sufficient preferences have
been generated so that one of the operators for state \soar{S2} can be
selected. When state \soar{S3} is removed, operator \soar{O9} will also be removed,
as will the acceptable
preferences for \soar{O7}, \soar{O8}, and \soar{O9}, and the
\soar{impasse}, \soar{attribute}, and \soar{choices} augmentations of state
\soar{S3}. These working memory elements are removed because they are no
longer linked to the subgoal stack. The acceptable preferences for
operators \soar{O4}, \soar{O5}, and \soar{O6} remain in working memory. They
were linked to state \soar{S3}, but since they are also linked to state
\soar{S2}, they will stay in working memory until \soar{S2} is removed (or
until they are retracted or rejected).

%-----------------------------------------------------
\subsection{Removal of Substates:  The Goal Dependency Set}
This subsection describes the Goal Dependency Set (GDS) with discussions 
on the motivation for the GDS and consequences of the GDS from a behavior 
developer/modeler's point of view. 

\subsubsection{Why the GDS was needed}

As a symbol system, Soar attempts to approximate the knowledge level
but will necessarily always fall short\cite{Newell90:UTC}.  We can
informally think of the way in which Soar falls short of the knowledge
level as its peculiar ``psychology.''  Those interested in using Soar
to model human psychology would like Soar's ``psychology'' to
approximate human psychology. Those using Soar to create agent
systems would like to make Soar's processing approximate the knowledge
level as closely as possible. However, Soar~7 had a number of
symbol-level ``quirks'' that appeared inconsistent with human
psychology and that made building large-scale, knowledge-based systems
in Soar more difficult than necessary.  Bob Wray's thesis 
\footnote{Robert E. Wray. \textit{Ensuring Reasoning Consistency in Hierarchical Architectures}. PhD thesis, University of Michigan, 1998.}
addressed many of these symbol-level problems
in Soar, among them logical inconsistency in symbol manipulations,
non-contemporaneous constraints in chunks \cite{Wray96:Compilation},
race conditions in rule firings and in the decision process, and
contention between original task knowledge and learned knowledge
\cite{Wray01:Resolving}.

The Goal Dependency Set implements a solution to logical
inconsistencies between persistent (o-supported) working memory
elements (WMEs) in a substate and its ``context''. The context
consists of all the WMEs in any superstates above the local
goal/state\footnote{This subsection will use ``state,'' not ``goal.''  At
	the kernel level, states are still called ``goals'' and ``goal'' is often
	still used to refer to states. As a result, a confusion in terminology results, 
	with ``\textbf{Goal} Dependency Set'' a specific example, even though ``goals'' 
	have not been an explicit, behavior-level Soar construct since Soar~6.}. In Soar, any
action (application) of an operator receives an o-support preference.
This preference makes the resulting WME persistent: it will remain in
memory until explicitly removed or until its local state is removed,
regardless of whether it continues to be justified.

Persistent WMEs are pervasive in Soar, because operators are the main
unit of problem solving. Persistence is necessary for taking any
non-monotonic step in a problem space. However, persistent WMEs also
are dependent on WMEs in the superstate context. The problem in
Soar prior to GDS, especially when trying to create a large-scale system\cite{Jones99:Automated}, is that the knowledge developer
must always think about which dependencies can be ``ignored'' and
which may affect the persistent WME. For
example, imagine an exploration robot that makes a persistent decision
to travel to some distant destination based, in part, on its power
reserves.  Now suppose that the agent notices that its power reserves
have failed.  If this change is not communicated to the state where
the travel decision was made, the agent will continue to act as if its
full power reserves were still available.

Of course, for this specific example, the knowledge designer can
encode some knowledge to react to this inconsistency. The fundamental
problem is that the knowledge designer has to consider \emph{all}
possible interactions between all o-supported WMEs and all contexts.
Soar systems often use the architecture's impasse mechanism to realize
a form of decomposition. These potential interactions mean that the
knowledge developer cannot focus on individual problem spaces in isolation when
creating knowledge, which makes knowledge development more difficult.
Further, in all but the simplest systems, the knowledge designer will
miss some potential interactions. The result is agents are that were
unnecessarily brittle, failing in difficult-to-understand,
difficult-to-duplicate ways.  

The GDS also solves the the problem of non-contemporaneous constraints
in chunks. A non-contemporaneous constraint refers to two or more
conditions that never co-occur simultaneously. An example might be a
driving robot that learned a rule that attempted to match ``red
light'' and ``green light'' simultaneously. Obviously, for functioning
traffic lights, this rule would never fire. By ensuring that local
persistent elements are always consistent with the higher-level
context, non-contemporaneous constraints in chunks are
\emph{guaranteed} not to happen.


The GDS captures context dependencies during processing, meaning the
architecture will identify and respond to inconsistencies
automatically.  The knowledge designer then does not have to consider
potential inconsistencies between local, o-supported WMEs and the
context.


\subsubsection{Behavior-level view of the Goal Dependency Set}

The following discussion covers what the GDS does, and how that impacts
production knowledge design and implementation.

\paragraph{Operation of the Goal Dependency Set:}


\begin{figure}
	\insertfigure{simple-ncc}{3in}
	\caption{Simplified Representation of the context dependencies (above the line), local os-upported WMEs (below the line), and the generation of a result.  Prior to GDS, this situation led to non-contemporaneous constraints in the chunk that generates {\bf 3}.}
	\label{'ncc'}
\end{figure}

Whenever a feature is created (added to working memory) in the Soar
architecture, that feature will persist for some time.  The
persistence of features may differ with respect to how long the
features remain in memory, and more importantly, what circumstances
cause the feature to be removed.  The Soar architecture utilizes
two primary types of persistence: i-support and o-support.

\index{production!instantiation}
The weakest persistence is instantiation support. An i-supported
feature exists in memory only as long as the production which lead to
the feature's creation remains instantiated. Thus, the WME depends
upon this production instantiation (and, more specifically, the
features the instantiation tests). When one of the conditions in the
production instantiation no longer matches, the instantiation is
retracted, resulting in the loss of the acceptable preference for the
WME.\footnote{Importantly, in a technical sense, the WME is only
	retracted when it loses instantiation support, not when the creating
	production is retracting.  For example, a WME could receive i-support
	from several different instantiations and the retraction of one would
	not lead to the retraction of the WME.  However, the the following
	generally discusses direct dependency unmediated by preferences,
	ignoring this complication for clarity.}  I-support is illustrated in
Figure~\ref{'ncc'}. A copy of {\bf A} in the subgoal, {\bf A$_s$}, is
retracted automatically when {\bf A} changes to {\bf A'}.  The
substate WME persists only as long as it remains justified by {\bf A}.
This is called ``instantiation support'' (i-support) in
Soar.

In the broadest sense, we can say that some feature $<$b$>$ is
``dependent'' upon another element $<$a$>$ if $<$a$>$ was used in the
creation of $<$b$>$, i.e., if $<$a$>$ was tested in the production
instantiation that created $<$b$>$. Further, a dependent change with
respect to feature $<$b$>$ is a change to any of its instantiating
features.  In Figure~\ref{'ncc'}, the change from {\bf A} to {\bf A'}
is a dependent change for feature {\bf 1} because {\bf A} was used to
create {\bf 1}.

In Soar, some features are insensitive to dependent changes. These
features are often referred to as ``persistent WMEs'' because, unlike
i-supported WMEs, they remain in memory until explicitly removed. Any feature created by the action of an operator
receives ``operator support.'' An o-supported feature remains in
memory until explicitly rejected (or until the superstructure to which
it is attached is removed). Removal is architecturally
independent of the WME's instantiating conditions.

The GDS provides a solution to the first problem. When {\bf A}
changes, the persistent WME {\bf 1} may be no longer consistent with
its context (e.g., {\bf A'}).  The specific solution is inspired by
the dependency analysis portion of the chunking algorithm. Whenever an o-supported WME is
created in the local state, the superstate dependencies of that new
feature are determined and added to the {\em goal dependency set}
(GDS) of that state. Conceptually speaking, whenever a working memory
change occurs, the dependency sets for every state in the context
hierarchy are compared to working memory changes. \textit{If a removed element 
is found in a GDS, the state is removed from memory (along with all existing
substructure).} The dependency set includes only dependencies for
o-supported features.  For example, in Figure~\ref{'gds'}, at time
$t_0$, because only i-supported features have been created in the
subgoal, the dependency set is empty.

\begin{figure}
	\insertfigure{gomor-o-support}{3in}
	\caption{The Dependency Set in Soar.}
	\label{'gds'}
\end{figure}


Three types of features can be tested in the creation of an
o-supported feature.  Each requires a slightly different type of
update to the dependency set.\vspace{-10pt}
\begin{enumerate}
	\item \textbf{Elements in the superstate:} WMEs in the superstate are added
	directly to the goal's dependency set. In Figure~\ref{'gds'}, the
	persistent subgoal item {\bf 3} is dependent upon {\bf A} and {\bf
		D}. These superstate WMEs are added to the subgoal's dependency set when
	{\bf 3} is added to working memory at time $t_1$. It does not matter
	that {\bf A} is i-supported and {\bf D} o-supported.
	\item \textbf{Local i-supported features:} Local i-supported features are not
	added to the goal dependency set.  Instead, the superstate WMEs that
	led to the creation of the i-supported feature are determined and
	added to the GDS. In the example, when {\bf 4} is created, {\bf A},
	{\bf B} and {\bf C} must be added to the dependency set because they
	are the superstate features that led to {\bf 1}, which in turn led to
	{\bf 2} and finally {\bf 4}. However, because item {\bf A} was
	previously added to the dependency set at $t_1$, it is unnecessary to
	add it again.
	\item \textbf{Local o-supported features:} The dependencies of a local
	o-supported feature have already been added to the state's GDS. Thus,
	tests of local o-supported WMEs do not require additions to the
	dependency set. In Figure~\ref{'gds'}, the creation of element {\bf
		5} does not change the dependency set because it is dependent only
	upon persistent items {\bf 3} and {\bf 4}, whose features had been
	previously added to the GDS.
\end{enumerate}

Any change to the current dependency set will cause
the retraction of all subgoal structure. Thus, any time after time
$t_1$, either the {\bf D} to {\bf D'} or {\bf A} to {\bf A'}
transition would cause the removal of the entire subgoal. The {\bf E}
to {\bf E'} transition causes no retraction because {\bf E} is not in
the goal's dependency set.

\paragraph{The role of the GDS in agent design:}


The GDS places some design time constraints on operator implementation.
These constraints are:
\begin{itemize} \vspace{-10pt}
	\item Operator actions that are used to remember a previous state/situation should be asserted in the top state \vspace{-8pt}
	\item All operator elaborations should be i-supported \vspace{-8pt}
	\item Any operator with local actions should be designed to be re-entrant
\end{itemize}

Soar says any operator effect is o-supported, regardless of whether
that assertion is entailed by the current situation, or whether it
reflects an assumption about it. The GDS adds additional (needed)
constraint.  Because any context dependencies for subgoal, o-supported
assertions will be added to the GDS, the developer must decide if an
o-supported element should be represented in a substate or the top
state.

This decision is straightforward if the functional role of the
persistent element is considered. Four important capabilities that
require persistence are: \vspace{-8pt}
\begin{enumerate}
	
	\item \textbf{Reasoning hypothetically:} Some assertions may need to
	reflect hypothetical states.  Such assertions are ``assumptions''
	because a hypothetical inference cannot always be grounded in the
	current context.  In other problem solvers with truth maintenance,
	only assumptions are persistent.
	\vspace{-8pt}
	\item \textbf{Reasoning non-monotonically:} 
	Sometimes the result of an inference changes one of the assertions on
	which the inference is dependent.  As an example, consider the task of
	counting.  Each newly counted item replaces the old value of the
	count. 
	\vspace{-8pt}
	\item \textbf{Remembering:} 
	Agents oftentimes need to remember an external situation or stimulus,
	even when that perception is no longer available.  
	\vspace{-8pt}
	\item \textbf{Avoiding Expensive Computations:}  In some situations,
	an agent may have the information needed to assert some belief in a
	new world state but the expense of performing the computation
	necessary for the assertion, given what is already known, makes the
	computation avoidable.  For example, in dynamic, complex domains,
	determining when to make an expensive calculation is often formulated
	as an explicit agent task\cite{Jones99:Automated}.
\end{enumerate}

When remembering or avoiding an expensive computation, the
agent/designer is making a commitment to retain something even though
it might not be supported in the current context. \textbf{These
	WMEs should be asserted in the top state. \emph{For many Soar systems,
		especially those focused on execution in a dynamic environment, 
		most o-supported elements will need to be stored on the top state.}} 

For any kind of local, non-monotonic reasoning about the context
(counting, projection planning), features should be stored locally.
When a dependent context change occurs, the GDS interrupts the
processing by removing the state. While this may seem like a severe
over-reaction, formal and empirical analysis have suggested that this
solution is less computationally expensive than attempting to identify
the specific dependent assumption \cite{Wray03:Ensuring}.

%-----------------------------------------------------
\subsection{Soar's Cycle: With Substates}
\label{ARCH-decision-substates}

When there are multiple substates, Soar's cycle remains basically the
same but has a few minor changes.  


The first change is that during the decision procedure, Soar will detect
impasses and create new substates.  For example, following the proposal
phase, the decision phase will detect if a decision cannot be made given
the current preferences.  If an impasse arises, a new substate is
created and added to working memory.  

%The decision procedure will detect an operator no-change impasse as soon
%as an operator is selected and added to working memory by checking to
%see whether or not productions will create o-supported actions during
%the next application phae.  If no o-supported actions will be created,
%the decision procedure will immediately create an operator no-change
%impasse, and then proceed to output, input, and so on.  In these cases,
%the operator no-change is made in the same decision as the operator
%selection.  There will be cases where the operator no-change happens on
%the following decisions, such as when there are o-supported productions
%that will fire, but do not lead to a change in the selected operator.

The second change when there are multiple substates is that at each
phase, Soar goes through the substates, from oldest (highest) to newest
(lowest), completing any necessary processing at that level for that
phase before doing any processing in the next substate.  When firing
productions for the proposal or application phases, Soar processes the
firing (and retraction) of rules, starting from those matching the
oldest substate to the newest.  Whenever a production fires or retracts,
changes are made to working memory and preference memory, possibly
changing which productions will match at the lower levels (productions
firing within a given level are fired in parallel -- simulated).
Productions firings at higher levels can resolve impasses and thus
eliminate lower states before the productions at the lower level ever
fire.  Thus, whenever a level in the state stack is reached, all
production activity is guaranteed to be consistent with any processing
that has occurred at higher levels.


% ----------------------------------------------------------------------------
\section{Learning Procedural Knowledge}
\label{ARCH-learning} 
\index{chunking}
\index{subgoal}

When an operator impasse is resolved, it means that Soar has, through problem 
solving, gained access to knowledge that was not readily available before. Therefore,
when an impasse is resolved, Soar has an opportunity to learn, by summarizing
and generalizing the processing in the substate.

\index{chunk}
One of Soar's learning mechanisms is called \textit{chunking} 
(See chapter \ref{CHUNKING} for more information); it attempts to
create a new production, called a chunk. The conditions of 
the chunk are the elements of the state that (through some chain of 
production firings) allowed the impasse to be resolved; the action of the 
production is the working memory element or preference that resolved the impasse
(the result of the impasse). The conditions and action are variablized so that this
new production may match in a similar situation in the future and
prevent an impasse from arising. 

\index{justification}
Chunks are very similar to justifications in that they are both
formed via the backtracing process and both create a result in their
actions. However, there are some important distinctions:
\vspace{-12pt}

\begin{enumerate}
\item Justifications disappear as soon as its conditions no longer match. 
\vspace{-8pt}
\item Chunks contain variables so that they may match working memory in other situations; justifications are similar to an instantiated chunk.
\end{enumerate}




% ----------------------------------------------------------------------------
\section{Input and Output}
\label{ARCH-io}	%\label{ch-abst-symb-inpu}
\index{I/O}

Many Soar users will want their programs to interact with a real or simulated
environment. For example, Soar programs may control a robot, receiving sensory
inputs and sending command outputs. Soar programs may also interact with
simulated environments, such as a flight simulator. Input is viewed as
Soar's perception and output is viewed as Soar's motor abilities.

When Soar interacts with an external environment, it must make use of
mechanisms that allow it to receive input from that environment and to effect
changes in that environment; the mechanisms provided in Soar are called
\textit{input functions} and \textit{output functions}.

\begin{description}
\item[Input functions] add and delete elements from working memory in response
	to changes in the external environment.
\item[Output functions] attempt to effect changes in the external
	environment. 
\end{description}

Input is processed at the beginning of each execution cycle and output
occurs at the end of each execution cycle.

For instructions on how to use input and output functions with Soar, refer to the
\textit{SML Quick Start Guide}.





% ----------------------------------------------------------------------------
\typeout{--------------- The SYNTAX of soar programs ------------------------}
\chapter{The Syntax of Soar Programs}
%\label{performance}
\label{SYNTAX}
\index{syntax!productions|see{production!syntax}}
\index{syntax!working memory elements|see{working memory element!syntax}}
\index{syntax!preferences|see{preference!syntax}}

This chapter describes in detail the syntax of elements in working
memory, preference memory, and production memory, and how impasses and
I/O are represented in working memory and in productions. Working memory
elements and preferences are created as Soar runs, while productions are
created by the user or through chunking. The bulk of this chapter
explains the syntax for writing productions.

The first section of this chapter describes the structure of working
memory elements in Soar; the second section describes the structure of
preferences; and the third section describes the structure of
productions. The fourth section describes the structure of impasses.
An overview of how input and output appear in working memory is
presented in the fifth section. Further discussion of Soar I/O can be found
on the Soar website.

This chapter assumes that you understand the operating principles of
Soar, as presented in Chapter \ref{ARCH}.

% ----------------------------------------------------------------------------
\section{Working Memory}
\label{SYNTAX-wm}
\index{working memory!syntax}
\index{working memory element!syntax}

Working memory contains \emph{working memory elements} (WME's). As
described in Section \ref{ARCH-wm}, WME's can be created by the actions of 
productions, the evaluation of preferences, the Soar
architecture, and via the input/output system.

\index{identifier}
\index{attribute}
\index{value}
\index{^ (carat symbol)}
A WME is a tuple consisting of three symbols: an {\em identifier}, an
\emph{attribute}, and a \emph{value}, where the entire WME is enclosed in
parentheses and the attribute is preceded by an up-arrow (\carat ).
A template for a working memory element is:
\begin{verbatim}
(identifier ^attribute value)
\end{verbatim}

The first position always holds an internal identifier symbol, generated by the Soar architecture as
it runs. The attribute and value positions can hold either identifiers or constants.
The term \emph{identifier} is used to refer both to the first position of
a WME, as well as to the symbols that occupy that position.
If a WME's attribute or value is an identifier, there is at least one WME that has 
that identifier symbol in its first position. 


% ----------------------------------------------------------------------------
\subsection{Symbols}
\label{SYNTAX-wm-symbols}
\index{symbol}
\index{attribute}
\index{value}

Soar distinguishes between two types of working memory symbols: \emph{identifiers} and {\em constants}. 

\paragraph{Identifiers:}
\index{identifier}

An identifier is a unique symbol, created at runtime when 
a new object is added to working memory. The names of 
identifiers are
created by Soar, and consist of a single uppercase letter followed by a string
of digits, such as \soar{G37} or \soar{O22}.

(The Soar user interface will also allow users to specify identifiers using
lowercase letters in a case-insensitive manner, for example, when using the \texttt{print} command.
But internally, they are actually uppercase letters.)

\paragraph{Constants:} 
\index{constant}
\index{symbolic constant}
\index{integer}
\index{floating-point constants}

There are three types of constants: integers,
floating-point, and symbolic constants:\vspace{-10pt}

\begin{itemize} 
\item Integer constants (numbers).  The range of values depends on the
        machine and implementation you're using, but it is at least $[$-2
        billion...+2 billion$]$.\vspace{-8pt}

\item Floating-point constants (numbers).  The range depends on
        the machine and implementation you're using.\vspace{-8pt}
        
\item Symbolic constants.  These are symbols with arbitrary names. A constant
        can use any combination of letters, digits, or \verb.$%&*+-/:<=>?_.
        Other characters (such as blank spaces) can be included by surrounding
        the complete constant name with vertical bars: \soar{|This is a
        constant|}.  (The vertical bars aren't part of the name; they're just
        notation.)  A vertical bar can be included by prefacing it with a
        backslash inside surrounding vertical bars:
        \verb.|Odd-symbol\|name|.\vspace{-8pt}
\end{itemize} 

Identifiers should not be confused with constants, although they may ``look
the same''; identifiers are generated (by the Soar architecture) at runtime
and will not necessarily be the same for repeated runs of the same program.
Constants are specified in the Soar program and will be the same for repeated
runs.

Even when a constant ``looks like'' an identifier, it will not act like
an identifier in terms of matching. A constant is printed surrounded by
vertical bars whenever there is a possibility of confusing it with an
identifier: \soar{|G37|} is a constant while \soar{G37} is an
identifier. To avoid possible confusion, you should not use
letter-number combinations as constants or for production names.

\subsection{Objects}
\index{object}

Recall from Section \ref{ARCH-wm} that all WME's that share an
identifier are collectively called an \textit{object} in working memory.
The individual working memory elements that make up an object are often
called \emph{augmentations}, because they augment the object.  A
template for an object in working memory is:

\begin{verbatim}
(identifier ^attribute-1 value-1 ^attribute-2 value-2 
            ^attribute-3 value-3... ^attribute-n value-n)
\end{verbatim}

For example, if you run Soar with the supplementary blocks-world program provided {\footnotesize
	\soar{\htmladdnormallink{online}{https://web.eecs.umich.edu/~soar/blocksworld.soar}}}, 
after one elaboration cycle, you can look at the top-level state object by using the \soar{print} command:

\label{example:prints1}
\begin{verbatim}
soar> print s1
(S1 ^io I1 ^ontop O2 ^ontop O3 ^ontop O1 ^problem-space blocks 
    ^superstate nil ^thing B3 ^thing T1 ^thing B1 ^thing B2 
    ^type state)
\end{verbatim} \vspace{12pt}

The attributes of an object are printed in alphabetical order to make it easier 
to find a specific attribute.

\index{attribute!multi-valued attribute}
\index{multi-attributes|see{attribute!multi-valued attribute}}
Working memory is a set, so that at any time, there are never duplicate
versions of working memory elements.
However, it is possible for several working memory elements to share the same identifier and
attribute but have different values.  Such attributes are called
multi-valued attributes or \emph{multi-attributes}.  For example, state
\soar{S1}, above, has two attributes that are multi-valued: \soar{thing} and 
\soar{ontop}. 


% ----------------------------------------------------------------------------
\subsection{Timetags}
\index{timetag}
\index{working memory element!timetag|see{timetag}}

When a working memory element is created, Soar assigns it a unique
integer \textit{timetag}. The timetag is a part of the working memory
element, and therefore, WME's are actually quadruples, rather than
triples. However, the timetags are not represented in working memory and
cannot be matched by productions. The timetags are used to distinguish
between multiple occurrences of the same WME. As preferences change and
elements are added and deleted from working memory, it is possible for
a WME to be created, removed, and created again. The second creation of
the WME --- which bears the same identifier, attribute, and value as the
first WME --- is \textit{different}, and therefore is assigned a
different timetag. This is important because a production will fire only
once for a given instantiation, and the instantiation is determined by
the timetags that match the production and not by the
identifier-attribute-value triples.

To look at the timetags of WMEs, the \soar{print --internal} command can be used:
\begin{verbatim}
soar> print --internal S1
(3: S1 ^io I1)
(10: S1 ^ontop O2)
(9: S1 ^ontop O3)
(11: S1 ^ontop O1)
(4: S1 ^problem-space blocks)
(2: S1 ^superstate nil)
(6: S1 ^thing B3)
(5: S1 ^thing T1)
(8: S1 ^thing B1)
(7: S1 ^thing B2)
(1: S1 ^type state)
\end{verbatim} \vspace{12pt}
This shows all the individual augmentations of \soar{S1}, each is preceded by
an integer \textit{timetag}.

% ----------------------------------------------------------------------------
\subsection{Acceptable preferences in working memory}
\label{SYNTAX-wm-preferences}
\index{working memory!acceptable preference}
\index{preference!acceptable(+)}
\index{preference}
\index{object}

The acceptable preferences for operators
appear in working memory as identifier-attribute-value-preference
quadruples. No other preferences appear in working memory. A template
for an acceptable preference in working memory is:
\begin{verbatim}
(identifier ^operator value +)
\end{verbatim} \vspace{12pt}

For example, if you run Soar with the example blocks-world program linked above, 
after the first operator has been selected, you can again look at the 
top-level state using the \soar{print --internal} command:

\begin{verbatim}
soar> print --internal s1
(3: S1 ^io I1)
(9: S1 ^ontop O3)
(10: S1 ^ontop O2)
(11: S1 ^ontop O1)
(48: S1 ^operator O4 +)
(49: S1 ^operator O5 +)
(50: S1 ^operator O6 +)
(51: S1 ^operator O7 +)
(54: S1 ^operator O7)
(52: S1 ^operator O8 +)
(53: S1 ^operator O9 +)
(4: S1 ^problem-space blocks)
(2: S1 ^superstate nil)
(5: S1 ^thing T1)
(8: S1 ^thing B1)
(6: S1 ^thing B3)
(7: S1 ^thing B2)
(1: S1 ^type state)
\end{verbatim} \vspace{12pt}

The state \soar{S1} has six augmentations of acceptable preferences for
different operators (\soar{O4} through \soar{O9}). These have plus signs
following the value to denote that they are acceptable preferences. The state
has exactly one operator, \soar{O7}. This state corresponds to the
illustration of working memory in Figure \ref{fig:ab-wmem2}.

% ----------------------------------------------------------------------------
\subsection{Working Memory as a Graph}
\index{link}
\index{identifier}
\index{object}

Not only is working memory a set, it is also a graph structure where the
identifiers are nodes, attributes are links, and constants are terminal
nodes.  Working memory is not an arbitrary graph, but a graph rooted in
the states (e.g. S1).  Therefore, all WMEs are \emph{linked} either directly or
indirectly to a state.  The impact of this constraint is that all WMEs
created by actions are linked to WMEs tested in the conditions.  The
link is one-way, from the identifier to the value. Less commonly, the
attribute of a WME may be an identifier.

\begin{figure}
\insertfigure{o43net}{4in}
\insertcaption{A semantic net illustration of four objects in working memory.}
\label{fig:o43net}
\end{figure}

Figure \ref{fig:o43net} illustrates four objects in working memory; the
object with identifier \soar{X44} has been linked to the object with
identifier \soar{O43}, using the attribute as the link, rather than the
value. The objects in working memory illustrated by this figure are:
\begin{verbatim}
(O43 ^isa apple ^color red ^inside O53 ^size small ^X44 200) 
(O87 ^isa ball ^color red ^inside O53 ^size big)
(O53 ^isa box ^size large ^color orange ^contains O43 O87)
(X44 ^unit grams ^property mass)
\end{verbatim} \vspace{12pt}

In this example, object \soar{O43} and object \soar{O87} are both linked to object \soar{O53} \\
through \soar{(O53 \carat contains O43)} and \soar{(O53
\carat contains O87)}, respectively (the \soar{contains} attribute
is a multi-valued attribute). Likewise, object \soar{O53} is linked to object
\soar{O43} through \soar{(O43 \carat inside O53)} and linked to object
\soar{O87} through \soar{(O87 \carat inside O53)}. Object \soar{X44} is linked
to object \soar{O43} through \soar{(O43 \carat X44 200)}.

Links are transitive so that \soar{O53} is linked to \soar{X44} (because
\soar{O53} is linked to \soar{O43} and \soar{O43} is linked to
\soar{X44}). However, since links are not symmetric, \soar{X44} is not
linked to \soar{O53}.


% ----------------------------------------------------------------------------
% ----------------------------------------------------------------------------
\section{Preference Memory}
\label{SYNTAX-prefmem}
\index{preference memory!syntax}
\index{preference!syntax}

Preferences are created by production firings and express the
relative or absolute merits for selecting an operator for a state.  When
preferences express an absolute rating, they are
identifier-attribute-value-preference quadruples; when preferences
express relative ratings, they are
identifier-attribute-value-preference-value quintuples

For example, 
\begin{verbatim}
(S1 ^operator O3 +)
\end{verbatim}
is a preference that asserts that operator O3 is an acceptable operator for
state S1, while
\begin{verbatim}
(S1 ^operator O3 > O4)
\end{verbatim}
is a preference that asserts that operator O3 is a better choice for the
operator of state S1 than operator O4.

The semantics of preferences and how they are processed were described in
Section \ref{ARCH-prefmem}, which also described each of the eleven different
types of preferences.  Multiple production instantiations may create identical 
preferences. Unlike working memory, preference memory is not a set: Duplicate 
preferences are allowed in preference memory.
% ----------------------------------------------------------------------------
% ----------------------------------------------------------------------------
\section{Production Memory}
\label{SYNTAX-pm}
\index{production!syntax}
\index{production memory!syntax}

\nocomment{XXXX start here with indexing}

Production memory contains productions, which can be entered in by a user
(typed in while Soar is running or \soar{load}ed from a file) or
generated by chunking while Soar is running. Productions (both
user-defined productions and chunks) may be examined using the
\soar{print} command, described in Section \ref{print} on page
\pageref{print}.

Each production has three required components: a name, a set of conditions
(also called the left-hand side, or LHS), and a set of actions (also called the
right-hand side, or RHS).  There are also two optional components: a 
documentation string and a type.

Syntactically, each production consists of the symbol \soar{sp}, followed
by: an opening curly brace, \soar{\{}; the production's name; the
documentation string (optional); the production type (optional);
comments (optional); the production's conditions; the symbol \soar{-->}
(literally: dash-dash-greaterthan); the production's actions; and a
closing curly brace, \soar{\}}.  Each element of a production is
separated by white space. Indentation and linefeeds are used by
convention, but are not necessary.

\begin{verbatim}
sp {production-name
    Documentation string
    :type
    CONDITIONS
    -->
    ACTIONS
    }
\end{verbatim}  \vspace{12pt}

\begin{figure}
\begin{verbatim}
sp {blocks-world*propose*move-block
   (state <s> ^problem-space blocks
              ^thing <thing1> {<> <thing1> <thing2>}
              ^ontop <ontop>)
   (<thing1> ^type block ^clear yes)
   (<thing2> ^clear yes)
   (<ontop> ^top-block <thing1>
            ^bottom-block <> <thing2>)
   -->
   (<s> ^operator <o> +)
   (<o> ^name move-block 
        ^moving-block <thing1> 
        ^destination <thing2>)}
\end{verbatim}
\insertcaption{An example production from the example blocks-world task.}
\label{fig:ex-prod}
\end{figure}

An example production, named ``\soar{blocks-world*propose*move-block}'', is
shown in Figure \ref{fig:ex-prod}. This production proposes operators named 
\soar{move-block} that move blocks
from one location to another. The details of this production will be described
in the following sections.

\subsubsection*{Conventions for indenting productions}

Productions in this manual are formatted using conventions designed to
improve their readability. These conventions are not part of the
required syntax. First, the name of the production immediately follows
the first curly bracket after the \soar{sp}.  All conditions are aligned
with the first letter after the first curly brace, and attributes of an
object are all aligned The arrow is indented to align with the
conditions and actions and the closing curly brace follows the last
action.

% ----------------------------------------------------------------------------
\subsection{Production Names}

The name of the production is  an almost arbitrary constant. (See Section
\ref{SYNTAX-wm-symbols} for a description of constants.) By convention, the
name describes the role of the production, but functionally, the name is
just a label primarily for the use of the programmer.  

A production name should never be a single letter followed by numbers, 
which is the format of identifiers.

The convention for naming productions is to separate important elements
with asterisks; the important elements that tend to appear in the name
are:\vspace{-12pt}
\begin{enumerate}
\item The name of the task or goal (e.g., \texttt{blocks-world}).\vspace{-10pt}
\item The name of the architectural function (e.g., \texttt{propose}).\vspace{-
10pt}
\item The name of the operator (or other object) at issue. (e.g.,
        \texttt{move-block})\vspace{-10pt} 
\item Any other relevant details.
\end{enumerate}


This name convention enables one to have a good idea of the function of
a production just by examining its name. This can help, for example,
when you are watching Soar run and looking at the specific productions
that are firing and retracting.  Since Soar uses white space to delimit
components of a production, if whitespace inadvertently occurs in the
production name, Soar will complain that an open parenthesis was
expected to start the first condition.

\subsection{Documentation string (optional)}

A production may contain an optional documentation string. The syntax
for a documentation string is that it is enclosed in double quotes and
appears after the name of the production and before the first condition
(and may carry over to multiple lines). The documentation string allows
the inclusion of internal documentation about the production; it will be
printed out when the production is printed using the \soar{print}
command.

% ----------------------------------------------------------------------------
\subsection{Production type (optional)}

A production may also include an optional \emph{production type}, which
may specify that the production should be considered a default
production (\soar{:default}) or a chunk (\soar{:chunk}), or may specify
that a production should be given o-support (\soar{:o-support}) or
i-support (\soar{:i-support}).  Users are discouraged from using these
types.  

Another flag (\soar{:template}) can be used to specify that a production should
be used to generate new reinforcement learning rules. See Section \ref{RL-templates} 
on page \pageref{RL-templates} for details.

There is one additional flag (\soar{:interrupt}) which can be placed at this location
in a production. However this flag does not specify a production type, but is
a signal that the production should be marked for special debugging capabilities. For more
information, see Section \ref{sp} on Page \pageref{sp}.

These types are described in Section \ref{sp}, which begins on Page \pageref{sp}.
% ----------------------------------------------------------------------------
\subsection{Comments (optional)}
\index{comments}

Productions may contain comments, which are not stored in Soar when the
production is loaded, and are therefore not printed out by the
\soar{print} command. A comment is begun with a pound sign character
\soar{\#} and ends at the end of the line.  Thus, everything following
the \soar{\#} is not considered part of the production, and comments
that run across multiple lines must each begin with a \soar{\#}.

For example:
\begin{verbatim}
sp {blocks-world*propose*move-block
   (state <s> ^problem-space blocks
              ^thing <thing1> {<> <thing1> <thing2>}
              ^ontop <ontop>)
   (<thing1> ^type block ^clear yes)
   (<thing2> ^clear yes)
#   (<ontop> ^top-block <thing1>
#           ^bottom-block <> <thing2>)
   -->
   (<s> ^operator <o> +)
   (<o> ^name move-block         # you can also use in-line comments
        ^moving-block <thing1>
        ^destination <thing2>)}
\end{verbatim}

When commenting out conditions or actions, be sure that all parentheses
remain balanced outside the comment.

\subsubsection*{External comments}

Comments may also appear in a file with Soar productions, outside
 the curly braces of the \soar{sp} command.  Comments
must either start a new line with a \soar{\#} or start with \soar{;\#}.
In both cases, the comment runs to the end of the line.

\begin{verbatim}
# imagine that this is part of a "Soar program" that contains 
# Soar productions as well as some other code.

load file blocks.soar      ;# this is also a comment
\end{verbatim}


% ----------------------------------------------------------------------------
% ----------------------------------------------------------------------------
% ----------------------------------------------------------------------------
\subsection{The condition side of productions (or LHS)}
\label{SYNTAX-pm-conditions}            %perf-cond
\index{production!condition side (LHS)}

The condition side of a production, also called the left-hand side (or
LHS) of the production, is a pattern for matching one or more WMEs. When
all of the conditions of a production match elements in working memory,
the production is said to be instantiated, and is ready to perform its
action. (Each instance binds the rule to specific WMEs.)

The following subsections describe the condition side of a production,
including predicates, disjunctions, conjunctions, negations, acceptable
preferences for operators, and a few advanced topics. 
% A grammar for the
% condition side is given in Appendix \ref{GRAMMARS}.

% ----------------------------------------------------------------------------
\subsubsection{Conditions}
\label{Conditions}
\index{production!condition side (LHS)}

The condition side of a production consists of a set of conditions.
Each condition tests for the existence or absence (explained later in
Section \ref{SYNTAX-pm-negated}) of working memory elements. Each
condition consists of a open parenthesis, followed by a test for the
identifier, and the tests for augmentations of that identifier, in terms
of attributes and values.  The condition is terminated with a close
parenthesis. A single condition might test properties of a single
working memory element, or properties of multiple working memory
elements that constitute an object.  
\begin{verbatim}
(identifier-test ^attribute1-test value1-test 
                 ^attribute2-test value2-test
                 ^attribute3-test value3-test
                 ...)
\end{verbatim}
The first condition in a production must match against a state in
working memory.  Thus, the first condition must begin with the
additional symbol ``state''.  All other conditions and actions must be
\textit{linked} directly or indirectly to this condition. This linkage
may be direct to the state, or it may be indirect, through objects
specified in the conditions.  If the identifiers of the actions are not
linked to the state, a warning is printed when the production is parsed,
and the production is not stored in production memory.  In the actions
of the example production shown in Figure \ref{fig:ex-prod}, the
operator preference is directly linked to the state and the remaining
actions are linked indirectly via the operator preference.

Although all of the attribute tests in the example condition above are followed
by value tests, it is possible to test for only the existence of an
attribute and not test any specific value by just including the
attribute and no value.  Another exception to the above template is
operator preferences, which have the following structure where a plus
sign follows the value test.
\begin{verbatim}
(state-identifier-test ^operator value1-test +
                 ...)
\end{verbatim}

In the remainder of this section, we describe the different tests that
can be used for identifiers, attributes, and values.  The simplest of
these is a constant, where the constant specified in the attribute or
value must match the same constant in a working memory element.

% ----------------------------------------------------------------------------
\subsubsection{Variables in productions}
\label{SYNTAX-pm-variables}
\index{variables}

Variables match against symbols in WMEs in the
identifier, attribute, or value positions.  Variables can be further
constrained by additional tests (described in later sections) or by
multiple occurrences in conditions.  If a variable occurs more than once
in the condition of a production, the production will match only if the
variables match the same identifier or constant.  However, there is no
restriction that prevents different variables from binding to the same
identifier or constant.

Because identifiers are generated by Soar at run time, it impossible to
include tests for specific identifiers in conditions.  Therefore,
variables are used in conditions whenever an identifier is to be
matched.

Variables also provide a mechanism for passing identifiers and constants
which match in conditions to the action side of a rule.

Syntactically, a variable is a symbol that begins with
a left angle-bracket (i.e., \soar{<}), ends with a right angle-bracket (i.e.,
\soar{>}), and contains at least one non-pipe (\soar{|}) character in between.

In the example production in Figure \ref{fig:ex-prod}, there are seven
variables: \soar{<s>}, \soar{<clear1>}, \soar{<clear2>}, \soar{<ontop>},
\soar{<block1>}, \soar{<block2>}, and \soar{<o>}.

The following table gives examples of legal and illegal variable names.

\begin{tabular}{| l | l |} \hline
\bf{Legal variables} &  \bf{Illegal variables} \\ \hline
\soar{<s>} &  \soar{<>} \\
\soar{<1>} & \soar{<1} \\
\soar{<variable1>} & \soar{variable>} \\
\soar{<abc1>} & \soar{<a b>} \\ \hline 
\end{tabular} \vspace{10pt}

% ----------------------------------------------------------------------------
\subsubsection{Predicates for values}
\label{SYNTAX-pm-predicates}    %perf-pred}
\index{predicates}
\index{<>}
\index{<}
\index{<=}
\index{>=}
\index{>}
\index{<=>}
\index{"@}
\index{"!"@}
\index{"@+}
\index{"@-}

A test for an identifier, attribute, or value in a condition (whether
constant or variable) can be modified by a preceding predicate. There
are six general predicates that can be used:
\soar{<>, <=>, <, <=, >=, >}.  

\begin{tabularx}{\textwidth}{| l | X |} \hline
\bf{Predicate} &  \bf{Semantics of Predicate} \\ \hline
\soar{<>}  & Not equal. Matches anything except the value immediately following it. \\
\soar{<=>} & Same type.  Matches any symbol that is the same type (identifier, 
             integer, floating-point, non-numeric constant) as the value 
             immediately following it. \\
\soar{<}   & Numerically less than the value immediately following it. \\
\soar{<=}  & Numerically less than or equal to the value immediately 
             following it. \\
\soar{>=}  & Numerically greater than or equal to the value immediately 
             following it. \\
\soar{>}   & Numerically greater than the value immediately following it. \\  
\hline 
\end{tabularx} \vspace{10pt}
\index{numeric comparisons}
\index{type comparisons}
\index{not equal test}
\index{LTI!comparisons}

The following table shows examples of legal and illegal predicates:

\begin{tabular}{| l | l |} \hline
	\bf{Legal predicates} &  \bf{Illegal predicates} \\ \hline
	\soar{> <valuex>} & \soar{> > <valuey>} \\
	\soar{< 1}  & \soar{1 >} \\
	\soar{<=> <y>} & \soar{= 10} \\  \hline
\end{tabular} \vspace{10pt}

There are also four special predicates that can be used to test Long-Term Indentifier (LTI) links
held by working memory identifiers:
\soar{@, !@, @+, @-}

\begin{tabularx}{\textwidth}{| l | X |} \hline
	\bf{Predicate} &  \bf{Semantics of Predicate} \\ \hline
	\soar{@}  & Same LTI. Matches when the two values are working memory identifiers linked to the same LTI. \\
	\soar{!@} & Different LTI.  Matches when the values are not both identifiers linked to the same LTI. \\
	\soar{@+} & Matches if the value is an identifier linked to some LTI. \\
	\soar{@-} & Matches if the value is not an identifier linked to some LTI. \\
	\hline 
\end{tabularx} 
\vspace{10pt}

See Section \ref{SMEM-kr} for more information on long-term semantic memory and LTIs.

\subsubsection*{Example Productions}

\begin{verbatim}
sp {propose-operator*to-show-example-predicate
   (state <s> ^car <c>)
   (<c> ^style convertible ^color <> rust)
   -->
   (<s> ^operator <o> +)
   (<o> ^name drive-car ^car <c>) }
\end{verbatim}

In this production, there must be a ``color'' attribute for the working memory
object that matches \verb+<c>+, and the value of that attribute must not be
``rust''. 

\begin{verbatim}
sp {example*lti*predicates
    (state <s> ^existing-item { @+ <orig-sti> }
               ^smem.result.retrieved { @ <orig-sti> <result-sti> })
-->
    ... }
\end{verbatim}

In this production, \verb|<orig-sti>|, is tested for whether it is linked to some LTI.
It is also compared against \verb|<result-sti>| (a working memory element retrieved from long-term memory and
known to be linked to an LTI) to see if the two elements point to the same long-term memory.
Note the the \verb|@+| in this example is actually unnecessary, since the \verb|{ @ <orig-sti> <result-sti> }| test will
fail to match if either value tested is not linked to an LTI.

% ----------------------------------------------------------------------------
\subsubsection{Disjunctions of values}
\label{SYNTAX-pm-disjuncts}      %perf-disj
\index{disjunction of constants}
\index{<< >>}

A test for an identifier, attribute, or value may also be for a
disjunction of constants. With a disjunction, there will be a match if any
one of the constants is found in a working memory element (and the other
parts of the working memory element matches). Variables and predicates
may not be used within disjunctive tests.

Syntactically, a disjunctive test is specified with double angle brackets
(i.e., \soar{ <<} and \soar{>>}). There must be spaces separating the brackets
from the constants. 

The following table provides examples of legal and illegal disjunctions:

\begin{tabular}{| l | l |} \hline
\bf{Legal disjunctions} &  \bf{Illegal disjunctions} \\ \hline
\soar{<< A B C 45 I17 >>} &  \soar{<< <var> A >>}  \\
\soar{<< 5 10 >>} &  \soar{<< < 5  > 10 >>}  \\
\soar{<< good-morning good-evening >>} & \soar{<<A B C >>} \\  \hline 
\end{tabular} \vspace{10pt}

\subsubsection*{Example Production}
For example, the third condition of the following
production contains a disjunction that restricts the color of the table to
\soar{red} or \soar{blue}:

\begin{verbatim}
sp {blocks*example-production-conditions
   (state ^operator <o> + ^table <t>)
   (<o> ^name move-block)
   (<t> ^type table ^color << red blue >> )
   -->
   ... }
\end{verbatim}

\subsubsection*{Note}
Disjunctions of complete conditions are not allowed in Soar.  Multiple
(similar) productions fulfill this role.


% ----------------------------------------------------------------------------
\subsubsection{Conjunctions of values}
\label{SYNTAX-pm-conjunctions}  %perf-conj}
\index{conjunctive!conditions}

A test for an identifier, attribute, or value in a condition may include
a conjunction of tests, all of which must hold for there to be a match.

Syntactically, conjuncts are contained within curly braces (i.e., \soar{\{}
and \soar{\}}). The following table shows some examples of legal and illegal
conjunctive tests:

\begin{tabular}{| l | l |} \hline
\bf{Legal conjunctions} &  \bf{Illegal conjunctions} \\  \hline
\soar{\{ <= <a> >= <b> \}} & \soar{\{ <x> < <a> + <b> \}} \\
\soar{\{ <x> > <y> \}}     & \soar{\{ > > <b> \}} \\
\soar{\{ <> <x> <y> \}}    & \soar{\{ <a> <b> \}} \\
\soar{\{ << A B C >> <x> \}} & \\
\soar{\{ <=> <x> > <y> << 1 2 3 4 >> <z> \}} & \\  \hline
\end{tabular} \vspace{10pt}

Because those examples are a bit difficult to interpret, let's go over the
legal examples one by one to understand what each is doing.

In the first example, the value must be less than or equal to the value bound
to variable \soar{<a>} and greater than or equal to the value bound to
variable \soar{<b>}.

In the second example, the value is bound to the variable \soar{<x>}, which
must also be greater than the value bound to variable \soar{<y>}. 

In the third example, the value must not be equal to the value bound to
variable \soar{<x>} and should be bound to variable \soar{<y>}.  Note the
importance of order when using conjunctions with predicates: in the second
example, the predicate modifies \soar{<y>}, but in the third
example, the predicate modifies \soar{<x>}.

In the fourth example, the value must be one of \soar{A}, \soar{B}, or
\soar{C}, and the second conjunctive test binds the value to variable
\soar{<x>}. 

In the fifth example, there are four conjunctive tests. First, the value must
be the same type as the value bound to variable \soar{<x>}. Second, the value
must be greater than the value bound to variable \soar{<y>}. Third, the value
must be equal to \soar{1}, \soar{2}, \soar{3}, or \soar{4}. Finally, the value
should be bound to variable \soar{<z>}.

In Figure \ref{fig:ex-prod}, a conjunctive test is used for the \soar{thing}
attribute in the first condition.

Note that it is illegal syntax for a condition to test the equality of two variables, 
as demonstrated in the last illegal conjunction above. Any such test can instead be 
coded in simpler terms by only using one variable in the places where either would be 
referenced throughout the rule.

% ----------------------------------------------------------------------------
\subsubsection{Negated conditions}
\label{SYNTAX-pm-negated}       %perf-nega-cond
\index{negated conditions}
\index{-}

In addition to the positive tests for elements in working memory, conditions
can also test for the absence of patterns.  A \emph{negated condition} will be
matched only if there does not exist a working memory element consistent with
its tests and variable bindings. Thus, it is a test for the \textit{absence}
of a working memory element.

Syntactically, a negated condition is specified by preceding a condition with a
dash (i.e., ``\soar{-}'').

For example, the following condition tests the absence of a working memory
element of the object bound to \soar{<p1> \carat type father}.

\begin{verbatim}
-(<p1> ^type father)
\end{verbatim} \vspace{12pt}

A negation can be used within an object with many attribute-value pairs by
having it precede a specific attribute:

\begin{verbatim}
(<p1> ^name john -^type father ^spouse <p2>)
\end{verbatim} \vspace{12pt}

In that example, the condition would match if there is a working memory
element that matches \soar{(<p1> \carat name john)} and another that matches 
\soar{(<p1> \carat spouse <p2>)}, but is no working memory element that matches 
\soar{(<p1> \carat type father)} (when \soar{p1} is bound to the same 
identifier).

On the other hand, the condition:
\begin{verbatim}
-(<p1> ^name john ^type father ^spouse <p2>)
\end{verbatim}

would match only if there is no object in working memory that matches all
three attribute-value tests.

\subsubsection*{Example Production}
\begin{verbatim}
sp {default*evaluate-object
   (state <ss> ^operator <so>)
   (<so> ^type evaluation 
         ^superproblem-space <p>)
  -(<p> ^default-state-copy no)
   -->
   (<so> ^default-state-copy yes) }
\end{verbatim}

\subsubsection*{Notes}

One use of negated conditions to avoid is testing for the absence of the
working memory element that a production creates with i-support; this
would lead to an ``infinite loop'' in your Soar program, as Soar would
repeatedly fire and retract the production. For example, the following rule's
actions will cause it to no longer match, which will cause the action to retract,
which will cause the rule to match, and so on:

\begin{verbatim}
sp {example*infinite-loop
    (state <s>  ^car <c>
               -^road )
    -->
    (<s> ^road |route-66|) }
\end{verbatim}

Also note that syntactically it is invalid for the first condition of a rule
to be a negated condition. For example, the following production would fail
to load:

\begin{verbatim}
sp {example*invalid-negated-first-condition
   (state <s> -^road <r>
               ^car <c>)
   -->
   ... }
\end{verbatim}

% ----------------------------------------------------------------------------
\subsubsection{Negated conjunctions of conditions}
\label{SYNTAX-pm-negaconj}      %perf-nega-conj}
\index{negated conjunctions}
\index{conjunctive!negation}

Conditions can be grouped into conjunctive sets by surrounding the set of
conditions with \soar{\{} and \soar{\}}. The production compiler groups the
test in these conditions together. This grouping allows for negated tests of
more than one working memory element at a time. In the example below, the
state is tested to ensure that it does not have an object on the table. 

\begin{verbatim}
sp {blocks*negated-conjunction-example
   (state <s> ^name top-state)
  -{(<s> ^ontop <on>)
    (<on> ^bottom-object <bo>)
    (<bo> ^type table)}
   -->
   (<s> ^nothing-ontop-table true) } 
\end{verbatim}

When using negated conjunctions of conditions, the production has
nested curly braces. One set of curly braces delimits the production, while
the other set delimits the conditions to be conjunctively negated.

If only the last condition, \soar{(<bo> \carat type table)} were negated, the
production would match only if the state \emph{had} an ontop relation, and the
ontop relation had a bottom-object, but the bottom object wasn't a table.
Using the negated conjunction, the production will also match when the state
has no ontop augmentation or when it has an ontop augmentation that doesn't
have a bottom-object augmentation.

The semantics of negated conjunctions can be thought of in terms of
mathematical logic, where the negation of $(A \wedge B \wedge C)$:

$\neg (A \wedge B \wedge C)$

can be rewritten as:

$(\neg A) \vee (\neg B) \vee (\neg C)$

That is, ``not (A and B and C)'' becomes ``(not A) or (not B) or (not C)''.



% ----------------------------------------------------------------------------
\subsubsection{Multi-valued attributes}
\label{SYNTAX-pm-multi}
\index{multi-valued attribute}

An object in working memory may have multiple augmentations that specify
the same attribute with different values; these are called multi-valued
attributes, or multi-attributes for short.  To shorten the specification
of a condition, tests for multi-valued attributes can be shortened so
that the value tests are together.

For example, the condition:
\begin{verbatim}
(<p1> ^type father ^child sally ^child sue)
\end{verbatim}

could also be written as:
\begin{verbatim}
(<p1> ^type father ^child sally sue)
\end{verbatim}


% ----------------------------------------------------------------------------
\subsubsection*{Multi-valued attributes and variables}

When variables are used with multi-valued attributes, remember that
variable bindings are not unique unless explicitly forced to be so. For
example, to test that an object has two values for attribute
\soar{child}, the variables in the following condition can match to the same
value.

\begin{verbatim}
(<p1> ^type father ^child <c1> <c2>)
\end{verbatim} \vspace{12pt}

To do tests for multi-valued attributes with variables correctly,
conjunctive tests must be used, as in:

\begin{verbatim}
(<p1> ^type father ^child <c1> {<> <c1> <c2>})
\end{verbatim} \vspace{12pt}

The conjunctive test \soar{ \{<> <c1> <c2>\} } ensures that \soar{<c2>} will
bind to a different value than \soar{<c1>} binds to.


% ----------------------------------------------------------------------------
\subsubsection*{Negated conditions and multi-valued attributes}

A negation can also precede an attribute with multiple values.  In this case
it tests for the absence of the conjunction of the values.  For example

\begin{verbatim}
(<p1> ^name john -^child oprah uma)
\end{verbatim}

is the same as 

\begin{verbatim}
(<p1> ^name john)
-{(<p1> ^child oprah)
  (<p1> ^child uma)}
\end{verbatim}

and the match is possible if either \soar{(<p1> \carat child oprah)} or
\soar{(<p1> \carat child uma)} cannot be found in working memory with the
binding for \soar{<p1>} (but not if both are present).

% ----------------------------------------------------------------------------
\subsubsection{Acceptable preferences for operators}
\label{SYNTAX-pm-acceptable}
\index{preference!acceptable as condition}
\index{acceptable preference|see{preference}}
\index{+}

The only preferences that can appear in working memory are acceptable
preferences for operators, and therefore, the only preferences that may appear
in the conditions of a production are acceptable preferences for operators.

Acceptable preferences for operators can be matched in a condition by testing
for a ``\soar{+}'' following the value.  This allows a production to test the
existence of a candidate operator and its properties, and possibly create a
preference for it, before it is selected.

In the example below, \soar{\carat operator <o> +} matches the acceptable
preference for the operator augmentation of the state. \emph{This does not
test that operator} \soar{<o>} \emph{has been selected as the current
operator}.

\begin{verbatim}
sp {blocks*example-production-conditions
   (state ^operator <o> + ^table <t>)
   (<o> ^name move-block)
   -->
   ... }
\end{verbatim}


In the example below, the production tests the state for acceptable
preferences for two different operators (and also tests that these operators
move different blocks):

\begin{verbatim}
sp {blocks*example-production-conditions
   (state ^operator <o1> + <o2> + ^table <t>)
   (<o1> ^name move-block ^moving-block <m1> ^destination <d1>)
   (<o2> ^name move-block ^moving-block {<m2> <> <m1>} 
         ^destination <d2>)
   -->
   ... }
\end{verbatim}

\subsubsection{Attribute tests}

The previous examples applied all of the different tests to the values of
working memory elements. 
All of the tests that can be used for values can also be used for
attributes and identifiers (except those including constants).

% ----------------------------------------------------------------------------
\subsubsection*{Variables in attributes}

Variables may be used with attributes, as in:

\begin{verbatim}
sp {blocks*example-production-conditions
   (state <s> ^operator <o> + 
              ^thing <t> {<> <t> <t2>} )
   (operator <o> ^name group 
                 ^by-attribute <a>
                 ^moving-block <t>
                 ^destination <t2>)
   (<t> ^type block ^<a> <x>)
   (<t2> ^type block ^<a> <x>)
   -->
   (<s> ^operator <o> >) }
\end{verbatim}

This production tests that there is acceptable operator that is trying to
group blocks according to some attribute, \soar{<a>}, and that block
\soar{<t>} and \soar{<t2>} both have this attribute (whatever it is), and have
the same value for the attribute.


% ----------------------------------------------------------------------------
\subsubsection*{Predicates in attributes}

Predicates may be used with attributes, as in:

\begin{verbatim}
sp {blocks*example-production-conditions
   (state ^operator <o> + ^table <t>)
   (<t> ^<> type table)
   -->
   ... }
\end{verbatim}

which tests that the object with its identifier bound to \soar{<t>} must have
an attribute whose value is \soar{table}, but the name of this attribute is
not \soar{type}.

% ----------------------------------------------------------------------------
\subsubsection*{Disjunctions of attributes}
\index{disjunctions of attributes}
\index{<< >>}

Disjunctions may also be used with attributes, as in:

\begin{verbatim}
sp {blocks*example-production-conditions
   (state ^operator <o> + ^table <t>)
   (<t> ^<< type name>> table)
   -->
   ... }
\end{verbatim}

which tests that the object with its identifier bound to \soar{<t>} must have
either an attribute \soar{type} whose value is \soar{table} or an attribute
\soar{name} whose value is \soar{table}.

% ----------------------------------------------------------------------------
\subsubsection*{Conjunctive tests for attributes}

Section \ref{SYNTAX-pm-conjunctions} illustrated the use of conjunctions for
the values in conditions. Conjunctive tests may also be used with attributes,
as in:

\begin{verbatim}
sp {blocks*example-production-conditions
   (state ^operator <o> + ^table <t>)
   (<t> ^{<ta> <> name} table)
   -->
   ... }
\end{verbatim}

which tests that the object with its identifier bound to \soar{<t>} must have
an attribute whose value is \soar{table}, and the name of this attribute is
not \soar{name}, and the name of this attribute (whatever it is) is bound to
the variable \soar{<ta>}.

When attribute predicates or attribute disjunctions are used with
multi-valued attributes, the production is rewritten internally to use a
conjunctive test for the attribute; the conjunctive test includes a
variable used to bind to the attribute name. Thus,

\begin{verbatim}
(<p1> ^type father ^ <> name sue sally)
\end{verbatim}

is interpreted to mean:

\begin{verbatim}
(<p1> ^type father 
      ^{<> name <a*1>} sue 
      ^<a*1> sally)
\end{verbatim}


% ----------------------------------------------------------------------------
\subsubsection{Attribute-path notation}
\label{SYNTAX-pm-path}
\index{dot notation}
\index{path notation}
%\index{.}

Often, variables appear in the conditions of productions only to link the value
of one attribute with the identifier of another attribute. Attribute-path
notation provides a shorthand so that these intermediate variables do not need
to be included.

Syntactically, path notation lists a sequence of attributes separated by dots
(.), after the \carat \ in a condition.

For example, using attribute path notation, the production:

\begin{verbatim}
sp {blocks-world*monitor*move-block
   (state <s> ^operator <o>)
   (<o> ^name move-block
        ^moving-block <block1>
        ^destination <block2>)
   (<block1> ^name <block1-name>)
   (<block2> ^name <block2-name>)   
   -->
   (write (crlf) |Moving Block: | <block1-name>
                 | to: | <block2-name> ) }
\end{verbatim}

could be written as:

\begin{verbatim}
sp {blocks-world*monitor*move-block
   (state <s> ^operator <o>)
   (<o> ^name move-block
        ^moving-block.name <block1-name>
        ^destination.name <block2-name>)   
   -->
   (write (crlf) |Moving Block: | <block1-name>
                 | to: | <block2-name> ) }
\end{verbatim}

Attribute-path notation yields shorter productions that are easier to
write, less prone to errors, and easier to understand.

When attribute-path notation is used, Soar internally expands the conditions
into the multiple Soar objects, creating its own variables as needed.
Therefore, when you print a production (using the \soar{print} command), the
production will not be represented using attribute-path notation.


%----------------------------------------------------------------------------
\subsubsection*{Negations and attribute path notation}

\nocomment{can't negations be used with structured values? there's no
        description of this (yes -- bobd)}

A negation may be used with attribute path notation, in which case it amounts
to a negated conjunction. For example, the production:

\begin{verbatim}
sp {blocks*negated-conjunction-example
   (state <s> ^name top-state)
  -{(<s> ^ontop <on>)
    (<on> ^bottom-object <bo>)
    (<bo> ^type table)}
   -->
   (<s> ^nothing-ontop-table true) } 
\end{verbatim}

could be rewritten as:

\begin{verbatim}
sp {blocks*negated-conjunction-example
   (state <s> ^name top-state -^ontop.bottom-object.type table)
   -->
   (<s> ^nothing-ontop-table true) }
\end{verbatim}


% ----------------------------------------------------------------------------
\subsubsection*{Multi-valued attributes and attribute path notation}

\nocomment{can't multi-attributes be used with structured values? there's no
        description of this (yes -- bobd)}

Attribute path notation may also be used with multi-valued attributes, such as:

\begin{verbatim}
sp {blocks-world*propose*move-block
   (state <s> ^problem-space blocks
              ^clear.block <block1> { <> <block1> <block2> }
              ^ontop <ontop>)
   (<block1> ^type block)
   (<ontop> ^top-block <block1>
            ^bottom-block <> <block2>)
   -->
   (<s> ^operator <o> +)
   (<o> ^name move-block +
        ^moving-block <block1> +
        ^destination <block2> +) }
\end{verbatim}


\subsubsection*{Multi-attributes and attribute-path notation}
\label{SYNTAX-pm-caveat}

\textbf{Note:} It would not be advisable to write the production in Figure
\ref{fig:ex-prod} using attribute-path notation as follows:

\begin{verbatim}
sp {blocks-world*propose*move-block*dont-do-this
   (state <s> ^problem-space blocks
              ^clear.block <block1>
              ^clear.block { <> <block1> <block2> }
              ^ontop.top-block <block1>
              ^ontop.bottom-block <> <block2>)
   (<block1> ^type block)
   -->
   ...
   }
\end{verbatim}

This is not advisable because it corresponds to a different set of conditions
than those in the original production (the \soar{top-block} and
\soar{bottom-block} need not correspond to the same \soar{ontop} relation).
To check this, we could print the original production at the Soar prompt:

\begin{verbatim}
soar> print blocks-world*propose*move-block*dont-do-this
sp {blocks-world*propose*move-block*dont-do-this
    (state <s> ^problem-space blocks ^thing <thing2>
          ^thing { <> <thing2> <thing1> } ^ontop <o*1> ^ontop <o*2>)
    (<thing2> ^clear yes)
    (<thing1> ^clear yes ^type block)
    (<o*1> ^top-block <thing1>)
    (<o*2> ^bottom-block { <> <thing2> <b*1> })
    -->
    (<s> ^operator <o> +)
    (<o> ^name move-block 
         ^moving-block <thing1> 
         ^destination <thing2>) }
\end{verbatim}

Soar has expanded the production into the longer form, and created two
distinctive variables, \soar{$<$o*1$>$} and \soar{$<$o*2$>$} to represent the
\soar{ontop} attribute. These two variables will not necessarily bind to the
same identifiers in working memory.

% ----------------------------------------------------------------------------
\subsubsection*{Negated multi-valued attributes and attribute-path notation}

Negations of multi-valued attributes can be combined with attribute-path
notation. However; it is very easy to make mistakes when using negated
multi-valued attributes with attribute-path notation. Although it is
possible to do it correctly, we strongly discourage its use.

For example, 

\begin{verbatim}
sp {blocks*negated-conjunction-example
   (state <s> ^name top-state -^ontop.bottom-object.name table A)
   -->
   (<s> ^nothing-ontop-A-or-table true) }
\end{verbatim}

gets expanded to:

\begin{verbatim}
sp {blocks*negated-conjunction-example
   (state <s> ^name top-state)
  -{(<s> ^ontop <o*1>)
    (<o*1> ^bottom-object <b*1>)
    (<b*1> ^name A)
    (<b*1> ^name table)}
   -->
   (<s> ^nothing-ontop-A-or-table true) }
\end{verbatim}

This example does not refer to two different blocks with different
names. It tests that there is not an \soar{ontop} relation with a
\soar{bottom-block} that is named \soar{A} and named \soar{table}. Thus, this
production probably should have been written as:

\begin{verbatim}
sp {blocks*negated-conjunction-example
   (state <s> ^name top-state 
              -^ontop.bottom-object.name table
              -^ontop.bottom-object.name A)
   -->
   (<s> ^nothing-ontop-A-or-table true) }
\end{verbatim}

which expands to: 
\begin{verbatim}
sp {blocks*negated-conjunction-example
   (state <s> ^name top-state)
  -{(<s> ^ontop <o*2>)
    (<o*2> ^bottom-object <b*2>)
    (<b*2> ^name a)}
  -{(<s> ^ontop <o*1>)
    (<o*1> ^bottom-object <b*1>)
    (<b*1> ^name table)}
   -->
   (<s> ^nothing-ontop-a-or-table true +) }
\end{verbatim}

\subsubsection*{Notes on attribute-path notation}\vspace{-12pt}
\begin{itemize}
\item Attributes specified in attribute-path notation may not start with a
        digit. For example, if you type \soar{\carat foo.3.bar}, Soar thinks
        the \soar{.3} is a floating-point number. (Attributes that don't
        appear in path notation can begin with a number.)

\item Attribute-path notation may be used to any depth.

\item Attribute-path notation may be combined with structured values,
        described in Section \ref{SYNTAX-pm-structured}.

\end{itemize}


% ----------------------------------------------------------------------------
\subsubsection{Structured-value notation}
\label{SYNTAX-pm-structured}    %pref-struc-cond}
\index{structured value notation}
\index{production!structured values}
\index{value!structured notation}

Another convenience that eliminates the use of intermediate variables is 
structured-value notation. 

Syntactically, the attributes and values of a condition may be written where a
variable would normally be written. The attribute-value structure is delimited
by parentheses.

Using structured-value notation, the production in Figure \ref{fig:ex-prod}
(on page \pageref{fig:ex-prod}) may also be written as:

\begin{verbatim}
sp {blocks-world*propose*move-block
   (state <s> ^problem-space blocks
              ^thing <thing1> 
              ^thing {<> <thing1> <thing2>}
              ^ontop (^top-block <thing1>
                      ^bottom-block <> <thing2>))
   (<thing1> ^type block ^clear yes)
   (<thing2> ^clear yes)
-->
   (<s> ^operator <o> +)
   (<o> ^name move-block
        ^moving-block <thing1>
        ^destination <thing2>) }
\end{verbatim}

Thus, several conditions may be ``collapsed'' into a single condition.


\subsubsection*{Using variables within structured-value notation}

Variables are allowed within the parentheses of structured-value notation to
specify an identifier to be matched elsewhere in the production. For example,
the variable \soar{<ontop>} could be added to the conditions (although it are
not referenced again, so this is not helpful in this instance):

\begin{verbatim}
sp {blocks-world*propose*move-block
   (state <s> ^problem-space blocks
              ^thing <thing1> 
              ^thing {<> <thing1> <thing2>}
              ^ontop (<ontop> 
                      ^top-block <thing1>
                      ^bottom-block <> <thing2>))
   (<thing1> ^type block ^clear yes)
   (<thing2> ^clear yes)
   -->
   (<s> ^operator <o> +)
   (<o> ^name move-block
        ^moving-block <thing1>
        ^destination <thing2>) }
\end{verbatim}

Structured values may be nested to any depth. Thus, it is possible to write
our example production using a single condition with multiple structured
values:

\begin{verbatim}
sp {blocks-world*propose*move-block
   (state <s> ^problem-space blocks
              ^thing <thing1> 
                     ({<> <thing1> <thing2>}
                      ^clear yes)
              ^ontop (^top-block 
                        (<thing1>
                         ^type block 
                         ^clear yes)
                      ^bottom-block <> <thing2>) )
   -->
   (<s> ^operator <o> +)
   (<o> ^name move-block
        ^moving-block <thing1>
        ^destination <thing2>) }
\end{verbatim}


\subsubsection*{Notes on structured-value notation}\vspace{-12pt}
\begin{itemize}
\item Attribute-path notation and structured-value notation are orthogonal and
        can be combined in any way. A structured value can contain an
        attribute path, or a structure can be given as the value for an
        attribute path. 

\item Structured-value notation can be combined with negations and with
        multi-attributes. 

\item Structured-value notation can not be used in the actions of productions.

\end{itemize}


% ----------------------------------------------------------------------------
% ----------------------------------------------------------------------------
\subsection{The action side of productions (or RHS)}
\label{SYNTAX-pm-action}
\index{RHS of production|see{production}}
\index{production!action side (RHS)}

The action side of a production, also called the right-hand side (or RHS) of
the production, consists of individual actions that can:
\begin{itemize}
\item Add new elements to working memory.
\item Remove elements from working memory.
\item Create preferences.
\item Perform other actions
\end{itemize}

When the conditions of a production match working memory, the production is
said to be instantiated, and the production will fire during the next
elaboration cycle. Firing the production involves performing the actions
\emph{using the same variable bindings} that formed the instantiation.

\subsubsection{Variables in Actions}
\index{variables}
Variables can be used in actions.  A variable that appeared in the
condition side will be replaced with the value that is was bound to in
the condition.  A variable that appears only in the action side will be
bound to a new identifier that begins with the first letter of that
variable (e.g., \soar{<o>} might be bound to \soar{o234}). This symbol is
guaranteed to be unique and it will be used for all occurrences of the
variable in the action side, appearing in all working memory elements
and preferences that are created by the production action.

\subsubsection{Creating Working Memory Elements}
An element is created in working memory by specifying it as an action.
Multiple augmentations of an object can be combined into a single
action, using the same syntax as in conditions, including path notation
and multi-valued attributes. 
\begin{verbatim}
   -->
   (<s> ^block.color red
        ^thing <t1> <t2>) }
\end{verbatim}
The action above is expanded to be:
\begin{verbatim}
   -->
   (<s> ^block <*b>)
   (<*b> ^color red)
   (<s> ^thing <t1>)
   (<s> ^thing <t2>) }
\end{verbatim}
This will add four elements to working memory with the variables replaced
with whatever values they were bound to on the condition side.

Since Soar is case sensitive, different combinations of upper- and
lowercase letters represent \emph{different} constants. For example,
``\soar{red}'', ``\soar{Red}'', and ``\soar{RED}'' are all distinct symbols in
Soar. In many cases, it is prudent to choose one of uppercase or lowercase and
write all constants in that case to avoid confusion (and bugs).

The constants that are used for attributes and values have a few
restrictions on them:\vspace{-12pt} 
\begin{enumerate}
\item There are a number of architecturally created augmentations for state
        and impasse objects; see Section \ref{SYNTAX-impasses} for a listing of 
        these special augmentations. User-defined productions can not create
        or remove augmentations of states that use these
        attribute names.\vspace{-8pt}
\item Attribute names should not begin with a number if these attributes will
        be used in attribute-path notation.
\end{enumerate}

\subsubsection{Removing Working Memory Elements}

A element is explicitly removed from working memory by following the
value with a dash: \soar{-}, also called a reject.  

\begin{verbatim}
   -->
   (<s> ^block <b> -)}
\end{verbatim}

If the removal of a working memory element removes the only link between
the state and working memory elements that had the value of the removed
element as an identifier, those working memory elements will be
removed. This is applied recursively, so that all item that become
unlinked are removed.

The removal should be used with an action that will be o-supported.
If removal is attempted with i-support, the working memory element will
reappear if the removal loses i-support and the element still has
support.  

% ----------------------------------------------------------------------------
\subsubsection{The syntax of preferences}
\index{preference}

Below are the eleven types of preferences as they can appear in the actions of a
production for the selection of operators:
\label{pref-list}

\begin{tabular}{| l | l |} \hline
\bf{RHS preferences}                        & \bf{Semantics} \\ \hline
\soar{(id \carat operator value)}          & acceptable  \\ 
\soar{(id \carat operator value +)}        & acceptable  \\ 
\soar{(id \carat operator value !)}        & require \\ 
\soar{(id \carat operator value \tild)}    & prohibit \\
\soar{(id \carat operator value -)}        & reject \\
\soar{(id \carat operator value > value2)} & better \\
\soar{(id \carat operator value < value2)} & worse \\
\soar{(id \carat operator value >)}        & best  \\
\soar{(id \carat operator value <)}        & worst \\
\soar{(id \carat operator value =)}        & unary indifferent  \\
\soar{(id \carat operator value = value2)} & binary indifferent  \\
\soar{(id \carat operator value = number)} & numeric indifferent \\
\hline
\end{tabular} \vspace{10pt}
\index{+}
\index{"!}
\index{~}
\index{-}
\index{>}
\index{<}
\index{=}
%\index{&}


The identifier and value will always be variables, such as
\soar{(<s1> \carat operator <o1> > <o2>)}.

The preference notation appears similar to the predicate tests that
appear on the left-hand side of productions, but has very different
meaning. Predicates cannot be used on the right-hand side of a
production and you cannot restrict the bindings of variables on the
right-hand side of a production. (Such restrictions can happen only in
the conditions.)

Also notice that the \soar{+} symbol is optional when specifying acceptable
preferences in the actions of a production, although using this symbol
will make the semantics of your productions clearer in many instances. The
\soar{+} symbol will always appear when you inspect preference memory (with
the \soar{preferences} command).

Productions are never needed to delete preferences because preferences
will be retracted when the production no longer matches.  Preferences
should never be created by operator application rules, and they should
always be created by rules that will give only i-support to their actions.

% ----------------------------------------------------------------------------
\subsubsection{Shorthand notations for preference creation}

There are a few shorthand notations allowed for the creation of operator
preferences on the right-hand side of productions.

Acceptable preferences do not need to be specified with a \soar{+}
symbol. \soar{(<s> \carat operator <op1>)} is assumed to mean \soar{(<s> \carat
operator <op1> +)}.

Ambiguity can easily arise when using a preference that can be
either binary or unary: \soar{> < =}. The default assumption is that if a
value follows the preference, then the preference is binary. It will be unary
if a carat (up-arrow), a closing parenthesis, another preference, or a comma follows it. 

Below are four examples of legal, although unrealistic, actions that have the
same effect.

\begin{verbatim}
(<s> ^operator <o1> <o2> + <o2> < <o1> <o3> =, <o4>)
(<s> ^operator <o1> + <o2> + 
            <o2> < <o1> <o3> =, <o4> +)
(<s> ^operator <o1> <o2> <o2> < <o1> <o4> <o3> =)
(<s> ^operator <o1> ^operator <o2>
           ^operator <o2> < <o1> ^operator <o4> <o3> =)
\end{verbatim}

Any one of those actions could be expanded to the following list of
preferences: 
\begin{verbatim}
(<s> ^operator <o1> +)
(<s> ^operator <o2> +)
(<s> ^operator <o2> < <o1>)
(<s> ^operator <o3> =)
(<s> ^operator <o4> +)
\end{verbatim}

Note that structured-value notation may not be used in the actions of 
productions.

The examples above also demonstrate the use of commas within the RHS of a rule.
Commas may only be used to separate items within an action. Most of the time, their use makes no 
difference, and is only for visual clarity. However, commas can be useful to 
disambiguate between unary and binary preferences.

For example, \verb|(<s> ^operator <o1> <o2> > <o3>)| would be interpreted as
\begin{verbatim}
(<s> ^operator <o1> +
     ^operator <o2> > <o3>)
\end{verbatim}
But \verb|(<s> ^operator <o1> <o2> >, <o3>)| would be interpreted as
\begin{verbatim}
(<s> ^operator <o1> +
     ^operator <o2> >
     ^operator <o3> +)
\end{verbatim}


% ----------------------------------------------------------------------------
\subsubsection{Righthand-side Functions}
\index{RHS Function}

The fourth type of action that can occur in productions is called a 
\emph{righthand-side function}.  Righthand-side functions allow productions
to create side effects other than changing working memory.  The RHS functions
are described below, organized by the type of side effect they have.

% ----------------------------------------------------------------------------
\subsubsection{Stopping and pausing Soar}
\label{RHS-stopping}

\begin{description}
\index{RHS Function!halt}
\item [\soarb{halt} ---] Terminates Soar's execution and returns to 
the user prompt.  A \soar{halt} action irreversibly terminates the
running of a Soar program.
It should not be used if the agent is to be restarted (see the
 \soar{interrupt} RHS action below.)
\begin{verbatim}
sp {
    ...
    -->
    (halt) }
\end{verbatim} 

\index{RHS Function!interrupt}
\item [\soarb{interrupt} --- ]
        Executing this function causes Soar to stop at the end of the
        current phase, and return to the user prompt. This is similar 
        to \soar{halt}, but does not terminate the run.
        The run may be continued by issuing a \soar{run} command from
	the user interface.  The \soar{interrupt} RHS function has the
	same effect as typing \soar{stop-soar} at the prompt, except
	that there is more control because it takes effect exactly
	at the end of the phase that fires the production.
\begin{verbatim}
sp {
    ...
    -->
    (interrupt) }
\end{verbatim}
	
	\label{interrupt-directive}
	Soar execution may also be stopped immediately before a production
	fires, using the \soar{:interrupt} directive. This functionality is
	called a matchtime interrupt and is very useful for debugging. See
	Section	\ref{sp} on Page \pageref{sp} for more information.
	
\begin{verbatim}
sp {production*name
    :interrupt
    ...
    -->
    ...
    }
\end{verbatim}

\index{RHS Function!wait}
\item [\soarb{wait} --- ]
	Executing this function causes the current Soar thread to sleep for the given integer
	number of milliseconds.
\begin{verbatim}
sp {
    ...
    -->
    (wait 1000) }
\end{verbatim}

	Note that use of this function is discouraged.
\end{description}

% ----------------------------------------------------------------------------
\subsubsection{Text input and output}

These functions are provided as production actions to do simple output of text in Soar. Soar applications that do extensive input and output of text should use Soar Markup Language (SML). To learn about SML, read the "SML Quick Start Guide" which should be located in the "Documentation"  folder of your Soar install.

 
\begin{description}
\index{RHS Function!write}
\item [\soarb{write} --- ] This function writes its arguments to the standard
        output. It does not automatically insert blanks, linefeeds, or carriage
        returns.  For example, if \soar{<o>} is bound to 4, then
\begin{verbatim}
sp {
    ...
    -->
    (write  <o> <o> <o> | x| <o> | | <o>) }
\end{verbatim}

prints

\begin{verbatim}
444 x4 4
\end{verbatim}

\index{RHS Function!carriage return, line feed (crlf)}
\item [\soarb{crlf} --- ] Short for ``carriage return, line feed'', this
        function can be called only within \soar{write}. It forces a new line
        at its position in the \soar{write} action. 
\begin{verbatim}
sp {
    ...
    -->
    (write <x> (crlf) <y>) }
\end{verbatim}


%\index{RHS Function!accept}
%\item [\soarb{accept} --- ] Suspends Soar's execution and waits for the user
%        to type a constant, followed by a carriage return. The result of
%        \soar{accept} is the constant. The accept function does not read 
%	in strings.  It accepts a
%        single constant (which may look like a string).
%        Soar applications that make extensive use of text input should be
%        implemented using Tcl and Tk functionality, described in the
%        \emph{Soar Advanced Applications Manual}.

%The \soarb{accept} function does not work properly under the TSI 
%(Tcl-Soar Interface), or any other Soar program that has a separate 
%``Agent Window'' instead of a Tcl or Wish Console.  In this instance, 
%users should employ the \soar{tcl} RHS function 
%(described on page \pageref{SYNTAX-pm-otheractions-tcl}) to get user
%input through a text widget.
%\begin{verbatim} 
%sp {
%    ...
%    -->
%    (<s> ^input (accept)) }
%\end{verbatim}

        \nocomment{Does this imply that a CR is not needed? I.e., will the
                constant be 'accepted' after a space is hit?
                }

\index{RHS Function!log}
\item [\soarb{log} --- ] This function is equivalent to the \soar{write} function,
except that it specifies the ``trace channel'' for output. It takes two arguments. First is an integer
corresponding to the channel level for output, second is the message to print. \\
See section \ref{trace} for information about trace channel levels.

\begin{verbatim}
sp {
	...
	-->
	(log 3 |This only prints when trace is set to 3 or higher!|) }
\end{verbatim}


\end{description}

% ----------------------------------------------------------------------------
\subsubsection{Mathematical functions}
\index{arithmetic operations}

The expressions described in this section can be nested to any depth. For all
of the functions in this section, missing or non-numeric arguments result 
in an error.


\begin{description}
\index{RHS Function!floating-point calculations}
\item [\soarb{+, -, *, /} --- ]
        These symbols provide prefix notation mathematical functions.
        These symbols work similarly to C functions.  They will take either 
        integer or real-number arguments. The first three functions return 
        an integer when all arguments are integers
        and otherwise return a real number, and the last two functions
        always return a real number. These functions can each take any number of arguments,
        and will return the result of sequentially operating on each argument.
        The \soar{-} symbol is also a
	unary function which, given a single argument, returns the
	product of the argument and \soar{-1}.  The \soar{/} symbol is
	also a unary function which, given a single argument, returns the
	reciprocal of the argument (1/x).

\begin{verbatim}
sp {
    ...
    -->
    (<s> ^sum (+ <x> <y>)
         ^product-sum (* (+ <v> <w>) (+ <x> <y>))
         ^big-sum (+ <x> <y> <z> 402)
         ^negative-x (- <x>))
}
\end{verbatim}

\index{RHS Function!div}
\index{RHS Function!mod}
\item [\soarb{div, mod} --- ]
        These symbols provide prefix notation binary mathematical functions
        (they each take two arguments). These symbols work similarly to C
        functions: They will take only integer arguments (using reals results
        in an error) and return an integer: \soar{div} takes two integers and
        returns their integer quotient; \soar{mod} returns their remainder.

\begin{verbatim}
sp {
    ...
    -->
    (<s> ^quotient (div <x> <y>)
         ^remainder (mod <x> <y>)) }
\end{verbatim}

\index{RHS Function!abs}
\index{RHS Function!atan2}
\index{RHS Function!sqrt}
\index{RHS Function!sin}
\index{RHS Function!cos}
\item [\soarb{abs, atan2, sqrt, sin, cos} --- ]   
        These provide prefix notation unary 
        mathematical functions (they each take one argument). 
        These symbols work similarly to C functions:
        They will take either integer or real-number arguments. The
        first function (\soar{abs}) returns an integer when its argument is an
        integer and otherwise returns a real number, and the last four
        functions always return a real number.  \soar{atan2} returns as
	a float in radians, the arctangent of (first\_arg / second\_arg).
	\soar{sin} and \soar{cos} take as arguments the angle in radians.

\begin{verbatim}
sp {
    ...
    -->
    (<s> ^abs-value (abs <x>)
         ^sqrt (sqrt <x>)) }
\end{verbatim}

\index{RHS Function!min}
\index{RHS Function!max}
\item [\soarb{min, max} --- ]   
		These symbols provide n-ary mathematical functions (they each take a list of symbols as arguments).
		These symbols work similarly to C functions.
		They take either integer or real-number arguments, and return a real-number value if any
		of their arguments are real-numbers. Otherwise they return integers.

\begin{verbatim}
sp {
	...
	-->
	(<s> ^max (max <x> 3.14 <z>)
	     ^min (min <a> <b> 42 <c>)) }
\end{verbatim}


% ----------------------------------------------------------------------------
\index{RHS Function!int}
\item [\soarb{int} --- ] Converts a single symbol to an integer constant. This
        function expects either an integer constant, symbolic constant, or
        floating point constant. The symbolic constant must be a string which
        can be interpreted as a single integer. The floating point constant is
        truncated to only the integer portion. This function essentially
        operates as a type casting function.

        For example, the expression \soar{2 + sqrt(6)} could be printed
        as an integer using the following:

\begin{verbatim}
sp {
    ...
    -->
    (write (+ 2 (int sqrt(6))) ) }
\end{verbatim}

% ----------------------------------------------------------------------------
\index{RHS Function!float}
\item [\soarb{float} --- ] Converts a single symbol to a floating point 
constant.
        This function expects either an integer constant, symbolic constant,
        or floating point constant. The symbolic constant must be a string
        which can be interpreted as a single floating point number. This
        function essentially operates as a type casting function. 

        For example, if you wanted to print out an integer expression as a
        floating-point number, you could do the following:

\begin{verbatim}
sp {
    ...
    -->
    (write (float (+ 2 3))) }
\end{verbatim}

% ----------------------------------------------------------------------------
\index{RHS Function!ifeq}
\item [\soarb{ifeq} --- ] Conditionally return a symbol.
        This function takes four arguments. It returns the third argument if
        the first two are equal and the fourth argument otherwise. Note that
        symbols of different types will always be considered unequal. For example,
        1.0 and 1 will be unequal because the first is a float and the second is
        an integer.

\begin{verbatim}
sp {example-rule
    (state <s> ^a <a> ^b <b>)
    ...
    -->
    (write (ifeq <a> <b> equal not-equal)) }
\end{verbatim}
\end{description}

% ----------------------------------------------------------------------------
\subsubsection{Generating and manipulating symbols}

A new symbol (an identifier) is generated on the right-hand side of a
production whenever a previously unbound variable is used. This section
describes other ways of generating and manipulating symbols on the right-hand
side. 

\begin{description}
\index{RHS Function!capitalize-symbol}
\item [\soarb{capitalize-symbol} --- ] Given a symbol, this function returns a
	new symbol with the first character capitalized. This function is
	provided primarily for text output, for example, to allow the first
        word in a sentence to be capitalized.

        \nocomment{This command is possibly obsolete, since Soar7 is case sensitive?}

\begin{verbatim}
(capitalize-symbol foo)
\end{verbatim}


\index{RHS Function!compute-heading}
\item [\soarb{compute-heading} --- ] This function takes four real-valued arguments of the form \\
($x_1, y_1, x_2, y_2$), and returns the direction (in degrees) from ($x_1, y_1$)
to ($x_2, y_2$), rounded to the nearest integer.

For example:

\begin{verbatim}
sp {
    ...
    -->
    (<s> ^heading (compute-heading 0 0.5 32.5 28)) }
\end{verbatim}

After this rule fires, working memory would look like: \\
\verb|(S1 ^heading 48)|.


\index{RHS Function!compute-range}
\item [\soarb{compute-range} --- ] This function takes four real-valued arguments of the form \\
($x_1, y_1, x_2, y_2$), and returns the distance from ($x_1, y_1$) to ($x_2, y_2$), rounded to the nearest integer.

For example:

\begin{verbatim}
sp {
    ...
    -->
    (<s> ^distance (compute-range 0 0.5 32.5 28)) }
\end{verbatim}

After this rule fires, working memory would look like: \\
\verb|(S1 ^distance 42)|.


\index{RHS Function!concat}
\item [\soarb{concat} --- ] Given an arbitrary number of symbols, this function
        concatenates them together into a single constant symbol. 

For example:

\begin{verbatim}
sp {example
    (state <s> ^type state)
    -->
    (<s> ^name (concat foo bar (+ 2 4))) }
\end{verbatim}

       After this rule fires, the WME \verb=(S1 ^name foobar6)= will be added.


\index{RHS Function!deep-copy}
\item [\soarb{deep-copy} --- ] This function returns a copy of the given symbol
along with linked copies of all descendant symbols. In other terms, a full copy is made of
the working memory subgraph that can be reached when starting from the given symbol.
All copied identifiers are created as new IDs, and all copied values remain the same.

For example:

\begin{verbatim}
sp {
    (state <s> ^tree <t>)
    (<t> ^branch1 foo ^branch2 <b>)
    (<b> ^branch3 <t>)
    -->
    (<s> ^tree-copy (deep-copy <t>)) }
\end{verbatim}

After this rule fires, the following structure would exist:

\begin{verbatim}
(S1 ^tree T1 ^tree-copy D1)
  (T1 ^branch1 foo ^branch2 B1)
    (B1 ^branch3 T1)
  (D1 ^branch1 foo ^branch2 B2)
    (B2 ^branch3 D1)
\end{verbatim}


\index{RHS Function!dc}
\item [\soarb{dc} --- ] This function takes no arguments, and returns
the integer number of the current decision cycle. 

For example:

\begin{verbatim}
sp {example
    (state <s> ^type state)
    -->
    (<s> ^dc-count (dc) }
\end{verbatim}


\index{RHS Function!"@}
\item [\soarb{@ (get)} --- ] This function returns the LTI number of the
given ID. If the given ID is not linked to an LTI, it does nothing.
For example:

\begin{verbatim}
sp {example
    (state <s> ^stm <l1>)
    -->
    (<s> ^lti-num (@ <l1>) }
\end{verbatim}

After this rule fires, the \verb|(S1 ^lti-num)| WME will have an integer value such as \soar{42}.


\index{RHS Function!link-stm-to-ltm}
\item [\soarb{link-stm-to-ltm} --- ] This function takes two arguments. 
It links the first given symbol to the LTI indicated by the second integer value.
For example:

\begin{verbatim}
sp {example
    (state <s> ^stm <l1>)
    -->
    (link-stm-to-ltm <l1> 42) }
\end{verbatim}

After this rule fires, the WME \verb=(S1 ^stm <l1>)= will be linked to \verb=@42=.


\index{RHS Function!make-constant-symbol}
\item [\soarb{make-constant-symbol} --- ] This function returns a new constant 
symbol
guaranteed to be different from all symbols currently present in the
system.  With no arguments, it returns a symbol whose name starts with
``\soar{constant}''.  With one or more arguments, it takes those
argument symbols, concatenates them, and uses that as the
prefix for the new symbol. (It may also append a number to the 
resulting symbol, 
if a symbol with that prefix as its name already exists.)

\begin{verbatim}
sp {
    ...
    -->
    (<s> ^new-symbol (make-constant-symbol)) }
\end{verbatim}

When this production fires, it will create an augmentation in working
memory such as:

\begin{verbatim}
(S1 ^new-symbol constant5)
\end{verbatim} \vspace{12pt}

The production:

\begin{verbatim}
sp {
    ...
    -->
    (<s> ^new-symbol (make-constant-symbol <s> )) }
\end{verbatim}

will create an augmentation in working memory such as:
\begin{verbatim}
(S1 ^new-symbol |S14|)
\end{verbatim}

when it fires. The vertical bars denote that the symbol is a
constant, rather than an identifier; in this example, the number 4 has
been appended to the symbol S1.

This can be particularly useful when used in conjunction with the
\soar{timestamp} function; by using \soar{timestamp} as an argument to
\soar{make-constant-symbol}, you can get a new symbol that is
guaranteed to be unique. For example:

\begin{verbatim}
sp {
    ...
    -->
    (<s> ^new-symbol (make-constant-symbol (timestamp))) }
\end{verbatim}

When this production fires, it will create an augmentation in working
memory such as:

\begin{verbatim}
(S1 ^new-symbol 8/1/96-15:22:49)
\end{verbatim}    


\index{RHS Function!rand-float}
\item [\soarb{rand-float} --- ] This function takes an optional positive real-valued argument.
If no argument (or a negative argument) is given, it returns a random real-valued number in the range $[0.0,1.0]$.
Otherwise, given a value $n$, it returns a number in the range $[0.0, n]$.

For example:

\begin{verbatim}
sp {
    ...
    -->
    (<s> ^fate (rand-float 1000)) }
\end{verbatim}

After this rule fires, working memory might look like: \\
\verb|(S1 ^fate 275.481802)|.


\index{RHS Function!rand-int}
\item [\soarb{rand-int} --- ] This function takes an optional positive integer argument.
If no argument (or a negative argument) is given, it returns a random integer number in the range $[-2^{31}, 2^{31}]$.
Otherwise, given a value $n$, it returns a number in the range $[0, n]$.

For example:

\begin{verbatim}
sp {
    ...
    -->
    (<s> ^fate (rand-int 1000)) }
\end{verbatim}

After this rule fires, working memory might look like: \\
\verb|(S1 ^fate 13)|.


\index{RHS Function!round-off}
\item [\soarb{round-off} --- ] This function returns the first given 
value rounded to the nearest multiple of the second given value.
Values must be integers or real-numbers.

For example:

\begin{verbatim}
sp {
    (state <s> ^pi <pi>
    -->
    (<s> ^pie (round-off <pi> 0.1)) }
\end{verbatim}

After this rule fires, working memory might look like: \\
\verb|(S1 ^pi 3.14159 ^pie 3.1)|.


\index{RHS Function!round-off-heading}
\item [\soarb{round-off-heading} --- ] This function is the same as \soar{round-off},
but additionally shifts the returned value by multiples of 360 such that $-360 \le value \le 360$.

For example:

\begin{verbatim}
sp {
    (state <s> ^heading <dir>
    -->
    (<s> ^true-heading (round-off-heading <dir> 0.5)) }
\end{verbatim}

After this rule fires, working memory might look like: \\
\verb|(S1 ^heading 526.432 ^true-heading 166.5)|.


\index{RHS Function!size}
\item [\soarb{size} --- ] This function returns an integer symbol whose value
is the count of WME augmentations on a given ID argument. Providing a non-ID
argument results in an error.

For example:

\begin{verbatim}
sp {
    (state <s> ^numbers <n>)
    (<n> ^1 1 ^10 10 ^100 100)
    -->
    (<s> ^augs (size <n>)) }
\end{verbatim}

After this rule fires, the value of \verb=S1 ^augs= would be $3$.

Note that some architecturally-maintained IDs such as \verb=(<s> ^epmem)= and \verb=(<s> ^io)=
are not counted by the \soar{size} function.

\index{RHS Function!strlen}
\item [\soarb{strlen} --- ] This function returns an integer symbol whose value
is the size of the given string symbol.

For example:

\begin{verbatim}
sp {
    (state <s> ^io.input-link.message <m>)
    ...
    -->
    (<s> ^message-len (strlen <m>)) }
\end{verbatim}


\index{RHS Function!timestamp}
\item [\soarb{timestamp} --- ] This function returns a symbol whose print name 
is a
representation of the current date and time. 

For example:

\begin{verbatim}
sp {
    ...
    -->
    (write (timestamp)) }
\end{verbatim}

When this production fires, it will print out a representation of the
current date and time, such as:
\begin{verbatim}
soar> run 1 e
8/1/96-15:22:49
\end{verbatim}


\index{RHS Function!trim}
\item [\soarb{trim} --- ] This function takes a single string symbol argument
and returns the same string with leading and trailing whitespace removed.

For example:

\begin{verbatim}
sp {
    (state <s> ^message <m>)
    -->
    (<s> ^trimmed (trim <m>)) }
\end{verbatim}

\end{description}

% ----------------------------------------------------------------------------
\subsubsection{User-defined functions and interface commands as RHS actions}
%\label{SYNTAX-pm-otheractions-tcl}

Any function which has a certain function signature may be registered with the
Kernel and called as a RHS function.  The function must have the following signature:

\begin{verbatim}
std::string MyFunction(smlRhsEventId id, void* pUserData, Agent* pAgent,
                  char const* pFunctionName, char const* pArgument);
\end{verbatim}

The Tcl and Java interfaces have similar function signatures. Any arguments passed
to the function on the RHS of a production are concatenated and passed to the function
in the pArgument argument.

Such a function can be registered with the kernel via the client interface by calling:

\begin{verbatim}
Kernel::AddRhsFunction(char const* pRhsFunctionName, RhsEventHandler 
                   handler, void* pUserData);
\end{verbatim}

The \soar{exec} and \soar{cmd} functions are used to call user-defined functions and interface
commands on the RHS of a production.

\begin{description}
\index{RHS Function!exec}
\item [\soarb{exec} --- ] Used to call user-defined registered functions. Any arguments are concatenated
without spaces. For example, if \soar{<o>} is bound to \soar{x}, then

\begin{verbatim}
sp {
   ...
   -->
   (exec MakeANote <o> 1) }
\end{verbatim}
   
will call the user-defined \soar{MakeANote} function with the argument "\soar{x1}".

The return value of the function, if any, may be placed in working memory or passed
to another RHS function. For example, the log of a number \soar{<x>} could be printed this way:

\begin{verbatim}
sp {
   ...
   -->
   (write |The log of | <x> | is: | (exec log(<x>))|) }
\end{verbatim}

where "\soar{log}" is a registered user-defined function.

\index{RHS Function!cmd}
\item[\soarb{cmd} --- ] Used to call built-in Soar commands. Spaces are inserted between concatenated 
arguments. For example, the production

\begin{verbatim}
sp {
   ...
   -->
   (write (cmd print --depth 2 <s>)) }
\end{verbatim}

will have the effect of printing the object bound to \soar{<s>} to depth 2.
\end{description}

%There are no safety nets with this function, and users are warned that they
%can get themselves into trouble if not careful.  Users should
%\emph{never} use the \soar{tcl} RHS function to invoke \soar{add-wme},
%\soar{remove-wme} or \soar{sp}.

% ----------------------------------------------------------------------------
\subsubsection{Controlling chunking}
\label{SYNTAX-pm-actions-learning}

\nocomment{These RHS actions have not been implemented as of this writing. The
        functionality is achieved using the user-interface functions
        ``chunky-problem-spaces'' and ``chunk-free-problem-spaces''; see
        online help or the web pages for details on these functions.}


Chunking is described in Chapter \ref{CHUNKING}.

The following two functions are provided as RHS actions to assist in
development of Soar programs; they are not intended to correspond to any
theory of learning in Soar. This functionality is provided as a development 
tool, so that learning may be turned off in specific problem spaces,
preventing otherwise buggy behavior.

The \soar{dont-learn} and \soar{force-learn} RHS actions are to be used with
specific settings for the \soar{chunk} command (see page \pageref{chunk}.)
Using the \soar{chunk} command, learning may be set to one of \soar{always},
\soar{never}, \soar{flagged}, or \soar{unflagged}; chunking must be set to
\soar{flagged} for the \soar{force-learn} RHS action to have any effect and
chunking must be set to \soar{unflagged} for the \soar{dont-learn} RHS action
to have any effect.

\begin{description}
\index{RHS Function!dont-learn}
\item [\soarb{dont-learn} --- ] When chunking is set to \soar{unflagged},
        by default chunks can be formed in all states; the \soar{dont-learn}
        RHS action will cause chunking to be turned off for the specified
        state.

\begin{verbatim}
sp {turn-learning-off
    (state <s> ^feature 1 ^feature 2 -^feature 3)
     -->
    (dont-learn <s>) }
\end{verbatim}

        The \soar{dont-learn} RHS action applies when \soar{chunk} is 
	    set to \soar{unflagged}, and has no effect when other settings for
        \soar{chunk} are used.


\index{RHS Function!force-learn}
\item [\soarb{force-learn} --- ] When learning is set to \soar{flagged},
        by default chunks are not formed in any state; the \soar{force-learn}
        RHS action will cause chunking to be turned on for the specified
        state.

\begin{verbatim}
sp {turn-learning-on
    (state <s> ^feature 1 ^feature 2 -^feature 3)
     -->
    (force-learn <s>) }
\end{verbatim}

        The \soar{force-learn} RHS action applies when \soar{chunk}
	    is set to \soar{flagged}, and has no effect when other settings for
        \soar{chunk} are used.

\end{description}
% ----------------------------------------------------------------------------
\subsection{Grammars for production syntax} 
\label{GRAMMARS}
\index{grammar}

This subsection contains the BNF grammars for the conditions and actions of
productions. (BNF stands for Backus-Naur form or Backus normal form; consult a
computer science book on theory, programming languages, or compilers for more
information. However, if you don't already know what a BNF grammar is, it's
unlikely that you have any need for this subsection.)

This information is provided for advanced Soar users, for example, those who
need to write their own parsers. Note that some terms (e.g. \soar{<sym\_constant>})
are undefined; as such, this grammar should only be used as a starting point.

\comment{this section still needs a disclaimer that what you can actually do
	is less restrictive than the way we described it in the main text } 

\comment{note that grammars are no longer consistent with new rhs actions}

\nocomment{John and I decided while talking about this that we just wouldn't let
	people know that they could omit the identifier of the state
	
	It is legal to omit the variable test for a state when that variable is not
	tested elsewhere in the production, nor used in the action.  For
	example: 
	\begin{verbatim}
	(state ^operator <o>)
	\end{verbatim}
	
	is equivalent to 
	\begin{verbatim}
	(state <s> ^operator <o>)
	\end{verbatim}
}

%-------------------------------------------------------
\subsubsection{Grammar of Soar productions}
\index{production!grammar}
\index{grammar}

A grammar for Soar productions is:
\begin{verbatim}
<soar-production>  ::= sp "{" <production-name> [<documentation>] [<flags>]
<condition-side> --> <action-side> "}"
<documentation>    ::= """ [<string>] """
<flags>            ::= ":" (o-support | i-support | chunk | default)
\end{verbatim}

% ----------------------------------------------------------------------------
\paragraph{Grammar for Condition Side:}
\label{SYNTAX-pm-condgrammar}

Below is a grammar for the condition sides of productions:
\begin{verbatim}
<condition-side>   ::= <state-imp-cond> <cond>*
<state-imp-cond>   ::= "(" (state | impasse) [<id_test>]
<attr_value_tests>+ ")"
<cond>             ::= <positive_cond> | "-" <positive_cond>
<positive_cond>    ::= <conds_for_one_id> | "{" <cond>+ "}"
<conds_for_one_id> ::= "(" [(state|impasse)] <id_test> 
<attr_value_tests>+ ")"
<id_test>          ::= <test>
<attr_value_tests> ::= ["-"] "^" <attr_test> ("." <attr_test>)*
<value_test>*
<attr_test>        ::= <test>
<value_test>       ::= <test> ["+"] | <conds_for_one_id> ["+"]  

<test>             ::= <conjunctive_test> | <simple_test>
<conjunctive_test> ::= "{" <simple_test>+ "}"
<simple_test>      ::= <disjunction_test> | <relational_test>
<disjunction_test> ::= "<<" <constant>+ ">>"
<relational_test>  ::= [<relation>] <single_test>
<relation>         ::= "<>" | "<" | ">" | "<=" | ">=" | "=" | "<=>"
<single_test>      ::= <variable> | <constant>
<variable>         ::= "<" <sym_constant> ">"
<constant>         ::= <sym_constant> | <int_constant> | <float_constant>
\end{verbatim}
\index{constant}
\index{variables}

\paragraph*{Notes on the Condition Side}\vspace{-12pt}
\begin{itemize}
	\item In an \soar{<id\_test>}, only a \soar{<variable>} may be used in a \soar{<single\_test>}.
\end{itemize}

\comment{I don't think that grammar is quite right -- e.g. should distinguish
	that acceptable preferences may appear for operators, but not other
	objects}

\comment{Grammar correctly describes Soar; it's just that you can actually do
	things that we've said can't be done. So in this section we'll mention
	that we lied before and that the grammar above is different, but
	correct.  see notes on difference on page 64 of June 7th draft}


% ----------------------------------------------------------------------------
\paragraph{Grammar for Action Side:}
\label{SYNTAX-pm-actgrammar}    %RHS grammar}
\index{grammar}

\comment{RD: this grammar is out of date}

Below is a grammar for the action sides of productions:
\begin{verbatim}
<rhs>                      ::= <rhs_action>*
<rhs_action>               ::= "(" <variable> <attr_value_make>+ ")" 
| <func_call>
<func_call>                ::= "(" <func_name> <rhs_value>* ")"
<func_name>                ::= <sym_constant> | "+" | "-" | "*" | "/"
<rhs_value>                ::= <constant> | <func_call> | <variable>
<attr_value_make>          ::= "^" <variable_or_sym_constant>
("." <variable_or_sym_constant>)* <value_make>+
<variable_or_sym_constant> ::= <variable> | <sym_constant>
<value_make>               ::= <rhs_value> <preference_specifier>*

<preference-specifier>     ::= <unary-preference> [","]
| <unary-or-binary-preference> [","]
| <unary-or-binary-preference> <rhs_value> [","]
<unary-pref>               ::= "+" | "-" | "!" | "~"
<unary-or-binary-pref>     ::= ">" | "=" | "<"
\end{verbatim}

\comment{I don't quite understand that last bit. 
	<forced-unary-pref>        ::= <binary-preference> {, | ) | ^}
	(but the parser doesn't consume the ")" or "^" here)}

\index{constant}
\index{variable}

% ----------------------------------------------------------------------------
%\subsection{Writing Productions that Create O-supported Preferences}

\nocomment{there's no discussion of o-support in this chapter, and probably
        there should be. maybe a quick separate section on the syntax of
        o-supported productions?

        [things to mention in this section: you can't always tell whether a
        preference will have o-support just by looking at the production
        (o-support is determined at runtime), and rules for determining
        o-support.]  
        }


% ----------------------------------------------------------------------------
% ----------------------------------------------------------------------------
\section{Impasses in Working Memory and in Productions}
\label{SYNTAX-impasses}
\index{subgoal}
\index{impasse}

When the preferences in preference memory cannot be resolved unambiguously,
Soar reaches an impasse, as described in Section \ref{ARCH-impasses}:
\vspace{-12pt}
\begin{itemize}
\item When Soar is unable to select a new operator (in the decision cycle), it
        is said to reach an operator impasse.
\end{itemize}\vspace{-8pt}

All impasses lead to the creation of a new substate in working memory, and appear 
as objects within that substate. These objects can be tested by productions.
This section describes the structure of state objects in working memory.

% ----------------------------------------------------------------------------
\subsection{Impasses in working memory}
\label{SYNTAX-impasseaug}       %perf-goal-impa}
\index{decision procedure}
\index{impasse}
\index{impasse!types}
\index{impasse!resolution}
\index{impasse!tie}
\index{impasse!conflict}
\index{impasse!constraint-failure}
\index{impasse!no-change}
\index{goal!termination}
\index{subgoal!termination}
\index{subgoal!augmentations}

There are four types of impasses. 

Below is a short description of the four types of impasses. (This was
described in more detail in Section \ref{ARCH-impasses} on page
\pageref{ARCH-impasses}.)

\vspace{-12pt}
\begin{enumerate}
\item \emph{tie}: when there is a collection of equally eligible operators
        competing for the value of a particular attribute;\vspace{-8pt}
\item \emph{conflict}: when two or more objects are better than each other,
        and they are not dominated by a third operator;\vspace{-8pt}
\item \emph{constraint-failure}: when there are conflicting necessity
        preferences; \vspace{-8pt}
\item \emph{no-change}: when the proposal phase runs to quiescence without 
        suggesting a new operator.
\end{enumerate}

The list below gives the seven augmentations that the architecture creates on the
substate generated when an impasse is reached, and the
values that each augmentation can contain:

\vspace{-12pt}
\begin{description} 
\item [\soar{\carat type state}] 
	\index{type (attribute)}
	\vspace{-8pt} 
\item [\soar{\carat impasse}] Contains the impasse type: 
	\soar{tie}, \soar{conflict}, \soar{constraint-failure}, or \soar{no-change}.
	\index{impasse (attribute)}
	\vspace{-8pt} 
\item [\soar{\carat choices}]Either \soar{multiple} (for tie and conflict impasses), 
	\soar{constraint-failure} \\ (for constraint-failure impasses), or 
	\soar{none} (for constraint-failure or no-change impasses).
	\index{choices (attribute)}
	\vspace{-8pt} 
\item [\soar{\carat superstate}] Contains the identifier of the state in which 
	the impasse arose.
	\index{superstate (attribute)}
	\vspace{-8pt}
\item [\soar{\carat attribute}] For multi-choice and constraint-failure impasses,
	this contains \soar{operator}. For no-change impasses, this contains 
	the attribute of the last decision with a value (\soar{state} or \soar{operator}).
	\index{attribute (attribute)}
	\vspace{-8pt}
\item [\soar{\carat item}] For multi-choice and constraint-failure impasses, this 
	contains all values involved in the tie, conflict, or constraint-failure. 
	If the set of items that tie or conflict changes during the impasse, the architecture 
	removes or adds the appropriate item augmentations without terminating the existing impasse.
	\index{item (attribute)}
	\vspace{-8pt}
\item [\soar{\carat item-count}] For multi-choice and constraint-failure impasses, this 
	contains the number of values listed under the item augmentation above.
	\index{item-count (attribute)}
	\vspace{-8pt}
\item [\soar{\carat non-numeric}] For tie impasses, this contains all operators that
	do not have numeric indifferent preferences associated with them. If the
	set of items that tie changes during the impasse, the architecture
	removes or adds the appropriate non-numeric augmentations without
	terminating the existing impasse.
	\index{non-numeric (attribute)} 
	\vspace{-8pt}
\item [\soar{\carat non-numeric-count}] For tie impasses, this contains the number of 
	operators listed under the non-numeric augmentation above.
	\index{non-numeric-count (attribute)}
	\vspace{-8pt}
\item [\soar{\carat quiescence}] States are the only objects with \soar{quiescence t}, 
	which is an explicit statement that quiescence (exhaustion of the elaboration cycle) 
	was reached in the superstate.  If problem solving in the subgoal is contingent on 
	quiescence having been reached, the substate should test this flag. The side-effect 
	is that no chunk will be built if it depended on that test. 
	See Section \ref{CHUNKING-creation} on page \pageref{CHUNKING-creation} for
	details. This attribute can be ignored when learning is turned off.
	\index{quiescence t (augmentation)}
	\index{elaboration cycle}
	\index{exhaustion}
\end{description} 

Knowing the names of these architecturally defined attributes and their
possible values will help you to write productions that test for the presence
of specific types of impasses so that you can attempt to resolve the impasse
in a manner appropriate to your program. Many of the default
productions in the \soar{demos/defaults} directory of the Soar distribution
 provide means for resolving
certain types of impasses. You may wish to make use of some of all of these
productions or merely use them as guides for writing your own set of
productions to respond to impasses.

\subsubsection*{Examples}

\nocomment{rewrite this section to show templates of what the objects look like
	in working memory for each different types of impasses}

The following is an example of a substate that is created for a tie among
three operators:
\index{goal!examples}
\index{impasse!examples}
\begin{verbatim}
(S12 ^type state ^impasse tie ^choices multiple ^attribute operator 
     ^superstate S3 ^item O9 O10 O11 ^quiescence t)
\end{verbatim} \vspace{12pt}

The following is an example of a substate that is created for a no-change
impasse to apply an operator:
\begin{verbatim}
(S12 ^type state ^impasse no-change ^choices none ^attribute operator 
     ^superstate S3 ^quiescence t)
(S3 ^operator O2)
\end{verbatim} \vspace{12pt}

% ----------------------------------------------------------------------------
\subsection{Testing for impasses in productions}

Since states appear in working memory, they may also be
tested for in the conditions of productions.

% There are numerous examples of this in the set of default productions (see
% Section \ref{default} or Appendix \ref{DEFAULT} for more information).

For example, the following production tests for a constraint-failure impasse
on the top-level state.

\begin{verbatim}
sp {default*top-goal*halt*operator*failure
    "Halt if no operator can be selected for the top goal."
    :default
    (state <ss> ^impasse constraint-failure ^superstate <s>)
    (<s> ^superstate nil)
-->
    (write (crlf) |No operator can be selected for top goal.| )
    (write (crlf) |Soar will halt now. Goodnight.| )
    (halt)
}
\end{verbatim}

% ----------------------------------------------------------------------------
\section{Soar I/O: Input and Output in Soar}
\label{SYNTAX-io}
\index{I/O}
\index{motor commands|see{I/O}}

Many Soar users will want their programs to interact with a real or simulated
environment. For example, Soar programs could control a robot, receiving sensory
\emph{inputs} and sending command \textit{outputs}. Soar programs might 
also interact with
simulated environments, such as a flight simulator. The mechanisms by which
Soar receives inputs and sends outputs to an external process is called
\emph{Soar I/O}.

\index{I/O!SML}
This section describes how input and output are represented in working memory
and in productions. Interfacing with a Soar agent through input and output can 
be done using the \textit{Soar Markup Language} (SML). The details of designing 
an external process that uses SML to create the input and respond to output from 
Soar are beyond the scope of this manual, but they are described 
\soar{\htmladdnormallink{online}{https://soar.eecs.umich.edu/articles/articles/soar-markup-language-sml}} 
on the Soar website. This section is provided for the sake of Soar users who will be making
use of a program that has already been implemented, or for those who would
simply like to understand how I/O works in Soar.
% A simple example
% of Soar I/O using Tcl is provided in Section (Appendix?) \ref{Interface-Tcl_I/O}.


% ----------------------------------------------------------------------------
\subsection{Overview of Soar I/O}

When Soar interacts with an external environment, it must make use of
mechanisms that allow it to receive input from that environment and to effect
changes in that environment. An external environment may be the real world or
a simulation; input is usually viewed as Soar's perception and output is
viewed as Soar's motor abilities.


\index{I/O!input functions}
\index{I/O!output functions}
\index{input functions|see{I/O}}
\index{output functions|see{I/O}}
Soar I/O is accomplished via \emph{input functions} and
\emph{output functions}. Input functions are called at the 
\emph{start}
of every execution cycle, and add elements directly to specific input
structures in working memory.  These changes to working memory
may change the set of productions that will fire or retract. 
Output functions are called
at the \emph{end} of every execution cycle and are processed in response to
changes to specific output structures in working memory.  An output function
is called only if changes have been made to the output-link structures in
working memory.

\index{I/O!io attribute}
\index{io attribute|see{I/O}}
\index{I/O!input links}
\index{I/O!output links}
\index{input links|see{I/O}}
\index{output links|see{I/O}}
The structures for manipulating input and output in Soar are linked
to a predefined attribute of the
top-level state, called the \soar{io} attribute.  The \soar{io} attribute has
substructure to represent sensor inputs from the environment called
\emph{input links}; because these are represented in working memory, Soar
productions can match against input links to respond to an external
situation. Likewise, the \soar{io} attribute has substructure to
represent motor commands, called \emph{output links}. Functions that 
execute motor commands in the environment use the values on the output links 
to determine when and how they should execute an action.  Generally,
input functions create and remove elements on the input link to update
Soar's perception of the environment.  Output functions respond to values
of working memory elements that appear on Soar's output link strucure.



% ----------------------------------------------------------------------------
\subsection{Input and output in working memory}
\label{ADVANCED-io-wm}

All input and output is represented in working memory as substructure of the
\soar{io} attribute of the top-level state.  By default, the architecture
creates an \soar{input-link} attribute of the \soar{io} object and
an \soar{output-link} attribute of the io object. 
The values of the \soar{input-link} and \soar{output-link} attributes
are identifiers whose augmentations are the complete set of input and
output working memory elements, respectively.  Some Soar systems may 
benefit from having multiple input and output links, or that use names
which are more
descriptive of the input or output function, such as \soar{vision-input-link},
\soar{text-input-link}, or \soar{motor-output-link}.  In addition to
providing  the default \soar{io} substructure, the architecture allows
users to create multiple input and output links via productions
and I/O functions.  Any identifiers for \soar{io} substructure created
by the user will be assigned at run time and are not guaranteed to be
the same from run to run.  Therefore users should always employ
variables when referring to input and output links in productions.

Suppose a blocks-world task is implemented using a robot to move
actual blocks around, with a camera creating input to Soar and a robotic arm
executing command outputs. 
\begin{figure}
\insertfigure{blocks-inputlink}{3.5in}
\insertcaption{An example portion of the input link for the blocks-world task.}
\label{fig:blocks-inputlink}
\end{figure}
The camera image might be analyzed by a separate vision program; this program
could have as its output the locations of blocks on an xy plane.  
The Soar input function could take the
output from the vision program and create the following working memory
elements on the input link (all identifiers are assigned at runtime; 
this is just an example of possible bindings):

\begin{verbatim}
(S1 ^io I1)          [A]
(I1 ^input-link I2)  [A]
(I2 ^block B1)
(I2 ^block B2)
(I2 ^block B3)
(B1 ^x-location 1)
(B1 ^y-location 0)
(B1 ^color red)
(B2 ^x-location 2)
(B2 ^y-location 0)
(B2 ^color blue)
(B3 ^x-location 3)
(B3 ^y-location 0)
(B3 ^color yellow)
\end{verbatim} \vspace{12pt}

The '[A]' notation in the example is used to indicate the working memory
elements that are created by the architecture and not by the input function.
This configuration of blocks corresponds to all blocks on the table, as
illustrated in the initial state in Figure \ref{fig:blocks}.

\begin{figure}
\insertfigure{blocks-outputlink}{3.5in}
\insertcaption{An example portion of the output link for the blocks-world task.}
\label{fig:blocks-outputlink}
\end{figure}

Then, during the Apply Phase of the execution cycle, Soar productions could 
respond to an operator, such as ``move the red block
ontop of the blue block'' by creating a structure on the output link, such as:

\begin{verbatim}
(S1 ^io I1)           [A]
(I1 ^output-link I3)  [A]
(I3 ^name move-block)
(I3 ^moving-block B1)
(I3 ^x-destination 2)
(I3 ^y-destination 1)
(B1 ^x-location 1)
(B1 ^y-location 0)
(B1 ^color red)
\end{verbatim}  \vspace{12pt}

An output function would look for specific structure in this output link and
translate this into the format required by the external program that controls
the robotic arm. Movement by the robotic arm would lead to changes in the 
vision system, which would later be reported on the input-link.

Input and output are viewed from Soar's perspective. An \emph{input
function} adds or deletes augmentations of the \soar{input-link} 
providing Soar with information about some occurrence external to Soar. An
\emph{output function} responds to substructure of the \soar{output-link}
produced by production firings, and causes some occurrence external to
Soar. Input and output occur through the \soar{io} attribute of the top-level
state exclusively.
\index{top-state!for I/O}

Structures placed on the input-link by an input function remain there until removed
by an input function. During this time, the structure continues to provide support for
any production that has matched against it. The structure does \emph{not} cause the production
to rematch and fire again on each cycle as long as it remains in working memory;
to get the production to refire, the structure must be removed and added again.



%The substructure of the input-link will remain in working memory until 
%the input function that
%created it removes it.  Thus working memory elements produced by an
%input function provide support for condition-matching
%in productions as long as the input persists in working memory, i.e.
%until the input function specifically removes the elements of the
%substructure.  However,
%a production that tests only a single element on the input structure will 
%result in instantiations that fire only once for each input element that
%matches.  The instantiation will not continue to fire for each matched
%input element, unless the element is removed and then added again.


% ----------------------------------------------------------------------------
\subsection{Input and output in production memory}
\label{ADVANCED-io-pm}

Productions involved in \emph{input} will test for specific attributes and
values on the input-link, while productions involved in \emph{output} will
create preferences for specific attributes and values on the output link.
For example, a simplified production that responds to the vision input 
for the blocks task might look like this:

\begin{verbatim}
sp {blocks-world*elaborate*input
    (state <s> ^io.input-link <in>)
    (<in> ^block <ib1>)
    (<ib1> ^x-location <x1> ^y-location <y1>)
    (<in> ^block {<ib2> <> <ib1>})
    (<ib2> ^x-location <x1> ^y-location {<y2> > <y1>})
    -->
    (<s> ^block <b1>)
    (<s> ^block <b2>)
    (<b1> ^x-location <x1>  ^y-location <y1> ^clear no)
    (<b2> ^x-location <x1>  ^y-location <y2> ^above <b1>)
}
\end{verbatim}  \vspace{12pt}

This production ``copies'' two blocks and their locations directly to 
the top-level state. 
%This is a generally a good idea when using input, since the input
%function may change the information on the link before the Soar program has
%finished using it.
It also adds information about the
relationship between the two blocks.  The variables used
for the blocks on the RHS of the production are deliberately different from the
variable name used for the block on the input-link in the LHS of the
production. If the variable were the same, the production would create 
a link into the structure of the input-link, rather than copy the information.
The attributes \soar{x-location} and
\soar{y-location} are assumed to be values and not identifiers, so the same
variable names may be used to do the copying.


A production that creates wmes on the output-link for the blocks task 
might look like this:

\begin{verbatim}
sp {blocks-world*apply*move-block*send-output-command
    (state <s> ^operator <o> ^io.output-link <out>)
    (<o> ^name move-block ^moving-block <b1> ^destination <b2>)
    (<b1> ^x-location <x1> ^y-location <y1>)
    (<b2> ^x-location <x2> ^y-location <y2>)
    -->
    (<out> ^move-block <b1>
           ^x-destination <x2> ^y-destination (+ <y2> 1))
}
\end{verbatim} \vspace{12pt}

This production would create substructure on the output-link that 
the output function could interpret as being a command to 
move the block to a new location.



% ----------------------------------------------------------------------------
\typeout{--------------- learning (CHUNKING) --------------------------------}
\chapter{Chunking}
\label{CHUNKING}
\index{chunking}
\index{learning}
\index{result}
\index{subgoal}

\nocomment{

	This chapter really needs a general explanation of learning in
	Soar before it dives into WHEN chunks are and are not formed,
	and HOW ... cf. Chpater 2-level discussions

	I'd like to add two figures to this chapter: \\
	1. an illustration of the backtracing process \\
	2. an illustration of how learning fits into the decision cycle.

	Also, I'm pretty sure this chapter hasn't changed much since
	justifications were added to Soar. It would be nice to connect
	justifications to chunks.
	}


\nocomment{
	\begin{figure}
	%\insertfigure{learn}{7.75in}
	\insertcaption{Will somehow be an illustration of learning in
		the decision cycle.} 
	\label{fig:learn}
	\end{figure}
	}

Chunking is Soar's mechanism to learn new procedural knowledge.
Chunking creates productions, called \emph{chunks}, that summarize the
processing required to produce the results of subgoals. When a chunk is built,
it is added to production memory, where it will be matched in similar
situations, avoiding the need for the subgoal. Chunks are created only when
results are formed in subgoals; since most Soar programs are continuously
subgoaling and returning results to higher-level states, chunks are typically
created continuously as Soar runs.

This chapter begins with a discussion of when chunks are built (Section
\ref{CHUNKING-creation} below), followed by a detailed discussion of
how Soar determines a chunk's conditions and actions (Section
\ref{CHUNKING-determining}). Sections \ref{CHUNKING-variablizing} through
\ref{CHUNKING-ordering} examine the construction of chunks in further
detail. Section \ref{CHUNKING-inhibition} explains how and why chunks are
prevented from matching with the WME's that led to their creation. Section
\ref{CHUNKING-problems} reviews the problem of overgeneral chunks.


% ----------------------------------------------------------------------------
\section{Chunk Creation}
\label{CHUNKING-creation}
\index{chunking!creation}

Several factors govern when chunks are built. Soar chunks the results of every
subgoal, \emph{unless} one of the following conditions is true:
\index{chunking!when active}

\index{learn}
\begin{enumerate}
\item Chunking is \soar{off}. (See Section \ref{chunk} on page \pageref{chunk}
	for details of \soar{chunk} and how to turn chunking on or off.
	When \soar{chunk} is set to \soar{always} chunks are built.  
	When \soar{chunk} is set to \soar{never}, chunks are not built.)

\item The chunking option \soar{bottom-only} is \soar{on} and a chunk has
	already been built for a subgoal of the state that generated the results. 
	(See Section \ref{chunk} on page \pageref{chunk} for details of 
	\soar{chunk} used with the \soar{bottom-only} setting.) 

	With bottom-only chunking, chunks are learned only in states in which no
	subgoal has yet generated a chunk. In this mode, chunks are learned
	only for the ``bottom'' of the subgoal hierarchy and not the
	intermediate levels. With experience, the subgoals at the bottom will
	be replaced by the chunks, allowing higher level subgoals to be
	chunked.\footnote{For some tasks, bottom-up chunking facilitates
	modeling power-law speedups, although its long-term theoretical
	status is problematic.}
	\index{bottom-only chunking}
	\index{chunking!bottom-only}
	
	\nocomment{this would be the appropriate place in the manual to discuss
		the rationale behind the existence of bottom-up chunking. I
		don't believe it's explained anywhere. 
		}

\item The chunk setting \soar{allow-local-negations} is \soar{off}, and the
  result is dependent on a test for the negation of a subgoal WME. Testing a
  local negation can result in an overgeneral chunk (see Section
  \ref{CHUNKING-problems} on page \pageref{CHUNKING-problems}). In this mode,
  such chunks are not created.

\item The chunk duplicates a production or chunk already in production memory.
	In some rare cases, a duplicate production will not be detected because the
	order of the conditions or actions is not the same as an existing production.  
	\index{chunking!duplicate chunks}

\item The augmentation, \soar{\carat quiescence t}, of the substate that
	produced the result is backtraced through.

	This mechanism is motivated by the \emph{chunking from exhaustion}
	problem, where the results of a subgoal are dependent on the
	exhaustion of alternatives (see Section \ref{CHUNKING-problems} on page
	\pageref{CHUNKING-problems}). If this substate augmentation is
	encountered when determining the conditions of a chunk, then no chunk
	will be built for the currently considered action. This is recursive, 
	so that if an un-chunked result is relevant to a second result, no 
	chunk will be built for the second result. This does not prevent the
	creation of a chunk that would include \soar{\carat quiescence t} as a
	condition.  
	\index{quiescence t (augmentation)} \index{exhaustion}
      
\item Chunking has been temporarily turned off via a call to the
	\soar{dont-learn} production action (described on page
	\pageref{SYNTAX-pm-actions-learning} in Section 
	\ref{SYNTAX-pm-actions-learning}).

	This capability is provided for debugging and system development, and
	it is not part of the theory of Soar.
\end{enumerate}

If a result is to be chunked, Soar builds the chunk \emph{as soon as the
result is created}, rather than waiting until subgoal termination.
\index{result}
\index{subgoal!result}

	\nocomment{CBC: As soon as it's identified as a result, I assume.
		E.g., for the case where a ``result'' is created
		first, and not linked to the superstate until later.
	
		BobD: it doesn't become a result until it's linked.}

% ----------------------------------------------------------------------------
\section{Determining Conditions and Actions}
\label{CHUNKING-determining}

Chunking is an experience-based learning mechanism that summarizes  as 
productions the problem solving that occurs within a state. In order to 
maintain a
history of the processing to be used for chunking, Soar builds a 
\emph{trace} of the productions that fire in the subgoals. This section
describes how the relevant actions are determined, how information is 
stored in a trace, and finally, how the trace and the actions together 
determine the conditions for the chunk.

In order for the chunk to apply at the appropriate time, its conditions must
test exactly those working memory elements that were necessary to produce the
results of the subgoal. Soar computes a chunk's conditions based on the
productions that fire in the subgoal, beginning with the results of the subgoal,
and then \emph{backtracing} through the productions that created each result. 
It recursively backtraces through the working memory elements that matched the
conditions of the productions, finding the actions that led to the WME's
creation, etc., until conditions are found that test elements that are linked to
a superstate. \index{backtracing}

\index{working memory!trace}
\index{trace!memory}

\nocomment{  This is what is used to say...
Soar computes a chunk's conditions based on the productions that 
fire in the subgoal. Chunking begins with the results of the subgoal,
and then \emph{backtraces} through the productions that created the preference
for each result. It then recursively backtraces through the working memory
elements that matched the conditions of the productions, finding the
acceptable preferences that led to their creation, etc., until conditions are
found that test elements that are linked to a superstate.}

% ----------------------------------------------------------------------------
\subsection{Determining a chunk's actions}
\index{result}
\index{subgoal result}
\index{chunking!determining actions}

A chunk's actions are built from the results of a subgoal.  A \emph{result} is
any working memory element created in the substate that is linked to a 
superstate.  A working memory element
is linked if its identifier is either the value of a superstate
WME, or the value of an augmentation  for an object that is linked to a
superstate.

\index{linked!chunk action}
\index{chunking!actions}

The results produced by a single production firing are the basis for creating
the actions of a chunk. A new result can lead to other results by linking a
superstate to a WME in the substate. This WME may in turn link
other WMEs in the substate to the superstate, making them results.
Therefore, the creation of a single WME that is linked to a superstate
can lead to the creation of a large number of results. All of the newly
created results become the basis of the chunk's actions.

% ----------------------------------------------------------------------------
\subsection{Tracing the creation and reference of working memory elements} 

Soar automatically maintains information on the creation of each 
working memory element in every state.  When a production fires, a
trace of the production is saved with the appropriate state. A \emph{trace} is
a list of the working memory elements matched by the production's conditions,
together with the actions created by the production.  The appropriate state
is the most recently created state (i.e., the state \emph{lowest} in the
subgoal hierarchy) that occurs in the production's matched working memory
elements.
\index{trace!memory}

Recall that when a subgoal is created, the \carat item augmentation lists all
values that lead to the impasse.
Chunking is complicated by the fact that the \soar{\carat item} augmentation
of the substate is created by the architecture and not by productions.
Backtracing cannot determine the cause of these substate augmentations in the
same way as other working memory elements. To overcome this, Soar maps these
augmentations onto the acceptable preferences for the operators in the 
\soar{\carat item} augmentations.


\subsubsection*{Negated conditions}
\index{negated conditions}
\index{chunking!negated conditions}

Negated conditions are included in a trace in the following way: when a
production fires, its negated conditions are fully instantiated with its
variables' appropriate values. This instantiation is based on the working
memory elements that matched the production's positive conditions. If the
variable is not used in any positive conditions, such as in a conjunctive
negation, a dummy variable is used that will later become a variable in a
chunk.

If the identifier used to instantiate a negated condition's identifier field
is linked to the superstate, then the instantiated negated condition is
added to the trace as a negated condition. In all other cases, the negated
condition is ignored because the system cannot determine why a working memory
element \emph{was not} produced in the subgoal and thus allowed the production
to fire. Ignoring these negations of conditions internal to the subgoal may
lead to overgeneralization in chunking (see Section \ref{CHUNKING-problems} on
page \pageref{CHUNKING-problems}). 
\index{overgeneral chunk}
     
% ----------------------------------------------------------------------------
\subsection{Determining a chunk's conditions}
\index{chunking!determining conditions}

The conditions of a chunk are determined by a dependency analysis of
production traces --- a process called \emph{backtracing}.  For each
instantiated production that creates a subgoal result, backtracing examines
the production trace to determine which working memory elements were matched.
If a matched working memory element is linked to a superstate, it is included
in the chunk's conditions. If it is not linked to a superstate, then
backtracing recursively examines the trace of the production that created the
working memory element. Thus, backtracing begins with a subgoal result, traces
backwards through all working memory elements that were used to produce that
result, and collects all of the working memory elements that are linked to a
superstate. This method ignores when the working memory elements were created,
thus allowing the conditions of one chunk to test the results of a chunk
learned earlier in the subgoal. The user can observe the backtracing process
by setting setting backtracing on, using the watch command: \soar{watch
backtracing -on} (see Section \ref{watch} on page \pageref{watch}). 
This prints out a trace of the conditions as they are collected.
\index{backtracing}
\index{chunking!conditions}

Given that results can be created at any point during a subgoal, it is
possible for one result to be relevant to another result. Whether or not the 
first result is included in the chunk for the second result depends on the
links that were used to match the first result in the subgoal. If the elements
are linked to the superstate, they are included as conditions. If the
elements are not linked to the superstate, then the result is traced through.
In some cases, there may be more than one set of links, so it is possible for
a result to be both backtraced through, and included as a condition.

\index{desirability preference} 
\index{preference!desirability}
\index{Context-Dependent Preference Set}
Some operator evaluations productions, i.e. rules that did not actually produce
the result but did lead to the selection of the operator, can also participate
in backtracing.  These rules produce a set of desirability preferences that we
call the context-dependent preference set (CDPS).  The rules that produced the
preferences in the CDPS are backtraced through to potentially add additional
conditions to the chunk.

Not all operator evaluation preferences are included in the CDPS.
Traditionally the argument has been that operator evaluation preferences should
affect only the \emph{efficiency} and not the \emph{correctness} of problem
solving, and therefore are not necessary to produce the results. As a result,
desirability preferences, namely \soar{reject}, (\soar{better},
\soar{worse}, \soar{best}, \soar{worst}, or \soar{indifferent}), were not included
in the CDPS and hence would not be backtraced through when creating chunks. (The
exception to this is that if the desirability or \soar{reject} preference is a
{\em result} of a subgoal, it will be in the chunk's actions.) The philosophy
behind this was that desirability and reject preferences should be used only as
search control for choosing between legal alternatives and should not be used to
guarantee the correctness of the problem solving. In contrast, necessity
preferences (\soar{require} or \soar{prohibit}) were used to enforce the
correctness of problem solving, and the productions that created these
preferences were always included in backtracing.
\index{require preference}
\index{prohibit preference}
\index{preference!require}
\index{preference!prohibit}
\index{preference!acceptable}
\index{preference!reject}
\index{preference!better}
\index{preference!worse}
\index{preference!best}
\index{preference!worst}
\index{preference!indifferent}
\index{preference!numeric-indifferent}

This is still the default behavior.  Only prohibit and require preference are
included in the CDPS. In some case, though, this may lead to over-general chunks
(see Section \ref{CHUNKING-problems} on page \pageref{CHUNKING-problems} for
details).  As of Soar 9.3.3, there is a new learning option (see Section
\ref{chunk} on page \pageref{chunk} for details) that will include a broader
range of desirability preferences in the CDPS. The rules
that determine which desirability preferences are included in the CDPS -- it
depends on whether a particular desirability preferences played a
role in the selection of the operator -- are documented in
Section \ref{CDPS} on page \pageref{CDPS}.

% ----------------------------------------------------------------------------
\section{The Context-Dependent Preference Set}
\label{CDPS}
\index{Context-Dependent Preference Set}

As described in the beginning of this chapter, chunking summarizes the
processing required to produce the results of sub-goals.  Traditionally, the
philosophy behind how an agent should be designed was
that the path of operator selections and applications from an initial state in a
sub-state to a result would always have all necessary tests in the operator
proposal conditions and any goal test, so only those items would need to be
“summarized”. The idea was that in a properly designed agent, a sub-state's
operator evaluation preferences lead to a more efficient search of the space but
do not influence the “correctness” of the result. As a result, the knowledge
used by rules that produce such evaluation preferences should not be included
in any chunks produced from that sub-state.

\index{chunking!overgeneral}
\index{chunking!incorrect chunks}
\index{incorrect chunks}
\index{overgeneral chunk}
In practice, however, a Soar program can be written so that search control does
affect the correctness of search.  A few examples can be found in Section
\ref{CHUNKING-problems} on page \pageref{CHUNKING-problems}.  Moreover,
reinforcement learning rules are often used to direct the search to those states
that provide more reward, consequently making the idea of correctness much
fuzzier.   As a result, there may be cases when it is useful to encode
goal-attainment knowledge into operator evaluation rules. Unfortunately, chunks
created in such a problem space will be overgeneral because important parts of
the superstate that were tested by operator evaluation rules do not appear as
conditions in the chunks that summarize the processing in that problem state.
The context-dependent preference set is a way to address this issue.

\emph{The context-dependent preference set (CDPS) is the set of
\textbf{relevant} operator evaluation preferences that led to the selection of
an operator in a sub-goal.} Whenever Soar creates either a justification or a
chunk, it recursively backtraces through rules to determine what conditions to include in the final chunk or justification.  With the CDPS, not only does Soar backtrace through each rule that created a matched working
memory element, but it also backtraces through every rule that
created a preferences in the CDPS for the selected operator when that rule fired. By
backtracing through that additional set of preferences at each step of the
backtrace, an agent will create more specific chunks that incorporate the
goal-attainment knowledge encoded in the operator evaluation rules.

\index{desirability preference}
\index{preference!desirability}
\index{necessity preference}
\index{preference!necessity}
All necessity preferences, i.e. prohibit and require preferences, are always
included in the CDPS since they inherently encode the correctness of whether an
operator is applicable in a problem space.  In contrast, desirability
preferences (rejects, betters, worses, bests, worsts and indifferents) are
included depending on the role they play in the selection of the operator (and
whether the add-desirability-prefs learn setting is active).

\index{decision!procedure}
How Soar determines which of those preferences to include in the CDPS is determined by the preference semantics it uses to choose an operator.
During the decision phase, operator preferences are evaluated in a sequence of
seven steps or filters, in an effort to select a single operator. Each step
handles a specific type of preference.  \emph{As the preference semantics are applied
at each step to incrementally filter the candidates to a potential selected operator, the CDPS
is incrementally built up based on the preferences that were instrumental in applying each filter.}

The following outline describes the logic that happens at each step.  For a more
detailed description of the various filters (but not the CDPS) see Appendix
\ref{PREFERENCES} on page \pageref{PREFERENCES}.  Note that depending on the set of preferences being processed, impasses may occur at some
of these stages, in which case, no operator is selected and the CDPS is emptied.
 Moreover, if the candidate set is reduced to zero or one, the decision process
will exit with a finalized CDPS. For simplicity's sake, this explanation assumes
that there are no impasses and the decision process continues.

\begin{itemize}
\index{preference!require}
\index{require preference}
\index{"!}
\item Require Filter: 
\begin{itemize}
\item If an operator is selected based on a require preference, that preference
is added to the CDPS.  The logic behind this step is straightforward, the
require preference directly resulted in the selection of the operator.
\end{itemize}

\index{preference!prohibit}
\index{prohibit preference}
\index{preference!reject}
\index{reject preference}
\item Prohibit/Reject Filters: 
\begin{itemize}
\item If there exists at least one prohibit or reject preference, all prohibit
and reject preferences for the eliminated candidates are added to the CDPS.  The logic
behind this stage is that the conditions that led to the exclusion of the
prohibited and rejected candidates is what allowed the final operator to be
selected from among that particular set of surviving candidates.
\end{itemize}

\index{preference!worse}
\index{worse preference}
\index{preference!better}
\index{better preference}
\item Better/Worse Filter: 
\begin{itemize}
\item For every candidate that is not worse than some other candidate, add all better/worse preferences involving the candidate.
\end{itemize}

\index{preference!best}
\index{best preference}
\item Best Filter: 
\begin{itemize}
\item Add any best preferences for remaining candidates to the CDPS. 
\end{itemize}

\index{preference!worst}
\index{worst preference}
\item Worst Filter:
\begin{itemize}
\item If any remaining candidate has a worst preference which leads to that
candidate being removed from consideration, that worst preference is added to
the CDPS.  Again, the logic is that the conditions that led to
that candidate not being selected allowed the final operator to be chosen.
\end{itemize}

\index{preference!indifferent}
\index{indifferent preference}
\item Indifferent Filter: 
\begin{itemize}
\item This is the final stage, so the operator is now selected based on the
agent's exploration policy.  How indifferent preferences are added to the CDPS
depends on whether any numeric indifferent preferences exist.
\begin{enumerate}
\item If there exists at least one numeric indifferent preference, then every
numeric preferences for the winning candidate is added to the CDPS.  There can
be multiple such preferences. Moreover, all binary indifferent preferences
between that winning candidate and candidates without a numeric preference are
added.
\item If all indifferent preferences are non-numeric, then any unary indifferent
preferences for the winning candidate are added to the CDPS.  Moreover, all
binary indifferent preferences between that winning candidate and other
candidates are added.
\end{enumerate}
\item The logic behind adding binary indifferent preferences between the
selected operator and the other final candidates is that those binary
indifferent preferences prevented a tie impasse and allowed the final
candidate to be chosen by the exploration policy from among those mutually
indifferent preferences.
\end{itemize}
\end{itemize}

Note that there may be cases where two or more rules create the same type of
preference for a particular candidate.  In those cases, only the first
preference encountered is added to the CDPS.  Adding all of them can produce
over-specific chunks.  It may still be
possible to learn similar chunks with those other preferences if the agent sub-goals
again in a similar context.

As of version 9.3.3, desirability preferences will not be added to the CDPS by
default.  The setting must be turned on via the learn command's
“add-desirability-prefs” setting.  See Section \ref{learn} on page \pageref{learn} for more
information.  Necessity preferences will always be added to the CDPS regardless
of setting.

\index{justification!conditions}
Note that the CDPS also affects the conditions of justifications, so the
add-desirability-prefs setting does have an effect on the agent even if learning
is turned off.

% ----------------------------------------------------------------------------
\section{Variablizing Identifiers}
\label{CHUNKING-variablizing}
\index{identifier!variablization of}
\index{chunking!variablization}
\index{variablization}

Chunks are constructed by examining the traces, which include working memory
elements and operator preferences. To achieve any useful generality in chunks,
identifiers of actual objects must be replaced by variables when the chunk is
created; otherwise chunks will only ever fire when the exact same objects
are matched.  However, a constant value is never variablized; the actual 
value always appears directly in the chunk.

When a chunk is built, all occurrences of the same identifier are replaced
with the same variable. This can lead to an overspecific chunk, when two
variables are forced to be the same in the chunk, even though distinct
variables in the original productions just happened to match the same
identifier.

A chunk's conditions are also constrained by any not-equal (\soar{<>}) tests
for pairs of indentifiers used in the conditions of productions that are
included in the chunk. These tests are saved in the production traces and then
added in to the chunk.
\index{chunking!conditions}

% ----------------------------------------------------------------------------
\section{Ordering Conditions}
\label{CHUNKING-ordering}
\index{matcher}
\index{ordering chunk conditions}

	\nocomment{I think we need an actual section (earlier in the
		manual), describing the rete matcher and the
		reordering of conditions (which also happens 
		internally for user-defined productions). Then this
		section would mention that it's the same reordering process.}

Since the efficiency of the Rete matcher  \cite{Forg81} depends
heavily upon the order of a production's conditions, the chunking mechanism
attempts to write the chunk's conditions in the most favorable order. At each
stage, the condition-ordering algorithm tries to determine which eligible
condition, if placed next, will lead to the fewest number of partial
instantiations when the chunk is matched. A condition that matches an object
with a multi-valued attribute will lead to multiple partial instantiations, so
it is generally more efficient to place these conditions later in the
ordering.
\index{chunking!ordering conditions}
\index{multi-valued attribute}

This is the same process that internally reorders the conditions in
user-defined productions, as mentioned briefly in Section \ref{ARCH-pm-structure}. 


% ----------------------------------------------------------------------------
\section{Inhibition of Chunks}
\label{CHUNKING-inhibition}
\index{chunking!refractory inhibition}
\index{refractory inhibition of chunks}

When a chunk is built, it may be able to match immediately with the same
working memory elements that participated in its creation. If the production's
actions include preferences for new operators, the production would immediately
fire and create a preference for a new operator, which duplicates the 
operator preference
that was the original result of the subgoal. To prevent this,
\emph{inhibition} is used. This means that each production that is built 
during chunking is considered to have already fired with the instantiation of
the exact set of working memory elements used to create it. This does not
prevent a newly learned chunk from matching other working memory elements
that are present and firing with those values.

	\nocomment{any insights to why its called ``refractory''?}

% ----------------------------------------------------------------------------
\section{Problems that May Arise with Chunking}
\label{CHUNKING-problems}
\index{chunking!overgeneral}
\index{chunking!incorrect chunks}
\index{incorrect chunks}
\index{overgeneral chunk}

\nocomment{Moved from chapter 2: If there are no variables in justifications, I
	don't quite understand how overgeneralization can occur. \\
        RD: can still be overgeneral due to lack of conditions, e.g.,
	chunking from exhaustion (from testing a negative condition in the
	subgoal) \\
	JEL: it's from losing the local negations}

\nocomment{BobD: there are more problems with chunking than this. See last two
	slides from ``guts of chunking'', 11th soar workshop (and talk to Bob
	and/or John for an explanation
	}

\nocomment{SBW 8/08: updating this section to cover the additional cases in the
talk referenced above (16 years later!) }

One of the weaknesses of Soar is that chunking can create overgeneral productions
that apply in inappropriate situations, or overspecific productions that will
never fire. These problems arise when chunking cannot accurately summarize the
processing that led to the creation of a result. Below is a description of
five known problems in chunking.

\subsection{Using search control to determine correctness}

Overgeneral chunks can be created if a result of problem solving in a subgoal
is dependent on search-control knowledge. Recall that desirability
preferences, such as \soar{better}, \soar{best}, and \soar{worst}, are not
included in the traces of problem solving used in chunking (Section
\ref{CHUNKING-determining} on page \pageref{CHUNKING-determining}). In theory,
these preferences do not affect the validity of search. In practice, however,
a Soar program can be written so that search control \emph{does} affect the
correctness of search. Here are two examples:\vspace{-12pt}

\begin{enumerate} 
\item Some of the tests for correctness of a result are included in
	productions that prefer operators that will produce correct results.
  	The system will work correctly only when those productions are loaded.\vspace{-8pt}
\item An operator is given a worst preference, indicating that it
  	should be used only when all other options have been exhausted.
  	Because of the semantics of worst, this operator will be selected
  	after all other operators; however, if this operator then produces a
  	result that is dependent on the operator occurring after all others,
  	this fact will not be captured in the conditions of the chunk.
\end{enumerate}

In both of these cases, part of the test for producing a result is {\em
implicit} in search control productions. This move allows the explicit state
test to be simpler because any state to which the test is applied is
guaranteed to satisfy some of the requirements for success. However, chunks
created in such a problem space will be overgeneral because the implicit parts
of the state test do not appear as conditions. 

\index{desirability preference}
\index{preference!desirability}
\index{necessity preference}
\index{preference!necessity}
\index{Context-Dependent Preference Set}
\textbf{Solution:} To avoid this problem, the context-dependent preference set
can be used to include any search-control knowledge that incorporates
goal-attainment knowledge in the backtrace used by chunking;  this will produce
chunks with additional conditions that resulted from the search-control
preferences, thereby avoiding overgenerality.  By default, all necessity
preferences (\soar{require} and \soar{prohibit}) are included in the CDPS.  If
goal-attainment knowledge is also included in desirability prefs, such as
\soar{better}, \soar{worse}, \soar{best}, \soar{worst}, or \soar{indifferent},
the add-desirability-prefs setting of the CDPS (Section \ref{chunk} on \pageref{chunk}) 
can be turned on to incorporate them into the backtrace and hence
any resulting chunks.   The rules that determine which desirability preferences
are included in the CDPS -- it depends on whether a particular desirability
preferences played an important role in the selection of an
operator that ultimately led to the result -- are documented in Section \ref{CDPS} on \pageref{CDPS}.
\index{necessity preference} 
\index{require preference}
\index{prohibit preference}
\index{preference!require}
\index{preference!prohibit}
\index{preference!acceptable}
\index{preference!reject}
\index{preference!better}
\index{preference!worse}
\index{preference!best}
\index{preference!worst}
\index{preference!indifferent}
\index{preference!numeric-indifferent}

\subsection{Testing for local negated conditions}

Overgeneral chunks can be created when negated conditions test for the absence
of a working memory element that, if it existed, would be local to the
substate.  Chunking has no mechanism for determining \textit{why} a given
working memory element does not exist, and thus a condition that occurred in a
production in the subgoal is not included in the chunk. For example, if a
production tests for the absence of a local flag, and that flag is copied down
to the substate from a superstate, then the chunk should include a test that
the flag in the superstate does not exist. 
Unfortunately, it is computationally expensive to determine why a given
working memory element does not exist. Chunking only includes negated tests if
they test for the absence of superstate working memory elements. 

\textbf{Solution:} To avoid using negated conditions for local data, the local
data can be made a result by attaching it to the superstate. This increases
the number of chunks learned, but a negated condition for the superstate can
be used that leads to correct chunks.

Alternatively, Soar's chunking mode can be set to reject chunks when the
backtrace encounters a local negation, by setting \soar{chunk
allow-local-negations on}. There are many cases where local negations
are safe to ignore (and hence this mode reduces performance), but it can
substantially reduce the number of overgeneral chunks in big agents (and aid in
debugging).

\index{negated!conditions}
\index{chunking!negated conditions}

\subsection{Testing for the substate}

Overgeneral chunks can be created if a result of a subgoal is dependent on the
creation of an impasse within the substate. For example, processing in a
subgoal may consist of exhaustively applying all the operators in the problem
space. If so, then a convenient way to recognize that all operators have
applied and processing is complete is to wait for a state no-change impasse to
occur. When the impasse occurs, a production can test for the resulting
substate and create a result for the original subgoal. This form of state test
builds overgeneral chunks because no pre-existing structure is relevant to the
result that terminates the subgoal. The result is dependent only on the
existence of the substate within a substate.
\index{quiescence t (augmentation)}
\index{exhaustion}

\textbf{Solution:} The current solution to this problem is to allow the
problem solving to signal the architecture that the test for a substate is
being made.  The signal used by Soar is a test for the \soar{\carat quiescence
t} augmentation of the subgoal.  The chunking mechanism recognizes this test
and does not build a chunk when it is found in a backtrace of a subgoal.  The
history of this test is maintained, so that if the result of the substate is
then used to produce further results for a superstate, no higher chunks will
be built.  However, if the result is used as search control (it is a
desirability preference), then it does not prevent the creation of chunks
because the original result is not included in the backtrace.  If the
\soar{\carat quiescence t} being tested is connected to a superstate, it will
not inhibit chunking and it will be included in the conditions of the chunk.

\subsection{Mapping multiple superstate WMEs to one local WME}

An agent may have several rule instantiations that match on different
structures in a superstate but create WMEs with the same attribute-value pairs
in a substate. For example, there may be a rule that matches several WMEs in a
superstate with the same multi-valued attribute and elaborates the local state
with a WME indicating that at least one WME with that attribute exists.  In
these cases, the total effect of those rule firings will be collapsed into
creating a single WME in the substate, because working memory is represented as
a set.  If this WME is then tested to create a result on the superstate, the
chunk that is subsequently created will be overgeneral: While the original
subgoal processing created only one result, the chunk will create a distinct
result for each superstate structure originally tested. This is because the
desired behavior cannot be reduced to a single rule.

\textbf{Solution:} If this type of behavior is needed, the single WME should go
in the top state, so that the chunks built can similarly map multiple
structures to one.

\subsection{Revising the substructure of a previous result}

This can occur when a subgoal creates a local structure, which is then linked to a
superstate, becoming a result. A new WME added to this structure is also a result, as
as it is linked to the superstate. However, if that WME is created
with a rule that matches the local state only (not the superstate), Soar cannot
build a chunk for the result, as it is unable to determine how the new WME is linked
to the superstate.

For example, assume that an agent builds up a structure consisting of an identifier
called \soar{\carat thing} attached to a substate, and then adds
\soar{\carat property foo} as an augmentation to \soar{thing}.
If the agent now matches \soar{thing} on the substate, and creates a WME on a
superstate linked to the same identifier, that identifier, along with its
augmentation \soar{\carat property foo}, becomes a result, and a chunk is formed.
Now, if a rule in the subgoal adds another augmentation to the \soar{thing}
identifier, (\soar{\carat property bar}, say), that augmentation will also be
a result, as it is linked to an identifier which is linked to a superstate.
However, if that rule matches the identifier through the substate, the chunking
process cannot determine how it is linked to the superstate, and a chunk cannot
be created.


\textbf{Solution:} If the substructure of a result must be revised, the rules
that modify it should match the result through the superstate, not through the
local state.

\nocomment{
	\subsection{Overuse of predicates}

	Moved from chapter 2:

	All of the predicate tests are lost in the chunk, and only the
	exact value is included. If the predicate is explicitly
	represented as a relation between two objects in working
	memory, chunking will capture that abstract relationship 
	and create a much more general chunk.

	(also needs clarification about the use of predicates in the
	blocks world task, where we have to say  that the block that's
	being moved is not the same as the block that's 
        being moved to)

	}

\chapter{Reinforcement Learning}
\label{RL}
\index{reinforcement learning}
\index{preference!numeric-indifferent}
\index{rl}

Soar has a reinforcement learning (RL) mechanism that tunes operator selection knowledge based on a given reward function.
This chapter describes the RL mechanism and how it is integrated with production memory, the decision cycle, and the state stack.
We assume that the reader is familiar with basic reinforcement learning concepts and notation. If not, we recommend first reading \emph{Reinforcement Learning: An Introduction} (1998) by Richard S. Sutton and Andrew G. Barto.
The detailed behavior of the RL mechanism is determined by numerous parameters that can be controlled and configured via the \soarb{rl} command.
Please refer to the documentation for that command in section \ref{rl} on page \pageref{rl}.

\section{RL Rules}
\label{RL-rules}

Soar's RL mechanism learns Q-values for state-operator\footnote{
In this context, the term ``state'' refers to the state of the task or environment, not a state identifier.
For the rest of this chapter, bold capital letter names such as \soarb{S1} will refer to identifiers and italic lowercase names such as $s_1$ will refer to task states.}
pairs.
Q-values are stored as numeric indifferent preferences created by specially formulated productions called \emph{RL rules}.
RL rules are identified by syntax.
A production is a RL rule if and only if its left hand side tests for a proposed operator, its right hand side creates a single numeric indifferent preference, and it is not a template rule (see \ref{RL-templates}).
These constraints ease the technical requirements of identifying/updating RL rules and makes it easy for the agent programmer to add/maintain RL capabilities within an agent.
We define an \emph{RL operator} as an operator with numeric indifferent preferences created by RL rules.

The following is an RL rule:

\begin{verbatim}
sp {rl*3*12*left
   (state <s> ^name task-name
              ^x 3
              ^y 12
	          ^operator <o> +)
   (<o> ^name move
	    ^direction left)
-->
   (<s> ^operator <o> = 1.5)
}
\end{verbatim}

Note that the LHS of the rule can test for anything as long as it contains a test for a proposed operator.
The RHS is constrained to exactly one action: creating a numeric indifferent preference for the proposed operator.

The following are not RL rules:

\begin{verbatim}
sp {multiple*preferences
   (state <s> ^operator <o> +)
-->
   (<s> ^operator <o> = 5, >)
}

sp {variable*binding
    (state <s> ^operator <o> +
               ^value <v>)
-->
    (<s> ^operator <o> = <v>)
}
\end{verbatim}

The first rule proposes multiple preferences for the proposed operator and thus does not comply with the rule format.
The second rule does not comply because it does not provide a \emph{constant} for the numeric indifferent preference value.

In the typical RL use case, the user intends for the agent to learn the best operator in each possible state of the environment.
The most straightforward way to achieve this is to give the agent a set of RL rules, each matching exactly one possible state-operator pair.
This approach is equivalent to a table-based RL algorithm, where the Q-value of each state-operator pair corresponds to the numeric indifferent preference created by exactly one RL rule.

In the more general case, multiple RL rules can match a single state-operator pair, and a single RL rule can match multiple state-operator pairs.
all numeric indifferent preferences for an operator are summed when calculating the operator's Q-value\footnote{
This is assuming the value of \soarb{numeric-indifferent-mode} is set to \soarb{sum}.
In general, the RL mechanism only works correctly when this is the case, and we assume this case in the rest of the chapter.
See page \pageref{decide-numeric-indifferent-mode} for more information about this parameter.}.
In this context, RL rules can be interpreted more generally as binary features in a linear approximator of each state-operator pair's Q-value, and their numeric indifferent preference values their weights.
In other words,
$$Q(s, a) = w_1 \phi_2 (s, a) + w_2 \phi_2 (s, a) + \ldots + w_n \phi_n (s, a)$$
where all RL rules in production memory are numbered $1 \dots n$, $Q(s, a)$ is the Q-value of the state-operator pair $(s, a)$, $w_i$ is the numeric indifferent preference value of RL rule $i$, $\phi_i (s, a) = 0$ if RL rule $i$ does not match $(s, a)$, and $\phi_i (s, a) = 1$ if it does.
This interpretation allows RL rules to simulate a number of popular function approximation schemes used in RL such as tile coding and sparse coding.

\section{Reward Representation}
\label{RL-reward}

RL updates are driven by reward signals.
In Soar, these reward signals are given to the RL mechanism through a working memory link called the \soarb{reward-link}.
Each state in Soar's state stack is automatically populated with a \soarb{reward-link} structure upon creation.
Soar will check this structure for a numeric reward signal for the last operator executed in the associated state at the beginning of every decision phase.
Reward is also collected when the agent is halted or a state is retracted.
% What happens when an agent with multiple states is halted? Do the rewards in the substates get collected?

In order to be recognized, the reward signal must follow this pattern:

\begin{verbatim}
(<r1> ^reward <r2>)
(<r2> ^value [val])
\end{verbatim}

where \verb=<r1>= is the \soarb{reward-link} identifier, \verb=<r2>= is some intermediate identifier, and \verb=[val]= is any constant numeric value.
Any structure that does not match this pattern is ignored.
If there are multiple valid reward signals, their values are summed into a single reward signal.
As an example, consider the following state:

\begin{verbatim}
(S1 ^reward-link R1)
  (R1 ^reward R2)
    (R2 ^value 1.0)
  (R1 ^reward R3)
    (R3 ^value -0.2)
\end{verbatim}  

In this state, there are two reward signals with values 1.0 and -0.2.
They will be summed together for a total reward of 0.8 and this will be the value given to the RL update algorithm.

There are two reasons for requiring the intermediate identifier.
The first is so that multiple reward signals with the same value can exist simultaneously.
Since working memory is a set, multiple WMEs with identical values in all three positions (identifier, attribute, value) cannot exist simultaneously.
Without an intermediate identifier, specifying two rewards with the same value would require a WME structure such as

\begin{verbatim}
(S1 ^reward-link R1)
  (R1 ^reward 1.0)
  (R1 ^reward 1.0)
\end{verbatim}

which is invalid. With the intermediate identifier, the rewards would be specified as

\begin{verbatim}
(S1 ^reward-link R1)
  (R1 ^reward R2)
    (R2 ^value 1.0)
  (R1 ^reward R3)
    (R3 ^value 1.0)
\end{verbatim}

which is valid.
The second reason for requiring an intermediate identifier in the reward signal is so that the rewards can be augmented with additional information, such as their source or how long they have existed.
Although this information will be ignored by the RL mechanism, it can be useful to the agent or programmer.
For example:

\begin{verbatim}
(S1 ^reward-link R1)
  (R1 ^reward R2)
    (R2 ^value 1.0)
    (R2 ^source environment)
  (R1 ^reward R3)
    (R3 ^value -0.2)
    (R3 ^source intrinsic)
    (R3 ^duration 5)
\end{verbatim}  

The \verb=(R2 ^source environment)=, \verb=(R3 ^source intrinsic)=, and \verb=(R3 ^duration 5)= \\
WMEs are arbitrary and ignored by RL, but were added by the agent to keep 
track of where the rewards came from and for how long.

Note that the \soarb{reward-link} is not part of the \soarb{io} structure and is not modified directly by the environment.
Reward information from the environment should be copied, via rules, from the \soarb{input-link} to the \soarb{reward-link}.
Also note that when collecting rewards, Soar simply scans the \soarb{reward-link} and sums the values of all valid reward WMEs.
The WMEs are not modified and no bookkeeping is done to keep track of previously seen WMEs.
This means that reward WMEs that exist for multiple decision cycles will be collected multiple times.

\section{Updating RL Rule Values}
\label{RL-algo}

Soar's RL mechanism is integrated naturally with the decision cycle and performs online updates of RL rules.
Whenever an RL operator is selected, the values of the corresponding RL rules will be updated.
The update can be on-policy (Sarsa) or off-policy (Q-Learning), as controlled by the \soarb{learning-policy} parameter of the \soarb{rl} command.
Let $\delta_t$ be the amount the Q-value of an RL operator changes in an update.
For Sarsa, we have
$$ \delta_t = \alpha \left[ r_{t+1} + \gamma Q(s_{t+1}, a_{t+1}) - Q(s_t, a_t) \right] $$
where 
\begin{itemize}
\item $Q(s_t, a_t)$ is the Q-value of the state and chosen operator in decision cycle $t$.
\item $Q(s_{t+1}, a_{t+1})$ is the Q-value of the state and chosen RL operator in the next decision cycle.
\item $r_{t+1}$ is the total reward collected in the next decision cycle.
\item $\alpha$ and $\gamma$ are the settings of the \soarb{learning-rate} and \soarb{discount-rate} parameters of the \soarb{rl} command, respectively.
\end{itemize}

Note that since $\delta_t$ depends on $Q(s_{t+1}, a_{t+1})$, the update for the operator selected in decision cycle $t$ is not applied until the next RL operator is chosen.
For Q-Learning, we have
$$ \delta_t = \alpha \left[ r_{t+1} + \gamma \underset{a \in A_{t+1}}{\max} Q(s_{t+1}, a) - Q(s_t, a_t) \right] $$
where $A_{t+1}$ is the set of RL operators proposed in the next decision cycle.

Finally, $\delta_t$ is divided by the number of RL rules comprising the Q-value for the operator and the numeric indifferent values for each RL rule is updated by that amount.

An example walkthrough of a Sarsa update with $\alpha = 0.3$ and $\gamma = 0.9$ (the default settings in Soar) follows.

\begin{enumerate}

\item In decision cycle $t$, an operator \soarb{O1} is proposed, and RL rules \soarb{rl-1} and \soarb{rl-2} create the following numeric indifferent preferences for it:
\begin{verbatim}
   rl-1: (S1 ^operator O1 = 2.3)
   rl-2: (S1 ^operator O1 =  -1)
\end{verbatim}  
	The Q-value for \soarb{O1} is $Q(s_t, \soarb{O1}) = 2.3 - 1 = 1.3$.
	 
\item \soarb{O1} is selected and executed, so $Q(s_t, a_t) = Q(s_t, \soarb{O1}) = 1.3$.

\item In decision cycle $t+1$, a total reward of 1.0 is collected on the \soarb{reward-link}, an operator \soarb{O2} is proposed, and another RL rule \soarb{rl-3} creates the following numeric indifferent preference for it:
\begin{verbatim}
	rl-3: (S1 ^operator O2 = 0.5)
\end{verbatim}
	So $Q(s_{t+1}, \soarb{O2}) = 0.5$.

\item \soarb{O2} is selected, so $Q(s_{t+1}, a_{t+1}) = Q(s_{t+1}, \soarb{O2}) = 0.5$
	Therefore, 
	$$\delta_t = \alpha \left[r_{t+1} + \gamma Q(s_{t+1}, a_{t+1}) - Q(s_t, a_t) \right] = 0.3 \times [ 1.0 + 0.9 \times 0.5 - 1.3 ] = 0.045$$
	Since \soarb{rl-1} and \soarb{rl-2} both contributed to the Q-value of \soarb{O1}, $\delta_t$ is evenly divided amongst them, resulting in updated values of
\begin{verbatim}
   rl-1: (<s> ^operator <o> = 2.3225)
   rl-2: (<s> ^operator <o> = -0.9775)
\end{verbatim}

\item \soarb{rl-3} will be updated when the next RL operator is selected.
\end{enumerate}

\subsection{Gaps in Rule Coverage}
\label{RL-gaps}

Call an operator with numeric indifferent preferences an RL operator.
The previous description had assumed that RL operators were selected in both decision cycles $t$ and $t+1$.
If the operator selected in $t+1$ is not an RL operator, then $Q(s_{t+1}, a_{t+1})$ would not be defined, and an update for the RL operator selected at time $t$ will be undefined.
We will call a sequence of one or more decision cycles in which RL operators are not selected between two decision cycles in which RL operators are selected a \emph{gap}.
Conceptually, it is desirable to use the temporal difference information from the RL operator after the gap to update the Q-value of the RL operator before the gap.
There are no intermediate storage locations for these updates.
Requiring that RL rules support operators at every decision can be difficult for agent programmers, particularly for operators that do not represent steps in a task, but instead perform generic maintenance functions, such as cleaning processed output-link structures.

To address this issue, Soar's RL mechanism supports automatic propagation of updates over gaps.
For a gap of length $n$, the Sarsa update is
$$\delta_t = \alpha \left[ \sum_{i=t}^{t+n}{\gamma^{i-t} r_i} + \gamma^{n+1} Q(s_{t+n+1}, a_{t+n+1}) - Q(s_t, a_t) \right]$$
and the Q-Learning update is
$$\delta_t = \alpha \left[ \sum_{i=t}^{t+n}{\gamma^{i-t} r_i} + \gamma^{n+1} \underset{a \in A_{t+n+1}}{\max} Q(s_{t+n+1}, a) - Q(s_t, a_t) \right]$$

Note that rewards will still be collected during the gap, but they are discounted based on the number of decisions they are removed from the initial RL operator.

Gap propagation can be disabled by setting the \soarb{temporal-extension} parameter of the \soarb{rl} command to \soarb{off}.
When gap propagation is disabled, the RL rules preceding a gap are updated using $Q(s_{t+1}, a_{t+1}) = 0$.
The \soarb{rl} setting of the \soarb{watch} command (see Section \ref{trace} on page \pageref{trace}) is useful in identifying gaps.


\subsection{RL and Substates}
\label{RL-substates}

When an agent has multiple states in its state stack, the RL mechanism will treat each substate independently.
As mentioned previously, each state has its own \soarb{reward-link}.
When an RL operator is selected in a state \soarb{S}, the RL updates for that operator are only affected by the rewards collected on the \soarb{reward-link} for \soarb{S} and the Q-values of subsequent RL operators selected in \soarb{S}.

The only exception to this independence is when a selected RL operator forces an operator-no-change impasse.
When this occurs, the number of decision cycles the RL operator at the superstate remains selected is dependent upon the processing in the impasse state.
Consider the operator trace in Figure \ref{fig:rl-optrace}.

\begin{itemize}
\item At decision cycle 1, RL operator \soarb{O1} is selected in \soarb{S1} and causes an operator-no-change impass for three decision cycles.
\item In the substate \soarb{S2}, operators \soarb{O2}, \soarb{O3}, and \soarb{O4} are selected and applied sequentially.
\item Meanwhile in \soarb{S1}, reward values $r_2$, $r_3$, and $r_4$ are put on the \soarb{reward-link} sequentially.
\item Finally, the impasse is resolved by \soarb{O4}, the proposal for \soarb{O1} is retracted, and RL operator \soarb{O5} is selected in \soarb{S1}.
\end{itemize}

\begin{figure}
\insertfigure{Figures/rl-optrace}{1.5in}
\insertcaption{Example Soar substate operator trace.}
\label{fig:rl-optrace}
\end{figure}

In this scenario, only the RL update for $Q(s_1, \soarb{O1})$ will be different from the ordinary case.
Its value depends on the setting of the \soarb{hrl-discount} parameter of the \soarb{rl} command.
When this parameter is set to the default value \soarb{on}, the rewards on \soarb{S1} and the Q-value of \soarb{O5} are discounted by the number of decision cycles they are removed from the selection of \soarb{O1}.
In this case the update for $Q(s_1, \soarb{O1})$ is
$$\delta_1 = \alpha \left[ r_2 + \gamma r_3 + \gamma^2 r_4 + \gamma^3 Q(s_5, \soarb{O5}) - Q(s_1, \soarb{O1}) \right]$$
which is equivalent to having a three decision gap separating \soarb{O1} and \soarb{O5}.

When \soarb{hrl-discount} is set to \soarb{off}, the number of cycles \soarb{O1} has been impassed will be ignored.
Thus the update would be
$$\delta_1 = \alpha \left[ r_2 + r_3 + r_4 + \gamma Q(s_5, \soarb{O5}) - Q(s_1, \soarb{O1}) \right]$$

For impasses other than operator no-change, RL acts as if the impasse hadn't occurred.
If \soarb{O1} is the last RL operator selected before the impasse, $r_2$ the reward received in the decision cycle immediately following, and \soarb{O}$_\soarb{n}$, the first operator selected after the impasse, then \soarb{O1} is updated with 
$$\delta_1 = \alpha \left[ r_2 + \gamma Q(s_n, \soarb{O}_\soarb{n}) - Q(s_1, \soarb{O1}) \right]$$

If an RL operator is selected in a substate immediately prior to the state's retraction, the RL rules will be updated based only on the reward signals present and not on the Q-values of future operators.
This point is not covered in traditional RL theory.
The retraction of a substate corresponds to a suspension of the RL task in that state rather than its termination, so the last update assumes the lack of information about future rewards rather than the discontinuation of future rewards.
To handle this case, the numeric indifferent preference value of each RL rule is stored as two separate values, the expected current reward (ECR) and expected future reward (EFR).
The ECR is an estimate of the expected immediate reward signal for executing the corresponding RL operator.
The EFR is an estimate of the time discounted Q-value of the next RL operator.
Normal updates correspond to traditional RL theory (showing the Sarsa case for simplicity):
\begin{align*}
\delta_{ECR} &= \alpha \left[ r_t - ECR(s_t, a_t) \right] \\
\delta_{EFR} &= \alpha \left[ \gamma Q(s_{t+1}, a_{t+1}) - EFR(s_t, a_t) \right] \\
\delta_t &= \delta_{ECR} + \delta_{EFR} \\
&= \alpha \left[ r_t + \gamma Q(s_{t+1}, a_{t+1}) - \left( ECR(s_t, a_t) + EFR(s_t, a_t) \right) \right] \\
&= \alpha \left[ r_t + \gamma Q(s_{t+1}, a_{t+1}) - Q(s_t, a_t) \right]
\end{align*}
During substate retraction, only the ECR is updated based on the reward signals present at the time of retraction, and the EFR is unchanged.

Soar's automatic subgoaling and RL mechanisms can be combined to naturally implement hierarchical reinforcement learning algorithms such as MAXQ and options.

\subsection{Eligibility Traces}
\label{RL-et}
The RL mechanism supports eligibility traces, which can improve the speed of learning by 
updating RL rules across multiple sequential steps. \\
The \soarb{eligibility-trace-decay-rate} and \soarb{eligibility-trace-tolerance} parameters control this mechanism.
By setting \soarb{eligibility-trace-decay-rate} to \soarb{0} (default), eligibility traces are in effect disabled.
When eligibility traces are enabled, the particular algorithm used is dependent upon the learning policy.
For Sarsa, the eligibility trace implementation is \emph{Sarsa($\lambda$)}. 
For Q-Learning, the eligibility trace implementation is \emph{Watkin's Q($\lambda$)}.

\subsubsection{Exploration}

The \soarb{indifferent-selection} command (page \pageref{decide-indifferent-selection}) determines how operators are selected based on their numeric indifferent preferences.
Although all the indifferent selection settings are valid regardless of how the numeric indifferent preferences were arrived at, the \soarb{epsilon-greedy} and \soarb{boltzmann} settings are specifically designed for use with RL and correspond to the two most common exploration strategies.
In an effort to maintain backwards compatibility, the default exploration policy is \soarb{softmax}.
As a result, one should change to \soarb{epsilon-greedy} or \soarb{boltzmann} when the reinforcement learning mechanism is enabled.

\subsection{GQ($\lambda$)}

\emph{Sarsa($\lambda$)} and \emph{Watkin's Q($\lambda$)} help agents to solve the temporal credit assignment problem more quickly.
However, if you wish to implement something akin to CMACs to generalize from experience, convergence is not guaranteed by these algorithms.
\emph(GQ($\lambda$)} is a gradient descent algorithm designed to ensure convergence when learning off-policy.
Soar provides both \soarb{on-policy-gq-lambda} and \soarb{off-policy-gq-lambda} to increase the likelihood of convergence when learning under these conditions.
If you should choose to use one of these algorithms, we recommend setting \soarb{step-size-parameter} to something small, such as $0.01$
in order to ensure that the secondary set of weights used by \emph(GQ($\lambda$)} change slowly enough for efficient convergence.

\section{Automatic Generation of RL Rules}

The number of RL rules required for an agent to accurately approximate operator Q-values is usually infeasibly large to write by hand, even for small domains.
Therefore, several methods exist to automate this.

\subsection{The gp Command}
The \soar{gp} command can be used to generate productions based on simple patterns.
This is useful if the states and operators of the environment can be distinguished by a fixed number of dimensions with finite domains.
An example is a grid world where the states are described by integer row/column coordinates, and the available operators are to move north, south, east, or west.
In this case, a single \soar{gp} command will generate all necessary RL rules:
	
\begin{verbatim}
gp {gen*rl*rules
   (state <s> ^name gridworld
              ^operator <o> +
              ^row [ 1 2 3 4 ]
              ^col [ 1 2 3 4 ])
   (<o> ^name move
        ^direction [ north south east west ])
-->
   (<s> ^operator <o> = 0.0)
}
\end{verbatim}
	
For more information see the documentation for this command on page \pageref{gp}.

\subsection{Rule Templates}
\label{RL-templates}

Rule templates allow Soar to dynamically generate new RL rules based on a predefined pattern as the agent encounters novel states.
This is useful when either the domains of environment dimensions are not known ahead of time, or when the enumerable state space of the environment is too large to capture in its entirety using \soar{gp}, but the agent will only encounter a small fraction of that space during its execution.
For example, consider the grid world example with 1000 rows and columns.
Attempting to generate RL rules for each grid cell and action a priori will result in $1000 \times 1000 \times 4 = 4 \times 10^6$ productions.
However, if most of those cells are unreachable due to walls, then the agent will never fire or update most of those productions.
Templates give the programmer the convenience of the \soar{gp} command without filling production memory with unnecessary rules.

Rule templates have variables that are filled in to generate RL rules as the agent encounters novel combinations of variable values.
A rule template is valid if and only if it is marked with the \soarb{:template} flag and, in all other respects, adheres to the format of an RL rule.
However, whereas an RL rule may only use constants as the numeric indifference preference value, a rule template may use a variable.
Consider the following rule template:

\begin{verbatim}
sp {sample*rule*template
    :template
    (state <s> ^operator <o> +
               ^value <v>)
-->
    (<s> ^operator <o> = <v>)
}
\end{verbatim}

During agent execution, this rule template will match working memory and create new productions by substituting all variables in the rule template that matched against constant values with the values themselves.
Suppose that the LHS of the rule template matched against the state

\begin{verbatim}
(S1 ^value 3.2)
(S1 ^operator O1 +)
\end{verbatim}

Then the following production will be added to production memory:

\begin{verbatim}
sp {rl*sample*rule*template*1
    (state <s> ^operator <o> +
               ^value 3.2)
-->
    (<s> ^operator <o> = 3.2)
}
\end{verbatim}

The variable \soar{<v>} is replaced by \soar{3.2} on both the LHS and the RHS, but \soar{<s>} and \soar{<o>} are not replaced because they matches against identifiers (\soar{S1} and \soar{O1}).
As with other RL rules, the value of \soar{3.2} on the RHS of this rule may be updated later by reinforcement learning, whereas the value of \soar{3.2} on the LHS will remain unchanged.
If \soar{<v>} had matched against a non-numeric constant, it will be replaced by that constant on the LHS, but the RHS numeric indifference preference value will be set to zero to make the new rule valid.

The new production's name adheres to the following pattern:
\soarb{rl*template-name*id}, where \soarb{template-name} is the name of the originating rule template and \soarb{id} is monotonically increasing integer that guarantees the uniqueness of the name.

If an identical production already exists in production memory, then the newly generate production is discarded.
It should be noted that the current process of identifying unique template match instances can become quite expensive in long agent runs.
Therefore, it is recommended to generate all necessary RL rules using the \soar{gp} command or via custom scripting when possible.

\subsection{Chunking}
Since RL rules are regular productions, they can be learned by chunking just like any other production.
This method is more general than using the \soar{gp} command or rule templates, and is useful if the environment state consists of arbitrarily complex relational structures that cannot be enumerated.

\chapter{Semantic Memory}
\label{SMEM}
\index{semantic memory}
\index{smem}

Soar's semantic memory is a repository for long-term declarative knowledge, supplementing what is contained in short-term working memory (and production memory). 
Episodic memory, which contains memories of the agent's experiences, is described in Chapter \ref{EPMEM}. 
The knowledge encoded in episodic memory is organized temporally, and specific information is embedded within the context of when it was experienced, whereas knowledge in semantic memory is independent of any specific context, representing more general facts about the world.

This chapter is organized as follows: semantic memory structures in working memory (\ref{SMEM-wm}); representation of knowledge in semantic memory (\ref{SMEM-kr}); storing semantic knowledge (\ref{SMEM-store}); retrieving semantic knowledge (\ref{SMEM-retrieve}); and a discussion of performance (\ref{SMEM-perf}). 
The detailed behavior of semantic memory is determined by numerous parameters that can be controlled and configured via the \soarb{smem} command. 
Please refer to the documentation for that command in Section \ref{smem} on page \pageref{smem}.


\section{Working Memory Structure}
\label{SMEM-wm}

Upon creation of a new state in working memory (see Section \ref{ARCH-impasses-types} on page \pageref{ARCH-impasses-types}; Section \ref{SYNTAX-impasses} on page \pageref{SYNTAX-impasses}), the architecture creates the following augmentations to facilitate agent interaction with semantic memory:

\begin{verbatim}
(<s> ^smem <smem>)
  (<smem> ^command <smem-c>)
  (<smem> ^result <smem-r>)
\end{verbatim}

As rules augment the \emph{command} structure in order to access/change semantic knowledge (\ref{SMEM-store}, \ref{SMEM-retrieve}), semantic memory augments the \emph{result} structure in response.
Production actions should not remove augmentations of the \emph{result} structure directly, as semantic memory will maintain these WMEs.



\section{Knowledge Representation}
\label{SMEM-kr}

The representation of knowledge in semantic memory is similar to that in working memory (see Section \ref{ARCH-wm} on page \pageref{ARCH-wm}) -- both include graph structures that are composed of symbolic elements consisting of an identifier, an attribute, and a value. 
It is important to note, however, key differences:

\begin{itemize}

\item 
Currently semantic memory only supports attributes that are symbolic constants (string, integer, or decimal), but \emph{not} attributes that are identifiers

\item 
Whereas working memory is a single, connected, directed graph, semantic memory can be disconnected, consisting of multiple directed, connected sub-graphs

\end{itemize}

\emph{Long-term} identifiers (LTIs) are defined as identifiers that exist in semantic memory.
The specific letter-number combination that labels an LTI (e.g. S5 or C7) is permanently associated with that long-term identifier: any retrievals of the long-term identifier are guaranteed to return the associated letter-number pair.  
For clarity, when printed, a long-term identifier is prefaced with the {@} symbol (e.g. {@}S5 or {@}C7). 
Also, when presented in a figure, long-term identifiers will be indicated by a double-circle. 
For instance, Figure \ref{fig:smem-concept} depicts the long-term identifier {@}A68, with four augmentations, representing the addition fact of ${6+7=13}$ (or, rather, 3, carry 1, in context of multi-column arithmetic).

\begin{figure}
\insertfigure{Figures/smem-concept}{1.5in}
\insertcaption{Example long-term identifier with four augmentations.}
\label{fig:smem-concept}
\end{figure}

\subsection{Integrating Long-Term Identifiers with Soar}
Integrating long-term identifiers in Soar presents a number of theoretical and implementation challenges.  
This section discusses the state of integration with each of Soar's memories/learning mechanisms.

\subsubsection{Working Memory}
Long-term identifiers exist as peers with short-term identifiers in Working Memory.

\subsubsection{Procedural Memory}
Soar's production parser (i.e. the \soarb{sp} command) has been modified to allow specification of long-term identifiers (prefaced with an {@} symbol) in any context where a variable is valid.
If a rule contains a long-term identifier that is not currently in semantic memory, a fatal error will be raised and Soar will quit.  
Once added to the rete, the long-term identifier is treated as a constant for matching purposes.  
If specified as the value of a WME in an action, a long-term identifier will be added to working memory if it does not already exist.  
There is also preliminary support for chunking over long-term identifiers.

It is currently possible to create production actions wherein the identifier of a new WME is a long-term identifier that exists neither in the production conditions, nor as the attribute or value of a prior action.  
Such rules will wreak havoc within Soar and are not supported.  
They will be detected and disallowed in future versions of semantic memory.

\subsubsection{Episodic Memory}
Episodic memory (see Section \ref{EPMEM} on page \pageref{EPMEM}) faithfully captures short- vs. long-term identifiers, including the episode of transition.  
Cues are handled in much the same way as cue-based retrievals, with respect to the differences in semantics of a short- vs. long-term identifier.

\section{Storing Semantic Knowledge}
\label{SMEM-store}

An agent stores a long-term identifier to semantic memory by creating a \emph{store} command: this is a WME whose identifier is the \emph{command} link of a state's \emph{smem} structure, the attribute is \emph{store}, and the value is an identifier (short or long).

\begin{verbatim}
<s> ^smem.command.store <identifier>
\end{verbatim}

Semantic memory will encode and store all WMEs whose identifier is the value of the store command.  
Storing deeper levels of working memory is achieved through multiple store commands.

Multiple store commands can be issued in parallel.  
Storage commands are processed on every state at the end of every phase of every decision cycle.  
Storage is guaranteed to succeed and a status WME will be created, where the identifier is the \emph{result} link of the \emph{smem} structure of that state, the attribute is \emph{success}, and the value is the value of the store command above.

\begin{verbatim}
<s> ^smem.result.success <identifier>
\end{verbatim}

Any short-term identifiers that compose the stored WMEs will be converted to long-term identifiers. 
If a long-term identifier is the value of a store command, the stored WMEs replace those associated with the LTI in semantic memory. 
It should be noted that between issuing store commands, it is possible that the augmentations of a long-term identifier in working memory are inconsistent with those in semantic memory.

\subsection{User-Initiated Storage}
Semantic memory provides agent designers the ability to store semantic knowledge via the \soarb{add} switch of the \soarb{smem} command (see Section \ref{smem} on page \pageref{smem}).  
The format of the command is nearly identical to the working memory manipulation components of the RHS of a production (i.e. no RHS-functions; see Section \ref{SYNTAX-pm-action} on page \pageref{SYNTAX-pm-action}).  
For instance:

\begin{verbatim}
smem --add {
   (<arithmetic> ^add10-facts <a01> <a02> <a03>)
   (<a01> ^digit1 1 ^digit-10 11)
   (<a02> ^digit1 2 ^digit-10 12)
   (<a03> ^digit1 3 ^digit-10 13)
}
\end{verbatim}

Unlike agent storage, declarative storage is automatically recursive.  
Thus, this command instance will add a new long-term identifier (represented by the temporary 'arithmetic' variable) with three augmentations.  
The value of each augmentation will each become an LTI with two constant attribute/value pairs.  
Manual storage can be arbitrarily complex and use standard dot-notation.

\subsection{Storage Location}
Semantic memory uses SQLite to facilitate efficient and standardized storage and querying of knowledge.  
The semantic store can be maintained in memory or on disk (per the \soarb{database} and \soarb{path} parameters). 
If the store is located on disk, users can use any standard SQLite programs/components to access/query its contents.
However, using a disk-based semantic store is very costly (performance is discussed in greater detail in Section \ref{SMEM-perf} on page \pageref{SMEM-perf}), and running in memory is recommended for most runs.

The \soarb{lazy-commit} parameter is a performance optimization. 
If set to \soarb{on} (default), disk databases will not reflect semantic memory changes until the Soar kernel shuts down. 
This improves performance by avoiding disk writes. 
The \soarb{optimization} parameter (see Section \ref{SMEM-perf} on page \pageref{SMEM-perf}) will have an affect on whether databases on disk can be opened while the Soar kernel is running.


\section{Retrieving Semantic Knowledge}
\label{SMEM-retrieve}

An agent retrieves knowledge from semantic memory by creating an appropriate command (we detail the types of commands below) on the \emph{command} link of a state's \emph{smem} structure. 
At the end of the output of each decision, semantic memory processes each state's \emph{smem} command structure.  
Results, meta-data, and errors are added to the \emph{result} structure of that state's \emph{smem} structure.

Only one type of retrieval command (which may include optional modifiers) can be issued per state in a single decision cycle.  
Malformed commands (including attempts at multiple retrieval types) will result in an error:

\begin{verbatim}
<s> ^smem.result.bad-cmd <smem-c>
\end{verbatim}

Where the \soarb{smem-c} variable refers to the \emph{command} structure of the state.

After a command has been processed, semantic memory will ignore it until some aspect of the command structure changes (via addition/removal of WMEs).  
When this occurs, the result structure is cleared and the new command (if one exists) is processed.

\subsection{Non-Cue-Based Retrievals}
A non-cue-based retrieval is a request by the agent to reflect in working memory the current augmentations of a long-term identifier in semantic memory. 
The command WME has a \emph{retrieve} attribute and a long-term identifier value:

\begin{verbatim}
<s> ^smem.command.retrieve <lti>
\end{verbatim}

If the value of the command is not a long-term identifier, an error will result: 

\begin{verbatim}
<s> ^smem.result.failure <lti>
\end{verbatim}

Otherwise, two new WMEs will be placed on the result structure:

\begin{verbatim}
<s> ^smem.result.success <lti>
<s> ^smem.result.retrieved <lti>
\end{verbatim}

All augmentations of the long-term identifier in semantic memory will be created as new WMEs in working memory.

\subsection{Cue-Based Retrievals}
A cue-based retrieval performs a search for a long-term identifier in semantic memory whose augmentations exactly match an agent-supplied cue, as well as optional cue modifiers.

A cue is composed of WMEs that describe the augmentations of a long-term identifier.  
A cue WME with a constant value denotes an exact match of both attribute and value.  
A cue WME with a long-term identifier as its value denotes an exact match as well.  
A cue WME with a short-term identifier as its value denotes an exact match of attribute, but with any value (constant or identifier).  

A cue-based retrieval command has a \emph{query} attribute and an identifier value, the cue:

\begin{verbatim}
<s> ^smem.command.query <cue>
\end{verbatim}

For instance, consider the following rule that creates a cue-based retrieval command:

\begin{verbatim}
sp {smem*sample*query
    (state <s> ^smem.command <sc>
               ^lti <lti>
               ^input-link.foo <bar>)
-->
    (<sc> ^query <q>)
    (<q> ^name <any-name>
         ^foo <bar>
         ^associate <lti>
         ^age 25)
}
\end{verbatim}

In this example, assume that the \soar{<lti>} variable will match a long-term identifier and the \soar{<bar>} variable will match a constant.  
Thus, the query requests retrieval of a long-term identifier from semantic memory with augmentations that satisfy ALL of the following requirements:

\begin{itemize}

\item 
Attribute \soar{name} and ANY value

\item 
Attribute \soar{foo} and value equal to the value of variable \soar{<bar>} at the time this rule fires

\item 
Attribute \soar{associate} and value equal to the long-term identifier \soar{<lti>} at the time this rule fires

\item 
Attribute \soar{age} and integer value \soar{25}

\end{itemize}

If no long-term identifier satisfies ALL of these requirements, an error is returned:

\begin{verbatim}
<s> ^smem.result.failure <cue>
\end{verbatim}

Otherwise, two WMEs are added:

\begin{verbatim}
<s> ^smem.result.success <cue>
<s> ^smem.result.retrieved <retrieved-lti>
\end{verbatim}

During a cue-based retrieval it is possible that the retrieved long-term identifier is not in working memory.  
If this is the case, semantic memory will add the long-term identifier to working memory with letter-number pair as was originally stored.

As with non-cue-based retrievals all of the augmentations of the long-term identifier in semantic memory are added as new WMEs to working memory.

It is possible that multiple long-term identifiers match the cue equally well. 
In this case, semantic memory will retrieve the long-term identifier that was most recently stored/retrieved.

The cue-based retrieval process can be further tempered using optional modifiers:

\begin{itemize}

\item 
The \emph{prohibit} command requires that the retrieved long-term identifier is not equal to a supplied long-term identifier:
\begin{verbatim}
<s> ^smem.command.prohibit <bad-lti>
\end{verbatim}
Multiple prohibit command WMEs may be issued as modifiers to a single cue-based retrieval.  
This method can be used to iterate over all matching long-term identifiers.

\item 
The \emph{neg-query} command requires that the retrieved long-term identifier does NOT contain a set of attributes/attribute-value pairs:
\begin{verbatim}
<s> ^smem.command.neg-query <cue>
\end{verbatim}
The syntax of this command is identical to that of regular/positive \emph{query} command.

\item
The \emph{math-query} command requires that the retrieved long term identifier contains an attribute value pair that meets a specified mathematical condition. 
This condition can either be a conditional query or a superlative query. 
Conditional queries are of the format:
\begin{verbatim}
<s> ^smem.command.math-query.<cue-attribute>.<condition-name> <cue-value>
\end{verbatim}
Superlative queries do not use a value argument and are of the format:
\begin{verbatim}
<s> ^smem.command.math-query.<cue-attribute>.<condition-name>
\end{verbatim}
Values used in math queries must be integer or float type values.
Currently supported condition names are:
\begin{description}
  \item[less] A value less than the given argument
  \item[greater] A value greater than the given argument
  \item[less-or-equal] A value less than or equal to the given argument
  \item[greater-or-equal] A value greater than or equal to the given argument
  \item[max] The maximum value for the attribute
  \item[min] The minimum value for the attribute
\end{description}
\end{itemize}

\section{Performance}
\label{SMEM-perf}

Initial empirical results with toy agents show that semantic memory queries carry up to a 40\% overhead as compared to comparable rete matching.  
However, the retrieval mechanism implements some basic query optimization: statistics are maintained about all stored knowledge.  
When a query is issued, semantic memory re-orders the cue such as to minimize expected query time.  
Because only perfect matches are acceptable, and there is no symbol variablization, semantic memory retrievals do not contend with the same combinatorial search space as the rete.  
Preliminary empirical study shows that semantic memory maintains sub-millisecond retrieval time for a large class of queries, even in very large stores (millions of nodes/edges).

Once the number of long-term identifiers overcomes initial overhead (about 1000 WMEs), initial empirical study shows that semantic storage requires far less than 1KB per stored WME.

\subsection{Math queries}
There are some additional performance considerations when using math queries during retrieval.
Initial testing indicates that conditional queries show the same time growth with respect to the number of memories as similar non-math restricted queries, however the actual time for retrieval may be slightly longer.
Superelative queries will often show a worse result than similar non-superelative queries, because the current implementation of semantic memory requires them to iterate over any memory that matches all other involved cues.

\subsection{Performance Tweaking}

When using a database stored to disk, several parameters become crucial to performance.  
The first is \soarb{lazy-commit}, which controls when database changes are written to disk.   
The default setting (\soarb{on}) will keep all writes in memory and only commit to disk upon re-initialization (quitting the agent or issuing the \soarb{init} command).  
The \soarb{off} setting will write each change to disk and thus incurs massive I/O delay.

The next parameter is \soarb{thresh}. 
This has to do with the locality of storing/updating activation information with semantic augmentations. 
By default, all WME augmentations are incrementally sorted by activation, such that cue-based retrievals need not sort large number of candidate long-term identifiers on demand, and thus retrieval time is independent of cue selectivity. 
However, each activation update (such as after a retrieval) incurs an update cost linear in the number of augmentations. 
If the number of augmentations for a long-term identifier is large, this cost can dominate. 
Thus, the \soarb{thresh} parameter sets the upper bound of augmentations, after which activation is stored with the long-term identifier. 
This allows the user to establish a balance between cost of updating augmentation activation and the number of long-term identifiers that must be pre-sorted during a cue-based retrieval. 
As long as the threshold is greater than the number of augmentations of most long-term identifiers, performance should be fine (as it will bound the effects of selectivity).

The next two parameters deal with the SQLite cache, which is a memory store used to speed operations like queries by keeping in memory structures like levels of index B+-trees. 
The first parameter, \soarb{page-size}, indicates the size, in bytes, of each cache page. 
The second parameter, \soarb{cache-size}, suggests to SQLite how many pages are available for the cache. 
Total cache size is the product of these two parameter settings. 
The cache memory is not pre-allocated, so short/small runs will not necessarily make use of this space. 
Generally speaking, a greater number of cache pages will benefit query time, as SQLite can keep necessary meta-data in memory. 
However, some documented situations have shown improved performance from decreasing cache pages to increase memory locality. 
This is of greater concern when dealing with file-based databases, versus in-memory. 
The size of each page, however, may be important whether databases are disk- or memory-based. 
This setting can have far-reaching consequences, such as index B+-tree depth. 
While this setting can be dependent upon a particular situation, a good heuristic is that short, simple runs should use small values of the page size (\soarb{1k}, \soarb{2k}, \soarb{4k}), whereas longer, more complicated runs will benefit from larger values (\soarb{8k}, \soarb{16k}, \soarb{32k}, \soarb{64k}). 
The episodic memory chapter (see Section \ref{EPMEM-perf} on page \pageref{EPMEM-perf}) has some further empirical evidence to assist in setting these parameters for very large stores.

The next parameter is \soarb{optimization}.  
The \soarb{safety} parameter setting will use SQLite default settings.  
If data integrity is of importance, this setting is ideal.  
The \soarb{performance} setting will make use of lesser data consistency guarantees for significantly greater performance.  
First, writes are no longer synchronous with the OS (synchronous pragma), thus semantic memory won't wait for writes to complete before continuing execution.  
Second, transaction journaling is turned off (journal\_mode pragma), thus groups of modifications to the semantic store are not atomic (and thus interruptions due to application/os/hardware failure could lead to inconsistent database state).  
Finally, upon initialization, semantic memory maintains a continuous exclusive lock to the database (locking\_mode pragma), thus other applications/agents cannot make simultaneous read/write calls to the database (thereby reducing the need for potentially expensive system calls to secure/release file locks).

Finally, maintaining accurate operation timers can be relatively expensive in Soar.  
Thus, these should be enabled with caution and understanding of their limitations.  
First, they will affect performance, depending on the level (set via the \soarb{timers} parameter).  
A level of \soarb{three}, for instance, times every modification to long-term identifier recency statistics.  
Furthermore, because these iterations are relatively cheap (typically a single step in the linked-list of a b+-tree), timer values are typically unreliable (depending upon the system, resolution is 1 microsecond or more).


\chapter{Episodic Memory}
\label{EPMEM}
\index{episodic memory}
\index{epmem}

Episodic memory is a record of an agent's stream of experience.
The episodic storage mechanism will automatically record episodes as a Soar agent executes.
The agent can later deliberately retrieve episodic knowledge to extract information and regularities that may not have been noticed during the original experience and combine them with current knowledge such as to improve performance on future tasks.

This chapter is organized as follows: episodic memory structures in working memory (\ref{EPMEM-wm}); episodic storage (\ref{EPMEM-storage}); retrieving episodes (\ref{EPMEM-retrieval}); and a discussion of performance (\ref{EPMEM-perf}).
The detailed behavior of episodic memory is determined by numerous parameters that can be controlled and configured via the \soarb{epmem} command.

Please refer to the documentation for that command in Section \ref{epmem} on page \pageref{epmem}.

\section{Working Memory Structure}
\label{EPMEM-wm}

Upon creation of a new state in working memory (see Section \ref{ARCH-impasses-types} on page \pageref{ARCH-impasses-types}; Section \ref{SYNTAX-impasses} on page \pageref{SYNTAX-impasses}), the architecture creates the following augmentations to facilitate agent interaction with episodic memory:

\begin{verbatim}
(<s> ^epmem <e>)
  (<e> ^command <e-c>)
  (<e> ^result <e-r>)
  (<e> ^present-id #)
\end{verbatim}

As rules augment the \soar{command} structure in order to retrieve episodes (\ref{EPMEM-retrieval}), episodic memory augments the \soar{result} structure in response.
Production actions should not remove augmentations of the \soar{result} structure directly, as episodic memory will maintain these WMEs.

The value of the \soar{present-id} augmentation is an integer and will update to expose to the agent the current episode number.
This information is identical to what is available via the \emph{time} statistic (see Section \ref{epmem} on page \pageref{epmem}) and the \emph{present-id} retrieval meta-data (\ref{EPMEM-meta}).

\section{Episodic Storage}
\label{EPMEM-storage}

Episodic memory records new episodes without deliberate action/consideration by the agent.
The timing and frequency of recording new episodes is controlled by the \soar{phase} and \soar{trigger} parameters.
The \soarb{phase} parameter sets the phase in the decision cycle (default: end of each decision cycle) during which episodic memory stores episodes and processes commands.
The value of the \soarb{trigger} parameter indicates to the architecture the event that concludes an episode: adding a new augmentation to the output-link (default) or each decision cycle.

For debugging purposes, the \soarb{force} parameter allows the user to manually request that an episode be recorded (or not) during the current decision cycle.
Behavior is as follows:

\vspace{-8pt}
\begin{itemize}
\item
	The value of the \soar{force} parameter is initialized to \soar{off} every decision cycle.
	\vspace{-6pt}
\item
	During the \soar{phase} of episodic storage, episodic memory tests the value of the \soar{force} parameter; if it has a value other than of off, episodic memory follows the \emph{forced} policy irrespective of the value of the \soar{trigger} parameter.
	\vspace{-6pt}
\end{itemize}

\subsection{Episode Contents}

When episodic memory stores a new episode, it captures the entire top-state of working memory.
There are currently two exceptions to this policy:

\begin{itemize}
\item
Episodic memory only supports WMEs whose attribute is a constant.
Behavior is currently undefined when attempting to store a WME that has an attribute that is an identifier.

\item
The \soarb{exclusions} parameter allows the user to specify a set of attributes for which Soar will not store WMEs.
The storage process currently walks the top-state of working memory in a breadth-first manner, and any WME that is not reachable other than via an excluded WME will not be stored.
By default, episodic memory excludes the \soar{epmem} and \soar{smem} structures, to prevent encoding of potentially large and/or frequently changing memory retrievals.

\end{itemize}

\subsection{Storage Location}
\index{epmem!storage}

Episodic memory uses SQLite to facilitate efficient and standardized storage and querying of episodes.
The episodic store can be maintained in memory or on disk (per the \soar{database} and \soar{path} parameters).
If the store is located on disk, users can use any standard SQLite programs/components to access/query its contents.
See the later discussion on performance (\ref{EPMEM-perf}) for additional parameters dealing with databases on disk.

Note that changes to storage parameters, for example \soar{database, path} and \soar{append} will not have an effect until the database is used after an initialization. This happens either shortly after launch (on first use) or after a database initialization command is issued. To switch databases or database storage types while running, set your new parameters and then perform an \soar{epmem --init} command.

The \soarb{path} parameter specifies the file system path the database is stored in. When \soar{path} is set to a valid file system path and \soar{database} mode is set to \emph{file}, then the SQLite database is written to that path.

The \soarb{append} parameter will determine whether all existing facts stored in a database on disk will be erased when episodic memory loads. Note that this affects \soar{init-soar} also.  In other words, if the \soar{append} setting is off, all episodes stored will be lost when an init-soar is performed. For episodic memory, \soar{append} mode is \soar{off} by default.

\soarit{Note}: As of version 9.3.3, Soar now uses a new schema for the episodic memory database. This means databases from 9.3.2 and below can no longer be loaded.  A conversion utility will be available in Soar 9.4 to convert from the old schema to the new one.

\section{Retrieving Episodes}
\label{EPMEM-retrieval}
\index{epmem!retrieve}

An agent retrieves episodes by creating an appropriate command (we detail the types of commands below) on the \soar{command} link of a state's \soar{epmem} structure.
At the end of the \soar{phase} of each decision, after episodic storage, episodic memory processes each state's \emph{epmem} command structure.
Results, meta-data, and errors are placed on the \soar{result} structure of that state's \soar{epmem} structure.

Only one type of retrieval command (which may include optional modifiers) can be issued per state in a single decision cycle.
Malformed commands (including attempts at multiple retrieval types) will result in an error:

\begin{verbatim}
<s> ^epmem.result.status bad-cmd
\end{verbatim}

After a command has been processed, episodic memory will ignore it until some aspect of the command structure changes (via addition/removal of WMEs).
When this occurs, the result structure is cleared and the new command (if one exists) is processed.

All retrieved episodes are recreated exactly as stored, except for any operators that have an acceptable preference, which are recreated with the attribute \soar{operator*}.
For example, if the original episode was:

\begin{verbatim}
(<s> ^operator <o1> +)
(<o1> ^name move)
\end{verbatim}

A retrieval of the episode would become:

\begin{verbatim}
(<s> ^operator* <o1>)
(<o1> ^name move)
\end{verbatim}

\subsection{Cue-Based Retrievals}
Cue-based retrieval commands are used to search for an episode in the store that best matches an agent-supplied cue, while adhering to optional modifiers.
A cue is composed of WMEs that partially describe a top-state of working memory in the retrieved episode.
All cue-based retrieval requests must contain a single \soarb{\carat query} cue and, optionally, a single \soarb{\carat neg-query} cue.

\begin{verbatim}
<s> ^epmem.command.query <required-cue>
<s> ^epmem.command.neg-query <optional-negative-cue>
\end{verbatim}

A \soar{\carat query} cue describes structures desired in the retrieved episode, whereas a \soar{\carat neg-query} cue describes non-desired structures.
For example, the following Soar production creates a \soar{\carat query} cue consisting of a particular state name and a copy of a current value on the \soar{input-link} structure:

\begin{verbatim}
sp {epmem*sample*query
    (state <s> ^epmem.command <ec>
               ^io.input-link.foo <bar>)
-->
    (<ec> ^query <q>)
    (<q> ^name my-state-name
         ^io.input-link.foo <bar>)
}
\end{verbatim}

\index{working memory activation}
As detailed below, multiple prior episodes may equally match the structure and contents of an agent's cue.
Nuxoll has produced initial evidence that in some tasks, retrieval quality improves when using \emph{activation} of cue WMEs as a form of feature weighting.
Thus, episodic memory supports integration with working memory activation (see Section \ref{wm-activation} on page \pageref{wm-activation}).
For a theoretical discussion of the Soar implementation of working memory activation, consider reading \emph{Comprehensive Working Memory Activation in Soar} (Nuxoll, A., Laird, J., James, M., ICCM 2004).

The cue-based retrieval process can be thought of conceptually as a nearest-neighbor search.
First, all candidate episodes, defined as episodes containing at least one leaf WME (a cue WME with no sub-structure) in at least one cue, are identified.
Two quantities are calculated for each candidate episode, with respect to the supplied cue(s): the cardinality of the match (defined as the number of matching leaf WMEs) and the activation of the match (defined as the sum of the activation values of each matching leaf WME).
Note that each of these values is negated when applied to a negative query.
To compute each candidate episode's match score, these quantities are combined with respect to the \soarb{balance} parameter as follows:

$$(balance)*(cardinality) + (1-balance)*(activation)$$

Performing a graph match on each candidate episode, with respect to the structure of the cue, could be very computationally expensive, so episodic memory implements a two-stage matching process.
An episode with perfect cardinality is considered a perfect \emph{surface} match and, per the \soarb{graph-match} parameter, is subjected to further \emph{structural} matching.
Whereas surface matching efficiently determines if all paths to leaf WMEs exist in a candidate episode, graph matching indicates whether or not the cue can be structurally unified with the candidate episode (paying special regard to the structural constraints imposed by shared identifiers).
Cue-based matching will return the most recent structural match, or the most recent candidate episode with the greatest match score.

A special note should be made with respect to how short- vs. long-term identifiers (see Section \ref{SMEM-kr} on page \pageref{SMEM-kr}) are interpreted in a cue.
Short-term identifiers are processed much as they are in working memory -- transient structures.
Cue matching will try to find any identifier in an episode (with respect to WME path from state) that can apply.
Long-term identifiers, however, are treated as constants.
Thus, when analyzing the cue, episodic memory will not consider long-term identifier augmentations, and will only match with the same long-term identifier (in the same context) in an episode.

The case-based retrieval process can be further controlled using optional modifiers:

\vspace{-8pt}
\begin{itemize}
\item
	The \soarb{before} command requires that the retrieved episode come relatively before a supplied time:
	\vspace{-6pt}
	\begin{verbatim}
	<s> ^epmem.command.before time
	\end{verbatim}
	\vspace{-6pt}
\item
	The \soarb{after} command requires that the retrieved episode come relatively after a supplied time:
	\vspace{-6pt}
	\begin{verbatim}
	<s> ^epmem.command.after time
	\end{verbatim}
	\vspace{-6pt}
\item
	The \soarb{prohibit} command requires that the time of the retrieved episode is not equal to a supplied time:
	\vspace{-6pt}
	\begin{verbatim}
	<s> ^epmem.command.prohibit time
	\end{verbatim}
	\vspace{-6pt}
	Multiple prohibit command WMEs may be issued as modifiers to a single CB retrieval.
	\vspace{-6pt}
\end{itemize}
\vspace{-12pt}

If no episode satisfies the cue(s) and optional modifiers an error is returned:

\begin{verbatim}
<s> ^epmem.result.failure <query> <optional-neg-query>
\end{verbatim}

If an episode is returned, there is additional meta-data supplied (\ref{EPMEM-meta}).

\subsection{Absolute Non-Cue-Based Retrieval}
At time of storage, each episode is attributed a unique \emph{time}.
This is the current value of \soarb{time} statistic and is provided as the \emph{memory-id} meta-data item of retrieved episodes (\ref{EPMEM-meta}).
An absolute non-cue-based retrieval is one that requests an episode by time.
An agent issues an absolute non-cue-based retrieval by creating a WME on the \soar{command} structure with attribute \emph{retrieve} and value equal to the desired time:

\begin{verbatim}
<s> ^epmem.command.retrieve time
\end{verbatim}

Supplying an invalid value for the \soar{retrieve} command will result in an error.

The time of the first episode in an episodic store will have value 1 and each subsequent episode's time will increase by 1.
Thus the desired time may be the mathematical result of operations performed on a known episode's time.

The current episodic memory implementation does not implement any episodic store dynamics, such as forgetting.
Thus any integer time greater than 0 and less than the current value of the \soar{time} statistic will be valid.
However, if forgetting is implemented in future versions, no such guarantee will be made.

\subsection{Relative Non-Cue-Based Retrieval}
Episodic memory supports the ability for an agent to ``play forward" episodes using relative non-cue-based retrievals.

Episodic memory stores the time of the last successful retrieval (non-cue-based or cue-based).
Agents can indirectly make use of this information by issuing \soarb{next} or \soarb{previous} commands.
Episodic memory executes these commands by attempting to retrieve the episode immediately proceeding/preceding the last successful retrieval (respectively).
To issue one of these commands, the agent must create a new WME on the \soar{command} link with the appropriate attribute (\soar{next} or \soar{previous}) and value of an arbitrary identifier:

\begin{verbatim}
<s> ^epmem.command.next <n>
<s> ^epmem.command.previous <p>
\end{verbatim}

If no such episode exists then an error is returned.

Currently, if the time of the last successfully retrieved episode is known to the agent (as could be the case by accessing result meta-data), these commands are identical to performing an absolute non-cue-based retrieval after adding/subtracting 1 to the last time (respectively).
However, if an episodic store dynamic like forgetting is implemented, these relative commands are guaranteed to return the next/previous valid episode (assuming one exists).

\subsection{Retrieval Meta-Data}
\label{EPMEM-meta}
\index{epmem!structures}

The following list details the WMEs that episodic memory creates in the \soar{result} link of the \soar{epmem} structure wherein a command was issued:

\begin{itemize}

\item \soarb{retrieved <retrieval-root>}
	If episodic memory retrieves an episode, that memory is placed here. This WME is an identifier that is treated as the root of the state that was used to create the episodic memory. If the \soar{retrieve} command was issued with an invalid time, the value of this WME will be \emph{no-memory}.
\item \soarb{success <query> <optional-neg-query>}
	If the cue-based retrieval was successful, the WME will have the status as the attribute and the value of the identifier of the query (and neg-query, if applicable).
\item \soarb{match-score}
	This WME is created whenever an episode is successfully retrieved from a cue-based retrieval command. The WME value is a decimal indicating the raw match score for that episode with respect to the cue(s).
\item \soarb{cue-size}
	This WME is created whenever an episode is successfully retrieved from a cue-based retrieval command. The WME value is an integer indicating the number of leaf WMEs in the cue(s).
\item \soarb{normalized-match-score}
	This WME is created whenever an episode is successfully retrieved from a cue-based retrieval command. The WME value is the decimal result of dividing the raw match score by the cue size. It can hypothetically be used as a measure of episodic memory's relative confidence in the retrieval.
\item \soarb{match-cardinality}
	This WME is created whenever an episode is successfully retrieved from a cue-based retrieval command. The WME value is an integer indicating the number of leaf WMEs matched in the \soar{\carat query} cue minus those matched in the \soar{\carat neg-query} cue.
\item \soarb{memory-id}
	This WME is created whenever an episode is successfully retrieved from a cue-based retrieval command. The WME value is an integer indicating the time of the retrieved episode.
\item \soarb{present-id}
	This WME is created whenever an episode is successfully retrieved from a cue-based retrieval command. The WME value is an integer indicating the current time, such as to provide a sense of ``now" in episodic memory terms. By comparing this value to the \soar{memory-id} value, the agent can gain a sense of the relative time that has passed since the retrieved episode was recorded.
\item \soarb{graph-match}
	This WME is created whenever an episode is successfully retrieved from a cue-based retrieval command and the \soar{graph-match} parameter was \soar{on}. The value is an integer with value 1 if graph matching was executed successfully and 0 otherwise.
\item \soarb{mapping <mapping-root>}
	This WME is created whenever an episode is successfully retrieved from a cue-based retrieval command, the \soar{graph-match} parameter was \soar{on}, and structural match was successful on the retrieved episode. This WME provides a mapping between identifiers in the cue and in the retrieved episode. For each identifier in the cue, there is a \soar{node} WME as an augmentation to the \soar{mapping} identifier. The node has a \soar{cue} augmentation, whose value is an identifier in the cue, and a \soar{retrieved} augmentation, whose value is an identifier in the retrieved episode. In a graph match it is possible to have multiple identifier mappings -- this map represents the ``first" unified mapping (with respect to episodic memory algorithms).
\end{itemize}

\section{Performance}
\label{EPMEM-perf}
\index{epmem!performance}

There are currently two sources of ``unbounded" computation: graph matching and cue-based queries.
Graph matching is combinatorial in the worst case.
Thus, if an episode presents a perfect surface match, but imperfect structural match (i.e. there is no way to unify the cue with the candidate episode), there is the potential for exhaustive search.
Each identifier in the cue can be assigned one of any historically consistent identifiers (with respect to the sequence of attributes that leads to the identifier from the root), termed a literal.
If the identifier is a multi-valued attribute, there will be more than one candidate literals and this situation can lead to a very expensive search process.
Currently there are no heuristics in place to attempt to combat the expensive backtracking.
Worst-case performance will be combinatorial in the total number of literals for each cue identifier (with respect to cue structure).

The cue-based query algorithm begins with the most recent candidate episode and will stop search as soon as a match is found (since this episode must be the most recent).
Given this procedure, it is trivial to create a two-WME cue that forces a linear search of the episodic store.
Episodic memory combats linear scan by only searching candidate episodes, i.e. only those that contain a change in at least one of the cue WMEs.
However, a cue that has no match and contains WMEs relevant to all episodes will force inspection of all episodes.
Thus, worst-case performance will be linear in the number of episodes.

\subsection{Performance Tweaking}
When using a database stored to disk, several parameters become crucial to performance.
The first is \soarb{commit}, which controls the number of episodes that occur between writes to disk.
If the total number of episodes (or a range) is known ahead of time, setting this value to a greater number will result in greatest performance (due to decreased I/O).

The next two parameters deal with the SQLite cache, which is a memory store used to speed operations like queries by keeping in memory structures like levels of index B+-trees.
The first parameter, \soarb{page-size}, indicates the size, in bytes, of each cache page.
The second parameter, \soarb{cache-size}, suggests to SQLite how many pages are available for the cache.
Total cache size is the product of these two parameter settings.
The cache memory is not pre-allocated, so short/small runs will not necessarily make use of this space.
Generally speaking, a greater number of cache pages will benefit query time, as SQLite can keep necessary meta-data in memory.
However, some documented situations have shown improved performance from decreasing cache pages to increase memory locality.
This is of greater concern when dealing with file-based databases, versus in-memory.
The size of each page, however, may be important whether databases are disk- or memory-based.
This setting can have far-reaching consequences, such as index B+-tree depth.
While this setting can be dependent upon a particular situation, a good heuristic is that short, simple runs should use small values of the page size (\soar{1k}, \soar{2k}, \soar{4k}), whereas longer, more complicated runs will benefit from larger values (\soar{8k}, \soar{16k}, \soar{32k}, \soar{64k}).
One known situation of concern is that as indexed tables accumulate many rows (\tild millions), insertion time of new rows can suffer an infrequent, but linearly increasing burst of computation.
In episodic memory, this situation will typically arise with many episodes and/or many working memory changes.
Increasing the page size will reduce the intensity of the spikes at the cost of increasing disk I/O and average/total time for episode storage.
Thus, the settings of page size for long, complicated runs establishes the desired balance of reactivity (i.e. max computation) and average speed.
To ground this discussion, the Figure \ref{fig:epmem-cache} depicts maximum and average episodic storage time (the value of the epmem\_storage timer, converted to milliseconds) with different page sizes after 10 million decisions (1 episode/decision) of a very basic agent (i.e. very few working memory changes per episode) running on a 2.8GHz Core i7 with Mac OS X 10.6.5.
While only a single use case, the cross-point of these data forms the basis for the decision to default the parameter at 8192 bytes.

\begin{figure}
\insertfigure{Figures/epmem-cache}{2.5in}
\insertcaption{Example episodic memory cache setting data.}
\label{fig:epmem-cache}
\end{figure}

The next parameter is \soarb{optimization}, which can be set to either \soar{safety} or \soar{performance}.
The \soar{safety} parameter setting will use SQLite default settings.
If data integrity is of importance, this setting is ideal.
The \soar{performance} setting will make use of lesser data consistency guarantees for significantly greater performance.
First, writes are no longer synchronous with the OS (synchronous pragma), thus episodic memory won't wait for writes to complete before continuing execution.
Second, transaction journaling is turned off (journal\_mode pragma), thus groups of modifications to the episodic store are not atomic (and thus interruptions due to application/os/hardware failure could lead to inconsistent database state).
Finally, upon initialization, episodic memory maintains a continuous exclusive lock to the database (locking\_mode pragma), thus other applications/agents cannot make simultaneous read/write calls to the database (thereby reducing the need for potentially expensive system calls to secure/release file locks).

Finally, maintaining accurate operation timers can be relatively expensive in Soar.
Thus, these should be enabled with caution and understanding of their limitations.
First, they will affect performance, depending on the level (set via the \soar{timers} parameter).
A level of \soar{three}, for instance, times every step in the cue-based retrieval candidate episode search.
Furthermore, because these iterations are relatively cheap (typically a single step in the linked-list of a b+-tree), timer values are typically unreliable (depending upon the system, resolution is 1 microsecond or more).

\chapter{Spatial Visual System}
\label{SVS}
\index{Spatial Visual System}
\index{SVS}
\index{svs}

\begin{figure}
\insertfigure{Figures/svs-setup}{4in}
\insertcaption{(a) Typical environment setup without using SVS. (b) Same environment using SVS.}
\label{fig:svs-setup}
\end{figure}

The Spatial Visual System (SVS) allows Soar to effectively represent and reason about continuous, three dimensional environments.
SVS maintains an internal representation of the environment as a collection of discrete objects with simple geometric shapes, called the scene graph.
The Soar agent can query for spatial relationships between the objects in the scene graph through a working memory interface similar to that of episodic and semantic memory.
Figure \ref{fig:svs-setup} illustrates the typical use case for SVS by contrasting it with an agent that does not utilize it.
The agent that does not use SVS (a. in the figure) relies on the environment to provide a symblic representation of the continuous state.
On the other hand, the agent that uses SVS (b) accepts a continuous representation of the environment state directly, and then performs queries on the scene graph to extract a symbolic representation internally.
This allows the agent to build more flexible symbolic representations without requiring modifications to the environment code.
Furthermore, it allows the agent to manipulate internal copies of the scene graph and then extract spatial relationships from the modified states, which is useful for look-ahead search and action modeling.
This type of imagery operation naturally captures and propogates the relationships implicit in spatial environments, and doesn't suffer from the frame problem that relational representations have.

\section{The scene graph}

The primary data structure of SVS is the \emph{scene graph}.
The scene graph is a tree in which the nodes represent objects in the scene and the edges represent ``part-of'' relationships between objects.
An example scene graph consisting of a car and a pole is shown in Figure \ref{fig:scene-graph}.
The scene graph's leaves are \emph{geometry nodes} and its interior nodes are \emph{group nodes}.
Geometry nodes represent atomic objects that have intrinsic shape, such as the wheels and chassis in the example.
Currently, the shapes supported by SVS are points, lines, convex polyhedrons, and spheres.
Group nodes represent objects that are the aggregates of their child nodes, such as the car object in the example.
The shape of a group node is the union of the shapes of its children.
Structuring complex objects in this way allows Soar to reason about them naturally at different levels of abstraction.
The agent can query SVS for relationships between the car as a whole with other objects (e.g. does it intersect the pole?), or the relationships between its parts (e.g. are the wheels pointing left or right with respect to the chassis?).
The scene graph always contains at least a root node: the \emph{world node}.

\begin{figure}
\insertfigure{Figures/scene_graph}{5in}
\insertcaption{(a) A 3D scene. (b) The scene graph representation.}
\label{fig:scene-graph}
\end{figure}

Each node other than the world node has a transform with respect to its parent.
A transform consists of three components:

\begin{description}
\item[position $(x,y,z)$]
Specifies the $x$, $y$, and $z$ offsets of the node's origin with respect to its parent's origin.

\item[rotation $(x,y,z)$]
Specifies the rotation of the node relative to its origin in Euler angles.
This means that the node is rotated the specified number of radians along each axis in the order $x-y-z$.
For more information, see \url{http://en.wikipedia.org/wiki/Euler_angles}.

\item[scaling $(x,y,z)$]
Specifies the factors by which the node is scaled along each axis.

\end{description}

The component transforms are applied in the order scaling, then rotation, then position.
Each node's transform is applied with respect to its parent's coordinate system, so the transforms accumulate down the tree.
A node's transform with respect to the world node, or its world transform, is the aggregate of all its ancestor transforms.
For example, if the car has a position transform of $(1,0,0)$ and a wheel on the car has a position transform of $(0,1,0)$, then the world position transform of the wheel is $(1,1,0)$.

SVS represents the scene graph structure in working memory under the \soarb{\^{}spatial-scene} link.
The working memory representation of the car and pole scene graph is

\begin{verbatim}
(S1 ^svs S3)
  (S3 ^command C3 ^spatial-scene S4)
    (S4 ^child C10 ^child C4 ^id world)
      (C10 ^id pole)
      (C4 ^child C9 ^child C8 ^child C7 ^child C6 ^child C5 ^id car)
        (C9 ^id chassis)
        (C8 ^id wheel3)
        (C7 ^id wheel2)
        (C6 ^id wheel1)
        (C5 ^id wheel0)
\end{verbatim}

Each state in working memory has its own scene graph.
When a new state is created, it will receive an independent copy of its parent's scene graph.
This is useful for performing look-ahead search, as it allows the agent to destructively modify the scene graph in a search state using mental imagery operations.

\subsection{svs\_viewer}

A viewer has been provided to show the scene graph visually. 
Run the program \texttt{svs\_viewer -s PORT} from the soar/out folder 
to launch the viewer listening on the given port. Once the viewer is running, 
from within soar use the command \texttt{svs connect\_viewer PORT} to connect 
to the viewer and begin drawing the scene graph. Any changes to the scene graph
will be reflected in the viewer. The viewer by default draws the topstate scene graph, 
to draw that on a substate first stop drawing the topstate with 
\texttt{svs S1.scene.draw off} and then \texttt{svs S7.scene.draw on}. 

\section{Scene Graph Edit Language}

The Scene Graph Edit Language (SGEL) is a simple, plain text, line oriented language that is used by SVS to modify the contents of the scene graph.
Typically, the scene graph is used to represent the state of the external environment, and the programmer sends SGEL commands reflecting changes in the environment to SVS via the Agent::SendSVSInput function in the SML API.
These commands are buffered by the agent and processed at the beginning of each input phase.
Therefore, it is common to send scene changes through SendSVSInput \emph{before} the input phase.
If you send SGEL commands at the end of the input phase, 
the results won't be processed until the following decison cycle.

Each SGEL command begins with a single word command type and ends with a newline.
The four command types are

\begin{description}
\item[\texttt{add ID PARENT\_ID [GEOMETRY] [TRANSFORM]}] \hfill \\
Add a node to the scene graph with the given \texttt{ID}, as a child of \texttt{PARENT\_ID}, 
and with type \texttt{TYPE} (usually object).
The \texttt{GEOMETRY} and \texttt{TRANSFORM} arguments are optional and described below.

\item[\texttt{change ID [GEOMETRY] [TRANSFORM]}] \hfill \\
Change the transform and/or geometry of the node with the given \texttt{ID}.

\item[\texttt{delete ID}] \hfill \\
  Delete the node with the given \texttt{ID}.

\item[\texttt{tag [add|change|delete] ID TAG\_NAME TAG\_VALUE}] \hfill \\
  Adds, changes, or deletes a tag from an object.
  A tag consists of a \texttt{TAG\_NAME}  
  and \texttt{TAG\_VALUE} pair and is added to the node with the given \texttt{ID}.
  Both \texttt{TAG\_NAME} and \texttt{TAG\_VALUE} must be strings.
  Tags can differentiate nodes (e.g. as a type field) and can be used in conjunction with 
  the \texttt{tag\_select} filter to choose a subset of the nodes. 

\end{description}

The \texttt{TRANSFORM} argument has the form \texttt{[p X Y Z] [r X Y Z] [s X Y Z]}, corresponding to the position, rotation, and scaling components of the transform, respectively.
All the components are optional; any combination of them can be excluded, and the included components can appear in any order.

The \texttt{GEOMETRY} argument has two forms:

\begin{description}

\item[\texttt{b RADIUS}] \hfill \\
Make the node a geometry node with sphere shape with radius \texttt{RADIUS}.

\item[\texttt{v X1 Y1 Z1 X2 Y2 Z2 ...}] \hfill \\
Make the node a geometry node with a convex polyhedron shape with the specified vertices.
Any number of vertices can be listed.

\end{description}

\subsection{Examples}

Creating a sphere called ball4 with radius 5 at location (4, 4, 0). \\
\texttt{add ball4 world b 5 p 4 4 0}

Creating a triangle in the xy plane, then rotate it vertically, double its size, and move it to (1, 1, 1).  \\
\texttt{add tri9 world v -1 -1 0 1 -1 0 0 0.5 0 p 1 1 1 r 1.507 0 0 s 2 2 2}

Creating a snowman shape of 3 spheres stacked on each other and located at (2, 2, 0). \\
\texttt{add snowman world p 2 2 0} \\
\texttt{add bottomball snowman b 3 p 0 0 3} \\
\texttt{add middleball snowman b 2 p 0 0 8} \\
\texttt{add topball snowman b 1 p 0 0 11} 

Set the rotation transform on box11 to 180 degrees around the z axis. \\
\texttt{change box11 r 0 0 3.14159}

Changing the color tag on box7 to green. \\
\texttt{tag change box7 color green}


\section{Commands}

The Soar agent initiates commands in SVS via the \soarb{\^{}command} link, 
similar to semantic and episodic memory. These commands allow the agent to 
modify the scene graph and extract filters. 
Commands are processed during the output phase and the results are added to 
working memory during the input phase. 
SVS supports the following commands:

\begin{description}
  \item{\textbf{add\_node}}
  Creates a new node and adds it to the scene graph
\item{\textbf{copy\_node}}
  Creates a copy of an existing node
\item{\textbf{delete\_node}}
  Removes a node from the scene graph and deletes it
\item{\textbf{set\_transform}}
  Changes the position, rotation, and/or scale of a node
\item{\textbf{set\_tag}}
  Adds or changes a tag on a node
\item{\textbf{delete\_tag}}
  Deletes a tag from a node
\item{\textbf{extract}}
	Compute the truth value of spatial relationships in the current scene graph.
\item{\textbf{extract\_once}}
  Same as extract, except it is only computed once and doesn't update when the scene changes.
\end{description}

\subsection{add\_node}

This commands adds a new node to the scene graph. 
\begin{description}
  \item{\soarb{\^{}id [string]}} The id of the node to create
  \item{\soarb{\^{}parent [string]}} The id of the node to attach the new node to (default is world)
  \item{\soarb{\^{}geometry << group point ball box >> }} The geometry the node should have 
  \item{\soarb{\^{}position.\{\^{}x \^{}y \^{}z\} }} Position of the node (optional)
  \item{\soarb{\^{}rotation.\{\^{}x \^{}y \^{}z\} }} Rotation of the node (optional)
  \item{\soarb{\^{}scale.\{\^{}x \^{}y \^{}z\} }} Scale of the node (optional)
\end{description}

The following example creates a node called \texttt{box5} and adds it to the world. 
The node has a box shape of side length 0.1 and is placed at position (1, 1, 0). 
\begin{verbatim}
(S1 ^svs S3)
  (S3 ^command C3 ^spatial-scene S4)
    (C3 ^add_node A1)
      (A1 ^id box5 ^parent world ^geometry box ^position P1 ^scale S6)
        (P1 ^x 1.0 ^y 1.0 ^z 0.0)
        (S6 ^x 0.1 ^y 0.1 ^z 0.1)
\end{verbatim}

\subsection{copy\_node}
This command creates a copy of an existing node and adds it to the scene graph. 
This copy is not recursive, it only copies the node itself, not its children. 
The position, rotation, and scale transforms are also copied from the source node
but they can be changed if desired. 
\begin{description}
  \item{\soarb{\^{}id [string]}} The id of the node to create
  \item{\soarb{\^{}source [string]}} The id of the node to copy
  \item{\soarb{\^{}parent [string]}} The id of the node to attach the new node to (default is world)
  \item{\soarb{\^{}position.\{\^{}x \^{}y \^{}z\} }} Position of the node (optional)
  \item{\soarb{\^{}rotation.\{\^{}x \^{}y \^{}z\} }} Rotation of the node (optional)
  \item{\soarb{\^{}scale.\{\^{}x \^{}y \^{}z\} }} Scale of the node (optional)
\end{description}

The following example copies a node called \texttt{box5} as new node \texttt{box6}
and moves it to position (2, 0, 2).
\begin{verbatim}
(S1 ^svs S3)
  (S3 ^command C3 ^spatial-scene S4)
    (C3 ^copy_node A1)
      (A1 ^id box6 ^source box5 ^position P1)
        (P1 ^x 2.0 ^y 0.0 ^z 2.0)
\end{verbatim}

\subsection{delete\_node}
This command deletes a node from the scene graph. Any children will also be deleted. 
\begin{description}
  \item{\soarb{\^{}id [string]}} The id of the node to delete
\end{description}

The following example deletes a node called \texttt{box5}
\begin{verbatim}
(S1 ^svs S3)
  (S3 ^command C3 ^spatial-scene S4)
    (C3 ^delete_node D1)
      (D1 ^id box5)
\end{verbatim}

\subsection{set\_transform}
This command allows you to change the position, rotation, and/or scale of an 
exisiting node. You can specify any combination of the three transforms. 
\begin{description}
  \item{\soarb{\^{}id [string]}} The id of the node to change
  \item{\soarb{\^{}position.\{\^{}x \^{}y \^{}z\} }} Position of the node (optional)
  \item{\soarb{\^{}rotation.\{\^{}x \^{}y \^{}z\} }} Rotation of the node (optional)
  \item{\soarb{\^{}scale.\{\^{}x \^{}y \^{}z\} }} Scale of the node (optional)
\end{description}

The following example moves and rotates a node called \texttt{box5}.
\begin{verbatim}
(S1 ^svs S3)
  (S3 ^command C3 ^spatial-scene S4)
    (C3 ^set_transform S6)
      (S6 ^id box5 ^position P1 ^rotation R1)
        (P1 ^x 2.0 ^y 2.0 ^z 0.0)
        (R1 ^x 0.0 ^y 0.0 ^z 1.57)
\end{verbatim}

\subsection{set\_tag}
This command allows you to add or change a tag on a node.
If a tag with the same id already exists, 
the existing value will be replaced with the new value.
\begin{description}
  \item{\soarb{\^{}id [string]}} The id of the node to set the tag on
  \item{\soarb{\^{}tag\_name [string]}} The name of the tag to add
  \item{\soarb{\^{}tag\_value [string]}} The value of the tag to add
\end{description}

The following example adds a shape tag to the node \texttt{box5}.
\begin{verbatim}
(S1 ^svs S3)
  (S3 ^command C3 ^spatial-scene S4)
    (C3 ^set_tag S6)
      (S6 ^id box5 ^tag_name shape ^tag_value cube)
\end{verbatim}

\subsection{delete\_tag}
This command allows you to delete a tag from a node.
\begin{description}
  \item{\soarb{\^{}id [string]}} The id of the node to delete the tag from
  \item{\soarb{\^{}tag\_name [string]}} The name of the tag to delete
\end{description}

The following example deletes the shape tag from the node \texttt{box5}.
\begin{verbatim}
(S1 ^svs S3)
  (S3 ^command C3 ^spatial-scene S4)
    (C3 ^delete_tag D1)
      (D1 ^name box5 ^tag_name shape)
\end{verbatim}

\subsection{extract and extract\_once}
This command is commonly used to compute spatial relationships in the scene graph.
More generally, it puts the result of a filter pipeline (described in section \ref{sec:svs-filters}) in working memory.
Its syntax is the same as filter pipeline syntax.
During the input phase, SVS will evaluate the filter and 
put a \soarb{\^{}result} attribute on the command's identifier.
Under the \soarb{\^{}result} attribute is a multi-valued \soarb{\^{}record} attribute.
Each record corresponds to an output value from the head of the filter pipeline, along with the parameters that produced the value.
With the regular \texttt{extract} command, these records will be updated as the scene graph
changes. With the \texttt{extract\_once} command, the records will be created once
and will not change. 
Note that you should not change the structure of a filter once it is created 
(SVS only processes a command once). 
Instead to extract something different you must create a new command. 
The following is an example of an extract command which tests whether the 
car and pole objects are intersecting. The \soar{\^{}status} and \soar{\^{}result} wmes are 
added by SVS when the command is finished. 

\begin{verbatim}
(S1 ^svs S3)
  (S3 ^command C3 ^spatial-scene S4)
    (C3 ^extract E2)
      (E2 ^a A1 ^b B1 ^result R7 ^status success ^type intersect)
        (A1 ^id car ^status success ^type node)
        (B1 ^id pole ^status success ^type node)
        (R7 ^record R17)
          (R17 ^params P1 ^value false)
            (P1 ^a car ^b pole)
\end{verbatim}

\section{Filters}
\label{sec:svs-filters}

Filters are the basic unit of computation in SVS.
They transform the continuous information in the scene graph into symbolic information that can be used by the rest of Soar.
Each filter accepts a number of labeled parameters as input, and produces a single output.
Filters can be arranged into pipelines in which the outputs of some filters are fed into the inputs of other filters.
The Soar agent creates filter pipelines by building an analogous structure in working memory as an argument to an "extract" command.
For example, the following structure defines a set of filters that reports whether the car intersects the pole:

\begin{verbatim}
(S1 ^svs S3)
  (S3 ^command C3 ^spatial-scene S4)
    (C3 ^extract E2)
      (E2 ^a A1 ^b B1 ^type intersect)
        (A1 ^id car ^type node)
        (B1 ^id pole ^type node)
\end{verbatim}

The \soarb{\^{}type} attribute specifies the type of filter to instantiate, and the other attributes specify parameters.
This command will create three filters: an \soarb{intersect} filter and two \soarb{node} filters.
A \soarb{node} filter take an \soarb{id} parameter and returns the scene graph node with that ID as its result.
Here, the outputs of the \soarb{car} and \soarb{pole} node filters are fed into the \soarb{\^{}a} and \soarb{\^{}b} parameters of the \soarb{intersect} filter.
SVS will update each filter's output once every decision cycle, at the end of the input phase.
The output of the \soarb{intersect} filter is a boolean value indicating whether the two objects are intersecting.
This is placed into working memory as the result of the extract command:

\begin{verbatim}
(S1 ^svs S3)
  (S3 ^command C3 ^spatial-scene S4)
    (C3 ^extract E2)
      (E2 ^a A1 ^b B1 ^result R7 ^status success ^type intersect)
        (A1 ^id car ^status success ^type node)
        (B1 ^id pole ^status success ^type node)
        (R7 ^record R17)
          (R17 ^params P1 ^value false)
            (P1 ^a car ^b pole)
\end{verbatim}

Notice that a \soarb{\^{}status} success is placed on each identifier corresponding to a filter.
A \soarb{\^{}result} WME is placed on the extract command with a single record with value \soarb{false}.

\subsection{Result lists}

Spatial queries often involve a large number of objects.
For example, the agent may want to compute whether an object intersects any others in the scene graph.
It would be inconvenient to build the extract command to process this query if the agent had to specify each object involved explicitly.
Too many WMEs would be required, which would slow down the production matcher as well as SVS because it must spend more time interpreting the command structure.
To handle these cases, all filter parameters and results can be lists of values.
For example, the query for whether one object intersects all others can be expressed as

\begin{verbatim}
(S1 ^svs S3)
  (S3 ^command C3)
    (C3 ^extract E2)
      (E2 ^a A1 ^b B1 ^result R7 ^status success ^type intersect)
        (A1 ^id car ^status success ^type node)
        (B1 ^status success ^type all_nodes)
        (R7 ^record R9 ^record R8)
          (R9 ^params P2 ^value false)
            (P2 ^a car ^b pole)
          (R8 ^params P1 ^value true)
            (P1 ^a car ^b car)
\end{verbatim}

The \soarb{all\_nodes} filter outputs a list of all nodes in the scene graph, and the \soarb{intersect} filter outputs a list of boolean values indicating whether the car intersects each node, represented by the multi-valued attribute \soarb{record}.
Notice that each \soarb{record} contains both the result of the query as well as the parameters that produced that result.
Not only is this approach more convenient than creating a separate command for each pair of nodes, but it also allows the \soarb{intersect} filter to answer the query more efficiently using special algorithms that can quickly rule out non-intersecting objects.

\subsection{Filter List}
The following is a list of all filters that are included in SVS. 
You can also get this list by using the cli command \texttt{svs filters} and 
get information about a specific filter using the command \texttt{svs filters.FILTER\_NAME}.
Many filters have a \texttt{\_select} version. The select version returns a subset
of the input nodes which pass a test. For example, the \texttt{intersect} filter returns
boolean values for each input (a, b) pair, while the \texttt{intersect\_select} filter
returns the nodes in set b which intersect the input node a. This is useful for passing
the results of one filter into another (e.g. take the nodes that intersect node a and find
the largest of them). 

\begin{description}
  \item{\soarb{node}} \\
    Given an \soarb{\^{}id}, outputs the node with that id.
  \item{\soarb{all\_nodes}} \\
    Outputs all the nodes in the scene
  \item{\soarb{combine\_nodes}} \\
    Given multiple node inputs as \soarb{\^{}a}, concates them into a single output set.
  \item{\soarb{remove\_node}} \\
    Removes node \soarb{\^{}id} from the input set \soarb{\^{}a} and outputs the rest. 
  \item{\soarb{node\_position}} \\
    Outputs the position of each node in input \soarb{\^{}a}.
  \item{\soarb{node\_rotation}} \\
    Outputs the rotation of each node in input \soarb{\^{}a}.
  \item{\soarb{node\_scale}} \\
    Outputs the scale of each node in input \soarb{\^{}a}.
  \item{\soarb{node\_bbox}} \\
    Outputs the bounding box of each node in input \soarb{\^{}a}.
\item{\soarb{distance} and \soarb{distance\_select}} \\
  Outputs the distance between input nodes \soarb{\^{}a} and \soarb{\^{}b}. 
  Distance can be specified by \soarb{\^{}distance\_type << centroid hull >>}, where
  \texttt{centroid} is the euclidean distance between the centers, and the \texttt{hull}
  is the minimum distance between the node surfaces. \texttt{distance\_select} chooses
  nodes in set b in which the distance to node a falls within the range \soarb{\^{}min} and \soarb{\^{}max}.
\item{\soarb{closest} and \soarb{farthest}} \\
  Outputs the node in set \soarb{\^{}b} closest to or farthest from \soarb{\^{}a}
  (also uses \soarb{distance\_type}).
\item{\soarb{axis\_distance} and \soarb{axis\_distance\_select}} \\
  Outputs the distance from input node \soarb{\^{}a} to \soarb{\^{}b} along
  a particular axis (\soarb{\^{}axis << x y z >>}). This distance is based on 
  bounding boxes. A value of 0 indicates the nodes overlap on the given axis, otherwise 
  the result is a signed value indicating whether node b is greater or less than 
  node a on the given axis.  
  The \texttt{axis\_distance\_select} filter also uses \soarb{\^{}min} and \soarb{\^{}max}
  to select nodes in set b. 
\item{\soarb{volume} and \soarb{volume\_select}} \\
  Outputs the bounding box volume of each node in set \soarb{\^{}a}. 
  For \texttt{volume\_select}, it outputs a subset of the nodes whose volumes
  fall within the range \soarb{\^{}min} and \soarb{\^{}max}.
\item{\soarb{largest} and \soarb{smallest}} \\
  Outputs the node in set \soarb{\^{}a} with the largest or smallest volume.
\item{\soarb{larger} and \soarb{larger\_select}}\\
  Outputs whether input node \soarb{\^{}a} is larger than each input node \soarb{\^{}b}, 
  or selects all nodes in b for which a is larger. 
\item{\soarb{smaller} and \soarb{smaller\_select}}\\
  Outputs whether input node \soarb{\^{}a} is smaller than each input node \soarb{\^{}b}, 
  or selects all nodes in b for which a is smaller. 
\item{\soarb{contain} and \soarb{contain\_select}} \\
  Outputs whether the bounding box of each input node \soarb{\^{}a} contains
  the bounding box of each input node \soarb{\^{}b}, or selects those nodes
  in b which are contained by node a. 
\item{\soarb{intersect} and \soarb{intersect\_select}} \\
  Outputs whether each input node \soarb{\^{}a} intersects each input node \soarb{\^{}b}, 
  or selects those nodes in b which intersect node a. Intersection is specified
  by \soarb{\^{}intersect\_type << hull box >>}; either the convex hull of the node
  or the axis-aligned bounding box. 
  \item{\soarb{tag\_select}} \\
    Outputs all the nodes in input set \soarb{\^{}a} which have the tag specified by 
    \soarb{\^{}tag\_name} and \soarb{\^{}tag\_value}. 
\end{description}

\subsection{Examples}

Select all the objects with a volume between 1 and 2. 
\begin{verbatim}
(S1 ^svs S3)
  (S3 ^command C3)
    (C3 ^extract E1)
      (E1 ^type volume_select ^a A1 ^min 1 ^max 2)
        (A1 ^type all_nodes)
\end{verbatim} 

Find the distance between the centroid of ball3 and all other objects. 
\begin{verbatim}
(S1 ^svs S3)
  (S3 ^command C3)
    (C3 ^extract E1)
      (E1 ^type distance ^a A1 ^b B1 ^distance_type centroid)
        (A1 ^type node ^id ball3)
        (B1 ^type all_nodes)
\end{verbatim} 

Test where ball2 intersects any red objects. 
\begin{verbatim}
(S1 ^svs S3)
  (S3 ^command C3)
    (C3 ^extract E1)
      (E1 ^type intersect ^a A1 ^b B1 ^intersect_type hull)
        (A1 ^type node ^id ball2)
        (B1 ^type tag_select ^a A2 ^tag_name color ^tag_value red)
          (A2 ^type all_nodes)
\end{verbatim}

Find all the objects on the table. This is done by selecting nodes 
where the distance between them and the table along the z axis is a small positive number. 
\begin{verbatim}
(S1 ^svs S3)
  (S3 ^command C3)
    (C3 ^extract E1)
      (E1 ^type axis_distance_select ^a A1 ^b B1 ^axis z ^min 0.0001 ^max 0.1)
        (A1 ^type node ^id table)
        (B1 ^type all_nodes)
\end{verbatim}

Find the smallest object that intersects the table (excluding itself). 
\begin{verbatim}
(S1 ^svs S3)
  (S3 ^command C3)
    (C3 ^extract E1)
      (E1 ^type smallest ^a A1)
        (A1 ^type intersect_select ^a A2 ^b B2 ^intersect_type hull)
          (A2 ^type node ^id table)
          (B1 ^type remove_node ^id table ^a A3)
            (A3 ^type all_nodes)
\end{verbatim}




\section{Writing new filters}

SVS contains a small set of generally useful filters, but many users will need additional specialized filters for their application.
Writing new filters for SVS is conceptually simple.

\begin{enumerate}
\item Write a C++ class that inherits from the appropriate filter subclass.
\item Register the new class in a global table of all filters (\texttt{filter\_table.cpp}).
\item Recompile the kernel. 
\end{enumerate}

\subsection{Filter subclasses}

The fact that filter inputs and outputs are lists rather than single values introduces some complexity to how filters are implemented.
Depending on the functionality of the filter, the multiple inputs into multiple parameters must be combined in different ways, and sets of inputs will map in different ways onto the output values.
Furthermore, the outputs of filters are cached so that the filter does not repeat computations on sets of inputs that do not change.
To shield the user from this complexity, a set of generally useful filter paradigms were implemented as subclasses of the basic \texttt{filter} class.
When writing custom filters, try to inherit from one of these classes instead of from \texttt{filter} directly.

\subsubsection{Map filter}
This is the most straightforward and useful class of filters.
A filter of this class takes the Cartesian product of all input values in all parameters,
and performs the same computation on each combination, generating one output.
In other words, this class implements a one-to-one mapping from input combinations to output values.

To write a new filter of this class, inherit from the \texttt{map\_filter} class, 
and define the \texttt{compute} function. Below is an example template:

\begin{verbatim}
class new_map_filter : public map_filter<double> // templated with output type
{
  public:
    new_map_filter(Symbol *root, soar_interface *si, filter_input *input, scene *scn)
    : map_filter<double>(root, si, input)   // call superclass constructor
    {}

    /* Compute
       Do the proper computation based on the input filter_params 
       and set the out parameter to the result 
       Return true if successful, false if an error occured */
    bool compute(const filter_params* p, double& out){
      sgnode* a;
      if(!get_filter_param(this, p, "a", a)){
        set_status("Need input node a");
        return false;
      }
      out = // Your computation here
    }
};
\end{verbatim}

\subsubsection{Select filter}
This is very similar to a map filter, except for each input combination from the 
Cartesian product the output is optional. This is useful for selecting and returning
a subset of the outputs. 

To write a new filter of this class, inherit from the \texttt{select\_filter} class, 
and define the \texttt{compute} function. Below is an example template:

\begin{verbatim}
class new_select_filter : public select_filter<double> // templated with output type
{
  public:
    new_select_filter(Symbol *root, soar_interface *si, filter_input *input, scene *scn)
    : select_filter<double>(root, si, input)   // call superclass constructor
    {}

    /* Compute
       Do the proper computation based on the input filter_params 
       and set the out parameter to the result (if desired)
       Also set the select bit to true if you want to the result to be output. 
       Return true if successful, false if an error occured */
    bool compute(const filter_params* p, double& out, bool& select){
      sgnode* a;
      if(!get_filter_param(this, p, "a", a)){
        set_status("Need input node a");
        return false;
      }
      out = // Your computation here
      select = // test for when to output the result of the computation
    }
};
\end{verbatim}

\subsubsection{Rank filter}
A filter where a ranking is computed for each combination from the Cartesian
product of the input and only the combination which results in the highest 
(or lowest) value is output. The default behavior is to select the highest, 
to select the lowest you can call \texttt{set\_select\_highest(false)} on the filter.

To write a new filter of this class, inherit from the \texttt{rank\_filter} class, 
and define the \texttt{rank} function. Below is an example template:

\begin{verbatim}
class new_rank_filter : public rank_filter
{
  public:
    new_rank_filter(Symbol *root, soar_interface *si, filter_input *input, scene *scn)
    : rank_filter(root, si, input)   // call superclass constructor
    {}

    /* Compute
       Do the proper computation based on the input filter_params 
       And set r to the ranking result. 
       Return true if successful, false if an error occured */
    bool compute(const filter_params* p, double& r){
      sgnode* a;
      if(!get_filter_param(this, p, "a", a)){
        set_status("Need input node a");
        return false;
      }
      r = // Ranking computation
    }
};
\end{verbatim}

\subsection{Generic Node Filters}
There are also a set of generic filters specialized for computations involving nodes. 
With these you only need to specify a predicate function involving nodes. 
Also see \texttt{filters/base\_node\_filters.h}.
There are three types of these filters. 

\subsubsection{Node Test Filters}
These filters involve a binary test between two nodes (e.g. intersection or larger). 
You must specify a test function of the following form:
\begin{verbatim}
bool node_test(sgnode* a, sgnode* b, const filter_params* p)
\end{verbatim}
For an example of how the following base filters are used, see \texttt{filters/intersect.cpp}.

\textbf{node\_test\_filter} \\
For each input pair (a, b) this outputs the boolean result of \texttt{node\_test(a, b)}.

\textbf{node\_test\_select\_filter} \\
For each input pair (a, b) this outputs node b if \texttt{node\_test(a, b) == true}. 
(Can choose to select b if the test is false by calling \texttt{set\_select\_true(false)}).

\subsubsection{Node Comparison Filters}
These filters involve a numerical comparison between two nodes (e.g. distance). 
You must specify a comparison function of the following form:
\begin{verbatim}
double node_comparison(sgnode* a, sgnode* b, const filter_params* p)
\end{verbatim}

For an example of how the following base filters are used, see \texttt{filters/distance.cpp}.

\textbf{node\_comparison\_filter} \\
For each input pair (a, b) this outputs the numerical result of \texttt{node\_comparison(a, b)}. 

\textbf{node\_comparison\_select\_filter} \\
For each input pair (a, b), this outputs node b if 
\texttt{min <= node\_comparison(a, b) <= max}. 
Min and max can be set through calling \texttt{set\_min(double)} 
and \texttt{set\_max(double)}, or as specified by the user through the filter\_params. 

\textbf{node\_comparison\_rank\_filter} \\
This outputs the input pair (a, b) for which \texttt{node\_comparison(a, b)} 
produces the highest value. To instead have the lowest value output call \texttt{set\_select\_highest(true)}.


\subsubsection{Node Evaluation Filters}
These filters involve a numerical evaluation of a single node (e.g. volume). 
You must specify an evaluation function of the following form:
\begin{verbatim}
double node_evaluation(sgnode* a, const filter_params* p)
\end{verbatim}

For an example of how the following base filters are used, see \texttt{filters/volume.cpp}.

\textbf{node\_evaluation\_filter} \\
For each input node a, this outputs the numerical result of \texttt{node\_evaluation(a)}. 

\textbf{node\_evaluation\_select\_filter} \\
For each input node a, this outputs the node if 
\texttt{min <= node\_evaluation(a) <= max}. 
Min and max can be set through calling \texttt{set\_min(double)} 
and \texttt{set\_max(double)}, or as specified by the user through the filter\_params. 

\textbf{node\_evaluation\_rank\_filter} \\
This outputs the input node a for which \texttt{node\_evaluation(a)} 
produces the highest value. To instead have the lowest value output call \texttt{set\_select\_highest(true)}.


\section{Command line interface}

The user can query and modify the runtime behavior of SVS using the \soarb{svs} command.
The syntax of this command differs from other Soar commands due to the complexity and object-oriented nature of the SVS implementation.
The basic idea is to allow the user to access each object in the SVS implementation (not to be confused with objects in the scene graph) at runtime.
Therefore, the command has the form \texttt{svs PATH [ARGUMENTS]}, where \texttt{PATH} uniquely identifies an object or the method of an object.
\texttt{ARGUMENTS} is a space separated list of strings that each object or function interprets in its own way.
For example, \texttt{svs S1.scene.world.car} identifies the car object 
in the scene graph of the top state.
As another example, \verb|svs connect_viewer 5999| calls the method to connect to the SVS visualizer with 5999 being the TCP port to connect on.
Every path has two special arguments.

\begin{itemize}
\item{\texttt{svs PATH dir}} prints all the children of the object at \texttt{PATH}.
\item{\texttt{svs PATH help}} prints text about how to use the object, if available.
\end{itemize}


% ----------------------------------------------------------------------------
\typeout{--------------- The Soar User INTERFACE -----------------------------}
\chapter{The Soar User Interface}
\label{INTERFACE}
\index{interface}
%\index{user interface}
%\index{function definitions}

\nocomment{for each command, use the 'funsum' command with a brief
	description. This writes to the manual.glo file which can be edited
	into the funtion summary and index (see that file for more
	instructions). This is a bit tedious, but the reason I've set it up
	this way is that the command set is in flux right now -- this lessens
	the chance that a command will be inadvertently omitted from the
	function summary (or that a defunct command will be inadvertently
	included). 
	}

\nocomment{\begin{figure}[h]
\psfig{figure=dilbert-living.ps,height=2.2in} \vspace{12pt}
\end{figure}
}
% ----------------------------------------------------------------------------


This chapter describes the set of user interface commands for Soar. All commands and examples are presented as 
if they are being entered at the Soar command prompt.

This chapter is organized into 7 sections:
\begin{enumerate}
\item Basic Commands for Running Soar
\item Examining Memory
\item Configuring Trace Information and Debugging
\item Configuring Soar's Run-Time Parameters
\item File System I/O Commands
\item Soar I/O commands
\item Miscellaneous Commands
\end{enumerate}

Each section begins with a summary description of the commands covered
in that section, including the role of the command and its importance
to the user.  Command syntax and usage are then described fully, in
alphabetical order.

The following pages were automatically generated from the git repository
at

\hspace{2em}\soar{\htmladdnormallink{https://github.com/SoarGroup/Soar/wiki}{https://github.com/SoarGroup/Soar/wiki}}

on the date listed on the title page of this manual.  Please consult
the repository directly for the most accurate and up-to-date information.

For a concise overview of the Soar interface functions, see the Function
Summary and Index on page \pageref{func-sum}. This index is intended to be a
quick reference into the commands described in this chapter.

\subsubsection*{Notation}

\nocomment{check for all commands that I've got the notation current}

The notation used to denote the syntax for each user-interface command follows
some general conventions:\vspace{-12pt}
\begin{itemize}
\item The command name itself is given in a \soarb{bold} font.\vspace{-8pt}
\item Optional command arguments are enclosed within square brackets,
	\soar{[} and \soar{]}.\vspace{-8pt}
\item A vertical bar, \soar{|}, separates alternatives.\vspace{-8pt}
\item Curly braces, \soar{\{\}}, are used to group arguments when at least
one argument from the set is required.
\item The commandline prompt that is printed by Soar, is normally
the agent name, followed by '\soar{>}'.  In the examples in this manual, 
we use ``\soar{soar>}''.
\item Comments in the examples are preceded by
a '\soar{\#}', and in-line comments are preceded by '\soar{;\#}'.
\end{itemize}

For many commands, there is some flexibility in the order in which the
arguments may be given. (See the online help for each command for more
information.)  We have not incorporated this flexible ordering into the syntax
specified for each command because doing so complicates the specification of
the command.  When the order of arguments will affect the output
produced by a command, the reader will be alerted.

Note that the command list was revamped and simplified in Soar 9.6.0.  While 
we encourage people to learn the new syntax, aliases and some special mechanism 
have been added to maintain backwards compatibility with old Soar commands.  As a 
result, many of the sub-commands of the newer commands may use different styles of 
arguments.

% ----------------------------------------------------------------------------
\section{Basic Commands for Running Soar}
\label{BASIC}

This section describes the commands used to start, run and stop a Soar 
program; to invoke on-line help information; and to create and 
delete Soar productions.  The specific commands described in this
section are:

\paragraph{Summary}
\begin{quote}
\begin{description}
\item[soar] - Commands and settings related to running Soar.  Use \textbf{soar ?} for a summary of sub-commands listed below.
\item[soar init] - Reinitialize Soar so a program can be rerun from scratch.
\item[soar stop] - Interrupt a running Soar program.
\item[soar max-chunks] - Limit the number of chunks created during a decision cycle.
\item[soar max-dc-time] - Set a wall-clock time limit such that the agent will be interrupted when a single decision cycle exceeds this limit.
\item[soar max-elaborations] - Limit the maximum number of elaboration cycles in a given phase.
\item[soar max-goal-depth] - Limit the sub-state stack depth.
\item[soar max-memory-usage] - Set the number of bytes that when exceeded by an agent, will trigger the memory usage exceeded event. 
\item[soar max-nil-output-cycles] - Limit the maximum number of decision cycles executed without producing output. 
\item[soar max-gp] - Set the upper-limit to the number of productions generated by the gp command.
\item[soar stop-phase] -  Controls the phase where agents stop when running by decision.
\item[soar tcl] -  Controls whether Soar Tcl mode is enabled.
\item[soar timers] - Toggle on or off the internal timers used to profile Soar.
\item[soar version] - Returns version number of Soar kernel.
\item[soar waitsnc] - Generate a wait state rather than a state-no-change impasse.
\item[gp] - Define a pattern used to generate and source a set of Soar productions.
\item[run] - Begin Soar's execution cycle.
\item[sp] - Create a production and add it to production memory.
\item[help] - Provide formatted, on-line information about Soar commands.
\end{description}
\end{quote}
These commands are all frequently used anytime Soar is run.

\input{wikicmd/tex/soar}
\input{wikicmd/tex/gp}
\subsection{\soarb{help}}
\label{help}
\index{help}
Provide formatted usage information about Soar commands. 
\subsubsection*{Synopsis}
help [command_name]
\end{verbatim}
\subsubsection*{Options}
\hline
\soar{\soar{\soar{\soar{ command\_name }}}} & Print usage syntax for the command.  \\
\hline
\end{tabular}
\subsubsection*{Description}
 This command prints formatted help for the given command name. 
\subsubsection*{Examples}
 To see the syntax for the \emph{excise}
help excise
\end{verbatim}
help
\end{verbatim}
\subsubsection*{Default Aliases}
\hline
\soar{\soar{\soar{\soar{ Alias }}}} & Maps to  \\
\hline
\soar{\soar{\soar{\soar{�? }}}} & help  \\
\hline
\soar{\soar{\soar{\soar{ h }}}} & help  \\
\hline
\soar{\soar{\soar{\soar{ man }}}} & help  \\
\hline
\end{tabular}

\subsection{\soarb{run}}
\label{run}
\index{run}
Begin Soar\~A�\^a�$\neg$\^a��s execution cycle. 
\subsubsection*{Synopsis}
run  [f|\emph{count}
]
run -[d|e|o|p][s][un] [f|\emph{count}
]
run -[d|e|o|p][un] \emph{count}
 [-i <e|p|d|o>]
\end{verbatim}
\subsubsection*{Options}
\hline
\soar{\soar{\soar{ -d, --decision }}} & Run Soar for count decision cycles.  \\
\hline
\soar{\soar{\soar{ -e, --elaboration }}} & Run Soar for count elaboration cycles.  \\
\hline
\soar{\soar{\soar{ -o, --output }}} & Run Soar until the nth time output is generated by the agent. Limited by the value of max-nil-output-cycles.  \\
\hline
\soar{\soar{\soar{ -p, --phase }}} & Run Soar by phases. A phase is either an input phase, proposal phase, decision phase, apply phase, or output phase.  \\
\hline
\soar{\soar{\soar{ -s, --self }}} & If other agents exist within the kernel, do not run them at this time.  \\
\hline
\soar{\soar{\soar{ -u, --update }}} & Sets a flag in the update event callback requesting that an environment updates. This is the default if --self is not specified.  \\
\hline
\soar{\soar{\soar{ -n, --noupdate }}} & Sets a flag in the update event callback requesting that an environment does not update. This is the default if --self is specified.  \\
\hline
\soar{\soar{\soar{ f, forever }}} & Run until halted by problem-solving completion or until stopped by an interrupt.  \\
\hline
\soar{\soar{\soar{ count }}} & A single integer which specifies the number of cycles to run Soar.  \\
\hline
\soar{\soar{\soar{ -i, --interleave }}} & Support round robin execution across agents at a finer grain than the run-size parameter. e = elaboration, p = phase, d = decision, o = output  \\
\hline
\end{tabular}
\paragraph*{Deprecated Options}
 These may be reimplemented in the future. 
\hline
\soar{\soar{\soar{ --operator }}} & Run Soar until the nth time an operator is selected.  \\
\hline
\soar{\soar{\soar{ --state }}} & Run Soar until the nth time a state is selected.  \\
\hline
\end{tabular}
\subsubsection*{Description}
 The \textbf{run}
 command starts the Soar execution cycle or continues any execution that was temporarily stopped. The default behavior of \textbf{run}
, with no arguments, is to cause Soar to execute until it is halted or interrupted by an action of a production, or until an external interrupt is issued by the user. The \textbf{run}
 command can also specify that Soar should run only for a specific number of Soar cycles or phases (which may also be prematurely stopped by a production action or the stop-soar command). This is helpful for debugging sessions, where users may want to pay careful attention to the specific productions that are firing and retracting. 
 The \textbf{run}
 command takes optional arguments: an integer, \emph{count}
, which specifies how many units to run; and a \emph{units}
 flag indicating what steps or increments to use. If \emph{count}
 is specified, but no \emph{units}
 are specified, then Soar is run by decision cycles. If \emph{units}
 are specified, but \emph{count}
 is unpecified, then \emph{count}
 defaults to '1'. The argument \textbf{forever}
 (can be shortened to \textbf{f}
) is a keyword used instead of an integer \emph{count}
 and indicates Soar should be run indefinitely, until halted by problem-solving completion, or stopped by an interrupt. 
 If there are multiple Soar agents that exist in the same Soar process, then issuing a \textbf{run}
 command in any agent will cause all agents to run with the same set of parameters, unless the flag \textbf{--self}
 is specified, in which case only that agent will execute. 
 If an environment is registered for the kernel's update event, then when the event it triggered, the environment will get information about how the \textbf{run}
 was executed. If a \textbf{run}
 was executed with the --update option, then then event sends a flag requesting that the environment actually update itself. If a \textbf{run}
 was executed with the --noupdate option, then the event sends a flag requesting that the environment not update itself. The --update option is the default when run is specified without the --self option is not specified. If the --self option is specified, then the --noupdate option is on by default. It is up to the environment to check for these flags and honor them. 
 Some use cases include: 
\hline
\soar{\soar{\soar{ run --self }}} & runs one agent but not the environment  \\
\hline
\soar{\soar{\soar{ run --self --update }}} & runs one agent and the environment  \\
\hline
\soar{\soar{\soar{ run }}} & runs all agents and the environment  \\
\hline
\soar{\soar{\soar{ run --noupdate }}} & runs all agents but not the environment  \\
\hline
\end{tabular}
\paragraph*{Setting an interleave size}
 When there are multiple agents running within the same process, it may be useful to keep agents more closely aligned in their execution cycle than the run increment (--elaboration, --phases, --decisions, --output) specifies. For instance, it may be necessary to keep agents in ``lock step'' at the phase level, eventhough the \textbf{run}
 command issued is for 5 decisions. Some use cases include: 
\hline
\soar{\soar{\soar{ run -d 5 -inteleave p }}} & run the agent one phase and then move to the next agent, \\ 
 looping over agents until they have run for 5 decision cycles  \\
\hline
\soar{\soar{\soar{ run -o 3 -interleave d }}} & run the agent one decision cycle and then move to the next agent. When an agent \\ 
generates output for the 3rd time, it no longer runs even if other agents continue.  \\
\hline
\end{tabular}
 The \textbf{interleave}
 parameter must always be equal to or smaller than the specified \textbf{run}
 parameter. This option is not currently compatible with the \textbf{forever}
 option. 
\paragraph*{Note}
 If Soar has been stopped due to a \textbf{halt}
 action, an \textbf{init-soar}
 command must be issued before Soar can be restarted with the \textbf{run}
 command. 
\subsubsection*{Default Aliases}
\hline
\soar{\soar{\soar{ Alias }}} & Maps to  \\
\hline
\soar{\soar{\soar{ d }}} & run -d 1  \\
\hline
\soar{\soar{\soar{ e }}} & run -e 1  \\
\hline
\soar{\soar{\soar{ step }}} & run 1  \\
\hline
\end{tabular}

\subsection{\soarb{sp}}
\label{sp}
\index{sp}
Define a Soar production. 
\subsubsection*{Synopsis}
sp {production_body}
\end{verbatim}
\subsubsection*{Options}
\hline
\soar{\soar{\soar{ production\_body }}} & A Soar production.  \\
\hline
\end{tabular}
\subsubsection*{Description}
 The \textbf{sp}
 command creates a new production and loads it into production memory. \emph{production\_body}
  name 
      ["documentation-string"] 
      [FLAG*]
      LHS
      -->
      RHS
\end{verbatim}
 The first element of a rule is its name. Conventions for names are given in the Soar Users Manual. If given, the documentation-string must be enclosed in double quotes. Optional flags define the type of rule and the form of support its right-hand side assertions will receive. The specific flags are listed in a separate section below. The LHS defines the left-hand side of the production and specifies the conditions under which the rule can be fired. Its syntax is given in detail in a subsequent section. The --$>$ symbol serves to separate the LHS and RHS portions. The RHS defines the right-hand side of the production and specifies the assertions to be made and the actions to be performed when the rule fires. The syntax of the allowable right-hand side actions are given in a later section. The Soar Users Manual gives an elaborate discussion of the design and coding of productions. Please see that reference for tutorial information about productions. 
 If the name of the new production is the same as an existing one, the old production will be overwritten (excised). 
 \textbf{RULE FLAGS}
\\ 
:o-support      specifies that all the RHS actions are to be given
                o-support when the production fires 
:no-support     specifies that all the RHS actions are only to be given
                i-support when the production fires 
:default        specifies that this production is a default production 
                (this matters for excise -task and watch task) 
:chunk          specifies that this production is a chunk 
                (this matters for learn trace)
:interrupt      specifies that Soar should stop running when this 
                production matches but before it fires
                (this is a useful debugging tool)
\end{verbatim}
 Multiple flags may be used, but not both of \textbf{o-support}
 and \textbf{no-support}
. 
 Although you could force your productions to provide O-support or I-support by using these commands --- regardless of the structure of the conditions and actions of the production --- this is not proper coding style. The \textbf{o-support}
 and \textbf{no-support}
 flags are included to help with debugging, but should not be used in a standard Soar program. 
\subsubsection*{Examples}
sp {blocks*create-problem-space   
     "This creates the top-level space"
     (state <s1> ^superstate nil)
     -->
     (<s1> ^name solve-blocks-world ^problem-space <p1>)
     (<p1> ^name blocks-world)
}
\end{verbatim}
\subsubsection*{See Also}
\hyperref[excise]{excise} \hyperref[learn]{learn} \hyperref[watch]{watch} 

\section{Examining Memory Systems}
\label{MEMORY}

This section describes the commands used to inspect production memory,
working memory, and preference memory; to see what productions will 
match and fire in the next Propose or Apply phase;  and to examine the 
goal dependency set.  These commands are particularly useful when
running or debugging Soar, as they let users see what Soar is ``thinking.''
The specific commands described in this section are:

\paragraph{Summary}
\begin{quote}
\begin{description}
\item[preferences] - Examine items in preference memory.
\item[production] - Commands to manipulate Soar rules and analyze their usage
\item[production break] - Set interrupt flag on specific productions.
\item[production excise] - This command removes productions from Soar's memory.
\item[production find] - Find productions that contain a given pattern.
\item[production firing-counts] - Print the number of times productions have fired.
\item[production matches] - Print information about the match set and partial matches.
\item[production memory-usage] - Print memory usage for production matches.
\item[production optimize-attribute] - Declare an attribute as multi-attributes so as to increase Rete production matching efficiency.
\item[production watch] - Trace firings and retractions of specific productions.
\item[print] - Print items in working, semantic and production memory.  Can also print the print the WMEs in the goal dependency set for each goal.
\item[wm] - Commands and settings related to working memory and working memory activation.
\item[wm activation] - Get/Set working memory activation parameters
\item[wm add] - Manually add an element to working memory.
\item[wm remove] - Manually remove an element from working memory.
\item[wm watch] - Print information about wmes that match a certain pattern as they are added and removed

\end{description}
\end{quote}

Of these commands, \textbf{print} is the most often used (and the most
complex) followed by \textbf{soar matches} and \textbf{soar memory-usage}. \textbf{print gds}
is useful for examining the goal dependecy set when subgoals seem to
be disappearing unexpectedly. \textbf{preferences}
is used to examine which candidate operators have been proposed.
\textbf{production find} is especially useful when the number of
productions loaded is high.  \soar{production watch} is related to \soar{watch}, but applies only 
to specific, named productions. \soar{production firing-counts} is used to see if how many times
certain rules fire.  

\subsection{\soarb{preferences}}
\label{preferences}
\index{preferences}
Examine details about the preferences that support the specified \emph{id}
 and \emph{attribute}
. 
\subsubsection*{Synopsis}
preferences [-0123nNtw] [[id] [[^]attribute]]
\end{verbatim}
\subsubsection*{Options}
\hline
\soar{\soar{\soar{ -0, -n, --none }}} & Print just the preferences themselves  \\
\hline
\soar{\soar{\soar{ -1, -N, --names }}} & Print the preferences and the names of the productions that generated them  \\
\hline
\soar{\soar{\soar{ -2, -t, --timetags }}} & Print the information for the --names option above plus the timetags of the wmes matched by the LHS of the indicated productions  \\
\hline
\soar{\soar{\soar{ -3, -w, --wmes }}} & Print the information for the --timetags option above plus the entire wme matched on the LHS.  \\
\hline
\soar{\soar{\soar{ -o, --object }}} & Print the support for all the wmes that comprise the object (the specified ID).  \\
\hline
\soar{\soar{\soar{id}}} & Must be an existing Soar object identifier.  \\
\hline
\soar{\soar{\soar{attribute}}} & Must be an existing \emph{\^{}attribute}
 of the specified identifier.  \\
\hline
\end{tabular}
\subsubsection*{Description}
 The \textbf{preferences}
 command prints all the preferences for the given object id and attribute. If \emph{id}
 and \emph{attribute}
 are not specified, they default to the current state and the current operator. The '\^{}' is optional when specifying the attribute. The optional arguments indicates the level of detail to print about each preference. 
 This command is useful for examining which candidate operators have been proposed and what relationships, if any, exist among them. If a preference has O-support, the string, ``:O'' will also be printed. 
 When only the ID is specified on the commandline, if the ID is a state, Soar uses the default attribute \^{}operator. If the ID is not a state, Soar prints the support information for all WMEs whose $<$value$>$ is the ID. 
 When an ID and the --object flag are specified, Soar prints the preferences / wme support for all WMEs comprising the specified ID. 
\subsection*{Note}
 For the time being, \textbf{numeric-indifferent}
 preferences are listed under the heading ``binary indifferents:''. 
\subsubsection*{Examples}
soar> preferences S1 operator --names
Preferences for S1 ^operator:
acceptables:
 O2 (fill) +
   From waterjug*propose*fill
 O3 (fill) +
   From waterjug*propose*fill
unary indifferents:
 O2 (fill) =
   From waterjug*propose*fill
 O3 (fill) =
   From waterjug*propose*fill
\end{verbatim}
 preferences -n
\end{verbatim}
soar> preferences s1 jug
Preferences for S1 ^jug:
  
acceptables:
  (S1 ^jug I4) �:O 
  (S1 ^jug J1) �:O 
\end{verbatim}
soar> pref J1 -1
 Support for (31: O3 ^jug J1)
   (O3 ^jug J1) 
     From water-jug*propose*fill
 Support for (11: S1 ^jug J1)
   (S1 ^jug J1) �:O 
     From water-jug*apply*initialize-water-jug-look-ahead
\end{verbatim}
 soar> pref -o s1
 Support for S1 ^problem-space:
   (S1 ^problem-space P1) 
 Support for S1 ^name:
   (S1 ^name water-jug) �:O 
 Support for S1 ^jug:
   (S1 ^jug I4) �:O 
   (S1 ^jug J1) �:O 
 Support for S1 ^desired:
   (S1 ^desired D1) �:O 
 Support for S1 ^superstate-set:
   (S1 ^superstate-set nil) 
 Preferences for S1 ^operator:
 acceptables:
   O2 (fill) +
   O3 (fill) +
 Arch-created wmes for S1�:
 (2: S1 ^superstate nil)
 (1: S1 ^type state)
 Input (IO) wmes for S1�:
 (3: S1 ^io I1)
\end{verbatim}
\subsubsection*{See Also}

\input{wikicmd/tex/production}
\subsection{\soarb{print}}
\label{print}
\index{print}
Print items in working memory or production memory. 
\subsubsection*{Synopsis}
print [-fFin] production_name
print -[a|c|D|j|u][fFin]
print [-i] [-d <depth>] \emph{identifier}
|\emph{timetag}
|\emph{pattern}
print -s[oS]
\end{verbatim}
\subsubsection*{Options}
\subsection*{Printing items in production memory}
\hline
\soar{\soar{\soar{ -a, --all }}} & print the names of all productions currently loaded  \\
\hline
\soar{\soar{\soar{ -c, --chunks }}} & print the names of all chunks currently loaded  \\
\hline
\soar{\soar{\soar{ -D, --defaults }}} & print the names of all default productions currently loaded  \\
\hline
\soar{\soar{\soar{ -f, --full }}} & When printing productions, print the whole production. This is the default when printing a named production.  \\
\hline
\soar{\soar{\soar{ -F, --filename }}} & also prints the name of the file that contains the production.  \\
\hline
\soar{\soar{\soar{ -i, --internal }}} & items should be printed in their internal form. For productions, this means leaving conditions in their reordered (rete net) form.  \\
\hline
\soar{\soar{\soar{ -j, --justifications }}} & print the names of all justifications currently loaded.  \\
\hline
\soar{\soar{\soar{ -n, --name }}} & When printing productions, print only the name and not the whole production. This is the default when printing any category of productions, as opposed to a named production.  \\
\hline
\soar{\soar{\soar{ -u, --user }}} & print the names of all user productions currently loaded  \\
\hline
\soar{\soar{\soar{production\_name}}} & print the production named production-name \\
\hline
\end{tabular}
\subsection*{Printing items in working memory}
\hline
 -d, --depth \emph{n}
 & This option overrides the default printing depth (see the default-wme-depth command for more detail).  \\
\hline
\soar{\soar{\soar{ -i, --internal }}} & items should be printed in their internal form. For working memory, this means printing the individual elements with their timetags, rather than the objects.  \\
\hline
\soar{\soar{\soar{ -v, --varprint }}} & Print identifiers enclosed in angle brackets.  \\
\hline
\emph{identifier}
 & print the object \emph{identifier}
. \emph{identifier}
 must be a valid Soar symbol such as \textbf{S1 }
\hline
\emph{pattern}
 & print the object whose working memory elements matching the given pattern. See Description for more information on printing objects matching a specific pattern.  \\
\hline
\emph{timetag}
 & print the object in working memory with the given \emph{timetag}
\hline
\end{tabular}
\subsection*{Printing the current subgoal stack}
\hline
\soar{\soar{\soar{ -s, --stack }}} & Specifies that the Soar goal stack should be printed. By default this includes both states and operators.  \\
\hline
\soar{\soar{\soar{ -o, --operators }}} & When printing the stack, print only \textbf{operators}
.  \\
\hline
\soar{\soar{\soar{ -S, --states }}} & When printing the stack, print only \textbf{states}
.  \\
\hline
\end{tabular}
\subsubsection*{Description}
 The \textbf{print}
(\emph{identifier}
 ^\emph{attribute value}
 [+])
\end{verbatim}
 The pattern is surrounded by parentheses. The \emph{identifier}
, \emph{attribute}
, and \emph{value}
 must be valid Soar symbols or the wildcard symbol * which matches all occurences. The optional \textbf{+}
 symbol restricts pattern matches to acceptable preferences. 
\subsubsection*{Examples}
print --internal (s1 ^* v2)
\end{verbatim}
print --stack
\end{verbatim}
print -if prodname
\end{verbatim}
print -u
\end{verbatim}
\subsubsection*{Default Aliases}
\hline
\soar{\soar{\soar{ Alias }}} & Maps to  \\
\hline
\soar{\soar{\soar{ p }}} & print  \\
\hline
\soar{\soar{\soar{ pc }}} & print --chunks  \\
\hline
\soar{\soar{\soar{ wmes }}} & print -i  \\
\hline
\end{tabular}
\subsubsection*{See Also}
\hyperref[default-wme-depth]{default-wme-depth} \hyperref[predefined-aliases]{predefined-aliases} 
\input{wikicmd/tex/wm}

% ****************************************************************************
% ----------------------------------------------------------------------------
\section{Configuring Trace Information and Output}
\label{DEBUG}

This section describes the commands used primarily for debugging or
to configure the trace output printed by Soar as it runs.  Users may:
specify the content of the runtime trace output; ask that
they be alerted when specific productions fire and retract; 
or request details on Soar's performance.

The specific commands described in this section are:


\paragraph{Summary}
\begin{quote}
\begin{description}
\item[echo] - Prints a string to the current output device.
\item[output] - Controls sub-commands and settings related to Soar's output.
\item[output enabled] - Toggles printing at the lowest level.
\item[output console] - Redirects printing to the the terminal.  Most users will not change this.
\item[output callbacks] - Toggles standard Soar agent callback-based printing.
\item[output log] - Record all user-interface input and output to a file. 
\item[output command-to-file] - Dump the printed output and results of a command to a file.
\item[output print-depth] - Set how many generations of an identifier's children that Soar will print
\item[output warnings] - Toggle whether or not warnings are printed.
\item[output verbose] - Control detailed information printed as Soar runs.
\item[output echo-commands] - Set whether or not commands are echoed to other connected debuggers. 
\item[stats] - Print information on Soar's runtime statistics.
\item[trace] - Control the information printed as Soar runs. \emph{(was \soar{watch})}
\item[visualize] - Creates graph visualizations of Soar's memory systems or processing.
\end{description}
\end{quote}

Of these commands, \soar{trace} is the most often used (and the most 
complex).  \textbf{output print-depth} is related to the \textbf{print} command. \soar{stats} 
is useful for understanding how much work Soar is doing. Soar applications that include a graphical interface or other
simulation environment will often require the use of \textbf{echo} 

\input{wikicmd/tex/echo}
\input{wikicmd/tex/output}
\subsection{\soarb{stats}}
\label{stats}
\index{stats}
Print information on Soar\~A�\^a�$\neg$\^a��s runtime statistics. 
\subsubsection*{Synopsis}
stats [-s|-m|-r]
\end{verbatim}
\subsubsection*{Options}
\hline
\soar{\soar{\soar{ -m, --memory }}} & report usage for Soar's memory pools  \\
\hline
\soar{\soar{\soar{ -r, --rete }}} & report statistics about the rete structure  \\
\hline
\soar{\soar{\soar{ -s, --system }}} & report the system (agent) statistics. This is the default if no args are specified.  \\
\hline
\end{tabular}
\subsubsection*{Description}
 This command prints Soar internal statistics. The argument indicates the component of interest. 
 With the \textbf{--system}
 flag, the \textbf{stats}
\item \textbf{Version}
 --- The Soar version number, hostname, and date of the run. 
\item \textbf{Number of productions}
 --- The total number of productions loaded in the system, including all chunks built during problem solving and all default productions. 
\item \textbf{Timing Information}
 --- Might be quite detailed depending on the flags set at compile time. See note on timers below. 
\item \textbf{Decision Cycles}
 --- The total number of decision cycles in the run and the average time-per-decision-cycle in milliseconds. 
\item \textbf{Elaboration cycles}
 --- The total number of elaboration cycles that were executed during the run, the average number of elaboration cycles per decision cycle, and the average time-per-elaboration-cycle in milliseconds. This is not the total number of production firings, as productions can fire in parallel. 
\item \textbf{Production Firings}
 --- The total number of productions that were fired. 
\item \textbf{Working Memory Changes}
 --- This is the total number of changes to working memory. This includes all additions and deletions from working memory. Also prints the average match time. 
\item \textbf{Working Memory Size}
 --- This gives the current, mean and maximum number of working memory elements. 
\end{itemize}
 The optional \textbf{stats}
 argument \textbf{--memory}
 provides information about memory usage and Soar's memory pools, which are used to allocate space for the various data structures used in Soar. 
 The optional \textbf{stats}
 argument \textbf{--rete}
 provides information about node usage in the Rete net, the large data structure used for efficient matching in Soar. 
\subsubsection*{Default Aliases}
\hline
\soar{\soar{\soar{ Alias }}} & Maps to  \\
\hline
\soar{\soar{\soar{ st }}} & stats  \\
\hline
\end{tabular}
\subsubsection*{See Also}
\hyperref[timers]{timers} \subsubsection*{A Note on Timers}
total CPU time
total kernel time
phase kernel time (per phase)
phase callbacks time (per phase)
input function time
output function time
\end{verbatim}
 Total CPU time is calculated from the time a decision cycle (or number of decision cycles) is initiated until stopped. Kernel time is the time spent in core Soar functions. In this case, kernel time is defined as the all functions other than the execution of callbacks and the input and output functions. The total kernel timer is only stopped for these functions. The phase timers (for the kernel and callbacks) track the execution time for individual phases of the decision cycle (i.e., input phase, preference phase, working memory phase, output phase, and decision phase). Because there is overhead associated with turning these timers on and off, the actual kernel time will always be greater than the derived kernel time (i.e., the sum of all the phase kernel timers). Similarly, the total CPU time will always be greater than the derived total (the sum of the other timers) because the overhead of turning these timers on and off is included in the total CPU time. In general, the times reported by the single timers should always be greater than than the corresponding derived time. Additionally, as execution time increases, the difference between these two values will also increase. For those concerned about the performance cost of the timers, all the run time timing calculations can be compiled out of the code by defining NO\_TIMING\_STUFF (in soarkernel.h) before compilation. 

\input{wikicmd/tex/trace}
\input{wikicmd/tex/visualize}

% ----------------------------------------------------------------------------
\section{Configuring Soar's Runtime Parameters}
\label{RUNTIME}

This section describes the commands that control Soar's Runtime Parameters.
Many of these commands provide options that simplify or restrict 
runtime behavior to enable easier and more localized debugging.
Others allow users to select alternative algorithms or methodologies.
Users can configure Soar's learning mechanism; examine the
backtracing information that supports chunks and justifications;
and configure options for selecting between mutually indifferent operators.

The specific commands described in this section are:

\paragraph{Summary}
\begin{quote}
\begin{description}
\item[chunk] - Set the parameters for explanation-based chunking, Soar's learning mechanism.
\item[explain] - Provides interactive exploration of why a rule was learned.
\item[decide ] - Commands and settings related to the selection of operators during the Soar decision process
\item[decide indifferent-selection] -  Controls indifferent preference arbitration.
\item[decide numeric-indifferent-mode] - Select method for combining numeric preferences.
\item[decide predict] - Predict the next selected operator 
\item[decide select] - Force the next selected operator 
\item[decide set-random-seed] - Seed the random number generator.
\item[epmem] - Get/Set episodic memory parameters and statistics
\item[rl] - Get/Set RL parameters and statistics 
\item[smem] - Get/Set semantic memory parameters and statistics
\item[svs] - Perform spatial visual system commands
\end{description}
\end{quote}

% ----------------------------------------------------------------------------
\input{wikicmd/tex/chunk}
\input{wikicmd/tex/explain}
\input{wikicmd/tex/decide}
\chapter{Episodic Memory}
\label{EPMEM}
\index{episodic memory}
\index{epmem}

Episodic memory is a record of an agent's stream of experience.
The episodic storage mechanism will automatically record episodes as a Soar agent executes.
The agent can later deliberately retrieve episodic knowledge to extract information and regularities that may not have been noticed during the original experience and combine them with current knowledge such as to improve performance on future tasks.

This chapter is organized as follows: episodic memory structures in working memory (\ref{EPMEM-wm}); episodic storage (\ref{EPMEM-storage}); retrieving episodes (\ref{EPMEM-retrieval}); and a discussion of performance (\ref{EPMEM-perf}).
The detailed behavior of episodic memory is determined by numerous parameters that can be controlled and configured via the \soarb{epmem} command.

Please refer to the documentation for that command in Section \ref{epmem} on page \pageref{epmem}.

\section{Working Memory Structure}
\label{EPMEM-wm}

Upon creation of a new state in working memory (see Section \ref{ARCH-impasses-types} on page \pageref{ARCH-impasses-types}; Section \ref{SYNTAX-impasses} on page \pageref{SYNTAX-impasses}), the architecture creates the following augmentations to facilitate agent interaction with episodic memory:

\begin{verbatim}
(<s> ^epmem <e>)
  (<e> ^command <e-c>)
  (<e> ^result <e-r>)
  (<e> ^present-id #)
\end{verbatim}

As rules augment the \soar{command} structure in order to retrieve episodes (\ref{EPMEM-retrieval}), episodic memory augments the \soar{result} structure in response.
Production actions should not remove augmentations of the \soar{result} structure directly, as episodic memory will maintain these WMEs.

The value of the \soar{present-id} augmentation is an integer and will update to expose to the agent the current episode number.
This information is identical to what is available via the \emph{time} statistic (see Section \ref{epmem} on page \pageref{epmem}) and the \emph{present-id} retrieval meta-data (\ref{EPMEM-meta}).

\section{Episodic Storage}
\label{EPMEM-storage}

Episodic memory records new episodes without deliberate action/consideration by the agent.
The timing and frequency of recording new episodes is controlled by the \soar{phase} and \soar{trigger} parameters.
The \soarb{phase} parameter sets the phase in the decision cycle (default: end of each decision cycle) during which episodic memory stores episodes and processes commands.
The value of the \soarb{trigger} parameter indicates to the architecture the event that concludes an episode: adding a new augmentation to the output-link (default) or each decision cycle.

For debugging purposes, the \soarb{force} parameter allows the user to manually request that an episode be recorded (or not) during the current decision cycle.
Behavior is as follows:

\vspace{-8pt}
\begin{itemize}
\item
	The value of the \soar{force} parameter is initialized to \soar{off} every decision cycle.
	\vspace{-6pt}
\item
	During the \soar{phase} of episodic storage, episodic memory tests the value of the \soar{force} parameter; if it has a value other than of off, episodic memory follows the \emph{forced} policy irrespective of the value of the \soar{trigger} parameter.
	\vspace{-6pt}
\end{itemize}

\subsection{Episode Contents}

When episodic memory stores a new episode, it captures the entire top-state of working memory.
There are currently two exceptions to this policy:

\begin{itemize}
\item
Episodic memory only supports WMEs whose attribute is a constant.
Behavior is currently undefined when attempting to store a WME that has an attribute that is an identifier.

\item
The \soarb{exclusions} parameter allows the user to specify a set of attributes for which Soar will not store WMEs.
The storage process currently walks the top-state of working memory in a breadth-first manner, and any WME that is not reachable other than via an excluded WME will not be stored.
By default, episodic memory excludes the \soar{epmem} and \soar{smem} structures, to prevent encoding of potentially large and/or frequently changing memory retrievals.

\end{itemize}

\subsection{Storage Location}
\index{epmem!storage}

Episodic memory uses SQLite to facilitate efficient and standardized storage and querying of episodes.
The episodic store can be maintained in memory or on disk (per the \soar{database} and \soar{path} parameters).
If the store is located on disk, users can use any standard SQLite programs/components to access/query its contents.
See the later discussion on performance (\ref{EPMEM-perf}) for additional parameters dealing with databases on disk.

Note that changes to storage parameters, for example \soar{database, path} and \soar{append} will not have an effect until the database is used after an initialization. This happens either shortly after launch (on first use) or after a database initialization command is issued. To switch databases or database storage types while running, set your new parameters and then perform an \soar{epmem --init} command.

The \soarb{path} parameter specifies the file system path the database is stored in. When \soar{path} is set to a valid file system path and \soar{database} mode is set to \emph{file}, then the SQLite database is written to that path.

The \soarb{append} parameter will determine whether all existing facts stored in a database on disk will be erased when episodic memory loads. Note that this affects \soar{init-soar} also.  In other words, if the \soar{append} setting is off, all episodes stored will be lost when an init-soar is performed. For episodic memory, \soar{append} mode is \soar{off} by default.

\soarit{Note}: As of version 9.3.3, Soar now uses a new schema for the episodic memory database. This means databases from 9.3.2 and below can no longer be loaded.  A conversion utility will be available in Soar 9.4 to convert from the old schema to the new one.

\section{Retrieving Episodes}
\label{EPMEM-retrieval}
\index{epmem!retrieve}

An agent retrieves episodes by creating an appropriate command (we detail the types of commands below) on the \soar{command} link of a state's \soar{epmem} structure.
At the end of the \soar{phase} of each decision, after episodic storage, episodic memory processes each state's \emph{epmem} command structure.
Results, meta-data, and errors are placed on the \soar{result} structure of that state's \soar{epmem} structure.

Only one type of retrieval command (which may include optional modifiers) can be issued per state in a single decision cycle.
Malformed commands (including attempts at multiple retrieval types) will result in an error:

\begin{verbatim}
<s> ^epmem.result.status bad-cmd
\end{verbatim}

After a command has been processed, episodic memory will ignore it until some aspect of the command structure changes (via addition/removal of WMEs).
When this occurs, the result structure is cleared and the new command (if one exists) is processed.

All retrieved episodes are recreated exactly as stored, except for any operators that have an acceptable preference, which are recreated with the attribute \soar{operator*}.
For example, if the original episode was:

\begin{verbatim}
(<s> ^operator <o1> +)
(<o1> ^name move)
\end{verbatim}

A retrieval of the episode would become:

\begin{verbatim}
(<s> ^operator* <o1>)
(<o1> ^name move)
\end{verbatim}

\subsection{Cue-Based Retrievals}
Cue-based retrieval commands are used to search for an episode in the store that best matches an agent-supplied cue, while adhering to optional modifiers.
A cue is composed of WMEs that partially describe a top-state of working memory in the retrieved episode.
All cue-based retrieval requests must contain a single \soarb{\carat query} cue and, optionally, a single \soarb{\carat neg-query} cue.

\begin{verbatim}
<s> ^epmem.command.query <required-cue>
<s> ^epmem.command.neg-query <optional-negative-cue>
\end{verbatim}

A \soar{\carat query} cue describes structures desired in the retrieved episode, whereas a \soar{\carat neg-query} cue describes non-desired structures.
For example, the following Soar production creates a \soar{\carat query} cue consisting of a particular state name and a copy of a current value on the \soar{input-link} structure:

\begin{verbatim}
sp {epmem*sample*query
    (state <s> ^epmem.command <ec>
               ^io.input-link.foo <bar>)
-->
    (<ec> ^query <q>)
    (<q> ^name my-state-name
         ^io.input-link.foo <bar>)
}
\end{verbatim}

\index{working memory activation}
As detailed below, multiple prior episodes may equally match the structure and contents of an agent's cue.
Nuxoll has produced initial evidence that in some tasks, retrieval quality improves when using \emph{activation} of cue WMEs as a form of feature weighting.
Thus, episodic memory supports integration with working memory activation (see Section \ref{wm-activation} on page \pageref{wm-activation}).
For a theoretical discussion of the Soar implementation of working memory activation, consider reading \emph{Comprehensive Working Memory Activation in Soar} (Nuxoll, A., Laird, J., James, M., ICCM 2004).

The cue-based retrieval process can be thought of conceptually as a nearest-neighbor search.
First, all candidate episodes, defined as episodes containing at least one leaf WME (a cue WME with no sub-structure) in at least one cue, are identified.
Two quantities are calculated for each candidate episode, with respect to the supplied cue(s): the cardinality of the match (defined as the number of matching leaf WMEs) and the activation of the match (defined as the sum of the activation values of each matching leaf WME).
Note that each of these values is negated when applied to a negative query.
To compute each candidate episode's match score, these quantities are combined with respect to the \soarb{balance} parameter as follows:

$$(balance)*(cardinality) + (1-balance)*(activation)$$

Performing a graph match on each candidate episode, with respect to the structure of the cue, could be very computationally expensive, so episodic memory implements a two-stage matching process.
An episode with perfect cardinality is considered a perfect \emph{surface} match and, per the \soarb{graph-match} parameter, is subjected to further \emph{structural} matching.
Whereas surface matching efficiently determines if all paths to leaf WMEs exist in a candidate episode, graph matching indicates whether or not the cue can be structurally unified with the candidate episode (paying special regard to the structural constraints imposed by shared identifiers).
Cue-based matching will return the most recent structural match, or the most recent candidate episode with the greatest match score.

A special note should be made with respect to how short- vs. long-term identifiers (see Section \ref{SMEM-kr} on page \pageref{SMEM-kr}) are interpreted in a cue.
Short-term identifiers are processed much as they are in working memory -- transient structures.
Cue matching will try to find any identifier in an episode (with respect to WME path from state) that can apply.
Long-term identifiers, however, are treated as constants.
Thus, when analyzing the cue, episodic memory will not consider long-term identifier augmentations, and will only match with the same long-term identifier (in the same context) in an episode.

The case-based retrieval process can be further controlled using optional modifiers:

\vspace{-8pt}
\begin{itemize}
\item
	The \soarb{before} command requires that the retrieved episode come relatively before a supplied time:
	\vspace{-6pt}
	\begin{verbatim}
	<s> ^epmem.command.before time
	\end{verbatim}
	\vspace{-6pt}
\item
	The \soarb{after} command requires that the retrieved episode come relatively after a supplied time:
	\vspace{-6pt}
	\begin{verbatim}
	<s> ^epmem.command.after time
	\end{verbatim}
	\vspace{-6pt}
\item
	The \soarb{prohibit} command requires that the time of the retrieved episode is not equal to a supplied time:
	\vspace{-6pt}
	\begin{verbatim}
	<s> ^epmem.command.prohibit time
	\end{verbatim}
	\vspace{-6pt}
	Multiple prohibit command WMEs may be issued as modifiers to a single CB retrieval.
	\vspace{-6pt}
\end{itemize}
\vspace{-12pt}

If no episode satisfies the cue(s) and optional modifiers an error is returned:

\begin{verbatim}
<s> ^epmem.result.failure <query> <optional-neg-query>
\end{verbatim}

If an episode is returned, there is additional meta-data supplied (\ref{EPMEM-meta}).

\subsection{Absolute Non-Cue-Based Retrieval}
At time of storage, each episode is attributed a unique \emph{time}.
This is the current value of \soarb{time} statistic and is provided as the \emph{memory-id} meta-data item of retrieved episodes (\ref{EPMEM-meta}).
An absolute non-cue-based retrieval is one that requests an episode by time.
An agent issues an absolute non-cue-based retrieval by creating a WME on the \soar{command} structure with attribute \emph{retrieve} and value equal to the desired time:

\begin{verbatim}
<s> ^epmem.command.retrieve time
\end{verbatim}

Supplying an invalid value for the \soar{retrieve} command will result in an error.

The time of the first episode in an episodic store will have value 1 and each subsequent episode's time will increase by 1.
Thus the desired time may be the mathematical result of operations performed on a known episode's time.

The current episodic memory implementation does not implement any episodic store dynamics, such as forgetting.
Thus any integer time greater than 0 and less than the current value of the \soar{time} statistic will be valid.
However, if forgetting is implemented in future versions, no such guarantee will be made.

\subsection{Relative Non-Cue-Based Retrieval}
Episodic memory supports the ability for an agent to ``play forward" episodes using relative non-cue-based retrievals.

Episodic memory stores the time of the last successful retrieval (non-cue-based or cue-based).
Agents can indirectly make use of this information by issuing \soarb{next} or \soarb{previous} commands.
Episodic memory executes these commands by attempting to retrieve the episode immediately proceeding/preceding the last successful retrieval (respectively).
To issue one of these commands, the agent must create a new WME on the \soar{command} link with the appropriate attribute (\soar{next} or \soar{previous}) and value of an arbitrary identifier:

\begin{verbatim}
<s> ^epmem.command.next <n>
<s> ^epmem.command.previous <p>
\end{verbatim}

If no such episode exists then an error is returned.

Currently, if the time of the last successfully retrieved episode is known to the agent (as could be the case by accessing result meta-data), these commands are identical to performing an absolute non-cue-based retrieval after adding/subtracting 1 to the last time (respectively).
However, if an episodic store dynamic like forgetting is implemented, these relative commands are guaranteed to return the next/previous valid episode (assuming one exists).

\subsection{Retrieval Meta-Data}
\label{EPMEM-meta}
\index{epmem!structures}

The following list details the WMEs that episodic memory creates in the \soar{result} link of the \soar{epmem} structure wherein a command was issued:

\begin{itemize}

\item \soarb{retrieved <retrieval-root>}
	If episodic memory retrieves an episode, that memory is placed here. This WME is an identifier that is treated as the root of the state that was used to create the episodic memory. If the \soar{retrieve} command was issued with an invalid time, the value of this WME will be \emph{no-memory}.
\item \soarb{success <query> <optional-neg-query>}
	If the cue-based retrieval was successful, the WME will have the status as the attribute and the value of the identifier of the query (and neg-query, if applicable).
\item \soarb{match-score}
	This WME is created whenever an episode is successfully retrieved from a cue-based retrieval command. The WME value is a decimal indicating the raw match score for that episode with respect to the cue(s).
\item \soarb{cue-size}
	This WME is created whenever an episode is successfully retrieved from a cue-based retrieval command. The WME value is an integer indicating the number of leaf WMEs in the cue(s).
\item \soarb{normalized-match-score}
	This WME is created whenever an episode is successfully retrieved from a cue-based retrieval command. The WME value is the decimal result of dividing the raw match score by the cue size. It can hypothetically be used as a measure of episodic memory's relative confidence in the retrieval.
\item \soarb{match-cardinality}
	This WME is created whenever an episode is successfully retrieved from a cue-based retrieval command. The WME value is an integer indicating the number of leaf WMEs matched in the \soar{\carat query} cue minus those matched in the \soar{\carat neg-query} cue.
\item \soarb{memory-id}
	This WME is created whenever an episode is successfully retrieved from a cue-based retrieval command. The WME value is an integer indicating the time of the retrieved episode.
\item \soarb{present-id}
	This WME is created whenever an episode is successfully retrieved from a cue-based retrieval command. The WME value is an integer indicating the current time, such as to provide a sense of ``now" in episodic memory terms. By comparing this value to the \soar{memory-id} value, the agent can gain a sense of the relative time that has passed since the retrieved episode was recorded.
\item \soarb{graph-match}
	This WME is created whenever an episode is successfully retrieved from a cue-based retrieval command and the \soar{graph-match} parameter was \soar{on}. The value is an integer with value 1 if graph matching was executed successfully and 0 otherwise.
\item \soarb{mapping <mapping-root>}
	This WME is created whenever an episode is successfully retrieved from a cue-based retrieval command, the \soar{graph-match} parameter was \soar{on}, and structural match was successful on the retrieved episode. This WME provides a mapping between identifiers in the cue and in the retrieved episode. For each identifier in the cue, there is a \soar{node} WME as an augmentation to the \soar{mapping} identifier. The node has a \soar{cue} augmentation, whose value is an identifier in the cue, and a \soar{retrieved} augmentation, whose value is an identifier in the retrieved episode. In a graph match it is possible to have multiple identifier mappings -- this map represents the ``first" unified mapping (with respect to episodic memory algorithms).
\end{itemize}

\section{Performance}
\label{EPMEM-perf}
\index{epmem!performance}

There are currently two sources of ``unbounded" computation: graph matching and cue-based queries.
Graph matching is combinatorial in the worst case.
Thus, if an episode presents a perfect surface match, but imperfect structural match (i.e. there is no way to unify the cue with the candidate episode), there is the potential for exhaustive search.
Each identifier in the cue can be assigned one of any historically consistent identifiers (with respect to the sequence of attributes that leads to the identifier from the root), termed a literal.
If the identifier is a multi-valued attribute, there will be more than one candidate literals and this situation can lead to a very expensive search process.
Currently there are no heuristics in place to attempt to combat the expensive backtracking.
Worst-case performance will be combinatorial in the total number of literals for each cue identifier (with respect to cue structure).

The cue-based query algorithm begins with the most recent candidate episode and will stop search as soon as a match is found (since this episode must be the most recent).
Given this procedure, it is trivial to create a two-WME cue that forces a linear search of the episodic store.
Episodic memory combats linear scan by only searching candidate episodes, i.e. only those that contain a change in at least one of the cue WMEs.
However, a cue that has no match and contains WMEs relevant to all episodes will force inspection of all episodes.
Thus, worst-case performance will be linear in the number of episodes.

\subsection{Performance Tweaking}
When using a database stored to disk, several parameters become crucial to performance.
The first is \soarb{commit}, which controls the number of episodes that occur between writes to disk.
If the total number of episodes (or a range) is known ahead of time, setting this value to a greater number will result in greatest performance (due to decreased I/O).

The next two parameters deal with the SQLite cache, which is a memory store used to speed operations like queries by keeping in memory structures like levels of index B+-trees.
The first parameter, \soarb{page-size}, indicates the size, in bytes, of each cache page.
The second parameter, \soarb{cache-size}, suggests to SQLite how many pages are available for the cache.
Total cache size is the product of these two parameter settings.
The cache memory is not pre-allocated, so short/small runs will not necessarily make use of this space.
Generally speaking, a greater number of cache pages will benefit query time, as SQLite can keep necessary meta-data in memory.
However, some documented situations have shown improved performance from decreasing cache pages to increase memory locality.
This is of greater concern when dealing with file-based databases, versus in-memory.
The size of each page, however, may be important whether databases are disk- or memory-based.
This setting can have far-reaching consequences, such as index B+-tree depth.
While this setting can be dependent upon a particular situation, a good heuristic is that short, simple runs should use small values of the page size (\soar{1k}, \soar{2k}, \soar{4k}), whereas longer, more complicated runs will benefit from larger values (\soar{8k}, \soar{16k}, \soar{32k}, \soar{64k}).
One known situation of concern is that as indexed tables accumulate many rows (\tild millions), insertion time of new rows can suffer an infrequent, but linearly increasing burst of computation.
In episodic memory, this situation will typically arise with many episodes and/or many working memory changes.
Increasing the page size will reduce the intensity of the spikes at the cost of increasing disk I/O and average/total time for episode storage.
Thus, the settings of page size for long, complicated runs establishes the desired balance of reactivity (i.e. max computation) and average speed.
To ground this discussion, the Figure \ref{fig:epmem-cache} depicts maximum and average episodic storage time (the value of the epmem\_storage timer, converted to milliseconds) with different page sizes after 10 million decisions (1 episode/decision) of a very basic agent (i.e. very few working memory changes per episode) running on a 2.8GHz Core i7 with Mac OS X 10.6.5.
While only a single use case, the cross-point of these data forms the basis for the decision to default the parameter at 8192 bytes.

\begin{figure}
\insertfigure{Figures/epmem-cache}{2.5in}
\insertcaption{Example episodic memory cache setting data.}
\label{fig:epmem-cache}
\end{figure}

The next parameter is \soarb{optimization}, which can be set to either \soar{safety} or \soar{performance}.
The \soar{safety} parameter setting will use SQLite default settings.
If data integrity is of importance, this setting is ideal.
The \soar{performance} setting will make use of lesser data consistency guarantees for significantly greater performance.
First, writes are no longer synchronous with the OS (synchronous pragma), thus episodic memory won't wait for writes to complete before continuing execution.
Second, transaction journaling is turned off (journal\_mode pragma), thus groups of modifications to the episodic store are not atomic (and thus interruptions due to application/os/hardware failure could lead to inconsistent database state).
Finally, upon initialization, episodic memory maintains a continuous exclusive lock to the database (locking\_mode pragma), thus other applications/agents cannot make simultaneous read/write calls to the database (thereby reducing the need for potentially expensive system calls to secure/release file locks).

Finally, maintaining accurate operation timers can be relatively expensive in Soar.
Thus, these should be enabled with caution and understanding of their limitations.
First, they will affect performance, depending on the level (set via the \soar{timers} parameter).
A level of \soar{three}, for instance, times every step in the cue-based retrieval candidate episode search.
Furthermore, because these iterations are relatively cheap (typically a single step in the linked-list of a b+-tree), timer values are typically unreliable (depending upon the system, resolution is 1 microsecond or more).

\chapter{Reinforcement Learning}
\label{RL}
\index{reinforcement learning}
\index{preference!numeric-indifferent}
\index{rl}

Soar has a reinforcement learning (RL) mechanism that tunes operator selection knowledge based on a given reward function.
This chapter describes the RL mechanism and how it is integrated with production memory, the decision cycle, and the state stack.
We assume that the reader is familiar with basic reinforcement learning concepts and notation. If not, we recommend first reading \emph{Reinforcement Learning: An Introduction} (1998) by Richard S. Sutton and Andrew G. Barto.
The detailed behavior of the RL mechanism is determined by numerous parameters that can be controlled and configured via the \soarb{rl} command.
Please refer to the documentation for that command in section \ref{rl} on page \pageref{rl}.

\section{RL Rules}
\label{RL-rules}

Soar's RL mechanism learns Q-values for state-operator\footnote{
In this context, the term ``state'' refers to the state of the task or environment, not a state identifier.
For the rest of this chapter, bold capital letter names such as \soarb{S1} will refer to identifiers and italic lowercase names such as $s_1$ will refer to task states.}
pairs.
Q-values are stored as numeric indifferent preferences created by specially formulated productions called \emph{RL rules}.
RL rules are identified by syntax.
A production is a RL rule if and only if its left hand side tests for a proposed operator, its right hand side creates a single numeric indifferent preference, and it is not a template rule (see \ref{RL-templates}).
These constraints ease the technical requirements of identifying/updating RL rules and makes it easy for the agent programmer to add/maintain RL capabilities within an agent.
We define an \emph{RL operator} as an operator with numeric indifferent preferences created by RL rules.

The following is an RL rule:

\begin{verbatim}
sp {rl*3*12*left
   (state <s> ^name task-name
              ^x 3
              ^y 12
	          ^operator <o> +)
   (<o> ^name move
	    ^direction left)
-->
   (<s> ^operator <o> = 1.5)
}
\end{verbatim}

Note that the LHS of the rule can test for anything as long as it contains a test for a proposed operator.
The RHS is constrained to exactly one action: creating a numeric indifferent preference for the proposed operator.

The following are not RL rules:

\begin{verbatim}
sp {multiple*preferences
   (state <s> ^operator <o> +)
-->
   (<s> ^operator <o> = 5, >)
}

sp {variable*binding
    (state <s> ^operator <o> +
               ^value <v>)
-->
    (<s> ^operator <o> = <v>)
}
\end{verbatim}

The first rule proposes multiple preferences for the proposed operator and thus does not comply with the rule format.
The second rule does not comply because it does not provide a \emph{constant} for the numeric indifferent preference value.

In the typical RL use case, the user intends for the agent to learn the best operator in each possible state of the environment.
The most straightforward way to achieve this is to give the agent a set of RL rules, each matching exactly one possible state-operator pair.
This approach is equivalent to a table-based RL algorithm, where the Q-value of each state-operator pair corresponds to the numeric indifferent preference created by exactly one RL rule.

In the more general case, multiple RL rules can match a single state-operator pair, and a single RL rule can match multiple state-operator pairs.
all numeric indifferent preferences for an operator are summed when calculating the operator's Q-value\footnote{
This is assuming the value of \soarb{numeric-indifferent-mode} is set to \soarb{sum}.
In general, the RL mechanism only works correctly when this is the case, and we assume this case in the rest of the chapter.
See page \pageref{decide-numeric-indifferent-mode} for more information about this parameter.}.
In this context, RL rules can be interpreted more generally as binary features in a linear approximator of each state-operator pair's Q-value, and their numeric indifferent preference values their weights.
In other words,
$$Q(s, a) = w_1 \phi_2 (s, a) + w_2 \phi_2 (s, a) + \ldots + w_n \phi_n (s, a)$$
where all RL rules in production memory are numbered $1 \dots n$, $Q(s, a)$ is the Q-value of the state-operator pair $(s, a)$, $w_i$ is the numeric indifferent preference value of RL rule $i$, $\phi_i (s, a) = 0$ if RL rule $i$ does not match $(s, a)$, and $\phi_i (s, a) = 1$ if it does.
This interpretation allows RL rules to simulate a number of popular function approximation schemes used in RL such as tile coding and sparse coding.

\section{Reward Representation}
\label{RL-reward}

RL updates are driven by reward signals.
In Soar, these reward signals are given to the RL mechanism through a working memory link called the \soarb{reward-link}.
Each state in Soar's state stack is automatically populated with a \soarb{reward-link} structure upon creation.
Soar will check this structure for a numeric reward signal for the last operator executed in the associated state at the beginning of every decision phase.
Reward is also collected when the agent is halted or a state is retracted.
% What happens when an agent with multiple states is halted? Do the rewards in the substates get collected?

In order to be recognized, the reward signal must follow this pattern:

\begin{verbatim}
(<r1> ^reward <r2>)
(<r2> ^value [val])
\end{verbatim}

where \verb=<r1>= is the \soarb{reward-link} identifier, \verb=<r2>= is some intermediate identifier, and \verb=[val]= is any constant numeric value.
Any structure that does not match this pattern is ignored.
If there are multiple valid reward signals, their values are summed into a single reward signal.
As an example, consider the following state:

\begin{verbatim}
(S1 ^reward-link R1)
  (R1 ^reward R2)
    (R2 ^value 1.0)
  (R1 ^reward R3)
    (R3 ^value -0.2)
\end{verbatim}  

In this state, there are two reward signals with values 1.0 and -0.2.
They will be summed together for a total reward of 0.8 and this will be the value given to the RL update algorithm.

There are two reasons for requiring the intermediate identifier.
The first is so that multiple reward signals with the same value can exist simultaneously.
Since working memory is a set, multiple WMEs with identical values in all three positions (identifier, attribute, value) cannot exist simultaneously.
Without an intermediate identifier, specifying two rewards with the same value would require a WME structure such as

\begin{verbatim}
(S1 ^reward-link R1)
  (R1 ^reward 1.0)
  (R1 ^reward 1.0)
\end{verbatim}

which is invalid. With the intermediate identifier, the rewards would be specified as

\begin{verbatim}
(S1 ^reward-link R1)
  (R1 ^reward R2)
    (R2 ^value 1.0)
  (R1 ^reward R3)
    (R3 ^value 1.0)
\end{verbatim}

which is valid.
The second reason for requiring an intermediate identifier in the reward signal is so that the rewards can be augmented with additional information, such as their source or how long they have existed.
Although this information will be ignored by the RL mechanism, it can be useful to the agent or programmer.
For example:

\begin{verbatim}
(S1 ^reward-link R1)
  (R1 ^reward R2)
    (R2 ^value 1.0)
    (R2 ^source environment)
  (R1 ^reward R3)
    (R3 ^value -0.2)
    (R3 ^source intrinsic)
    (R3 ^duration 5)
\end{verbatim}  

The \verb=(R2 ^source environment)=, \verb=(R3 ^source intrinsic)=, and \verb=(R3 ^duration 5)= \\
WMEs are arbitrary and ignored by RL, but were added by the agent to keep 
track of where the rewards came from and for how long.

Note that the \soarb{reward-link} is not part of the \soarb{io} structure and is not modified directly by the environment.
Reward information from the environment should be copied, via rules, from the \soarb{input-link} to the \soarb{reward-link}.
Also note that when collecting rewards, Soar simply scans the \soarb{reward-link} and sums the values of all valid reward WMEs.
The WMEs are not modified and no bookkeeping is done to keep track of previously seen WMEs.
This means that reward WMEs that exist for multiple decision cycles will be collected multiple times.

\section{Updating RL Rule Values}
\label{RL-algo}

Soar's RL mechanism is integrated naturally with the decision cycle and performs online updates of RL rules.
Whenever an RL operator is selected, the values of the corresponding RL rules will be updated.
The update can be on-policy (Sarsa) or off-policy (Q-Learning), as controlled by the \soarb{learning-policy} parameter of the \soarb{rl} command.
Let $\delta_t$ be the amount the Q-value of an RL operator changes in an update.
For Sarsa, we have
$$ \delta_t = \alpha \left[ r_{t+1} + \gamma Q(s_{t+1}, a_{t+1}) - Q(s_t, a_t) \right] $$
where 
\begin{itemize}
\item $Q(s_t, a_t)$ is the Q-value of the state and chosen operator in decision cycle $t$.
\item $Q(s_{t+1}, a_{t+1})$ is the Q-value of the state and chosen RL operator in the next decision cycle.
\item $r_{t+1}$ is the total reward collected in the next decision cycle.
\item $\alpha$ and $\gamma$ are the settings of the \soarb{learning-rate} and \soarb{discount-rate} parameters of the \soarb{rl} command, respectively.
\end{itemize}

Note that since $\delta_t$ depends on $Q(s_{t+1}, a_{t+1})$, the update for the operator selected in decision cycle $t$ is not applied until the next RL operator is chosen.
For Q-Learning, we have
$$ \delta_t = \alpha \left[ r_{t+1} + \gamma \underset{a \in A_{t+1}}{\max} Q(s_{t+1}, a) - Q(s_t, a_t) \right] $$
where $A_{t+1}$ is the set of RL operators proposed in the next decision cycle.

Finally, $\delta_t$ is divided by the number of RL rules comprising the Q-value for the operator and the numeric indifferent values for each RL rule is updated by that amount.

An example walkthrough of a Sarsa update with $\alpha = 0.3$ and $\gamma = 0.9$ (the default settings in Soar) follows.

\begin{enumerate}

\item In decision cycle $t$, an operator \soarb{O1} is proposed, and RL rules \soarb{rl-1} and \soarb{rl-2} create the following numeric indifferent preferences for it:
\begin{verbatim}
   rl-1: (S1 ^operator O1 = 2.3)
   rl-2: (S1 ^operator O1 =  -1)
\end{verbatim}  
	The Q-value for \soarb{O1} is $Q(s_t, \soarb{O1}) = 2.3 - 1 = 1.3$.
	 
\item \soarb{O1} is selected and executed, so $Q(s_t, a_t) = Q(s_t, \soarb{O1}) = 1.3$.

\item In decision cycle $t+1$, a total reward of 1.0 is collected on the \soarb{reward-link}, an operator \soarb{O2} is proposed, and another RL rule \soarb{rl-3} creates the following numeric indifferent preference for it:
\begin{verbatim}
	rl-3: (S1 ^operator O2 = 0.5)
\end{verbatim}
	So $Q(s_{t+1}, \soarb{O2}) = 0.5$.

\item \soarb{O2} is selected, so $Q(s_{t+1}, a_{t+1}) = Q(s_{t+1}, \soarb{O2}) = 0.5$
	Therefore, 
	$$\delta_t = \alpha \left[r_{t+1} + \gamma Q(s_{t+1}, a_{t+1}) - Q(s_t, a_t) \right] = 0.3 \times [ 1.0 + 0.9 \times 0.5 - 1.3 ] = 0.045$$
	Since \soarb{rl-1} and \soarb{rl-2} both contributed to the Q-value of \soarb{O1}, $\delta_t$ is evenly divided amongst them, resulting in updated values of
\begin{verbatim}
   rl-1: (<s> ^operator <o> = 2.3225)
   rl-2: (<s> ^operator <o> = -0.9775)
\end{verbatim}

\item \soarb{rl-3} will be updated when the next RL operator is selected.
\end{enumerate}

\subsection{Gaps in Rule Coverage}
\label{RL-gaps}

Call an operator with numeric indifferent preferences an RL operator.
The previous description had assumed that RL operators were selected in both decision cycles $t$ and $t+1$.
If the operator selected in $t+1$ is not an RL operator, then $Q(s_{t+1}, a_{t+1})$ would not be defined, and an update for the RL operator selected at time $t$ will be undefined.
We will call a sequence of one or more decision cycles in which RL operators are not selected between two decision cycles in which RL operators are selected a \emph{gap}.
Conceptually, it is desirable to use the temporal difference information from the RL operator after the gap to update the Q-value of the RL operator before the gap.
There are no intermediate storage locations for these updates.
Requiring that RL rules support operators at every decision can be difficult for agent programmers, particularly for operators that do not represent steps in a task, but instead perform generic maintenance functions, such as cleaning processed output-link structures.

To address this issue, Soar's RL mechanism supports automatic propagation of updates over gaps.
For a gap of length $n$, the Sarsa update is
$$\delta_t = \alpha \left[ \sum_{i=t}^{t+n}{\gamma^{i-t} r_i} + \gamma^{n+1} Q(s_{t+n+1}, a_{t+n+1}) - Q(s_t, a_t) \right]$$
and the Q-Learning update is
$$\delta_t = \alpha \left[ \sum_{i=t}^{t+n}{\gamma^{i-t} r_i} + \gamma^{n+1} \underset{a \in A_{t+n+1}}{\max} Q(s_{t+n+1}, a) - Q(s_t, a_t) \right]$$

Note that rewards will still be collected during the gap, but they are discounted based on the number of decisions they are removed from the initial RL operator.

Gap propagation can be disabled by setting the \soarb{temporal-extension} parameter of the \soarb{rl} command to \soarb{off}.
When gap propagation is disabled, the RL rules preceding a gap are updated using $Q(s_{t+1}, a_{t+1}) = 0$.
The \soarb{rl} setting of the \soarb{watch} command (see Section \ref{trace} on page \pageref{trace}) is useful in identifying gaps.


\subsection{RL and Substates}
\label{RL-substates}

When an agent has multiple states in its state stack, the RL mechanism will treat each substate independently.
As mentioned previously, each state has its own \soarb{reward-link}.
When an RL operator is selected in a state \soarb{S}, the RL updates for that operator are only affected by the rewards collected on the \soarb{reward-link} for \soarb{S} and the Q-values of subsequent RL operators selected in \soarb{S}.

The only exception to this independence is when a selected RL operator forces an operator-no-change impasse.
When this occurs, the number of decision cycles the RL operator at the superstate remains selected is dependent upon the processing in the impasse state.
Consider the operator trace in Figure \ref{fig:rl-optrace}.

\begin{itemize}
\item At decision cycle 1, RL operator \soarb{O1} is selected in \soarb{S1} and causes an operator-no-change impass for three decision cycles.
\item In the substate \soarb{S2}, operators \soarb{O2}, \soarb{O3}, and \soarb{O4} are selected and applied sequentially.
\item Meanwhile in \soarb{S1}, reward values $r_2$, $r_3$, and $r_4$ are put on the \soarb{reward-link} sequentially.
\item Finally, the impasse is resolved by \soarb{O4}, the proposal for \soarb{O1} is retracted, and RL operator \soarb{O5} is selected in \soarb{S1}.
\end{itemize}

\begin{figure}
\insertfigure{Figures/rl-optrace}{1.5in}
\insertcaption{Example Soar substate operator trace.}
\label{fig:rl-optrace}
\end{figure}

In this scenario, only the RL update for $Q(s_1, \soarb{O1})$ will be different from the ordinary case.
Its value depends on the setting of the \soarb{hrl-discount} parameter of the \soarb{rl} command.
When this parameter is set to the default value \soarb{on}, the rewards on \soarb{S1} and the Q-value of \soarb{O5} are discounted by the number of decision cycles they are removed from the selection of \soarb{O1}.
In this case the update for $Q(s_1, \soarb{O1})$ is
$$\delta_1 = \alpha \left[ r_2 + \gamma r_3 + \gamma^2 r_4 + \gamma^3 Q(s_5, \soarb{O5}) - Q(s_1, \soarb{O1}) \right]$$
which is equivalent to having a three decision gap separating \soarb{O1} and \soarb{O5}.

When \soarb{hrl-discount} is set to \soarb{off}, the number of cycles \soarb{O1} has been impassed will be ignored.
Thus the update would be
$$\delta_1 = \alpha \left[ r_2 + r_3 + r_4 + \gamma Q(s_5, \soarb{O5}) - Q(s_1, \soarb{O1}) \right]$$

For impasses other than operator no-change, RL acts as if the impasse hadn't occurred.
If \soarb{O1} is the last RL operator selected before the impasse, $r_2$ the reward received in the decision cycle immediately following, and \soarb{O}$_\soarb{n}$, the first operator selected after the impasse, then \soarb{O1} is updated with 
$$\delta_1 = \alpha \left[ r_2 + \gamma Q(s_n, \soarb{O}_\soarb{n}) - Q(s_1, \soarb{O1}) \right]$$

If an RL operator is selected in a substate immediately prior to the state's retraction, the RL rules will be updated based only on the reward signals present and not on the Q-values of future operators.
This point is not covered in traditional RL theory.
The retraction of a substate corresponds to a suspension of the RL task in that state rather than its termination, so the last update assumes the lack of information about future rewards rather than the discontinuation of future rewards.
To handle this case, the numeric indifferent preference value of each RL rule is stored as two separate values, the expected current reward (ECR) and expected future reward (EFR).
The ECR is an estimate of the expected immediate reward signal for executing the corresponding RL operator.
The EFR is an estimate of the time discounted Q-value of the next RL operator.
Normal updates correspond to traditional RL theory (showing the Sarsa case for simplicity):
\begin{align*}
\delta_{ECR} &= \alpha \left[ r_t - ECR(s_t, a_t) \right] \\
\delta_{EFR} &= \alpha \left[ \gamma Q(s_{t+1}, a_{t+1}) - EFR(s_t, a_t) \right] \\
\delta_t &= \delta_{ECR} + \delta_{EFR} \\
&= \alpha \left[ r_t + \gamma Q(s_{t+1}, a_{t+1}) - \left( ECR(s_t, a_t) + EFR(s_t, a_t) \right) \right] \\
&= \alpha \left[ r_t + \gamma Q(s_{t+1}, a_{t+1}) - Q(s_t, a_t) \right]
\end{align*}
During substate retraction, only the ECR is updated based on the reward signals present at the time of retraction, and the EFR is unchanged.

Soar's automatic subgoaling and RL mechanisms can be combined to naturally implement hierarchical reinforcement learning algorithms such as MAXQ and options.

\subsection{Eligibility Traces}
\label{RL-et}
The RL mechanism supports eligibility traces, which can improve the speed of learning by 
updating RL rules across multiple sequential steps. \\
The \soarb{eligibility-trace-decay-rate} and \soarb{eligibility-trace-tolerance} parameters control this mechanism.
By setting \soarb{eligibility-trace-decay-rate} to \soarb{0} (default), eligibility traces are in effect disabled.
When eligibility traces are enabled, the particular algorithm used is dependent upon the learning policy.
For Sarsa, the eligibility trace implementation is \emph{Sarsa($\lambda$)}. 
For Q-Learning, the eligibility trace implementation is \emph{Watkin's Q($\lambda$)}.

\subsubsection{Exploration}

The \soarb{indifferent-selection} command (page \pageref{decide-indifferent-selection}) determines how operators are selected based on their numeric indifferent preferences.
Although all the indifferent selection settings are valid regardless of how the numeric indifferent preferences were arrived at, the \soarb{epsilon-greedy} and \soarb{boltzmann} settings are specifically designed for use with RL and correspond to the two most common exploration strategies.
In an effort to maintain backwards compatibility, the default exploration policy is \soarb{softmax}.
As a result, one should change to \soarb{epsilon-greedy} or \soarb{boltzmann} when the reinforcement learning mechanism is enabled.

\subsection{GQ($\lambda$)}

\emph{Sarsa($\lambda$)} and \emph{Watkin's Q($\lambda$)} help agents to solve the temporal credit assignment problem more quickly.
However, if you wish to implement something akin to CMACs to generalize from experience, convergence is not guaranteed by these algorithms.
\emph(GQ($\lambda$)} is a gradient descent algorithm designed to ensure convergence when learning off-policy.
Soar provides both \soarb{on-policy-gq-lambda} and \soarb{off-policy-gq-lambda} to increase the likelihood of convergence when learning under these conditions.
If you should choose to use one of these algorithms, we recommend setting \soarb{step-size-parameter} to something small, such as $0.01$
in order to ensure that the secondary set of weights used by \emph(GQ($\lambda$)} change slowly enough for efficient convergence.

\section{Automatic Generation of RL Rules}

The number of RL rules required for an agent to accurately approximate operator Q-values is usually infeasibly large to write by hand, even for small domains.
Therefore, several methods exist to automate this.

\subsection{The gp Command}
The \soar{gp} command can be used to generate productions based on simple patterns.
This is useful if the states and operators of the environment can be distinguished by a fixed number of dimensions with finite domains.
An example is a grid world where the states are described by integer row/column coordinates, and the available operators are to move north, south, east, or west.
In this case, a single \soar{gp} command will generate all necessary RL rules:
	
\begin{verbatim}
gp {gen*rl*rules
   (state <s> ^name gridworld
              ^operator <o> +
              ^row [ 1 2 3 4 ]
              ^col [ 1 2 3 4 ])
   (<o> ^name move
        ^direction [ north south east west ])
-->
   (<s> ^operator <o> = 0.0)
}
\end{verbatim}
	
For more information see the documentation for this command on page \pageref{gp}.

\subsection{Rule Templates}
\label{RL-templates}

Rule templates allow Soar to dynamically generate new RL rules based on a predefined pattern as the agent encounters novel states.
This is useful when either the domains of environment dimensions are not known ahead of time, or when the enumerable state space of the environment is too large to capture in its entirety using \soar{gp}, but the agent will only encounter a small fraction of that space during its execution.
For example, consider the grid world example with 1000 rows and columns.
Attempting to generate RL rules for each grid cell and action a priori will result in $1000 \times 1000 \times 4 = 4 \times 10^6$ productions.
However, if most of those cells are unreachable due to walls, then the agent will never fire or update most of those productions.
Templates give the programmer the convenience of the \soar{gp} command without filling production memory with unnecessary rules.

Rule templates have variables that are filled in to generate RL rules as the agent encounters novel combinations of variable values.
A rule template is valid if and only if it is marked with the \soarb{:template} flag and, in all other respects, adheres to the format of an RL rule.
However, whereas an RL rule may only use constants as the numeric indifference preference value, a rule template may use a variable.
Consider the following rule template:

\begin{verbatim}
sp {sample*rule*template
    :template
    (state <s> ^operator <o> +
               ^value <v>)
-->
    (<s> ^operator <o> = <v>)
}
\end{verbatim}

During agent execution, this rule template will match working memory and create new productions by substituting all variables in the rule template that matched against constant values with the values themselves.
Suppose that the LHS of the rule template matched against the state

\begin{verbatim}
(S1 ^value 3.2)
(S1 ^operator O1 +)
\end{verbatim}

Then the following production will be added to production memory:

\begin{verbatim}
sp {rl*sample*rule*template*1
    (state <s> ^operator <o> +
               ^value 3.2)
-->
    (<s> ^operator <o> = 3.2)
}
\end{verbatim}

The variable \soar{<v>} is replaced by \soar{3.2} on both the LHS and the RHS, but \soar{<s>} and \soar{<o>} are not replaced because they matches against identifiers (\soar{S1} and \soar{O1}).
As with other RL rules, the value of \soar{3.2} on the RHS of this rule may be updated later by reinforcement learning, whereas the value of \soar{3.2} on the LHS will remain unchanged.
If \soar{<v>} had matched against a non-numeric constant, it will be replaced by that constant on the LHS, but the RHS numeric indifference preference value will be set to zero to make the new rule valid.

The new production's name adheres to the following pattern:
\soarb{rl*template-name*id}, where \soarb{template-name} is the name of the originating rule template and \soarb{id} is monotonically increasing integer that guarantees the uniqueness of the name.

If an identical production already exists in production memory, then the newly generate production is discarded.
It should be noted that the current process of identifying unique template match instances can become quite expensive in long agent runs.
Therefore, it is recommended to generate all necessary RL rules using the \soar{gp} command or via custom scripting when possible.

\subsection{Chunking}
Since RL rules are regular productions, they can be learned by chunking just like any other production.
This method is more general than using the \soar{gp} command or rule templates, and is useful if the environment state consists of arbitrarily complex relational structures that cannot be enumerated.

\chapter{Semantic Memory}
\label{SMEM}
\index{semantic memory}
\index{smem}

Soar's semantic memory is a repository for long-term declarative knowledge, supplementing what is contained in short-term working memory (and production memory). 
Episodic memory, which contains memories of the agent's experiences, is described in Chapter \ref{EPMEM}. 
The knowledge encoded in episodic memory is organized temporally, and specific information is embedded within the context of when it was experienced, whereas knowledge in semantic memory is independent of any specific context, representing more general facts about the world.

This chapter is organized as follows: semantic memory structures in working memory (\ref{SMEM-wm}); representation of knowledge in semantic memory (\ref{SMEM-kr}); storing semantic knowledge (\ref{SMEM-store}); retrieving semantic knowledge (\ref{SMEM-retrieve}); and a discussion of performance (\ref{SMEM-perf}). 
The detailed behavior of semantic memory is determined by numerous parameters that can be controlled and configured via the \soarb{smem} command. 
Please refer to the documentation for that command in Section \ref{smem} on page \pageref{smem}.


\section{Working Memory Structure}
\label{SMEM-wm}

Upon creation of a new state in working memory (see Section \ref{ARCH-impasses-types} on page \pageref{ARCH-impasses-types}; Section \ref{SYNTAX-impasses} on page \pageref{SYNTAX-impasses}), the architecture creates the following augmentations to facilitate agent interaction with semantic memory:

\begin{verbatim}
(<s> ^smem <smem>)
  (<smem> ^command <smem-c>)
  (<smem> ^result <smem-r>)
\end{verbatim}

As rules augment the \emph{command} structure in order to access/change semantic knowledge (\ref{SMEM-store}, \ref{SMEM-retrieve}), semantic memory augments the \emph{result} structure in response.
Production actions should not remove augmentations of the \emph{result} structure directly, as semantic memory will maintain these WMEs.



\section{Knowledge Representation}
\label{SMEM-kr}

The representation of knowledge in semantic memory is similar to that in working memory (see Section \ref{ARCH-wm} on page \pageref{ARCH-wm}) -- both include graph structures that are composed of symbolic elements consisting of an identifier, an attribute, and a value. 
It is important to note, however, key differences:

\begin{itemize}

\item 
Currently semantic memory only supports attributes that are symbolic constants (string, integer, or decimal), but \emph{not} attributes that are identifiers

\item 
Whereas working memory is a single, connected, directed graph, semantic memory can be disconnected, consisting of multiple directed, connected sub-graphs

\end{itemize}

\emph{Long-term} identifiers (LTIs) are defined as identifiers that exist in semantic memory.
The specific letter-number combination that labels an LTI (e.g. S5 or C7) is permanently associated with that long-term identifier: any retrievals of the long-term identifier are guaranteed to return the associated letter-number pair.  
For clarity, when printed, a long-term identifier is prefaced with the {@} symbol (e.g. {@}S5 or {@}C7). 
Also, when presented in a figure, long-term identifiers will be indicated by a double-circle. 
For instance, Figure \ref{fig:smem-concept} depicts the long-term identifier {@}A68, with four augmentations, representing the addition fact of ${6+7=13}$ (or, rather, 3, carry 1, in context of multi-column arithmetic).

\begin{figure}
\insertfigure{Figures/smem-concept}{1.5in}
\insertcaption{Example long-term identifier with four augmentations.}
\label{fig:smem-concept}
\end{figure}

\subsection{Integrating Long-Term Identifiers with Soar}
Integrating long-term identifiers in Soar presents a number of theoretical and implementation challenges.  
This section discusses the state of integration with each of Soar's memories/learning mechanisms.

\subsubsection{Working Memory}
Long-term identifiers exist as peers with short-term identifiers in Working Memory.

\subsubsection{Procedural Memory}
Soar's production parser (i.e. the \soarb{sp} command) has been modified to allow specification of long-term identifiers (prefaced with an {@} symbol) in any context where a variable is valid.
If a rule contains a long-term identifier that is not currently in semantic memory, a fatal error will be raised and Soar will quit.  
Once added to the rete, the long-term identifier is treated as a constant for matching purposes.  
If specified as the value of a WME in an action, a long-term identifier will be added to working memory if it does not already exist.  
There is also preliminary support for chunking over long-term identifiers.

It is currently possible to create production actions wherein the identifier of a new WME is a long-term identifier that exists neither in the production conditions, nor as the attribute or value of a prior action.  
Such rules will wreak havoc within Soar and are not supported.  
They will be detected and disallowed in future versions of semantic memory.

\subsubsection{Episodic Memory}
Episodic memory (see Section \ref{EPMEM} on page \pageref{EPMEM}) faithfully captures short- vs. long-term identifiers, including the episode of transition.  
Cues are handled in much the same way as cue-based retrievals, with respect to the differences in semantics of a short- vs. long-term identifier.

\section{Storing Semantic Knowledge}
\label{SMEM-store}

An agent stores a long-term identifier to semantic memory by creating a \emph{store} command: this is a WME whose identifier is the \emph{command} link of a state's \emph{smem} structure, the attribute is \emph{store}, and the value is an identifier (short or long).

\begin{verbatim}
<s> ^smem.command.store <identifier>
\end{verbatim}

Semantic memory will encode and store all WMEs whose identifier is the value of the store command.  
Storing deeper levels of working memory is achieved through multiple store commands.

Multiple store commands can be issued in parallel.  
Storage commands are processed on every state at the end of every phase of every decision cycle.  
Storage is guaranteed to succeed and a status WME will be created, where the identifier is the \emph{result} link of the \emph{smem} structure of that state, the attribute is \emph{success}, and the value is the value of the store command above.

\begin{verbatim}
<s> ^smem.result.success <identifier>
\end{verbatim}

Any short-term identifiers that compose the stored WMEs will be converted to long-term identifiers. 
If a long-term identifier is the value of a store command, the stored WMEs replace those associated with the LTI in semantic memory. 
It should be noted that between issuing store commands, it is possible that the augmentations of a long-term identifier in working memory are inconsistent with those in semantic memory.

\subsection{User-Initiated Storage}
Semantic memory provides agent designers the ability to store semantic knowledge via the \soarb{add} switch of the \soarb{smem} command (see Section \ref{smem} on page \pageref{smem}).  
The format of the command is nearly identical to the working memory manipulation components of the RHS of a production (i.e. no RHS-functions; see Section \ref{SYNTAX-pm-action} on page \pageref{SYNTAX-pm-action}).  
For instance:

\begin{verbatim}
smem --add {
   (<arithmetic> ^add10-facts <a01> <a02> <a03>)
   (<a01> ^digit1 1 ^digit-10 11)
   (<a02> ^digit1 2 ^digit-10 12)
   (<a03> ^digit1 3 ^digit-10 13)
}
\end{verbatim}

Unlike agent storage, declarative storage is automatically recursive.  
Thus, this command instance will add a new long-term identifier (represented by the temporary 'arithmetic' variable) with three augmentations.  
The value of each augmentation will each become an LTI with two constant attribute/value pairs.  
Manual storage can be arbitrarily complex and use standard dot-notation.

\subsection{Storage Location}
Semantic memory uses SQLite to facilitate efficient and standardized storage and querying of knowledge.  
The semantic store can be maintained in memory or on disk (per the \soarb{database} and \soarb{path} parameters). 
If the store is located on disk, users can use any standard SQLite programs/components to access/query its contents.
However, using a disk-based semantic store is very costly (performance is discussed in greater detail in Section \ref{SMEM-perf} on page \pageref{SMEM-perf}), and running in memory is recommended for most runs.

The \soarb{lazy-commit} parameter is a performance optimization. 
If set to \soarb{on} (default), disk databases will not reflect semantic memory changes until the Soar kernel shuts down. 
This improves performance by avoiding disk writes. 
The \soarb{optimization} parameter (see Section \ref{SMEM-perf} on page \pageref{SMEM-perf}) will have an affect on whether databases on disk can be opened while the Soar kernel is running.


\section{Retrieving Semantic Knowledge}
\label{SMEM-retrieve}

An agent retrieves knowledge from semantic memory by creating an appropriate command (we detail the types of commands below) on the \emph{command} link of a state's \emph{smem} structure. 
At the end of the output of each decision, semantic memory processes each state's \emph{smem} command structure.  
Results, meta-data, and errors are added to the \emph{result} structure of that state's \emph{smem} structure.

Only one type of retrieval command (which may include optional modifiers) can be issued per state in a single decision cycle.  
Malformed commands (including attempts at multiple retrieval types) will result in an error:

\begin{verbatim}
<s> ^smem.result.bad-cmd <smem-c>
\end{verbatim}

Where the \soarb{smem-c} variable refers to the \emph{command} structure of the state.

After a command has been processed, semantic memory will ignore it until some aspect of the command structure changes (via addition/removal of WMEs).  
When this occurs, the result structure is cleared and the new command (if one exists) is processed.

\subsection{Non-Cue-Based Retrievals}
A non-cue-based retrieval is a request by the agent to reflect in working memory the current augmentations of a long-term identifier in semantic memory. 
The command WME has a \emph{retrieve} attribute and a long-term identifier value:

\begin{verbatim}
<s> ^smem.command.retrieve <lti>
\end{verbatim}

If the value of the command is not a long-term identifier, an error will result: 

\begin{verbatim}
<s> ^smem.result.failure <lti>
\end{verbatim}

Otherwise, two new WMEs will be placed on the result structure:

\begin{verbatim}
<s> ^smem.result.success <lti>
<s> ^smem.result.retrieved <lti>
\end{verbatim}

All augmentations of the long-term identifier in semantic memory will be created as new WMEs in working memory.

\subsection{Cue-Based Retrievals}
A cue-based retrieval performs a search for a long-term identifier in semantic memory whose augmentations exactly match an agent-supplied cue, as well as optional cue modifiers.

A cue is composed of WMEs that describe the augmentations of a long-term identifier.  
A cue WME with a constant value denotes an exact match of both attribute and value.  
A cue WME with a long-term identifier as its value denotes an exact match as well.  
A cue WME with a short-term identifier as its value denotes an exact match of attribute, but with any value (constant or identifier).  

A cue-based retrieval command has a \emph{query} attribute and an identifier value, the cue:

\begin{verbatim}
<s> ^smem.command.query <cue>
\end{verbatim}

For instance, consider the following rule that creates a cue-based retrieval command:

\begin{verbatim}
sp {smem*sample*query
    (state <s> ^smem.command <sc>
               ^lti <lti>
               ^input-link.foo <bar>)
-->
    (<sc> ^query <q>)
    (<q> ^name <any-name>
         ^foo <bar>
         ^associate <lti>
         ^age 25)
}
\end{verbatim}

In this example, assume that the \soar{<lti>} variable will match a long-term identifier and the \soar{<bar>} variable will match a constant.  
Thus, the query requests retrieval of a long-term identifier from semantic memory with augmentations that satisfy ALL of the following requirements:

\begin{itemize}

\item 
Attribute \soar{name} and ANY value

\item 
Attribute \soar{foo} and value equal to the value of variable \soar{<bar>} at the time this rule fires

\item 
Attribute \soar{associate} and value equal to the long-term identifier \soar{<lti>} at the time this rule fires

\item 
Attribute \soar{age} and integer value \soar{25}

\end{itemize}

If no long-term identifier satisfies ALL of these requirements, an error is returned:

\begin{verbatim}
<s> ^smem.result.failure <cue>
\end{verbatim}

Otherwise, two WMEs are added:

\begin{verbatim}
<s> ^smem.result.success <cue>
<s> ^smem.result.retrieved <retrieved-lti>
\end{verbatim}

During a cue-based retrieval it is possible that the retrieved long-term identifier is not in working memory.  
If this is the case, semantic memory will add the long-term identifier to working memory with letter-number pair as was originally stored.

As with non-cue-based retrievals all of the augmentations of the long-term identifier in semantic memory are added as new WMEs to working memory.

It is possible that multiple long-term identifiers match the cue equally well. 
In this case, semantic memory will retrieve the long-term identifier that was most recently stored/retrieved.

The cue-based retrieval process can be further tempered using optional modifiers:

\begin{itemize}

\item 
The \emph{prohibit} command requires that the retrieved long-term identifier is not equal to a supplied long-term identifier:
\begin{verbatim}
<s> ^smem.command.prohibit <bad-lti>
\end{verbatim}
Multiple prohibit command WMEs may be issued as modifiers to a single cue-based retrieval.  
This method can be used to iterate over all matching long-term identifiers.

\item 
The \emph{neg-query} command requires that the retrieved long-term identifier does NOT contain a set of attributes/attribute-value pairs:
\begin{verbatim}
<s> ^smem.command.neg-query <cue>
\end{verbatim}
The syntax of this command is identical to that of regular/positive \emph{query} command.

\item
The \emph{math-query} command requires that the retrieved long term identifier contains an attribute value pair that meets a specified mathematical condition. 
This condition can either be a conditional query or a superlative query. 
Conditional queries are of the format:
\begin{verbatim}
<s> ^smem.command.math-query.<cue-attribute>.<condition-name> <cue-value>
\end{verbatim}
Superlative queries do not use a value argument and are of the format:
\begin{verbatim}
<s> ^smem.command.math-query.<cue-attribute>.<condition-name>
\end{verbatim}
Values used in math queries must be integer or float type values.
Currently supported condition names are:
\begin{description}
  \item[less] A value less than the given argument
  \item[greater] A value greater than the given argument
  \item[less-or-equal] A value less than or equal to the given argument
  \item[greater-or-equal] A value greater than or equal to the given argument
  \item[max] The maximum value for the attribute
  \item[min] The minimum value for the attribute
\end{description}
\end{itemize}

\section{Performance}
\label{SMEM-perf}

Initial empirical results with toy agents show that semantic memory queries carry up to a 40\% overhead as compared to comparable rete matching.  
However, the retrieval mechanism implements some basic query optimization: statistics are maintained about all stored knowledge.  
When a query is issued, semantic memory re-orders the cue such as to minimize expected query time.  
Because only perfect matches are acceptable, and there is no symbol variablization, semantic memory retrievals do not contend with the same combinatorial search space as the rete.  
Preliminary empirical study shows that semantic memory maintains sub-millisecond retrieval time for a large class of queries, even in very large stores (millions of nodes/edges).

Once the number of long-term identifiers overcomes initial overhead (about 1000 WMEs), initial empirical study shows that semantic storage requires far less than 1KB per stored WME.

\subsection{Math queries}
There are some additional performance considerations when using math queries during retrieval.
Initial testing indicates that conditional queries show the same time growth with respect to the number of memories as similar non-math restricted queries, however the actual time for retrieval may be slightly longer.
Superelative queries will often show a worse result than similar non-superelative queries, because the current implementation of semantic memory requires them to iterate over any memory that matches all other involved cues.

\subsection{Performance Tweaking}

When using a database stored to disk, several parameters become crucial to performance.  
The first is \soarb{lazy-commit}, which controls when database changes are written to disk.   
The default setting (\soarb{on}) will keep all writes in memory and only commit to disk upon re-initialization (quitting the agent or issuing the \soarb{init} command).  
The \soarb{off} setting will write each change to disk and thus incurs massive I/O delay.

The next parameter is \soarb{thresh}. 
This has to do with the locality of storing/updating activation information with semantic augmentations. 
By default, all WME augmentations are incrementally sorted by activation, such that cue-based retrievals need not sort large number of candidate long-term identifiers on demand, and thus retrieval time is independent of cue selectivity. 
However, each activation update (such as after a retrieval) incurs an update cost linear in the number of augmentations. 
If the number of augmentations for a long-term identifier is large, this cost can dominate. 
Thus, the \soarb{thresh} parameter sets the upper bound of augmentations, after which activation is stored with the long-term identifier. 
This allows the user to establish a balance between cost of updating augmentation activation and the number of long-term identifiers that must be pre-sorted during a cue-based retrieval. 
As long as the threshold is greater than the number of augmentations of most long-term identifiers, performance should be fine (as it will bound the effects of selectivity).

The next two parameters deal with the SQLite cache, which is a memory store used to speed operations like queries by keeping in memory structures like levels of index B+-trees. 
The first parameter, \soarb{page-size}, indicates the size, in bytes, of each cache page. 
The second parameter, \soarb{cache-size}, suggests to SQLite how many pages are available for the cache. 
Total cache size is the product of these two parameter settings. 
The cache memory is not pre-allocated, so short/small runs will not necessarily make use of this space. 
Generally speaking, a greater number of cache pages will benefit query time, as SQLite can keep necessary meta-data in memory. 
However, some documented situations have shown improved performance from decreasing cache pages to increase memory locality. 
This is of greater concern when dealing with file-based databases, versus in-memory. 
The size of each page, however, may be important whether databases are disk- or memory-based. 
This setting can have far-reaching consequences, such as index B+-tree depth. 
While this setting can be dependent upon a particular situation, a good heuristic is that short, simple runs should use small values of the page size (\soarb{1k}, \soarb{2k}, \soarb{4k}), whereas longer, more complicated runs will benefit from larger values (\soarb{8k}, \soarb{16k}, \soarb{32k}, \soarb{64k}). 
The episodic memory chapter (see Section \ref{EPMEM-perf} on page \pageref{EPMEM-perf}) has some further empirical evidence to assist in setting these parameters for very large stores.

The next parameter is \soarb{optimization}.  
The \soarb{safety} parameter setting will use SQLite default settings.  
If data integrity is of importance, this setting is ideal.  
The \soarb{performance} setting will make use of lesser data consistency guarantees for significantly greater performance.  
First, writes are no longer synchronous with the OS (synchronous pragma), thus semantic memory won't wait for writes to complete before continuing execution.  
Second, transaction journaling is turned off (journal\_mode pragma), thus groups of modifications to the semantic store are not atomic (and thus interruptions due to application/os/hardware failure could lead to inconsistent database state).  
Finally, upon initialization, semantic memory maintains a continuous exclusive lock to the database (locking\_mode pragma), thus other applications/agents cannot make simultaneous read/write calls to the database (thereby reducing the need for potentially expensive system calls to secure/release file locks).

Finally, maintaining accurate operation timers can be relatively expensive in Soar.  
Thus, these should be enabled with caution and understanding of their limitations.  
First, they will affect performance, depending on the level (set via the \soarb{timers} parameter).  
A level of \soarb{three}, for instance, times every modification to long-term identifier recency statistics.  
Furthermore, because these iterations are relatively cheap (typically a single step in the linked-list of a b+-tree), timer values are typically unreliable (depending upon the system, resolution is 1 microsecond or more).


\chapter{Spatial Visual System}
\label{SVS}
\index{Spatial Visual System}
\index{SVS}
\index{svs}

\begin{figure}
\insertfigure{Figures/svs-setup}{4in}
\insertcaption{(a) Typical environment setup without using SVS. (b) Same environment using SVS.}
\label{fig:svs-setup}
\end{figure}

The Spatial Visual System (SVS) allows Soar to effectively represent and reason about continuous, three dimensional environments.
SVS maintains an internal representation of the environment as a collection of discrete objects with simple geometric shapes, called the scene graph.
The Soar agent can query for spatial relationships between the objects in the scene graph through a working memory interface similar to that of episodic and semantic memory.
Figure \ref{fig:svs-setup} illustrates the typical use case for SVS by contrasting it with an agent that does not utilize it.
The agent that does not use SVS (a. in the figure) relies on the environment to provide a symblic representation of the continuous state.
On the other hand, the agent that uses SVS (b) accepts a continuous representation of the environment state directly, and then performs queries on the scene graph to extract a symbolic representation internally.
This allows the agent to build more flexible symbolic representations without requiring modifications to the environment code.
Furthermore, it allows the agent to manipulate internal copies of the scene graph and then extract spatial relationships from the modified states, which is useful for look-ahead search and action modeling.
This type of imagery operation naturally captures and propogates the relationships implicit in spatial environments, and doesn't suffer from the frame problem that relational representations have.

\section{The scene graph}

The primary data structure of SVS is the \emph{scene graph}.
The scene graph is a tree in which the nodes represent objects in the scene and the edges represent ``part-of'' relationships between objects.
An example scene graph consisting of a car and a pole is shown in Figure \ref{fig:scene-graph}.
The scene graph's leaves are \emph{geometry nodes} and its interior nodes are \emph{group nodes}.
Geometry nodes represent atomic objects that have intrinsic shape, such as the wheels and chassis in the example.
Currently, the shapes supported by SVS are points, lines, convex polyhedrons, and spheres.
Group nodes represent objects that are the aggregates of their child nodes, such as the car object in the example.
The shape of a group node is the union of the shapes of its children.
Structuring complex objects in this way allows Soar to reason about them naturally at different levels of abstraction.
The agent can query SVS for relationships between the car as a whole with other objects (e.g. does it intersect the pole?), or the relationships between its parts (e.g. are the wheels pointing left or right with respect to the chassis?).
The scene graph always contains at least a root node: the \emph{world node}.

\begin{figure}
\insertfigure{Figures/scene_graph}{5in}
\insertcaption{(a) A 3D scene. (b) The scene graph representation.}
\label{fig:scene-graph}
\end{figure}

Each node other than the world node has a transform with respect to its parent.
A transform consists of three components:

\begin{description}
\item[position $(x,y,z)$]
Specifies the $x$, $y$, and $z$ offsets of the node's origin with respect to its parent's origin.

\item[rotation $(x,y,z)$]
Specifies the rotation of the node relative to its origin in Euler angles.
This means that the node is rotated the specified number of radians along each axis in the order $x-y-z$.
For more information, see \url{http://en.wikipedia.org/wiki/Euler_angles}.

\item[scaling $(x,y,z)$]
Specifies the factors by which the node is scaled along each axis.

\end{description}

The component transforms are applied in the order scaling, then rotation, then position.
Each node's transform is applied with respect to its parent's coordinate system, so the transforms accumulate down the tree.
A node's transform with respect to the world node, or its world transform, is the aggregate of all its ancestor transforms.
For example, if the car has a position transform of $(1,0,0)$ and a wheel on the car has a position transform of $(0,1,0)$, then the world position transform of the wheel is $(1,1,0)$.

SVS represents the scene graph structure in working memory under the \soarb{\^{}spatial-scene} link.
The working memory representation of the car and pole scene graph is

\begin{verbatim}
(S1 ^svs S3)
  (S3 ^command C3 ^spatial-scene S4)
    (S4 ^child C10 ^child C4 ^id world)
      (C10 ^id pole)
      (C4 ^child C9 ^child C8 ^child C7 ^child C6 ^child C5 ^id car)
        (C9 ^id chassis)
        (C8 ^id wheel3)
        (C7 ^id wheel2)
        (C6 ^id wheel1)
        (C5 ^id wheel0)
\end{verbatim}

Each state in working memory has its own scene graph.
When a new state is created, it will receive an independent copy of its parent's scene graph.
This is useful for performing look-ahead search, as it allows the agent to destructively modify the scene graph in a search state using mental imagery operations.

\subsection{svs\_viewer}

A viewer has been provided to show the scene graph visually. 
Run the program \texttt{svs\_viewer -s PORT} from the soar/out folder 
to launch the viewer listening on the given port. Once the viewer is running, 
from within soar use the command \texttt{svs connect\_viewer PORT} to connect 
to the viewer and begin drawing the scene graph. Any changes to the scene graph
will be reflected in the viewer. The viewer by default draws the topstate scene graph, 
to draw that on a substate first stop drawing the topstate with 
\texttt{svs S1.scene.draw off} and then \texttt{svs S7.scene.draw on}. 

\section{Scene Graph Edit Language}

The Scene Graph Edit Language (SGEL) is a simple, plain text, line oriented language that is used by SVS to modify the contents of the scene graph.
Typically, the scene graph is used to represent the state of the external environment, and the programmer sends SGEL commands reflecting changes in the environment to SVS via the Agent::SendSVSInput function in the SML API.
These commands are buffered by the agent and processed at the beginning of each input phase.
Therefore, it is common to send scene changes through SendSVSInput \emph{before} the input phase.
If you send SGEL commands at the end of the input phase, 
the results won't be processed until the following decison cycle.

Each SGEL command begins with a single word command type and ends with a newline.
The four command types are

\begin{description}
\item[\texttt{add ID PARENT\_ID [GEOMETRY] [TRANSFORM]}] \hfill \\
Add a node to the scene graph with the given \texttt{ID}, as a child of \texttt{PARENT\_ID}, 
and with type \texttt{TYPE} (usually object).
The \texttt{GEOMETRY} and \texttt{TRANSFORM} arguments are optional and described below.

\item[\texttt{change ID [GEOMETRY] [TRANSFORM]}] \hfill \\
Change the transform and/or geometry of the node with the given \texttt{ID}.

\item[\texttt{delete ID}] \hfill \\
  Delete the node with the given \texttt{ID}.

\item[\texttt{tag [add|change|delete] ID TAG\_NAME TAG\_VALUE}] \hfill \\
  Adds, changes, or deletes a tag from an object.
  A tag consists of a \texttt{TAG\_NAME}  
  and \texttt{TAG\_VALUE} pair and is added to the node with the given \texttt{ID}.
  Both \texttt{TAG\_NAME} and \texttt{TAG\_VALUE} must be strings.
  Tags can differentiate nodes (e.g. as a type field) and can be used in conjunction with 
  the \texttt{tag\_select} filter to choose a subset of the nodes. 

\end{description}

The \texttt{TRANSFORM} argument has the form \texttt{[p X Y Z] [r X Y Z] [s X Y Z]}, corresponding to the position, rotation, and scaling components of the transform, respectively.
All the components are optional; any combination of them can be excluded, and the included components can appear in any order.

The \texttt{GEOMETRY} argument has two forms:

\begin{description}

\item[\texttt{b RADIUS}] \hfill \\
Make the node a geometry node with sphere shape with radius \texttt{RADIUS}.

\item[\texttt{v X1 Y1 Z1 X2 Y2 Z2 ...}] \hfill \\
Make the node a geometry node with a convex polyhedron shape with the specified vertices.
Any number of vertices can be listed.

\end{description}

\subsection{Examples}

Creating a sphere called ball4 with radius 5 at location (4, 4, 0). \\
\texttt{add ball4 world b 5 p 4 4 0}

Creating a triangle in the xy plane, then rotate it vertically, double its size, and move it to (1, 1, 1).  \\
\texttt{add tri9 world v -1 -1 0 1 -1 0 0 0.5 0 p 1 1 1 r 1.507 0 0 s 2 2 2}

Creating a snowman shape of 3 spheres stacked on each other and located at (2, 2, 0). \\
\texttt{add snowman world p 2 2 0} \\
\texttt{add bottomball snowman b 3 p 0 0 3} \\
\texttt{add middleball snowman b 2 p 0 0 8} \\
\texttt{add topball snowman b 1 p 0 0 11} 

Set the rotation transform on box11 to 180 degrees around the z axis. \\
\texttt{change box11 r 0 0 3.14159}

Changing the color tag on box7 to green. \\
\texttt{tag change box7 color green}


\section{Commands}

The Soar agent initiates commands in SVS via the \soarb{\^{}command} link, 
similar to semantic and episodic memory. These commands allow the agent to 
modify the scene graph and extract filters. 
Commands are processed during the output phase and the results are added to 
working memory during the input phase. 
SVS supports the following commands:

\begin{description}
  \item{\textbf{add\_node}}
  Creates a new node and adds it to the scene graph
\item{\textbf{copy\_node}}
  Creates a copy of an existing node
\item{\textbf{delete\_node}}
  Removes a node from the scene graph and deletes it
\item{\textbf{set\_transform}}
  Changes the position, rotation, and/or scale of a node
\item{\textbf{set\_tag}}
  Adds or changes a tag on a node
\item{\textbf{delete\_tag}}
  Deletes a tag from a node
\item{\textbf{extract}}
	Compute the truth value of spatial relationships in the current scene graph.
\item{\textbf{extract\_once}}
  Same as extract, except it is only computed once and doesn't update when the scene changes.
\end{description}

\subsection{add\_node}

This commands adds a new node to the scene graph. 
\begin{description}
  \item{\soarb{\^{}id [string]}} The id of the node to create
  \item{\soarb{\^{}parent [string]}} The id of the node to attach the new node to (default is world)
  \item{\soarb{\^{}geometry << group point ball box >> }} The geometry the node should have 
  \item{\soarb{\^{}position.\{\^{}x \^{}y \^{}z\} }} Position of the node (optional)
  \item{\soarb{\^{}rotation.\{\^{}x \^{}y \^{}z\} }} Rotation of the node (optional)
  \item{\soarb{\^{}scale.\{\^{}x \^{}y \^{}z\} }} Scale of the node (optional)
\end{description}

The following example creates a node called \texttt{box5} and adds it to the world. 
The node has a box shape of side length 0.1 and is placed at position (1, 1, 0). 
\begin{verbatim}
(S1 ^svs S3)
  (S3 ^command C3 ^spatial-scene S4)
    (C3 ^add_node A1)
      (A1 ^id box5 ^parent world ^geometry box ^position P1 ^scale S6)
        (P1 ^x 1.0 ^y 1.0 ^z 0.0)
        (S6 ^x 0.1 ^y 0.1 ^z 0.1)
\end{verbatim}

\subsection{copy\_node}
This command creates a copy of an existing node and adds it to the scene graph. 
This copy is not recursive, it only copies the node itself, not its children. 
The position, rotation, and scale transforms are also copied from the source node
but they can be changed if desired. 
\begin{description}
  \item{\soarb{\^{}id [string]}} The id of the node to create
  \item{\soarb{\^{}source [string]}} The id of the node to copy
  \item{\soarb{\^{}parent [string]}} The id of the node to attach the new node to (default is world)
  \item{\soarb{\^{}position.\{\^{}x \^{}y \^{}z\} }} Position of the node (optional)
  \item{\soarb{\^{}rotation.\{\^{}x \^{}y \^{}z\} }} Rotation of the node (optional)
  \item{\soarb{\^{}scale.\{\^{}x \^{}y \^{}z\} }} Scale of the node (optional)
\end{description}

The following example copies a node called \texttt{box5} as new node \texttt{box6}
and moves it to position (2, 0, 2).
\begin{verbatim}
(S1 ^svs S3)
  (S3 ^command C3 ^spatial-scene S4)
    (C3 ^copy_node A1)
      (A1 ^id box6 ^source box5 ^position P1)
        (P1 ^x 2.0 ^y 0.0 ^z 2.0)
\end{verbatim}

\subsection{delete\_node}
This command deletes a node from the scene graph. Any children will also be deleted. 
\begin{description}
  \item{\soarb{\^{}id [string]}} The id of the node to delete
\end{description}

The following example deletes a node called \texttt{box5}
\begin{verbatim}
(S1 ^svs S3)
  (S3 ^command C3 ^spatial-scene S4)
    (C3 ^delete_node D1)
      (D1 ^id box5)
\end{verbatim}

\subsection{set\_transform}
This command allows you to change the position, rotation, and/or scale of an 
exisiting node. You can specify any combination of the three transforms. 
\begin{description}
  \item{\soarb{\^{}id [string]}} The id of the node to change
  \item{\soarb{\^{}position.\{\^{}x \^{}y \^{}z\} }} Position of the node (optional)
  \item{\soarb{\^{}rotation.\{\^{}x \^{}y \^{}z\} }} Rotation of the node (optional)
  \item{\soarb{\^{}scale.\{\^{}x \^{}y \^{}z\} }} Scale of the node (optional)
\end{description}

The following example moves and rotates a node called \texttt{box5}.
\begin{verbatim}
(S1 ^svs S3)
  (S3 ^command C3 ^spatial-scene S4)
    (C3 ^set_transform S6)
      (S6 ^id box5 ^position P1 ^rotation R1)
        (P1 ^x 2.0 ^y 2.0 ^z 0.0)
        (R1 ^x 0.0 ^y 0.0 ^z 1.57)
\end{verbatim}

\subsection{set\_tag}
This command allows you to add or change a tag on a node.
If a tag with the same id already exists, 
the existing value will be replaced with the new value.
\begin{description}
  \item{\soarb{\^{}id [string]}} The id of the node to set the tag on
  \item{\soarb{\^{}tag\_name [string]}} The name of the tag to add
  \item{\soarb{\^{}tag\_value [string]}} The value of the tag to add
\end{description}

The following example adds a shape tag to the node \texttt{box5}.
\begin{verbatim}
(S1 ^svs S3)
  (S3 ^command C3 ^spatial-scene S4)
    (C3 ^set_tag S6)
      (S6 ^id box5 ^tag_name shape ^tag_value cube)
\end{verbatim}

\subsection{delete\_tag}
This command allows you to delete a tag from a node.
\begin{description}
  \item{\soarb{\^{}id [string]}} The id of the node to delete the tag from
  \item{\soarb{\^{}tag\_name [string]}} The name of the tag to delete
\end{description}

The following example deletes the shape tag from the node \texttt{box5}.
\begin{verbatim}
(S1 ^svs S3)
  (S3 ^command C3 ^spatial-scene S4)
    (C3 ^delete_tag D1)
      (D1 ^name box5 ^tag_name shape)
\end{verbatim}

\subsection{extract and extract\_once}
This command is commonly used to compute spatial relationships in the scene graph.
More generally, it puts the result of a filter pipeline (described in section \ref{sec:svs-filters}) in working memory.
Its syntax is the same as filter pipeline syntax.
During the input phase, SVS will evaluate the filter and 
put a \soarb{\^{}result} attribute on the command's identifier.
Under the \soarb{\^{}result} attribute is a multi-valued \soarb{\^{}record} attribute.
Each record corresponds to an output value from the head of the filter pipeline, along with the parameters that produced the value.
With the regular \texttt{extract} command, these records will be updated as the scene graph
changes. With the \texttt{extract\_once} command, the records will be created once
and will not change. 
Note that you should not change the structure of a filter once it is created 
(SVS only processes a command once). 
Instead to extract something different you must create a new command. 
The following is an example of an extract command which tests whether the 
car and pole objects are intersecting. The \soar{\^{}status} and \soar{\^{}result} wmes are 
added by SVS when the command is finished. 

\begin{verbatim}
(S1 ^svs S3)
  (S3 ^command C3 ^spatial-scene S4)
    (C3 ^extract E2)
      (E2 ^a A1 ^b B1 ^result R7 ^status success ^type intersect)
        (A1 ^id car ^status success ^type node)
        (B1 ^id pole ^status success ^type node)
        (R7 ^record R17)
          (R17 ^params P1 ^value false)
            (P1 ^a car ^b pole)
\end{verbatim}

\section{Filters}
\label{sec:svs-filters}

Filters are the basic unit of computation in SVS.
They transform the continuous information in the scene graph into symbolic information that can be used by the rest of Soar.
Each filter accepts a number of labeled parameters as input, and produces a single output.
Filters can be arranged into pipelines in which the outputs of some filters are fed into the inputs of other filters.
The Soar agent creates filter pipelines by building an analogous structure in working memory as an argument to an "extract" command.
For example, the following structure defines a set of filters that reports whether the car intersects the pole:

\begin{verbatim}
(S1 ^svs S3)
  (S3 ^command C3 ^spatial-scene S4)
    (C3 ^extract E2)
      (E2 ^a A1 ^b B1 ^type intersect)
        (A1 ^id car ^type node)
        (B1 ^id pole ^type node)
\end{verbatim}

The \soarb{\^{}type} attribute specifies the type of filter to instantiate, and the other attributes specify parameters.
This command will create three filters: an \soarb{intersect} filter and two \soarb{node} filters.
A \soarb{node} filter take an \soarb{id} parameter and returns the scene graph node with that ID as its result.
Here, the outputs of the \soarb{car} and \soarb{pole} node filters are fed into the \soarb{\^{}a} and \soarb{\^{}b} parameters of the \soarb{intersect} filter.
SVS will update each filter's output once every decision cycle, at the end of the input phase.
The output of the \soarb{intersect} filter is a boolean value indicating whether the two objects are intersecting.
This is placed into working memory as the result of the extract command:

\begin{verbatim}
(S1 ^svs S3)
  (S3 ^command C3 ^spatial-scene S4)
    (C3 ^extract E2)
      (E2 ^a A1 ^b B1 ^result R7 ^status success ^type intersect)
        (A1 ^id car ^status success ^type node)
        (B1 ^id pole ^status success ^type node)
        (R7 ^record R17)
          (R17 ^params P1 ^value false)
            (P1 ^a car ^b pole)
\end{verbatim}

Notice that a \soarb{\^{}status} success is placed on each identifier corresponding to a filter.
A \soarb{\^{}result} WME is placed on the extract command with a single record with value \soarb{false}.

\subsection{Result lists}

Spatial queries often involve a large number of objects.
For example, the agent may want to compute whether an object intersects any others in the scene graph.
It would be inconvenient to build the extract command to process this query if the agent had to specify each object involved explicitly.
Too many WMEs would be required, which would slow down the production matcher as well as SVS because it must spend more time interpreting the command structure.
To handle these cases, all filter parameters and results can be lists of values.
For example, the query for whether one object intersects all others can be expressed as

\begin{verbatim}
(S1 ^svs S3)
  (S3 ^command C3)
    (C3 ^extract E2)
      (E2 ^a A1 ^b B1 ^result R7 ^status success ^type intersect)
        (A1 ^id car ^status success ^type node)
        (B1 ^status success ^type all_nodes)
        (R7 ^record R9 ^record R8)
          (R9 ^params P2 ^value false)
            (P2 ^a car ^b pole)
          (R8 ^params P1 ^value true)
            (P1 ^a car ^b car)
\end{verbatim}

The \soarb{all\_nodes} filter outputs a list of all nodes in the scene graph, and the \soarb{intersect} filter outputs a list of boolean values indicating whether the car intersects each node, represented by the multi-valued attribute \soarb{record}.
Notice that each \soarb{record} contains both the result of the query as well as the parameters that produced that result.
Not only is this approach more convenient than creating a separate command for each pair of nodes, but it also allows the \soarb{intersect} filter to answer the query more efficiently using special algorithms that can quickly rule out non-intersecting objects.

\subsection{Filter List}
The following is a list of all filters that are included in SVS. 
You can also get this list by using the cli command \texttt{svs filters} and 
get information about a specific filter using the command \texttt{svs filters.FILTER\_NAME}.
Many filters have a \texttt{\_select} version. The select version returns a subset
of the input nodes which pass a test. For example, the \texttt{intersect} filter returns
boolean values for each input (a, b) pair, while the \texttt{intersect\_select} filter
returns the nodes in set b which intersect the input node a. This is useful for passing
the results of one filter into another (e.g. take the nodes that intersect node a and find
the largest of them). 

\begin{description}
  \item{\soarb{node}} \\
    Given an \soarb{\^{}id}, outputs the node with that id.
  \item{\soarb{all\_nodes}} \\
    Outputs all the nodes in the scene
  \item{\soarb{combine\_nodes}} \\
    Given multiple node inputs as \soarb{\^{}a}, concates them into a single output set.
  \item{\soarb{remove\_node}} \\
    Removes node \soarb{\^{}id} from the input set \soarb{\^{}a} and outputs the rest. 
  \item{\soarb{node\_position}} \\
    Outputs the position of each node in input \soarb{\^{}a}.
  \item{\soarb{node\_rotation}} \\
    Outputs the rotation of each node in input \soarb{\^{}a}.
  \item{\soarb{node\_scale}} \\
    Outputs the scale of each node in input \soarb{\^{}a}.
  \item{\soarb{node\_bbox}} \\
    Outputs the bounding box of each node in input \soarb{\^{}a}.
\item{\soarb{distance} and \soarb{distance\_select}} \\
  Outputs the distance between input nodes \soarb{\^{}a} and \soarb{\^{}b}. 
  Distance can be specified by \soarb{\^{}distance\_type << centroid hull >>}, where
  \texttt{centroid} is the euclidean distance between the centers, and the \texttt{hull}
  is the minimum distance between the node surfaces. \texttt{distance\_select} chooses
  nodes in set b in which the distance to node a falls within the range \soarb{\^{}min} and \soarb{\^{}max}.
\item{\soarb{closest} and \soarb{farthest}} \\
  Outputs the node in set \soarb{\^{}b} closest to or farthest from \soarb{\^{}a}
  (also uses \soarb{distance\_type}).
\item{\soarb{axis\_distance} and \soarb{axis\_distance\_select}} \\
  Outputs the distance from input node \soarb{\^{}a} to \soarb{\^{}b} along
  a particular axis (\soarb{\^{}axis << x y z >>}). This distance is based on 
  bounding boxes. A value of 0 indicates the nodes overlap on the given axis, otherwise 
  the result is a signed value indicating whether node b is greater or less than 
  node a on the given axis.  
  The \texttt{axis\_distance\_select} filter also uses \soarb{\^{}min} and \soarb{\^{}max}
  to select nodes in set b. 
\item{\soarb{volume} and \soarb{volume\_select}} \\
  Outputs the bounding box volume of each node in set \soarb{\^{}a}. 
  For \texttt{volume\_select}, it outputs a subset of the nodes whose volumes
  fall within the range \soarb{\^{}min} and \soarb{\^{}max}.
\item{\soarb{largest} and \soarb{smallest}} \\
  Outputs the node in set \soarb{\^{}a} with the largest or smallest volume.
\item{\soarb{larger} and \soarb{larger\_select}}\\
  Outputs whether input node \soarb{\^{}a} is larger than each input node \soarb{\^{}b}, 
  or selects all nodes in b for which a is larger. 
\item{\soarb{smaller} and \soarb{smaller\_select}}\\
  Outputs whether input node \soarb{\^{}a} is smaller than each input node \soarb{\^{}b}, 
  or selects all nodes in b for which a is smaller. 
\item{\soarb{contain} and \soarb{contain\_select}} \\
  Outputs whether the bounding box of each input node \soarb{\^{}a} contains
  the bounding box of each input node \soarb{\^{}b}, or selects those nodes
  in b which are contained by node a. 
\item{\soarb{intersect} and \soarb{intersect\_select}} \\
  Outputs whether each input node \soarb{\^{}a} intersects each input node \soarb{\^{}b}, 
  or selects those nodes in b which intersect node a. Intersection is specified
  by \soarb{\^{}intersect\_type << hull box >>}; either the convex hull of the node
  or the axis-aligned bounding box. 
  \item{\soarb{tag\_select}} \\
    Outputs all the nodes in input set \soarb{\^{}a} which have the tag specified by 
    \soarb{\^{}tag\_name} and \soarb{\^{}tag\_value}. 
\end{description}

\subsection{Examples}

Select all the objects with a volume between 1 and 2. 
\begin{verbatim}
(S1 ^svs S3)
  (S3 ^command C3)
    (C3 ^extract E1)
      (E1 ^type volume_select ^a A1 ^min 1 ^max 2)
        (A1 ^type all_nodes)
\end{verbatim} 

Find the distance between the centroid of ball3 and all other objects. 
\begin{verbatim}
(S1 ^svs S3)
  (S3 ^command C3)
    (C3 ^extract E1)
      (E1 ^type distance ^a A1 ^b B1 ^distance_type centroid)
        (A1 ^type node ^id ball3)
        (B1 ^type all_nodes)
\end{verbatim} 

Test where ball2 intersects any red objects. 
\begin{verbatim}
(S1 ^svs S3)
  (S3 ^command C3)
    (C3 ^extract E1)
      (E1 ^type intersect ^a A1 ^b B1 ^intersect_type hull)
        (A1 ^type node ^id ball2)
        (B1 ^type tag_select ^a A2 ^tag_name color ^tag_value red)
          (A2 ^type all_nodes)
\end{verbatim}

Find all the objects on the table. This is done by selecting nodes 
where the distance between them and the table along the z axis is a small positive number. 
\begin{verbatim}
(S1 ^svs S3)
  (S3 ^command C3)
    (C3 ^extract E1)
      (E1 ^type axis_distance_select ^a A1 ^b B1 ^axis z ^min 0.0001 ^max 0.1)
        (A1 ^type node ^id table)
        (B1 ^type all_nodes)
\end{verbatim}

Find the smallest object that intersects the table (excluding itself). 
\begin{verbatim}
(S1 ^svs S3)
  (S3 ^command C3)
    (C3 ^extract E1)
      (E1 ^type smallest ^a A1)
        (A1 ^type intersect_select ^a A2 ^b B2 ^intersect_type hull)
          (A2 ^type node ^id table)
          (B1 ^type remove_node ^id table ^a A3)
            (A3 ^type all_nodes)
\end{verbatim}




\section{Writing new filters}

SVS contains a small set of generally useful filters, but many users will need additional specialized filters for their application.
Writing new filters for SVS is conceptually simple.

\begin{enumerate}
\item Write a C++ class that inherits from the appropriate filter subclass.
\item Register the new class in a global table of all filters (\texttt{filter\_table.cpp}).
\item Recompile the kernel. 
\end{enumerate}

\subsection{Filter subclasses}

The fact that filter inputs and outputs are lists rather than single values introduces some complexity to how filters are implemented.
Depending on the functionality of the filter, the multiple inputs into multiple parameters must be combined in different ways, and sets of inputs will map in different ways onto the output values.
Furthermore, the outputs of filters are cached so that the filter does not repeat computations on sets of inputs that do not change.
To shield the user from this complexity, a set of generally useful filter paradigms were implemented as subclasses of the basic \texttt{filter} class.
When writing custom filters, try to inherit from one of these classes instead of from \texttt{filter} directly.

\subsubsection{Map filter}
This is the most straightforward and useful class of filters.
A filter of this class takes the Cartesian product of all input values in all parameters,
and performs the same computation on each combination, generating one output.
In other words, this class implements a one-to-one mapping from input combinations to output values.

To write a new filter of this class, inherit from the \texttt{map\_filter} class, 
and define the \texttt{compute} function. Below is an example template:

\begin{verbatim}
class new_map_filter : public map_filter<double> // templated with output type
{
  public:
    new_map_filter(Symbol *root, soar_interface *si, filter_input *input, scene *scn)
    : map_filter<double>(root, si, input)   // call superclass constructor
    {}

    /* Compute
       Do the proper computation based on the input filter_params 
       and set the out parameter to the result 
       Return true if successful, false if an error occured */
    bool compute(const filter_params* p, double& out){
      sgnode* a;
      if(!get_filter_param(this, p, "a", a)){
        set_status("Need input node a");
        return false;
      }
      out = // Your computation here
    }
};
\end{verbatim}

\subsubsection{Select filter}
This is very similar to a map filter, except for each input combination from the 
Cartesian product the output is optional. This is useful for selecting and returning
a subset of the outputs. 

To write a new filter of this class, inherit from the \texttt{select\_filter} class, 
and define the \texttt{compute} function. Below is an example template:

\begin{verbatim}
class new_select_filter : public select_filter<double> // templated with output type
{
  public:
    new_select_filter(Symbol *root, soar_interface *si, filter_input *input, scene *scn)
    : select_filter<double>(root, si, input)   // call superclass constructor
    {}

    /* Compute
       Do the proper computation based on the input filter_params 
       and set the out parameter to the result (if desired)
       Also set the select bit to true if you want to the result to be output. 
       Return true if successful, false if an error occured */
    bool compute(const filter_params* p, double& out, bool& select){
      sgnode* a;
      if(!get_filter_param(this, p, "a", a)){
        set_status("Need input node a");
        return false;
      }
      out = // Your computation here
      select = // test for when to output the result of the computation
    }
};
\end{verbatim}

\subsubsection{Rank filter}
A filter where a ranking is computed for each combination from the Cartesian
product of the input and only the combination which results in the highest 
(or lowest) value is output. The default behavior is to select the highest, 
to select the lowest you can call \texttt{set\_select\_highest(false)} on the filter.

To write a new filter of this class, inherit from the \texttt{rank\_filter} class, 
and define the \texttt{rank} function. Below is an example template:

\begin{verbatim}
class new_rank_filter : public rank_filter
{
  public:
    new_rank_filter(Symbol *root, soar_interface *si, filter_input *input, scene *scn)
    : rank_filter(root, si, input)   // call superclass constructor
    {}

    /* Compute
       Do the proper computation based on the input filter_params 
       And set r to the ranking result. 
       Return true if successful, false if an error occured */
    bool compute(const filter_params* p, double& r){
      sgnode* a;
      if(!get_filter_param(this, p, "a", a)){
        set_status("Need input node a");
        return false;
      }
      r = // Ranking computation
    }
};
\end{verbatim}

\subsection{Generic Node Filters}
There are also a set of generic filters specialized for computations involving nodes. 
With these you only need to specify a predicate function involving nodes. 
Also see \texttt{filters/base\_node\_filters.h}.
There are three types of these filters. 

\subsubsection{Node Test Filters}
These filters involve a binary test between two nodes (e.g. intersection or larger). 
You must specify a test function of the following form:
\begin{verbatim}
bool node_test(sgnode* a, sgnode* b, const filter_params* p)
\end{verbatim}
For an example of how the following base filters are used, see \texttt{filters/intersect.cpp}.

\textbf{node\_test\_filter} \\
For each input pair (a, b) this outputs the boolean result of \texttt{node\_test(a, b)}.

\textbf{node\_test\_select\_filter} \\
For each input pair (a, b) this outputs node b if \texttt{node\_test(a, b) == true}. 
(Can choose to select b if the test is false by calling \texttt{set\_select\_true(false)}).

\subsubsection{Node Comparison Filters}
These filters involve a numerical comparison between two nodes (e.g. distance). 
You must specify a comparison function of the following form:
\begin{verbatim}
double node_comparison(sgnode* a, sgnode* b, const filter_params* p)
\end{verbatim}

For an example of how the following base filters are used, see \texttt{filters/distance.cpp}.

\textbf{node\_comparison\_filter} \\
For each input pair (a, b) this outputs the numerical result of \texttt{node\_comparison(a, b)}. 

\textbf{node\_comparison\_select\_filter} \\
For each input pair (a, b), this outputs node b if 
\texttt{min <= node\_comparison(a, b) <= max}. 
Min and max can be set through calling \texttt{set\_min(double)} 
and \texttt{set\_max(double)}, or as specified by the user through the filter\_params. 

\textbf{node\_comparison\_rank\_filter} \\
This outputs the input pair (a, b) for which \texttt{node\_comparison(a, b)} 
produces the highest value. To instead have the lowest value output call \texttt{set\_select\_highest(true)}.


\subsubsection{Node Evaluation Filters}
These filters involve a numerical evaluation of a single node (e.g. volume). 
You must specify an evaluation function of the following form:
\begin{verbatim}
double node_evaluation(sgnode* a, const filter_params* p)
\end{verbatim}

For an example of how the following base filters are used, see \texttt{filters/volume.cpp}.

\textbf{node\_evaluation\_filter} \\
For each input node a, this outputs the numerical result of \texttt{node\_evaluation(a)}. 

\textbf{node\_evaluation\_select\_filter} \\
For each input node a, this outputs the node if 
\texttt{min <= node\_evaluation(a) <= max}. 
Min and max can be set through calling \texttt{set\_min(double)} 
and \texttt{set\_max(double)}, or as specified by the user through the filter\_params. 

\textbf{node\_evaluation\_rank\_filter} \\
This outputs the input node a for which \texttt{node\_evaluation(a)} 
produces the highest value. To instead have the lowest value output call \texttt{set\_select\_highest(true)}.


\section{Command line interface}

The user can query and modify the runtime behavior of SVS using the \soarb{svs} command.
The syntax of this command differs from other Soar commands due to the complexity and object-oriented nature of the SVS implementation.
The basic idea is to allow the user to access each object in the SVS implementation (not to be confused with objects in the scene graph) at runtime.
Therefore, the command has the form \texttt{svs PATH [ARGUMENTS]}, where \texttt{PATH} uniquely identifies an object or the method of an object.
\texttt{ARGUMENTS} is a space separated list of strings that each object or function interprets in its own way.
For example, \texttt{svs S1.scene.world.car} identifies the car object 
in the scene graph of the top state.
As another example, \verb|svs connect_viewer 5999| calls the method to connect to the SVS visualizer with 5999 being the TCP port to connect on.
Every path has two special arguments.

\begin{itemize}
\item{\texttt{svs PATH dir}} prints all the children of the object at \texttt{PATH}.
\item{\texttt{svs PATH help}} prints text about how to use the object, if available.
\end{itemize}



% ----------------------------------------------------------------------------

\section{File System I/O Commands}
\label{FILE-IO}

This section describes commands which interact in one way or another
with operating system input and output, or file I/O.  Users can
save/retrieve information to/from files, redirect the information
printed by Soar as it runs, and save and load the binary representation
of productions.
The specific commands described in this section are:

\paragraph{Summary}
\begin{quote}
\begin{description}
\item[cd] - Change directory.
\item[dirs] - List the directory stack.
\item[load] - Loads soar files, rete networks, saved percept streams and external libraries.
\item[load file] - Sources a file containing soar commands and productions.  May also contain Tcl code if Tcl mode is enabled.
\item[load library] - Loads an external library that extends functionality of Soar.
\item[load rete-network] - Loads a rete network that represents rules loaded in production memory.
\item[load library] - Loads soar files, rete networks, saved percept streams and external libraries.
\item[ls] - List the contents of the current working directory.
\item[popd] - Pop the current working directory off the stack and change to the next directory on the stack.
\item[pushd] - Push a directory onto the directory stack, changing to it.
\item[pwd] - Print the current working directory.
\item[save] - Saves chunks, rete networks and percept streams.
\item[save chunks] - Saves chunks into a file.
\item[save percepts] - Saves future input link structures into a file.
\item[save rete-network] - Saves the current rete networks that represents rules loaded in production memory.
\end{description}
\end{quote}

The \textbf{load file} command, previously know as \textbf{soar} is used for nearly every Soar program.  The
directory functions are important to understand so that users can
navigate directories/folders to load/save the files of interest.  
Saving and loading percept streams are used mainly  when Soar needs to interact with an
external environment.  Users might take advantage of these commands when
debugging agents, but care should be used in adding and removing wmes this
way as they do not fall under Soar's truth maintenance system.

\input{wikicmd/tex/file-system}
\input{wikicmd/tex/output}
\input{wikicmd/tex/load}
\input{wikicmd/tex/save}

% ***************************************************************************
% ----------------------------------------------------------------------------
\section{Miscellaneous}
\label{MISC}


\comment{this section still needs to be rewritten...}

\nocomment{This section describes the commands used to inspect production memory,
working memory, and preference memory; to see what productions will 
match and fire in the next Propose or Apply phase;  and to examine the 
goal dependency set.  These commands are particularly useful when
running or debugging Soar, as they let users see what Soar is ``thinking.''}
The specific commands described in this section are:


\paragraph{Summary}
\begin{quote}
\begin{description}
\item[alias] - Define a new alias, or command, using existing commands and arguments.
\item[debug] - Contains commands that provide access to Soar's internals. Most users will not need to access these commands
\item[debug allocate] - Allocate additional 32 kilobyte blocks of memory for a specified memory pool without running Soar.
\item[debug port] - Returns the port the kernel instance is listening on.
\item[debug time] - Uses a default system clock timer to record the wall time required while executing a command.
\item[debug internal-symbols] - Print information about the Soar symbol table.
\end{description}
\end{quote}

\input{wikicmd/tex/alias}
\input{wikicmd/tex/debug}


\cleardoublepage
\phantomsection
\addcontentsline{toc}{chapter}{Appendices}
\appendix

% change 'include's to 'input' for final version
%  (use 'include' instead if you're only printing part of the manual)
%\include{a-glossary}
%% ----------------------------------------------------------------------------
\typeout{--------------- BLOCKSCODE ------------------------------------------}
\chapter{The Blocks-World Program}
\label{BLOCKSCODE}

\footnotesize
\begin{verbatim}
###############################################################################
###
### File              : blocks.soar
### Original author(s): John E. Laird <laird@eecs.umich.edu>
### Organization      : University of Michigan AI Lab
### Created on        : 15 Mar 1995, 13:53:46
### Last Modified By  : Clare Bates Congdon <congdon@eecs.umich.edu>
### Last Modified On  : 17 Jul 1996, 16:35:14
### Soar Version      : 7
###
### Description : A new, simpler implementation of the blocks world
###               with just three blocks being moved at random.
###
### Notes: 
###   CBC, 6/27: Converted to Tcl syntax
###   CBC, 6/27: Added extensive comments
###############################################################################

 
###############################################################################
# Create the initial state with blocks A, B, and C on the table.
#
# This is the first production that will fire; Soar creates the initial state
#   as an architectural function (in the 'zeroth' decision cycle), which will
#   match against this production.
# This production does a lot of work because it is creating (preferences for)
# all the structure for the initial state:
# 1. The state has a problem-space named 'blocks'. The problem-space limits
#    the operators that will be selected for a task. In this simple problem,
#    it isn't really necessary (there is only one operator), but it's a
#    programming convention that you should get used to.
# 2. The state has four 'things' -- three blocks and the table.
# 3. The state has three 'ontop' relations
# 4. Each of the things has substructure: their type and their names. Note that
#    the fourth thing is actually a 'table'.
# 5. Each of the ontop relations has substructure: the top thing and the
#    bottom thing.
# Finally, the production writes a message for the user.
#
# Note that this production will fire exactly once and will never retract.

sp {blocks-world*elaborate*initial-state
   (state <s> ^superstate nil)
-->
   (<s> ^problem-space blocks
        ^thing <block-A> <block-B> <block-C> <table>
        ^ontop <ontop-A> <ontop-B> <ontop-C>)
   (<block-A> ^type block ^name A)
   (<block-B> ^type block ^name B)
   (<block-C> ^type block ^name C)
   (<table> ^type table ^name TABLE)
   (<ontop-A> ^top-block <block-A> ^bottom-block <table>)
   (<ontop-B> ^top-block <block-B> ^bottom-block <table>)
   (<ontop-C> ^top-block <block-C> ^bottom-block <table>)
   (write (crlf) |Initial state has A, B, and C on the table.|)}


###############################################################################
# State elaborations - keep track of which objects are clear
# There are two productions - one for blocks and one for the table.
###############################################################################

###############################################################################
# Assert table always clear
#
# The conditions establish that:
#  1. The state has a problem-space named 'blocks'.
#  2. The state has a thing of type table.
# The action:
#  1. creates an acceptable preference for an attribute-value pair asserting
#     the table is clear.
#
# This production will also fire once and never retract.

sp {elaborate*table*clear
   (state <s> ^problem-space blocks
              ^thing <table>)
   (<table> ^type table)
-->
   (<table> ^clear yes)}

###############################################################################
# Calculate whether a block is clear
#
# The conditions establish that:
#  1. The state has a problem-space named 'blocks'.
#  2. The state has a thing of type block.
#  3. There is no 'ontop' relation having the block as its 'bottom-block'.
# The action:
#  1. create an acceptable preference for an attribute-value pair asserting
#     the block is clear.
#
# This production will retract whenever an 'ontop' relation for the given block
#  is created. Since the (<block> ^clear yes) wme only has i-support, it will
#  be removed from working memory automatically when the production retracts.

sp {elaborate*block*clear
   (state <s> ^problem-space blocks
              ^thing <block>)
   (<block> ^type block)
   -(<ontop> ^bottom-block <block>)
-->
   (<block> ^clear yes)}


###############################################################################
# Suggest MOVE-BLOCK operators
#
# This production proposes operators that move one block ontop of another block.  
# The conditions establish that:
#  1. The state has a problem-space named 'blocks'
#  2. The block moved and the block moved TO must be both be clear.
#  3. The block moved is different from the block moved to.
#  4. The block moved must be type block.
#  5. The block moved must not already be ontop the block being moved to.
# The actions:
#  1. create an acceptable preference for an operator.
#  2. create acceptable preferences for the substructure of the operator (its
#     name, its 'moving-block' and the 'destination).

sp {blocks-world*propose*move-block
   (state <s> ^problem-space blocks
              ^thing <thing1> {<> <thing1> <thing2>}
              ^ontop <ontop>)
   (<thing1> ^type block ^clear yes)
   (<thing2> ^clear yes)
   (<ontop> ^top-block <thing1>
            ^bottom-block <> <thing2>)
-->
   (<s> ^operator <o> +)
   (<o> ^name move-block
        ^moving-block <thing1>
        ^destination <thing2>)}

###############################################################################
# Make all acceptable move-block operators also indifferent
#
# The conditions establish that:
#  1. the state has an acceptable preference for an operator
#  2. the operator is named move-block
# The actions:
#  1. create an indifferent prefererence for the operator

sp {blocks-world*compare*move-block*indifferent
   (state <s> ^operator <o> +)
   (<o> ^name move-block)
-->
   (<s> ^operator <o> =)}



###############################################################################
# Apply a MOVE-BLOCK operator
# 
# There are two productions that are part of applying the operator.
# Both will fire in parallel.
###############################################################################

###############################################################################
# Apply a MOVE-BLOCK operator
#   (the block is no longer ontop of the thing it used to be ontop of)
#
# This production is part of the application of a move-block operator.
# The conditions establish that:
#  1. An operator has been selected for the current state
#     a. the operator is named move-block
#     b. the operator has a 'moving-block' and a 'destination'
#  2. The state has an ontop relation
#     a. the ontop relation has a 'top-block' that is the same as the
#        'moving-block' of the operator
#     b. the ontop relation has a 'bottom-block' that is different from the 
#        'destination' of the operator
# The actions:
#  1. create a reject preference for the ontop relation

sp {blocks-world*apply*move-block*remove-old-ontop
   (state <s> ^operator <o>
              ^ontop <ontop>)
   (<o> ^name move-block 
        ^moving-block <block1> 
        ^destination <block2>)
   (<ontop> ^top-block <block1> 
            ^bottom-block { <> <block2> <block3> })
-->
   (<s> ^ontop <ontop> -)}
 

###############################################################################
# Apply a MOVE-BLOCK operator
#   (the block is now ontop of the destination)
#
# This production is part of the application of a move-block operator.
# The conditions establish that:
#  1. An operator has been selected for the current state
#     a. the operator is named move-block
#     b. the operator has a 'moving-block' and a 'destination'
# The actions:
#  1. create an acceptable preference for a new ontop relation
#  2. create (acceptable preferences for) the substructure of the ontop
#     relation: the top block and the bottom block

sp {blocks-world*apply*move-block*add-new-ontop
   (state <s> ^operator <o>)
   (<o> ^name move-block
        ^moving-block <block1>
        ^destination <block2>)
-->
   (<s> ^ontop <ontop>)
   (<ontop> ^top-block <block1>
            ^bottom-block <block2>)}


###############################################################################
###############################################################################
# Detect that the goal has been achieved 
#
# The conditions establish that:
#  1. The state has a problem-space named 'blocks'
#  2. The state has three ontop relations
#     a. a block named A is ontop a block named B
#     b. a block named B is ontop a block named C
#     c. a block named C is ontop a block named TABLE
# The actions:
#  1. print a message for the user that the A,B,C tower has been built
#  2. halt Soar

sp {blocks-world*detect*goal
   (state <s> ^problem-space blocks
              ^ontop <AB> 
               { <> <AB> <BC>}
               { <> <AB> <> <BC> <CT> } )
   (<AB> ^top-block <A> ^bottom-block <B>)
   (<BC> ^top-block <B> ^bottom-block <C>)
   (<CT> ^top-block <C> ^bottom-block <T>)
   (<A> ^type block ^name A)
   (<B> ^type block ^name B)
   (<C> ^type block ^name C)
   (<T> ^type table ^name TABLE)
-->
   (write (crlf) |Achieved A, B, C|)
   (halt)}


###############################################################################
###############################################################################
# Monitor the state: Print a message every time a block is moved
#
# The conditions establish that:
#  1. An operator has been selected for the current state
#     a. the operator is named move-block
#     b. the operator has a 'moving-block' and a 'destination'
#  2. each block has a name
# The actions:
#  1. print a message for the user that the block has been moved to the
#     destination. 

sp {blocks-world*monitor*move-block
   (state <s> ^operator <o>)
   (<o> ^name move-block
        ^moving-block <block1>
        ^destination <block2>)
   (<block1> ^name <block1-name>)
   (<block2> ^name <block2-name>)   
-->
   (write (crlf) |Moving Block: | <block1-name>
                 | to: | <block2-name> ) }
\end{verbatim}
\normalsize

% ----------------------------------------------------------------------------
\typeout{--------------- appendix: GRAMMARS for productions ------------------}
\chapter{Grammars for production syntax}
\label{GRAMMARS}
\index{grammar}

This appendix contains the BNF grammars for the conditions and actions of
productions. (BNF stands for Backus-Naur form or Backus normal form; consult a
computer science book on theory, programming languages, or compilers for more
information. However, if you don't already know what a BNF grammar is, it's
unlikely that you have any need for this appendix.)

This information is provided for advanced Soar users, for example, those who
need to write their own parsers. Note that some terms (e.g. \soar{<sym\_constant>})
are undefined; as such, this grammar should only be used as a starting point.

\comment{this section still needs a disclaimer that what you can actually do
	is less restrictive than the way we described it in the main text } 

\comment{note that grammars are no longer consistent with new rhs actions}

\nocomment{John and I decided while talking about this that we just wouldn't let
	people know that they could omit the identifier of the state

	It is legal to omit the variable test for a state when that variable is not
	tested elsewhere in the production, nor used in the action.  For
	example: 
	\begin{verbatim}
	(state ^operator <o>)
	\end{verbatim}

	is equivalent to 
	\begin{verbatim}
	(state <s> ^operator <o>)
	\end{verbatim}
	}

%-------------------------------------------------------
\section{Grammar of Soar productions}

A grammar for Soar productions is:
\begin{verbatim}
<soar-production>  ::= sp "{" <production-name> [<documentation>] [<flags>]
                     <condition-side> --> <action-side> "}"
<documentation>    ::= """ [<string>] """
<flags>            ::= ":" (o-support | i-support | chunk | default)
\end{verbatim}

% ----------------------------------------------------------------------------
\subsection{Grammar for Condition Side}
\label{SYNTAX-pm-condgrammar}
\index{condition-side grammar}
\index{grammar!condition side}

Below is a grammar for the condition sides of productions:
\begin{verbatim}
<condition-side>   ::= <state-imp-cond> <cond>*
<state-imp-cond>   ::= "(" (state | impasse) [<id_test>]
                     <attr_value_tests>+ ")"
<cond>             ::= <positive_cond> | "-" <positive_cond>
<positive_cond>    ::= <conds_for_one_id> | "{" <cond>+ "}"
<conds_for_one_id> ::= "(" [(state|impasse)] <id_test> 
                     <attr_value_tests>+ ")"
<id_test>          ::= <test>
<attr_value_tests> ::= ["-"] "^" <attr_test> ("." <attr_test>)*
                     <value_test>*
<attr_test>        ::= <test>
<value_test>       ::= <test> ["+"] | <conds_for_one_id> ["+"]  

<test>             ::= <conjunctive_test> | <simple_test>
<conjunctive_test> ::= "{" <simple_test>+ "}"
<simple_test>      ::= <disjunction_test> | <relational_test>
<disjunction_test> ::= "<<" <constant>+ ">>"
<relational_test>  ::= [<relation>] <single_test>
<relation>         ::= "<>" | "<" | ">" | "<=" | ">=" | "=" | "<=>"
<single_test>      ::= <variable> | <constant>
<variable>         ::= "<" <sym_constant> ">"
<constant>         ::= <sym_constant> | <int_constant> | <float_constant>
\end{verbatim}
\index{constant}
\index{variable}

\subsubsection*{Notes on the Condition Side}\vspace{-12pt}
\begin{itemize}
\item In an \soar{<id\_test>}, only a \soar{<variable>} may be used in a \soar{<single\_test>}.
\end{itemize}

\comment{I don't think that grammar is quite right -- e.g. should distinguish
        that acceptable preferences may appear for operators, but not other
        objects}

\comment{Grammar correctly describes Soar; it's just that you can actually do
	things that we've said can't be done. So in this section we'll mention
	that we lied before and that the grammar above is different, but
	correct.  see notes on difference on page 64 of June 7th draft}


% ----------------------------------------------------------------------------
\subsection{Grammar for Action Side}
\label{SYNTAX-pm-actgrammar}    %RHS grammar}
\index{action-side grammar}
\index{grammar!action side}

\comment{RD: this grammar is out of date}

Below is a grammar for the action sides of productions:
\begin{verbatim}
<rhs>                      ::= <rhs_action>*
<rhs_action>               ::= "(" <variable> <attr_value_make>+ ")" 
                             | <func_call>
<func_call>                ::= "(" <func_name> <rhs_value>* ")"
<func_name>                ::= <sym_constant> | "+" | "-" | "*" | "/"
<rhs_value>                ::= <constant> | <func_call> | <variable>
<attr_value_make>          ::= "^" <variable_or_sym_constant>
                             ("." <variable_or_sym_constant>)* <value_make>+
<variable_or_sym_constant> ::= <variable> | <sym_constant>
<value_make>               ::= <rhs_value> <preference_specifier>*

<preference-specifier>     ::= <unary-preference> [","]
                             | <unary-or-binary-preference> [","]
                             | <unary-or-binary-preference> <rhs_value> [","]
<unary-pref>               ::= "+" | "-" | "!" | "~" | "@"
<unary-or-binary-pref>     ::= ">" | "=" | "<" | "&"
\end{verbatim}

\comment{I don't quite understand that last bit. 
<forced-unary-pref>        ::= <binary-preference> {, | ) | ^}
       (but the parser doesn't consume the ")" or "^" here)}

\index{constant}
\index{variable}

%% ----------------------------------------------------------------------------
\typeout{--------------- appendix: calculation of o-SUPPORT -----------------}
\chapter{The Calculation of O-Support}
\label{SUPPORT}
\index{support}
\index{i-support}
\index{o-support}
\index{persistence}

This appendix provides a description of when a preference is given o-support by an instantiation (a preference that is not given o-support will have i-support). Soar has four possible procedures for deciding support, which can be selected among with the o-support-mode command (see page \pageref{o-support-mode}). However, only o-support modes 3 \& 4 can be considered current to Soar 8, and o-support mode 4 should be considered an improved version of mode 3.   The default o-support mode is mode 4.

In o-support modes 3 \& 4, support is given by production; that is, all preferences generated by the RHS of a single instantiated production will have the same support. 


In both modes, a production must meet the following two requirements to create o-supported preferences:
\begin{enumerate}
\item The RHS has no operator proposals, i.e. nothing of the form \begin{verbatim}(<s> ^operator <o> +) \end{verbatim}
\item The LHS has a condition that tests the current operator, i.e. something of the form
\footnote{Sometimes, o-support mode 3 does not notice that this condition is true. This is a bug, which is unlikely to be fixed, since users are encouraged to use mode 4.}
\begin{verbatim}(<s> ^operator <o>)\end{verbatim}
\comment{this is only true if mode 3's checks are improved}
\end{enumerate}



In condition 1, the variable \soar{<s>} must be bound to a state identifier.
In condition 2, the variable \soar{<s>} must be bound to the lowest state identifier. That is to say, each (positive) condition on the LHS takes the form \soar{(id \carat attr value)}, some of these id's match state identifiers, and the system looks for the deepest matched state identifier. The tested current operator must be on this state. For example, in the production-

\begin{verbatim}
sp {elaborate*state*operator*name
  (state <s> ^superstate <s1>)
  (<s1> ^operator <o>)
  (<o> ^name <name>)
-->
  (<s> ^name something)}
\end{verbatim}


the RHS action gets i-support. Of course, the state bound to \soar{<s>} is destroyed when \soar{(<s1> \carat operator <o>)} retracts, so o-support would make little difference. On the other hand, the production- 

\begin{verbatim}
sp {operator*superstate*application
   (state <s> ^superstate <s1>)
   (<s> ^operator <o>)
   (<o> ^name <name>)
 -->
   (<s1> ^sub-operator-name <name>)}
\end{verbatim}

gives o-support to its RHS action, which remains after the substate bound to \soar{<s>} is destroyed. 


There is a third condition that determines support, and it is in this condition that modes 3 \& 4 differ. An extension of condition 1 is that operator augmentations should always receive i-support. Soar has been written to recognize augmentations directly off the operator (ie, \soar{(<o> \carat augmentation value)}), and to attempt to give them i-support. However, there was some confusion about what to do about a production that simultaneously tests an operator, doesn't propose an operator, adds an operator augmentation, and adds a non-operator augmentation, such as-

\begin{verbatim}
sp {operator*augmentation*application
  (state <s> ^task test-support
  	      ^operator <o>)
-->
   (<o> ^new augmentation)
   (<s> ^new augmentation)}
\end{verbatim}


In o-support mode 3, both RHS actions receive o-support; in o-support mode 4, both receive i-support. In either case, Soar will print a warning on firing this production, because this is considered bad coding style.

\nocomment{Support calculations are done at run time, as each production is fired. Could these decisions be done at compile time? Much of the decision is based on the structure of the production, which could be analyzed once as the production was loaded or chunked. However, it may be impossible to guarantee that a variable will be bound to a state id just by examining production syntax. Another issue is whether the state tested in condition 2 is the lowest state - this potentially could differ from instantiation to instantiation. For instance the operator*augmentation*application production above could match against multiple states in the state stack. 

 
%-----------------------------------------------------------
\section{Possible problems with implementation of modes 3 \& 4}

\begin{enumerate}
\item Default mode is actually o-support mode 3. Do we not want 4 to be default?
\item There is still the bug Andy pointed out. In condition 1, the variable \soar{<s>} is \textit{supposed} to be bound to a state variable, but the code does not actually check for this.
\item There is one additional, strange difference between modes 3 \& 4. In condition 3, the \soar{id} of each RHS action is tested to see if it is the id of the operator. This id is represented either as a symbol or as a rete location. Mode 4 tests the id both as a symbol and as a rete location, while mode 3 does only the symbol test. The rete test should be added to mode 3.
\end{enumerate}


\section{O-support modes 1 \& 2}

In o-support modes 1 \& 2, there are some of the same calculations as in 3 \& 4 when a production is matched (which occurs when a wme is added to the rete). In particular, if it is an operator proposal, it is set as IE\_PRODS. Otherwise, if it tests the current operator, it is set as PE\_PRODS, without testing for operator  elaborations. The match is placed on the appropriate dll, according to IE\_PRODS or PE\_PRODS.

Later, when the production is instantiated and the new preferences are built, there are no support calculations for 3 \& 4. But 1 \& 2 have support calculations. I suppose that the purpose of the earlier support calculations is that it places the production on the proper list to be fired during apply or propose,that is, whether it is an IE\_PROD or a PE\_PROD.

During this instantiation process, the function calculate\_support\_for\_instantiation\_preferences() is called to redo support IF the variable need\_to\_do\_support\_calculations is set to TRUE. This variable can be true only when-

\begin{enumerate}
\item  called from chunk\_instantiation OR
\item  \#ifndef SOAR\_8\_ONLY
SOAR\_8\_ONLY is a compile option, which is not defined by default. I think that its purpose is that, when defined, there is no run-time option to switch out of Soar 8. This allows a significant portion of code to be left out. Check out function Soar\_Operand2. 
\end{enumerate}


Mode 2 computes support in what is called 'Doug Pearson's way', which is described as-

 \begin{verbatim}
 For a particular preference p=(id ^attr ...) on the RHS of an
   instantiation [LHS,RHS]:

   RULE #1 (Context pref's): If id is the match state and attr="operator", 
   then p does NOT get o-support.  This rule overrides all other rules.

   RULE #2 (O-A support):  If LHS includes (match-state ^operator ...),
   then p gets o-support.

   RULE #3 (O-M support):  If LHS includes (match-state ^operator ... +),
   then p gets o-support.

   RULE #4 (O-C support): If RHS creates (match-state ^operator ... +/!),
   and p is in TC(RHS-operators, RHS), then p gets o-support.

   Here "TC" means transitive closure; the starting points for the TC are 
   all operators the RHS creates an acceptable/require preference for (i.e., 
   if the RHS includes (match-state ^operator such-and-such +/!), then 
   "such-and-such" is one of the starting points for the TC).  The TC
   is computed only through the preferences created by the RHS, not
   through any other existing preferences or WMEs.

   If none of rules 1-4 apply, then p does NOT get o-support.

   Note that rules 1 through 3 can be handled in linear time (linear in 
   the size of the LHS and RHS); rule 4 can be handled in time quadratic 
   in the size of the RHS (and typical behavior will probably be linear).
   
   
   What is 'match state'? The match goal for the instantiation.
   Match goal - (a match goal is associated with an instantiation).
   Look through instantiated LHS conditions.
   Find the lowest goal state matched to one of the condition's ids.
 \end{verbatim}  
   
O-support mode 1 computes Doug's support and compares it to the poor cousin of mode 3 \& 4 support calculations, ie calculation without checking for operator elaboration. It prints any differences it finds.

}
   
   
 \nocomment{
   
   
   
   
   
   
   

3. the RHS has no direct elaborations of the current operator, ie no actions of the form 
(<o> ^augmentation value).
However, an indirect elaboration such as
   (<o> ^name <d>)
   -->
   (<d> ^augmentation value)
will not prevent o-support.


In mode 3, an instantiation will generate o-supported preferences iff
1. the RHS has no operator proposals (nothing of the form (<s> ^operator <o> +))
2. the LHS has a condition that tests the current operator (something of the form 
(<s> ^operator <o>))
3. 


Operator proposal - a production whose RHS has action (<s> ^operator <o> +))
Operator test - 
	LHS has condition of the form (<s> ^operator <o>)
Operator elaboration -
	o_support_mode 3:
		
	o_support_mode 4:
	

o_support_mode 4: 
1. if an operator proposal - i-support
2. if not an operator test - i-support
3. if an operator test with no elaborations - o-support
4. if an operator test with some elaborations and some non-elaboration, non-function RHS action - i-support (warns)
5. if an operator test with only elaborations - i-support

o_support_mode 3:
1. if an operator proposal - i-support
2. if not an operator test - i-support
3. if an operator test with no elaborations - o-support
4. if an operator test with some elaborations and some non-elaboration, non-function RHS action - o-support (warns)
5. if an operator test with only elaborations - i-support


o_support_mode 0:
1. if an operator proposal - i-support
2. if test operator - o-support
3. else - i-support

}


%% ----------------------------------------------------------------------------
\typeout{--------------- appendix: evaluation of PREFERENCES -----------------}
\chapter{The Resolution of Operator Preferences}
\label{PREFERENCES}
\index{preferences}
% This is a technical discussion of the filtering done to evaluate preferences;
% it might belong in a different version of the manual, but not 492

\comment{what's not clear in the following discussion is what happens in the
	usual case, that is, when there's a single acceptable preference.}

During the decision phase, operator preferences are evaluated in a sequence 
of eight steps, in an effort to select a single operator. 
Each step handles a specific type of preference, as illustrated in Figure 
\ref{fig:prefsem}. (The figure should be read starting at the top
where all the operator preferences are collected and passed into the procedure. At
each step, the procedure either exits through a arrow to the right, or passes to 
the next step through an arrow to the left.)

Input to the procedure are the set of current operator preferences, and the output
consists of:
\begin{enumerate}
\item a subset of the candidate operators, either the empty set, a set consisting of a single, 
winning candidate, or a larger set of candidates that may be conflicting,
tied, or indifferent.
\item an impasse-type, possibly NONE\_IMPASSE\_TYPE.
\end{enumerate}
The procedure has several potential exit points. Some occur when the procedure
has detected a particular type of impasse. The others occur when the number of
candidates has been reduced to 
one (necessarily the winner) or zero (a no-change impasse).

\nocomment{
There are nine filter-like operations involved in evaluating the preferences
available for a particular identifier and attribute. These filters are
executed in a specific order to determine the correct values for the working
memory augmentation, as illustrated in Figure \ref{fig:prefsem}. (The figure
should be read starting at the top left where all the values for an attribute
are collected and passed to the first filter.) Each filter reduces the number
of preferences that need to be considered. If a conflict is found, then an
impasse is generated and the filtering process is halted. The impasse
generation is handled as a special exit from a filter and is indicated with a
grey line in Figure \ref{fig:prefsem}.

The preference semantics module takes as input one or more preferences for a
given identifier and attribute; its output includes: \vspace{-10pt}
\begin{enumerate}
\item a possibly empty set of candidate augmentations that may be conflicting,
	indifferent, or parallel\vspace{-10pt}
\item possibly, an impasse type (if the
	candidates are conflicting)
\end{enumerate}
}

\index{decision!procedure}

\begin{figure}
\insertfigure{newprefsem}{7in}
\insertcaption{An illustration of the preference resolution process. There are eight
	steps; only five of these provide exits from the  resolution process.}
\label{fig:prefsem}
\end{figure}

Each step in Figure \ref{fig:prefsem} is described below:

\index{preference!require}
\index{require preference}
\index{"!}
\index{constraint-failure impasse}
\begin{description}
\item[RequireTest (!)]
This test checks for required candidates in preference memory and
also constraint-failure impasses involving require preferences (see
Section \ref{ARCH-impasses} on page \pageref{ARCH-impasses}).

\begin{itemize}
\item If there is exactly one candidate operator with a require preference and
	that candidate does not have a prohibit preference, then that candidate
	is the winner and preference semantics terminates.
\item Otherwise ---
	If there is more than one required candidate, then a constraint-
	failure impasse is recognized and preference semantics terminates 
	by returning the set of required candidates.
\item Otherwise ---
	If there exists a required candidate that is also prohibited, a
	constraint-failure impasse with the required/prohibited value is
	recognized and preference semantics terminates.
\item Otherwise ---
	The candidates are passed to AcceptableCollect.
\end{itemize}

\item[AcceptableCollect (+) ] This operation builds a list of operators
	for which there is an acceptable preference in preference memory.
	This list of candidate operators is passed to the ProhibitFilter.\index{+}
\nocomment{
\begin{itemize}
\item If there are no acceptable preferences in memory for the value of an
	attribute then exit preference semantics with no items picked. 
	(This is an efficiency termination, and does not apply to other filters.)
\item Otherwise ---
	The candidates are passed to the ProhibitFilter.
\end{itemize}
}
\index{preference!acceptable}
\index{acceptable preference}


\item[ProhibitFilter ($\sim$) ] This filter removes the candidates that
	have prohibit preferences in memory. The rest of the candidates are passed to
	the RejectFilter.
\index{preference!prohibit}
\index{prohibit preference}
\index{~}

\item[RejectFilter ($-$) ] This filter removes the candidates that have
	reject preferences in memory. 
\index{preference!reject}
\index{reject preference}
\index{-}

\item[Exit Point 1]
	\begin{itemize}
	\item At this point, if the set of remaining candidates is either empty or has one
	member, preference semantics terminates and this set is returned.
	\item Otherwise, the remaining candidates are passed to the
	BetterWorseFilter.
	\end{itemize}
\index{-}

\item[BetterWorseFilter ($>$), ($<$) ] This filter removes any candidates that are worse
	than another candidate.
\index{preference!worse}
\index{worse preference}
\index{preference!better}
\index{better preference}
\index{<}
\index{>}

\item[Exit Point 2]
	\begin{itemize}
	\item If the set of remaining candidates is empty, a conflict impasse is created
	returning the set of all candidates passed into this filter, i.e. all of the
	conflicted operators.
	\item If the set of remaining candidates has one
	member, preference semantics terminates and this set is returned.
	\item Otherwise, the remaining candidates are passed to the
	BestFilter.
	\end{itemize}
\index{-}

\item[BestFilter ($>$) ] If some remaining candidate has a best preference,
	this filter removes any candidates that do not have
	a best preference. If there are no best preferences for any of the current
	candidates, the filter has no effect. The remaining candidates are passed
	to the WorstFilter.
\index{preference!best}
\index{best preference}

\item[Exit Point 3]
	\begin{itemize}
	\item At this point, if the set of remaining candidates is either empty or has one
	member, preference semantics terminates and this set is returned.
	\item Otherwise, the remaining candidates are passed to the
	WorstFilter.
	\end{itemize}
\index{-}

\item[WorstFilter ($<$) ] This filter removes any candidates that have
	a worst preference. If all remaining candidates have worst preferences or there
	are no worst preferences, this filter has no effect.
\index{preference!worst}
\index{worst preference}

\item[Exit Point 4]
	\begin{itemize}
	\item At this point, if the set of remaining candidates is either empty or has one
	member, preference semantics terminates and this set is returned.
	\item Otherwise, the remaining candidates are passed to the
	IndifferentFilter.
	\end{itemize}
\index{-}

\index{=}
\item[IndifferentFilter (=) ] This operation traverses the remaining candidates and marks 
	each candidate for which one of the following is true:
	\begin{itemize}
	\item the candidate has a unary indifferent preference
	\item the candidate has a numeric indifferent preference
	\end{itemize}
	This filter then checks every candidate that is not one of the above two types
	to see if it has a binary indifferent preference with every other candidate.
	If one of the candidates fails this test, then the procedure signals a tie impasse
	and returns the complete set of candidates that were passed into the 
	IndifferentFilter. Otherwise, the candidates are mutually indifferent, in which case 
	an operator is chosen according to the method set by the 
	\textbf{indifferent-selection} command, described on 
	page \pageref{indifferent-selection}.
\index{preference!indifferent}
\index{indifferent preference}
\end{description}


%\include{a-using}
%\include{a-default}
%% ----------------------------------------------------------------------------
\typeout{--------------- appendix: calculation of o-SUPPORT -----------------}
\chapter{The Calculation of O-Support}
\label{SUPPORT}
\index{support}
\index{i-support}
\index{o-support}
\index{persistence}

This appendix provides a description of when a preference is given o-support by an instantiation (a preference that is not given o-support will have i-support). Soar has four possible procedures for deciding support, which can be selected among with the o-support-mode command (see page \pageref{o-support-mode}). However, only o-support modes 3 \& 4 can be considered current to Soar 8, and o-support mode 4 should be considered an improved version of mode 3.   The default o-support mode is mode 4.

In o-support modes 3 \& 4, support is given by production; that is, all preferences generated by the RHS of a single instantiated production will have the same support. 


In both modes, a production must meet the following two requirements to create o-supported preferences:
\begin{enumerate}
\item The RHS has no operator proposals, i.e. nothing of the form \begin{verbatim}(<s> ^operator <o> +) \end{verbatim}
\item The LHS has a condition that tests the current operator, i.e. something of the form
\footnote{Sometimes, o-support mode 3 does not notice that this condition is true. This is a bug, which is unlikely to be fixed, since users are encouraged to use mode 4.}
\begin{verbatim}(<s> ^operator <o>)\end{verbatim}
\comment{this is only true if mode 3's checks are improved}
\end{enumerate}



In condition 1, the variable \soar{<s>} must be bound to a state identifier.
In condition 2, the variable \soar{<s>} must be bound to the lowest state identifier. That is to say, each (positive) condition on the LHS takes the form \soar{(id \carat attr value)}, some of these id's match state identifiers, and the system looks for the deepest matched state identifier. The tested current operator must be on this state. For example, in the production-

\begin{verbatim}
sp {elaborate*state*operator*name
  (state <s> ^superstate <s1>)
  (<s1> ^operator <o>)
  (<o> ^name <name>)
-->
  (<s> ^name something)}
\end{verbatim}


the RHS action gets i-support. Of course, the state bound to \soar{<s>} is destroyed when \soar{(<s1> \carat operator <o>)} retracts, so o-support would make little difference. On the other hand, the production- 

\begin{verbatim}
sp {operator*superstate*application
   (state <s> ^superstate <s1>)
   (<s> ^operator <o>)
   (<o> ^name <name>)
 -->
   (<s1> ^sub-operator-name <name>)}
\end{verbatim}

gives o-support to its RHS action, which remains after the substate bound to \soar{<s>} is destroyed. 


There is a third condition that determines support, and it is in this condition that modes 3 \& 4 differ. An extension of condition 1 is that operator augmentations should always receive i-support. Soar has been written to recognize augmentations directly off the operator (ie, \soar{(<o> \carat augmentation value)}), and to attempt to give them i-support. However, there was some confusion about what to do about a production that simultaneously tests an operator, doesn't propose an operator, adds an operator augmentation, and adds a non-operator augmentation, such as-

\begin{verbatim}
sp {operator*augmentation*application
  (state <s> ^task test-support
  	      ^operator <o>)
-->
   (<o> ^new augmentation)
   (<s> ^new augmentation)}
\end{verbatim}


In o-support mode 3, both RHS actions receive o-support; in o-support mode 4, both receive i-support. In either case, Soar will print a warning on firing this production, because this is considered bad coding style.

\nocomment{Support calculations are done at run time, as each production is fired. Could these decisions be done at compile time? Much of the decision is based on the structure of the production, which could be analyzed once as the production was loaded or chunked. However, it may be impossible to guarantee that a variable will be bound to a state id just by examining production syntax. Another issue is whether the state tested in condition 2 is the lowest state - this potentially could differ from instantiation to instantiation. For instance the operator*augmentation*application production above could match against multiple states in the state stack. 

 
%-----------------------------------------------------------
\section{Possible problems with implementation of modes 3 \& 4}

\begin{enumerate}
\item Default mode is actually o-support mode 3. Do we not want 4 to be default?
\item There is still the bug Andy pointed out. In condition 1, the variable \soar{<s>} is \textit{supposed} to be bound to a state variable, but the code does not actually check for this.
\item There is one additional, strange difference between modes 3 \& 4. In condition 3, the \soar{id} of each RHS action is tested to see if it is the id of the operator. This id is represented either as a symbol or as a rete location. Mode 4 tests the id both as a symbol and as a rete location, while mode 3 does only the symbol test. The rete test should be added to mode 3.
\end{enumerate}


\section{O-support modes 1 \& 2}

In o-support modes 1 \& 2, there are some of the same calculations as in 3 \& 4 when a production is matched (which occurs when a wme is added to the rete). In particular, if it is an operator proposal, it is set as IE\_PRODS. Otherwise, if it tests the current operator, it is set as PE\_PRODS, without testing for operator  elaborations. The match is placed on the appropriate dll, according to IE\_PRODS or PE\_PRODS.

Later, when the production is instantiated and the new preferences are built, there are no support calculations for 3 \& 4. But 1 \& 2 have support calculations. I suppose that the purpose of the earlier support calculations is that it places the production on the proper list to be fired during apply or propose,that is, whether it is an IE\_PROD or a PE\_PROD.

During this instantiation process, the function calculate\_support\_for\_instantiation\_preferences() is called to redo support IF the variable need\_to\_do\_support\_calculations is set to TRUE. This variable can be true only when-

\begin{enumerate}
\item  called from chunk\_instantiation OR
\item  \#ifndef SOAR\_8\_ONLY
SOAR\_8\_ONLY is a compile option, which is not defined by default. I think that its purpose is that, when defined, there is no run-time option to switch out of Soar 8. This allows a significant portion of code to be left out. Check out function Soar\_Operand2. 
\end{enumerate}


Mode 2 computes support in what is called 'Doug Pearson's way', which is described as-

 \begin{verbatim}
 For a particular preference p=(id ^attr ...) on the RHS of an
   instantiation [LHS,RHS]:

   RULE #1 (Context pref's): If id is the match state and attr="operator", 
   then p does NOT get o-support.  This rule overrides all other rules.

   RULE #2 (O-A support):  If LHS includes (match-state ^operator ...),
   then p gets o-support.

   RULE #3 (O-M support):  If LHS includes (match-state ^operator ... +),
   then p gets o-support.

   RULE #4 (O-C support): If RHS creates (match-state ^operator ... +/!),
   and p is in TC(RHS-operators, RHS), then p gets o-support.

   Here "TC" means transitive closure; the starting points for the TC are 
   all operators the RHS creates an acceptable/require preference for (i.e., 
   if the RHS includes (match-state ^operator such-and-such +/!), then 
   "such-and-such" is one of the starting points for the TC).  The TC
   is computed only through the preferences created by the RHS, not
   through any other existing preferences or WMEs.

   If none of rules 1-4 apply, then p does NOT get o-support.

   Note that rules 1 through 3 can be handled in linear time (linear in 
   the size of the LHS and RHS); rule 4 can be handled in time quadratic 
   in the size of the RHS (and typical behavior will probably be linear).
   
   
   What is 'match state'? The match goal for the instantiation.
   Match goal - (a match goal is associated with an instantiation).
   Look through instantiated LHS conditions.
   Find the lowest goal state matched to one of the condition's ids.
 \end{verbatim}  
   
O-support mode 1 computes Doug's support and compares it to the poor cousin of mode 3 \& 4 support calculations, ie calculation without checking for operator elaboration. It prints any differences it finds.

}
   
   
 \nocomment{
   
   
   
   
   
   
   

3. the RHS has no direct elaborations of the current operator, ie no actions of the form 
(<o> ^augmentation value).
However, an indirect elaboration such as
   (<o> ^name <d>)
   -->
   (<d> ^augmentation value)
will not prevent o-support.


In mode 3, an instantiation will generate o-supported preferences iff
1. the RHS has no operator proposals (nothing of the form (<s> ^operator <o> +))
2. the LHS has a condition that tests the current operator (something of the form 
(<s> ^operator <o>))
3. 


Operator proposal - a production whose RHS has action (<s> ^operator <o> +))
Operator test - 
	LHS has condition of the form (<s> ^operator <o>)
Operator elaboration -
	o_support_mode 3:
		
	o_support_mode 4:
	

o_support_mode 4: 
1. if an operator proposal - i-support
2. if not an operator test - i-support
3. if an operator test with no elaborations - o-support
4. if an operator test with some elaborations and some non-elaboration, non-function RHS action - i-support (warns)
5. if an operator test with only elaborations - i-support

o_support_mode 3:
1. if an operator proposal - i-support
2. if not an operator test - i-support
3. if an operator test with no elaborations - o-support
4. if an operator test with some elaborations and some non-elaboration, non-function RHS action - o-support (warns)
5. if an operator test with only elaborations - i-support


o_support_mode 0:
1. if an operator proposal - i-support
2. if test operator - o-support
3. else - i-support

}


%\include{SAN-preferences}
%\include{SAN-tcl-io}
% ----------------------------------------------------------------------------
\typeout{--------------- appendix: GDS ------------------}
\chapter[A Goal Dependency Set Primer]{
A Goal Dependency Set Primer\footnote{A preliminary draft by Robert Wray, contact at \texttt{wrayre@acm.org}.
}}

\label{GDS}
\index{GDS}


% a list of hyphenation points for re-occuring words in the document
\hyphenation{con-temp-or-an-e-ous}
\hyphenation{OP-ER-AND}
\hyphenation{Mich-i-gan}

%\pagestyle{myheadings}
%\markboth{GDS Primer}{DRAFT: Not for Quotation or Distribution}

%\input{macros}


      

% use optional labels to link authors explicitly to addresses:
% \author[label1,label2]{}
% \address[label1]{}
% \address[label2]{}
%\author{Robert Wray  \\  Soar Technology \\ 3600 Green Road Suite 600 \\ Ann Arbor, MI 48105 \\ (734)327-8000 \\ \texttt{wrayre@acm.org}  
%        }


%\maketitle                        %%%% To set Title and Author names.
%\thispagestyle{empty}

%%%% Replace with your Abstract.

%%%%%%%%%%%%%%%%%%%%%%%%%%%%%%%%%

This document briefly describes the Goal Dependency Set (GDS), which
was introduced with Soar~8.  There are three sections: a brief
discussion of the motivation for the GDS, a discussion of the
consequences of the GDS from a behavior developer/modeler's point of
view, and some details on the kernel implementation of the GDS, for
anyone working at the architecture level.  This document is by no
means complete, but introduces the GDS in Soar-specific terms.

\section*{Why the GDS was needed}

As a symbol system, Soar attempts to approximate the knowledge level
but will necessarily always fall short \cite{Newell90:UTC}.  We can
informally think of the way in which Soar falls short of the knowledge
level as its peculiar ``psychology.''  Those interested in using Soar
to model human psychology would like Soar's ``psychology'' to
approximate human psychology.  Those using Soar to create agent
systems would like to make Soar's processing approximate the knowledge
level as closely as possible.  However, Soar~7 had a number of
symbol-level ``quirks'' that appeared inconsistent with human
psychology and that made building large-scale, knowledge-based systems
in Soar more difficult than necessary.  Bob Wray's thesis 
\footnote{Robert E. Wray. \textit{Ensuring Reasoning Consistency in Hierarchical Architectures}. PhD thesis, University of Michigan, 1998.}
addressed many of these symbol-level problems
in Soar, among them logical inconsistency in symbol manipulations,
non-contemporaneous constraints in chunks \cite{Wray96:Compilation},
race conditions in rule firings and in the decision process, and
contention between original task knowledge and learned knowledge
\cite{Wray01:Resolving}.

The Goal Dependency Set implements a solution to logical
inconsistencies between persistent (o-supported) working memory
elements (WMEs) in a substate and its ``context''.  The context
consists of all the WMEs in any superstates above the local
goal/state\footnote{This report will use ``state,'' not ``goal.''  At
the kernel level, states are still called ``goals'' and ``goal'' is often
still used to refer to states.    As a result, a
confusion in terminology results, with ``\textbf{Goal} Dependency Set'' a 
specific example, even though ``goals'' have not been
an explicit, behavior-level Soar construct since Soar~6}.  In Soar, any
action (application) of an operator receives an o-support preference.
This preference makes the resulting WME persistent: it will remain in
memory until explicitly removed (or until its local state is removed),
regardless of whether it continues to be justified.

Persistent WMEs are pervasive in Soar, because operators are the main
unit of problem solving.  Persistence is necessary for taking any
non-monotonic step in a problem space.  However, persistent WMEs also
are dependent on WMEs in the superstate context.  The problem in
Soar~7, especially when trying to create large-scale systems like
TacAir-Soar \cite{Jones99:Automated}, is that the knowledge developer
must always think about which dependencies can be ``ignored'' and
which need to result in a reconsideration of the persistent WME.  For
example, imagine an exploration robot that makes a persistent decision
to travel to some distant destination based, in part, on its power
reserves.  Now suppose that the agent notices that its power reserves
have failed.  If this change is not communicated to the state where
the travel decision was made, the agent will continue to act as if its
full power reserves were still available.

Of course, for this specific example, the knowledge designer can
encode some knowledge to react to this inconsistency.  The fundamental
problem is that the knowledge designer has to consider \emph{all}
possible interactions between all o-supported WMEs and all contexts.
Soar systems often use the architecture's impasse mechanism to realize
a form of decomposition.  These potential interactions mean that the
knowledge developer cannot focus on individual problem spaces when
creating knowledge, which makes knowledge development more difficult.
Further, in all but the simplest systems, the knowledge designer will
miss some potential interactions.  The result is agents are that were
unnecessarily brittle, failing in difficult-to-understand,
difficult-to-duplicate ways.  

The GDS also solves the the problem of non-contemporaneous constraints
in chunks.  A non-contemporaneous constraint refers to two or more
conditions that never co-occur simultaneously.  An example might be a
driving robot that learned a rule that attempted to match ``red
light'' and ``green light'' simultaneously. Obviously, for functioning
traffic lights, this rule would never fire.  By ensuring that local
persistent elements are always consistent with the higher-level
context, non-contemporaneous constraints in chunks are
\emph{guaranteed} not to happen.


The GDS captures context dependencies during processing, meaning the
architecture will identify and respond to inconsistencies
automatically.  The knowledge designer then does not have to consider
potential inconsistencies between local, o-supported WMEs and the
context.  The following sections describe further how the GDS works
and how to use the GDS in behavior systems, as well as how the GDS is
implemented in the Soar kernel.


\section*{Behavior-level view of the Goal Dependency Set}

This section discusses what the GDS does, and how that impacts
production knowledge design and implementation.

\subsection*{Operation of the Goal Dependency Set}


\begin{figure}
\insertfigure{simple-ncc}{3in}
\caption{Simplified Representation of the context dependencies (above the line), local os-upported WMEs (below the line), and the generation of a result.  In Soar~7, this situation led to non-contemporaneous constraints in the chunk that generates {\bf 3}.}
\label{'ncc'}
\end{figure}

Whenever a feature is created (added to working memory) in the Soar~7
architecture, that feature will persist for some time.  The
persistence of features may differ with respect to how long the
features remain in memory, and more importantly, what circumstances
cause the feature to be removed.  The Soar~7 architecture utilizes
three primary types of persistence: i-support, o-support, and
c-support.

The weakest persistence is instantiation support.  An i-supported
feature exists in memory only as long as the production which lead to
the feature's creation remains instantiated.  Thus, the WME depends
upon this production instantiation (and, more specifically, the
features the instantiation tests).  When one of the conditions in the
production instantiation no longer matches, the instantiation is
retracted, resulting in the loss of the acceptable preference for the
WME.\footnote{Importantly, in a technical sense, the WME is only
retracted when it loses instantiation support, not when the creating
production is retracting.  For example, a WME could receive i-support
from several different instantiations and the retraction of one would
not lead to the retraction of the WME.  However, the the following
generally discusses direct dependency unmediated by preferences,
ignoring this complication for clarity.}  I-support is illustrated in
Figure~\ref{'ncc'}. A copy of {\bf A} in the subgoal, {\bf A$_s$}, is
retracted automatically when {\bf A} changes to {\bf A'}.  The
substate WME persists only as long as it remains justified by {\bf A}.
This justification is called ``instantiation support'' (I-support) in
Soar (and should not be confused with result \emph{justifications}.)

In the broadest sense, we can say that some feature $<$b$>$ is
``dependent'' upon another element $<$a$>$ if $<$a$>$ was used in the
creation of $<$b$>$, i.e., if $<$a$>$ was tested in the production
instantiation that created $<$b$>$.  Further, a dependent change with
respect to feature $<$b$>$ is a change to any of its instantiating
features.  In Figure~\ref{'ncc'}, the change from {\bf A} to {\bf A'}
is a dependent change for feature {\bf 1} because {\bf A} was used to
create {\bf 1}.

In Soar 7, some features are insensitive to dependent changes.  These
features are often referred to as ``persistent WMEs'' because, unlike
i-supported WMEs, they remain in memory until explicitly removed.
There are two different types of this stronger persistence: o-support
and c-support.  

Any feature created by the action of an operator
receives ``operator support.''  An o-supported feature remains in
memory until explicitly rejected (or until the superstructure to which
it is attached is removed).  Removal is architecturally
independent of the WME's instantiating conditions.

Context-support affects the persistence of an operator itself, rather
than its effects.  Once a unique operator has been chosen by the
decision procedure, the choice persists until explicitly re-decided
(via a reconsider preference).  C-support ensures that the WME for a
selected operator remains available even if the production that
proposed the operator is no longer instantiated.  Soar~8 eliminates
c-support, so that operators now persist only as long as they receive
instantiation support.  This change was integral to the overall
solution Soar~8 provides, but is distinct from the GDS.

The GDS provides a solution to the first problem.  When {\bf A}
changes, the persistent WME {\bf 1} may be no longer consistent with
its context (e.g., {\bf A'}).  The specific solution is inspired by
the chunking algorithm.  In Soar~8, whenever an o-supported WME is
created in the local state, the superstate dependencies of that new
feature are determined and added to the {\em goal dependency set}
(GDS) of that state. Conceptually speaking, whenever a working memory
change occurs, the dependency sets for every state in the context
hierarchy are compared to working memory changes.\footnote{The
implementation is slightly different, trading additional memory
overhead to avoid scanning all the goal dependency sets after each WM
change.  See the next section.  }  If a removed element is found in a
GDS, the state is removed from memory (along with all existing
substructure). The dependency set includes only dependencies for
o-supported features.  For example, in Figure~\ref{'gds'}, at time
$t_0$, because only i-supported features have been created in the
subgoal, the dependency set is empty.

\begin{figure}
\insertfigure{gomor-o-support}{3in}
\caption{The Dependency Set in Soar~8.}
\label{'gds'}
\end{figure}


Three types of features can be tested in the creation of an
o-supported feature.  Each requires a slightly different type of
update to the dependency set.
\begin{description}
\item [Elements in the superstate:] WMEs in the superstate are added
directly to the goal's dependency set.  In Figure~\ref{'gds'}, the
persistent subgoal item {\bf 3} is dependent upon {\bf A} and {\bf
D}. These superstate WMEs are added to the subgoal's dependency set when
{\bf 3} is added to working memory at time $t_1$.  It does not matter
that {\bf A} is i-supported and {\bf D} o-supported.\footnote{In addition,
superstate WMEs can also include context slot preferences, which 
are represented in the architecture as working memory elements.}
\item [Local I-Supported Features:] Local i-supported features are not
added to the goal dependency set.  Instead, the superstate WMEs that
led to the creation of the i-supported feature are determined and
added to the GDS.  In the example, when {\bf 4} is created, {\bf A},
{\bf B} and {\bf C} must be added to the dependency set because they
are the superstate features that led to {\bf 1}, which in turn led to
{\bf 2} and finally {\bf 4}.  However, because item {\bf A} was
previously added to the dependency set at $t_1$, it is unnecessary to
add it again.
\item [Local O-Supported Features:] The dependencies of a local
o-supported feature have already been added to the state's GDS.  Thus,
tests of local o-supported WMEs do not require additions to the
dependency set.  In Figure~\ref{'gds'}, the creation of element {\bf
5} does not change the dependency set because it is dependent only
upon persistent items {\bf 3} and {\bf 4}, whose features had been
previously added to the GDS.
\end{description}

In Soar~8, any change to the current dependency set will cause
the retraction of all subgoal structure.  Thus, any time after time
$t_1$, either the {\bf D} to {\bf D'} or {\bf A} to {\bf A'}
transition would cause the removal of the entire subgoal. The {\bf E}
to {\bf E'} transition causes no retraction because {\bf E} is not in
the goal's dependency set.

\subsection*{The role of the GDS in agent design}

The GDS places some design time constraints on operator implementation.
These constraints are:
\begin{itemize}
\item Operator actions that are used to remember a previous state/situation should be asserted in the top state
\item All operator elaborations should be i-supported
\item Any operator with local actions should be designed to be re-entrant
\end{itemize}
This section describes these issues.

Soar says any operator effect is o-supported, regardless of whether
that assertion is entailed by the current situation, or whether it
reflects an assumption about it.  The GDS adds additional (needed)
constraint.  Because any context dependencies for subgoal, o-supported
assertions will be added to the GDS, the developer must decide if an
o-supported element should be represented in a substate or the top
state.

This decision is straightforward if the functional role of the
persistent element is considered.  Four important capabilities that
require persistence are:
\begin{enumerate}

\item \textbf{Reasoning hypothetically:} ~ Some assertions may need to
reflect hypothetical states.  Such assertions are ``assumptions''
because a hypothetical inference cannot always be grounded in the
current context.  In other problem solvers with truth maintenance,
only assumptions are persistent.

\item \textbf{Reasoning non-monotonically:} ~
Sometimes the result of an inference changes one of the assertions on
which the inference is dependent.  As an example, consider the task of
counting.  Each newly counted item replaces the old value of the
count. 

\item \textbf{Remembering:} ~
Agents oftentimes need to remember an external situation or stimulus,
even when that perception is no longer available.  

\item \textbf{Avoiding Expensive Computations:} ~ In some situations,
an agent may have the information needed to assert some belief in a
new world state but the expense of performing the computation
necessary for the assertion, given what is already known, makes the
computation avoidable.  For example, in dynamic, complex domains,
determining when to make an expensive calculation is often formulated
as an explicit agent task \cite{Jones99:Automated}.
\end{enumerate}

When remembering or avoiding an expensive computation, the
agent/designer is making a commitment to retain something even though
it might not be supported in the current context.  In Soar~8, these
WMEs should be asserted in the top state.  \emph{For many Soar systems,
especially those focused on execution in a dynamic environment, 
most o-supported elements will need to be stored on the top state.} 

For any kind of local, non-monotonic reasoning about the context
(counting, projection planning), features should be stored locally.
When a dependent context change occurs, the GDS interrupts the
processing by removing the state.  While this may seem like a severe
over-reaction, formal and empirical analysis have suggested that this
solution is less computationally expensive than attempting to identify
the specific dependent assumption \cite{Wray03:Ensuring}.


\subsection*{Operator Elaborations}

Operator elaborations (i.e., placing some information on an operator
WME) should be i-supported when using Soar~8, since this information
is, by definition, temporary/not persistent (because it's located on
the non-persistent operator).  However, the kernel itself hasn't kept
up with this change.  Prior to Soar~8.5, Soar's o-support modes
computed operator elaborations as o-supported, resulting in the
context conditions being added to the GDS.  This often leads to
unwanted/unnecessary retractions.  If you are using a version prior to
Soar~8, you should declare any operator elaborations i-supported (i.e.,
using :i-support).




\section*{Kernel-level view of the Goal Dependency Set}


The actual implementation of the GDS in the Soar kernel is slightly
more complex than the conceptual description of the previous section
(but not significantly so).  

Elements are added to the GDS via \verb+elaborate_gds()+, a procedure in
decide.c that mimics the chunking backtrace function.  The algorithm
is shown in Figure~\ref{tab:dhj:proc}.  When an o-supported preference
is asserted, elaborate\_gds() is called.  Conditions in a production
instantiation that are located in a higher context can be added
directly to the GDS (1).  For local conditions, \verb+elaborate_gds()+ first
checks whether the tested WME is o-supported, or if it has been
previously been back traced through (2). If either of these are true,
the WME can be ignored because it's dependencies have been added to
the GDS previously.  If not, \verb+elaborate_gds()+ is called recursively,
to find the context dependencies for the local, contributing WME,~$c$
(3).

\begin{figure}[h]
%\rule{\textwidth}{.5mm}
\framebox[\textwidth]{
\begin{minipage}{\textwidth}
\begin{tabbing}
xxx\=xxx\=xxx\=xxx\=xxx\=xxxxxxxxxxxxxxxxxxxx\= \kill

\textbf{PROC} $create\_new\_assertion(\ldots)$ \\
\> Whenever a new o-supported element is asserted, the GDS is updated \\ 
\> to include any new context dependencies.  \\
\> $\ldots$\\
\> $A_{inst} \leftarrow $ instantiation that asserted acceptable preference for A  \\
\> \textbf{IF} A is an o-supported WME\\
%\>\>$A_{goal_{GDS}}$ : = $append(A\rightarrow goal\rightarrow GDS$ \\
%\>\>$G \leftarrow A_{goal} \quad$ 
\>\>G is the goal/state in which A is asserted \\
\>\>$G_{GDS} \leftarrow append(G_{GDS}, elaborate\_GDS(A)) $ \\

\>$\ldots$\\
\textbf{END} \\

\\
\textbf{PROC} $elaborate\_GDS(assertion\, A)$ \\
\> $S \leftarrow \{ NIL \} $ \\
        \>\textbf{FOR} Each assertion  $c$ in $A_{inst}$, the instantiation supporting A \\
$\bigcirc \! \! \! \! 1$      
          \>\>     \textbf{IF} $\left\{ GoalLevel(c) \quad\mbox{closer to top state than}\quad GoalLevel(A) \right\}$ \\

           \>\>\>\>              $append(c,S)\quad$ (append context dependency to GDS) \\
\\
$\bigcirc \! \! \! \! 2$
             \>\>   \textbf{ELSEIF} \{ \>\>\> $GoalLevel(c) \quad$ same as $GoalLevel(A)  \quad\mbox{AND}\quad $ \\
              \>\>\> \> \>    $c$ is NOT an o-supported WME $\quad\mbox{AND}\quad $ \\
\>\>\> \> \>  $c$ has not previously been inspected \} \\
$\bigcirc \! \! \! \! 3$               \>\>\>\>          $S \leftarrow append(S,elaborate\_GDS(c))$ \\
\>\>\>\>\>(compute GDS dependencies for $c$ and add to goal's GDS) \\
$\bigcirc \! \! \! \! 4$               \>\>\>\>          $c_{inspected} \leftarrow true \quad $ \\
\>\>\>\>\>($c$'s context dependencies have been added to the GDS;  \\
\>\>\>\>\>~~ no need to consider it again for this GDS)


\\
\> return S, the list of new dependencies in the GDS \\ 
\textbf{END} \\


%         \>\>\textbf{END(IF)} \\
%\>\textbf{END(FOR)} \\
%\textbf{END(PROC)} \\

\\
\textbf{PROC} $GoalLevel(assertion \quad A)$ \\
\> Return the goal level associated with assertion A

\end{tabbing}
\end{minipage}
}
%\rule{\textwidth}{.5mm}
\caption{The algorithm for determining members of the GDS.}
\label{tab:dhj:proc}
\end{figure}


When WME changes occur, each goal/state must be checked to determine
if the WME appeared on that goal's GDS. Because WME changes occur in
nearly every Soar elaboration cycle, we chose to extend the WME data
structure to avoid this scanning.  Figure~\ref{wme} illustrates the
relationship.  Each GDS structure consists of a pointer to its goal and a
pointer to a linked list of WMEs.  The \verb+gds_next+ and \verb+gds_prev+ pointers on
the WME structure define the GDS WMEs for a particular GDS and the GDS pointer
provides a link back from each GDS WME to the GDS data structure.

When a WME is removed, the GDS pointer can be checked to determine
immediately if the goal should be removed.  No scanning is necessary.


\begin{figure}
\insertfigure{gds_wme}{3in}
\caption{The GDS and WME data structures}.
\label{wme}
\end{figure}

\subsection*{Other implementation issues}

\begin{itemize}

\item Allocating memory for the GDS \\ The GDS memory is created for
each goal when the goal is created.  The GDS is deallocated when the
goal is removed.  A NIL WME pointer for the GDS indicates a goal has
no WMEs in its GDS.
\item Updating a WME GDS pointer \\ A WME should appear in only the
GDS of the highest goal for which it is dependent.  If a WME is
determined to already be in a GDS lower than the current goal, its GDS
pointer is updated to the higher goal, it is removed from the gds\_WME
DLL of the lower goal, and added to the higher one.  If there are no
other WMEs on the gds\_WME DLL of the lower goal, its WME pointer is
set to NIL (the GDS itself is retained, because we don't want to have
to reallocate memory for the GDS if we need to add to it later.)

\end{itemize}



%\bibliography{general,personal,soar}
%\bibliographystyle{acm}
 




% ----------------------------------------------------------------------------
% References
% ----------------------------------------------------------------------------
%\addcontentsline{toc}{chapter}{Bibliography}
%\bibliography{soarmanual} 

%---------------------------------------------------------------------------
%\vspace{\fill}
%\subsection*{Colophon}
%\addcontentsline{toc}{chapter}{Colophon}
%
%This document was produced on a Sun workstation using \LaTeX 2$_\epsilon$.
%Illustrations were created using idraw.
%

% ----------------------------------------------------------------------------
% Index
%   the file manual.idx is generated by latex; run 'makeindex manual' to
%     create the file manual.ind. However, this has a number of special
%     characters in it which will have to be put in verbose mode to be readable.
%   The characters that have to be changed are mostly at the top of the file;
%     you'll also have to look for all the underscores and change _ to \_ so
%     that latex won't choke.
%   Another option might be to do the whole index in typewriter font, but I
%     haven't tried this yet (maybe for draft versions, at least?). The carat
%     symbol still won't work (replace with \carat), but everything else
%     probably will.
%   In case it isn't obvious, generating the index is one of the last things
%     to do.
% ----------------------------------------------------------------------------
\cleardoublepage % had to add these things to get the clickable link in the pdf to 
\phantomsection  % link to the right page
\addcontentsline{toc}{chapter}{Index}
\small
\twocolumn
\printindex
\onecolumn

% ----------------------------------------------------------------------------
% Function Summary
% ----------------------------------------------------------------------------
\cleardoublepage
\phantomsection
\addcontentsline{toc}{chapter}{Summary of Soar Aliases, Variables, and Functions}
%\markboth{SUMMARY OF SOAR FUNCTIONS}{SUMMARY OF SOAR FUNCTIONS} ADD BACK IN
%\def\leftmark{\textit{SUMMARY OF SOAR FUNCTIONS}}
%\def\rightmark{\textit{SUMMARY OF SOAR FUNCTIONS}}
% ----------------------------------------------------------------------------
\typeout{--------------- FUNCTION SUMMARY AND INDEX -------------------------}
%\pagestyle{empty}
\markboth{}{}
\section*{Summary of Soar Aliases and Functions}
\label{FUNCTIONS}
\label{func-sum}

% ----------------------------------------------------------------------------
\subsection*{Predefined Aliases}\vspace{-5pt}
%\newpage
\label{predefined-aliases}

There are a number of Soar ``commands'' that are shorthand for other Soar
commands:

\begin{longtable}[l]{@{\extracolsep{\fill}}p{3cm}p{6cm}p{1cm}@{}}
	Alias  & Command & Page \\ \hline
\end{longtable}
\vspace{-18pt}
\begin{footnotesize}
%\begin{longtable}{ l l r }
\begin{longtable}[c]{@{}p{5cm}p{8cm}p{5cm}@{}}
\soar{?}  & \soar{help} & \pageref{help}\\
\soar{a}  & \soar{alias} & \pageref{alias}\\
\soar{add-wme} & \soar{wm add} & \pageref{wm-add}\\
\soar{allocate} & \soar{debug allocate} & \pageref{debug-allocate}\\
\soar{aw} & \soar{wm add} & \pageref{wm-add}\\
\soar{c} & \soar{explain chunk} & \pageref{explain-chunk}\\
\soar{capture-input} & \soar{save percepts} & \pageref{save-percepts}\\
\soar{chdir} & \soar{cd} & \pageref{cd}\\
\soar{chunk-name-format} & \soar{chunk naming-style} & \pageref{chunk}\\
\soar{cli} & \soar{soar tcl} & \pageref{soar-tcl}\\
\soar{clog} & \soar{output log} & \pageref{output-log}\\
\soar{command-to-file} & \soar{output command-to-file} & \pageref{output-command-to-file}\\
\soar{cs} & \soar{chunk stats} & \pageref{debugging-explanation-based-chunking}\\
\soar{cts} & \soar{output command-to-file} & \pageref{output-command-to-file}\\
\soar{d}  & \soar{run -d 1} & \pageref{run}\\
\soar{dir} & \soar{ls} & \pageref{ls}\\
\soar{e}  & \soar{run -e 1} & \pageref{run}\\
\soar{echo-commands} & \soar{output echo-commands} & \pageref{output-echo-commands}\\
\soar{ef} & \soar{explain formation} & \pageref{explain-formation}\\
\soar{ei} & \soar{explain identities} & \pageref{explain-identity}\\
\soar{es} & \soar{explain stats} & \pageref{explain-stats}\\
\soar{et} & \soar{explain explanation-trace} & \pageref{explain-explanation-trace-and-wm-trace}\\
\soar{excise} & \soar{production excise} & \pageref{production-excise}\\
\soar{fc} & \soar{production firing-counts} & \pageref{production-firing-counts}\\
\soar{firing-counts} & \soar{production firing-counts} & \pageref{production-firing-counts}\\
\soar{gds\_print} & \soar{print --gds} & \pageref{print}\\
\soar{gp-max} & \soar{soar max-gp} & \pageref{soar-max-gp}\\
\soar{h} & \soar{help} & \pageref{help}\\
\soar{i} & \soar{explain instantiation} & \pageref{explain-instantiation}\\
\soar{indifferent-selection} & \soar{decide indifferent-selection} & \pageref{decide-indifferent-selection}\\
\soar{inds} & \soar{decide indifferent selection} & \pageref{decide-indifferent-selection}\\
\soar{init} & \soar{soar init} & \pageref{soar-init}\\
\soar{internal-symbols} & \soar{debug internal-symbols} & \pageref{debug-internal-symbols}\\
\soar{interrupt} & \soar{soar stop} & \pageref{soar-stop}\\
\soar{is} & \soar{soar init} & \pageref{soar-init}\\
\soar{learn} & \soar{chunk} & \pageref{chunk}\\
\soar{load-library} & \soar{load library} & \pageref{load-library}\\
\soar{man} & \soar{help} & \pageref{help}\\
\soar{matches} & \soar{production matches} & \pageref{production-matches}\\
\soar{max-chunks} & \soar{chunk max-chunks} & \pageref{chunk-max-chunks}\\
\soar{max-dc-time} & \soar{soar max-dc-time} & \pageref{soar-max-dc-time}\\
\soar{max-elaborations} & \soar{soar max-elaborations} & \pageref{soar-max-elaborations}\\
\soar{max-goal-depth} & \soar{soar max-goal-depth} & \pageref{soar-max-goal-depth}\\
\soar{max-memory-usage} & \soar{soar max-memory-usage} & \pageref{soar-max-memory-usage}\\
\soar{max-nil-output-cycles} & \soar{soar max-nil-output-cycles} & \pageref{soar-max-nil-output-cycles}\\
\soar{memories} & \soar{production memory-usage} & \pageref{production-memory-usage}\\
\soar{multi-attributes} & \soar{production optimize-attribute} & \pageref{production-optimize-attribute}\\
\soar{numeric-indifferent-mode} & \soar{decide numeric-indifferent-mode} & \pageref{decide-numeric-indifferent-mode}\\
\soar{p}  & \soar{print} & \pageref{print}\\
\soar{pbreak} & \soar{production break} & \pageref{production-break}\\
\soar{pc} & \soar{print --chunks} & \pageref{print}\\
\soar{port} & \soar{debug port} & \pageref{debug-port}\\
\soar{predict} & \soar{decide predict} & \pageref{decide-predict}\\
\soar{production-find} & \soar{production find} & \pageref{production-find}\\
\soar{ps} & \soar{print --stack} & \pageref{print}\\
\soar{pw} & \soar{production watch} & \pageref{production-watch}\\
\soar{pwatch} & \soar{production watch} & \pageref{production-watch}\\
\soar{quit} & \soar{exit} & \pageref{exit}\\
\soar{r} & \soar{run} & \pageref{run}\\
\soar{remove-wme} & \soar{wm remove} & \pageref{wm-remove}\\
\soar{replay-input} & \soar{load percepts} & \pageref{load-percepts}\\
\soar{rete-net} & \soar{load rete-network} & \pageref{load-rete-network}\\
\soar{rn} & \soar{load rete-network} & \pageref{load-rete-network}\\
\soar{rw} & \soar{wm remove} & \pageref{wm-remove}\\
\soar{s} & \soar{run 1} & \pageref{run}\\
\soar{select} & \soar{decide select} & \pageref{decide-select}\\
\soar{set-default-depth} & \soar{output print-depth} & \pageref{output-print-depth}\\
\soar{set-stop-phase} & \soar{soar stop-phase} & \pageref{soar-stop-phase}\\
\soar{soarnews} & \soar{soar} & \pageref{soar}\\
\soar{source} & \soar{load file} & \pageref{load-file}\\
\soar{srand} & \soar{decide srand} & \pageref{decide-set-random-seed}\\
\soar{ss} & \soar{soar stop} & \pageref{soar-stop}\\
\soar{st} & \soar{stats} & \pageref{stats}\\
\soar{step} & \soar{run -d 1} & \pageref{run}\\
\soar{stop} & \soar{soar stop} & \pageref{soar-stop}\\
\soar{stop-soar} & \soar{soar-stop} & \pageref{soar-stop}\\
\soar{tcl} & \soar{soar tcl} & \pageref{soar-tcl}\\
\soar{time} & \soar{debug time} & \pageref{debug-time}\\
\soar{timers} & \soar{soar timers} & \pageref{soar-timers}\\
\soar{topd} & \soar{pwd} & \pageref{pwd}\\
\soar{un} & \soar{alias -r} & \pageref{alias}\\
\soar{unalias} & \soar{alias -r} & \pageref{alias}\\
\soar{varprint} & \soar{print -v -d 100} & \pageref{print}\\
\soar{verbose} & \soar{trace -A} & \pageref{trace}\\
\soar{version} & \soar{soar version} & \pageref{soar-version}\\
\soar{w}  & \soar{trace} & \pageref{trace}\\
\soar{waitsnc} & \soar{soar wait-snc} & \pageref{soar-wait-snc}\\
\soar{warnings} & \soar{output warnings} & \pageref{output-warnings}\\
\soar{watch} & \soar{trace} & \pageref{trace}\\
\soar{watch-wmes} & \soar{wm watch} & \pageref{wm-watch}\\
\soar{wma} & \soar{wm activation} & \pageref{wm-activation}\\
\soar{wmes} & \soar{print -depth 0 -internal} & \pageref{print}\\
\soar{wt} & \soar{explain wm-trace} & \pageref{explain-explanation-trace-and-wm-trace}\\
\end{longtable}
\end{footnotesize} \vspace{24pt}

\newpage
% ----------------------------------------------------------------------------
\newpage
\subsection*{Summary of Soar Functions}

The following table lists the commands in Soar. See the referenced page number
for a complete description of each command.

\begin{small}
\begin{longtable}{ l p{10cm} r }
Command  & Summary & Page \\  \hline
\input{funclist}
\end{longtable}
\end{small}




\end{document}
