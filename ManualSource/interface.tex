% ----------------------------------------------------------------------------
\typeout{--------------- The Soar User INTERFACE -----------------------------}
\chapter{The Soar User Interface}
\label{INTERFACE}
\index{interface}
%\index{user interface}
%\index{function definitions}

\nocomment{for each command, use the 'funsum' command with a brief
	description. This writes to the manual.glo file which can be edited
	into the funtion summary and index (see that file for more
	instructions). This is a bit tedious, but the reason I've set it up
	this way is that the command set is in flux right now -- this lessens
	the chance that a command will be inadvertently omitted from the
	function summary (or that a defunct command will be inadvertently
	included). 
	}

\nocomment{\begin{figure}[h]
\psfig{figure=dilbert-living.ps,height=2.2in} \vspace{12pt}
\end{figure}
}
% ----------------------------------------------------------------------------


This chapter describes the set of user interface commands for Soar. All commands and examples are presented as 
if they are being entered at the Soar command prompt.

This chapter is organized into 7 sections:
\begin{enumerate}
\item Basic Commands for Running Soar
\item Examining Memory
\item Configuring Trace Information and Debugging
\item Configuring Soar's Run-Time Parameters
\item File System I/O Commands
\item Soar I/O commands
\item Miscellaneous Commands
\end{enumerate}

Each section begins with a summary description of the commands covered
in that section, including the role of the command and its importance
to the user.  Command syntax and usage are then described fully, in
alphabetical order.

The following pages were automatically generated from the wiki version
at

\hspace{2em}\soar{\htmladdnormallink{http://code.google.com/p/soar/wiki/CommandIndex}{http://code.google.com/p/soar/wiki/CommandIndex}}


on the date listed on the title page of this manual.  Please consult
the wiki directly for the most accurate and up-to-date information.

For a concise overview of the Soar interface functions, see the Function
Summary and Index on page \pageref{func-sum}. This index is intended to be a
quick reference into the commands described in this chapter.

\subsubsection*{Notation}

\nocomment{check for all commands that I've got the notation current}

The notation used to denote the syntax for each user-interface command follows
some general conventions:\vspace{-12pt}
\begin{itemize}
\item The command name itself is given in a \soarb{bold} font.\vspace{-8pt}
\item Optional command arguments are enclosed within square brackets,
	\soar{[} and \soar{]}.\vspace{-8pt}
\item A vertical bar, \soar{|}, separates alternatives.\vspace{-8pt}
\item Curly braces, \soar{\{\}}, are used to group arguments when at least
one argument from the set is required.
\item The commandline prompt that is printed by Soar, is normally
the agent name, followed by '\soar{>}'.  In the examples in this manual, 
we use ``\soar{soar>}''.
\item Comments in the examples are preceded by
a '\soar{\#}', and in-line comments are preceded by '\soar{;\#}'.
\end{itemize}

For many commands, there is some flexibility in the order in which the
arguments may be given. (See the online help for each command for more
information.)  We have not incorporated this flexible ordering into the syntax
specified for each command because doing so complicates the specification of
the command.  When the order of arguments will affect the output
produced by a command, the reader will be alerted.

% ----------------------------------------------------------------------------
\section{Basic Commands for Running Soar}
\label{BASIC}

This section describes the commands used to start, run and stop a Soar 
program; to invoke on-line help information; and to create and 
delete Soar productions.  The specific commands described in this
section are:

\paragraph{Summary}
\begin{quote}
\begin{description}
%\item[d] - Run the Soar program for one decision cycle.
%\item[e] - Run the Soar program for one elaboration cycle.
\item[excise] - Delete Soar productions from production memory.
%\item[exit] - Terminate Soar and return to the operating system.
\item[gp] - Define a pattern used to generate and source a set of Soar productions.
\item[gp-max] - Set the upper-limit to the number of productions generated by the gp command.
\item[help] - Provide formatted, on-line information about Soar commands.
\item[init-soar] - Reinitialize Soar so a program can be rerun from scratch.
\item[run] - Begin Soar's execution cycle.
\item[sp] - Create a production and add it to production memory.
\item[stop-soar] - Interrupt a running Soar program.
\end{description}
\end{quote}
These commands are all frequently used anytime Soar is run.

\subsection{\soarb{excise}}
\label{excise}
\index{excise}
Delete Soar productions from production memory. 
\subsubsection*{Synopsis}
excise production_name [production_name ...]
excise -[acdtu]
\end{verbatim}
\subsubsection*{Options}
\hline
\soar{\soar{\soar{\soar{ -a, --all }}}} & Remove all productions from memory and perform an init-soar command  \\
\hline
\soar{\soar{\soar{\soar{ -c, --chunks }}}} & Remove all chunks (learned productions) and justifications from memory  \\
\hline
\soar{\soar{\soar{\soar{ -d, --default }}}} & Remove all default productions (:default) from memory  \\
\hline
\soar{\soar{\soar{\soar{ -t, --task }}}} & Remove chunks, justifications, and user productions from memory  \\
\hline
\soar{\soar{\soar{\soar{ -u, --user }}}} & Remove all user productions (but not chunks or default rules) from memory  \\
\hline
\soar{\soar{\soar{\soar{production\_name}}}} & Remove the specific production with this name.  \\
\hline
\end{tabular}
\subsubsection*{Description}
 This command removes productions from Soar's memory. The command must be called with either a specific production name or with a flag that indicates a particular group of productions to be removed. Using the flag \textbf{-a}
 or \textbf{--all}
 also causes an init-soar. 
\subsubsection*{Examples}
excise my*first*production --chunks
\end{verbatim}
excise --all
\end{verbatim}
\subsubsection*{Default Aliases}
\hline
\soar{\soar{\soar{\soar{ Alias }}}} & Maps to  \\
\hline
\soar{\soar{\soar{\soar{ ex }}}} & excise  \\
\hline
\end{tabular}
\subsubsection*{See Also}
\hyperref[init-soar]{init-soar} 
\input{wikicmd/tex/gp}
\input{wikicmd/tex/gp-max}
\subsection{\soarb{help}}
\label{help}
\index{help}
Provide formatted usage information about Soar commands. 
\subsubsection*{Synopsis}
help [command_name]
\end{verbatim}
\subsubsection*{Options}
\hline
\soar{\soar{\soar{\soar{ command\_name }}}} & Print usage syntax for the command.  \\
\hline
\end{tabular}
\subsubsection*{Description}
 This command prints formatted help for the given command name. 
\subsubsection*{Examples}
 To see the syntax for the \emph{excise}
help excise
\end{verbatim}
help
\end{verbatim}
\subsubsection*{Default Aliases}
\hline
\soar{\soar{\soar{\soar{ Alias }}}} & Maps to  \\
\hline
\soar{\soar{\soar{\soar{�? }}}} & help  \\
\hline
\soar{\soar{\soar{\soar{ h }}}} & help  \\
\hline
\soar{\soar{\soar{\soar{ man }}}} & help  \\
\hline
\end{tabular}

\subsection{\soarb{init-soar}}
\label{init-soar}
\index{init-soar}
empties working memory and resets run-time statistics. 
\subsubsection*{Synopsis}
init-soar
\end{verbatim}
\subsubsection*{Options}
 No options. 
\subsubsection*{Description}
 The \textbf{init-soar}
 command initializes Soar. It removes all elements from working memory, wiping out the goal stack, and resets all runtime statistics. The firing counts for all productions are reset to zero. The \textbf{init-soar}
 command allows a Soar program that has been halted to be reset and start its execution from the beginning. 
 \textbf{init-soar}
 does not remove any productions from production memory; to do this, use the \textbf{excise}
 command. Note however, that all justifications will be removed because they will no longer be supported. 
\subsubsection*{Default Aliases}
\hline
\soar{\soar{\soar{\soar{ Alias }}}} & Maps to  \\
\hline
\soar{\soar{\soar{\soar{ init }}}} & init-soar  \\
\hline
\soar{\soar{\soar{\soar{ is }}}} & init-soar  \\
\hline
\end{tabular}
\subsubsection*{See Also}
\hyperref[excise]{excise} 
\subsection{\soarb{run}}
\label{run}
\index{run}
Begin Soar\~A�\^a�$\neg$\^a��s execution cycle. 
\subsubsection*{Synopsis}
run  [f|\emph{count}
]
run -[d|e|o|p][s][un] [f|\emph{count}
]
run -[d|e|o|p][un] \emph{count}
 [-i <e|p|d|o>]
\end{verbatim}
\subsubsection*{Options}
\hline
\soar{\soar{\soar{ -d, --decision }}} & Run Soar for count decision cycles.  \\
\hline
\soar{\soar{\soar{ -e, --elaboration }}} & Run Soar for count elaboration cycles.  \\
\hline
\soar{\soar{\soar{ -o, --output }}} & Run Soar until the nth time output is generated by the agent. Limited by the value of max-nil-output-cycles.  \\
\hline
\soar{\soar{\soar{ -p, --phase }}} & Run Soar by phases. A phase is either an input phase, proposal phase, decision phase, apply phase, or output phase.  \\
\hline
\soar{\soar{\soar{ -s, --self }}} & If other agents exist within the kernel, do not run them at this time.  \\
\hline
\soar{\soar{\soar{ -u, --update }}} & Sets a flag in the update event callback requesting that an environment updates. This is the default if --self is not specified.  \\
\hline
\soar{\soar{\soar{ -n, --noupdate }}} & Sets a flag in the update event callback requesting that an environment does not update. This is the default if --self is specified.  \\
\hline
\soar{\soar{\soar{ f, forever }}} & Run until halted by problem-solving completion or until stopped by an interrupt.  \\
\hline
\soar{\soar{\soar{ count }}} & A single integer which specifies the number of cycles to run Soar.  \\
\hline
\soar{\soar{\soar{ -i, --interleave }}} & Support round robin execution across agents at a finer grain than the run-size parameter. e = elaboration, p = phase, d = decision, o = output  \\
\hline
\end{tabular}
\paragraph*{Deprecated Options}
 These may be reimplemented in the future. 
\hline
\soar{\soar{\soar{ --operator }}} & Run Soar until the nth time an operator is selected.  \\
\hline
\soar{\soar{\soar{ --state }}} & Run Soar until the nth time a state is selected.  \\
\hline
\end{tabular}
\subsubsection*{Description}
 The \textbf{run}
 command starts the Soar execution cycle or continues any execution that was temporarily stopped. The default behavior of \textbf{run}
, with no arguments, is to cause Soar to execute until it is halted or interrupted by an action of a production, or until an external interrupt is issued by the user. The \textbf{run}
 command can also specify that Soar should run only for a specific number of Soar cycles or phases (which may also be prematurely stopped by a production action or the stop-soar command). This is helpful for debugging sessions, where users may want to pay careful attention to the specific productions that are firing and retracting. 
 The \textbf{run}
 command takes optional arguments: an integer, \emph{count}
, which specifies how many units to run; and a \emph{units}
 flag indicating what steps or increments to use. If \emph{count}
 is specified, but no \emph{units}
 are specified, then Soar is run by decision cycles. If \emph{units}
 are specified, but \emph{count}
 is unpecified, then \emph{count}
 defaults to '1'. The argument \textbf{forever}
 (can be shortened to \textbf{f}
) is a keyword used instead of an integer \emph{count}
 and indicates Soar should be run indefinitely, until halted by problem-solving completion, or stopped by an interrupt. 
 If there are multiple Soar agents that exist in the same Soar process, then issuing a \textbf{run}
 command in any agent will cause all agents to run with the same set of parameters, unless the flag \textbf{--self}
 is specified, in which case only that agent will execute. 
 If an environment is registered for the kernel's update event, then when the event it triggered, the environment will get information about how the \textbf{run}
 was executed. If a \textbf{run}
 was executed with the --update option, then then event sends a flag requesting that the environment actually update itself. If a \textbf{run}
 was executed with the --noupdate option, then the event sends a flag requesting that the environment not update itself. The --update option is the default when run is specified without the --self option is not specified. If the --self option is specified, then the --noupdate option is on by default. It is up to the environment to check for these flags and honor them. 
 Some use cases include: 
\hline
\soar{\soar{\soar{ run --self }}} & runs one agent but not the environment  \\
\hline
\soar{\soar{\soar{ run --self --update }}} & runs one agent and the environment  \\
\hline
\soar{\soar{\soar{ run }}} & runs all agents and the environment  \\
\hline
\soar{\soar{\soar{ run --noupdate }}} & runs all agents but not the environment  \\
\hline
\end{tabular}
\paragraph*{Setting an interleave size}
 When there are multiple agents running within the same process, it may be useful to keep agents more closely aligned in their execution cycle than the run increment (--elaboration, --phases, --decisions, --output) specifies. For instance, it may be necessary to keep agents in ``lock step'' at the phase level, eventhough the \textbf{run}
 command issued is for 5 decisions. Some use cases include: 
\hline
\soar{\soar{\soar{ run -d 5 -inteleave p }}} & run the agent one phase and then move to the next agent, \\ 
 looping over agents until they have run for 5 decision cycles  \\
\hline
\soar{\soar{\soar{ run -o 3 -interleave d }}} & run the agent one decision cycle and then move to the next agent. When an agent \\ 
generates output for the 3rd time, it no longer runs even if other agents continue.  \\
\hline
\end{tabular}
 The \textbf{interleave}
 parameter must always be equal to or smaller than the specified \textbf{run}
 parameter. This option is not currently compatible with the \textbf{forever}
 option. 
\paragraph*{Note}
 If Soar has been stopped due to a \textbf{halt}
 action, an \textbf{init-soar}
 command must be issued before Soar can be restarted with the \textbf{run}
 command. 
\subsubsection*{Default Aliases}
\hline
\soar{\soar{\soar{ Alias }}} & Maps to  \\
\hline
\soar{\soar{\soar{ d }}} & run -d 1  \\
\hline
\soar{\soar{\soar{ e }}} & run -e 1  \\
\hline
\soar{\soar{\soar{ step }}} & run 1  \\
\hline
\end{tabular}

\subsection{\soarb{sp}}
\label{sp}
\index{sp}
Define a Soar production. 
\subsubsection*{Synopsis}
sp {production_body}
\end{verbatim}
\subsubsection*{Options}
\hline
\soar{\soar{\soar{ production\_body }}} & A Soar production.  \\
\hline
\end{tabular}
\subsubsection*{Description}
 The \textbf{sp}
 command creates a new production and loads it into production memory. \emph{production\_body}
  name 
      ["documentation-string"] 
      [FLAG*]
      LHS
      -->
      RHS
\end{verbatim}
 The first element of a rule is its name. Conventions for names are given in the Soar Users Manual. If given, the documentation-string must be enclosed in double quotes. Optional flags define the type of rule and the form of support its right-hand side assertions will receive. The specific flags are listed in a separate section below. The LHS defines the left-hand side of the production and specifies the conditions under which the rule can be fired. Its syntax is given in detail in a subsequent section. The --$>$ symbol serves to separate the LHS and RHS portions. The RHS defines the right-hand side of the production and specifies the assertions to be made and the actions to be performed when the rule fires. The syntax of the allowable right-hand side actions are given in a later section. The Soar Users Manual gives an elaborate discussion of the design and coding of productions. Please see that reference for tutorial information about productions. 
 If the name of the new production is the same as an existing one, the old production will be overwritten (excised). 
 \textbf{RULE FLAGS}
\\ 
:o-support      specifies that all the RHS actions are to be given
                o-support when the production fires 
:no-support     specifies that all the RHS actions are only to be given
                i-support when the production fires 
:default        specifies that this production is a default production 
                (this matters for excise -task and watch task) 
:chunk          specifies that this production is a chunk 
                (this matters for learn trace)
:interrupt      specifies that Soar should stop running when this 
                production matches but before it fires
                (this is a useful debugging tool)
\end{verbatim}
 Multiple flags may be used, but not both of \textbf{o-support}
 and \textbf{no-support}
. 
 Although you could force your productions to provide O-support or I-support by using these commands --- regardless of the structure of the conditions and actions of the production --- this is not proper coding style. The \textbf{o-support}
 and \textbf{no-support}
 flags are included to help with debugging, but should not be used in a standard Soar program. 
\subsubsection*{Examples}
sp {blocks*create-problem-space   
     "This creates the top-level space"
     (state <s1> ^superstate nil)
     -->
     (<s1> ^name solve-blocks-world ^problem-space <p1>)
     (<p1> ^name blocks-world)
}
\end{verbatim}
\subsubsection*{See Also}
\hyperref[excise]{excise} \hyperref[learn]{learn} \hyperref[watch]{watch} 
\subsection{\soarb{stop-soar}}
\label{stop-soar}
\index{stop-soar}
Pause Soar. 
\subsubsection*{Synopsis}
stop-soar [-s] [reason string]
\end{verbatim}
\subsubsection*{Options}
\hline
\soar{\soar{\soar{ -s, --self }}} & Stop only the soar agent where the command is issued. All other agents continue running as previously specified.  \\
\hline
\soar{\soar{\soar{ reason\_string }}} & An optional string which will be printed when Soar is stopped, to indicate why it was stopped. If left blank, no message will be printed when Soar is stopped.  \\
\hline
\end{tabular}
\subsubsection*{Description}
 The \textbf{stop-soar}
 command stops any running Soar agents. It sets a flag in the Soar kernel so that Soar will stop running at a ``safe'' point and return control to the user. This command is usually not issued at the command line prompt - a more common use of this command would be, for instance, as a side-effect of pressing a button on a Graphical User Interface (GUI). 
\subsubsection*{Default Aliases}
\hline
\soar{\soar{\soar{ Alias }}} & Maps to  \\
\hline
\soar{\soar{\soar{ interrupt }}} & stop-soar  \\
\hline
\soar{\soar{\soar{ ss }}} & stop-soar  \\
\hline
\soar{\soar{\soar{ stop }}} & stop-soar  \\
\hline
\end{tabular}
\subsubsection*{See Also}
\hyperref[run]{run} \subsubsection*{Warnings}
 If the graphical interface doesn't periodically do an ``update'' of flush the pending I/O, then it may not be possible to interrupt a Soar agent from the command line. 


\section{Examining Memory}
\label{MEMORY}

This section describes the commands used to inspect production memory,
working memory, and preference memory; to see what productions will 
match and fire in the next Propose or Apply phase;  and to examine the 
goal dependency set.  These commands are particularly useful when
running or debugging Soar, as they let users see what Soar is ``thinking.''
The specific commands described in this section are:

\paragraph{Summary}
\begin{quote}
\begin{description}
\item[default-wme-depth] - Set the level of detail used to print WMEs.
\item[gds-print] - Print the WMEs in the goal dependency set for each goal.
\item[internal-symbols] - Print information about the Soar symbol table.
\item[matches] - Print information about the match set and partial matches.
\item[memories] - Print memory usage for production matches.
\item[preferences] - Examine items in preference memory.
\item[print] - Print items in working memory or production memory.
\item[production-find] - Find productions that contain a given pattern.
%\item[wmes] - An alias for the print command; prints items in working memory.
\end{description}
\end{quote}

Of these commands, \textbf{print} is the most often used (and the most
complex) followed by \textbf{matches} and \textbf{memories}.  \textbf{preferences}
is used to examine which candidate operators have been proposed.
\textbf{production-find} is especially useful when the number of
productions loaded is high.  \textbf{gds-print}
is useful for examining the goal dependecy set when subgoals seem to
be disappearing unexpectedly.  \textbf{default-wme-depth} is related to the \textbf{print} command.
\textbf{internal-symbols} is not often used but is helpful when debugging Soar extensions or
trying to locate memory leaks.

\input{wikicmd/tex/default-wme-depth}
\subsection{\soarb{gds-print}}
\label{gds-print}
\index{gds-print}
Print the WMEs in the goal dependency set for each goal. 
\subsubsection*{Synopsis}
gds-print
\end{verbatim}
\subsubsection*{Options}
 No options. 
\subsubsection*{Description}
 The Goal Dependency Set (GDS) is described in an appendix of the Soar manual. This command is a debugging command for examining the GDS for each goal in the stack. First it steps through all the working memory elements in the rete, looking for any that are included in \emph{any}
 goal dependency set, and prints each one. Then it also lists each goal in the stack and prints the wmes in the goal dependency set for that particular goal. This command is useful when trying to determine why subgoals are disappearing unexpectedly: often something has changed in the goal dependency set, causing a subgoal to be regenerated prior to producing a result. 
\subsubsection*{Warnings}
 gds-print is horribly inefficient and should not generally be used except when something is going wrong and you need to examine the Goal Dependency Set. 
\subsubsection*{Default Aliases}
\hline
\soar{\soar{\soar{\soar{ Alias }}}} & Maps to  \\
\hline
\soar{\soar{\soar{\soar{ gds\_print }}}} & gds-print  \\
\hline
\end{tabular}

\subsection{\soarb{internal-symbols}}
\label{internal-symbols}
\index{internal-symbols}
Print information about the Soar symbol table. 
\subsubsection*{Synopsis}
internal-symbols
\end{verbatim}
\subsubsection*{Options}
 No options. 
\subsubsection*{Description}
 The \textbf{internal-symbols}
 command prints information about the Soar symbol table. Such information is typically only useful for users attempting to debug Soar by locating memory leaks or examining I/O structure. 
\subsubsection*{Example}
 soar> internal-symbols
 --- Symbolic Constants: ---
 operator
 accept
 evaluate-object
 problem-space
 sqrt
 interrupt
 mod
 goal
 io
 (...additional symbols deleted for brevity...)
 --- Integer Constants: ---
 --- Floating-Point Constants: ---
 --- Identifiers: ---
 --- Variables: ---  
 <o>
 <sso>
 <to>
 <ss>
 <ts>
 <so>
 <sss>
\end{verbatim}

\subsection{\soarb{matches}}
\label{matches}
\index{matches}
Prints information about partial matches and the match set. 
\subsubsection*{Synopsis}
matches [-nctw] production_name
matches -[a|r] [-nctw]
\end{verbatim}
\subsubsection*{Options}
\hline
\soar{\soar{\soar{production\_name}}} & Print partial match information for the named production.  \\
\hline
\soar{\soar{\soar{ -n, --names, -c, --count }}} & For the match set, print only the names of the productions that are about to fire or retract (the default). If printing partial matches for a production, just list the partial match counts.  \\
\hline
\soar{\soar{\soar{ -t, --timetags }}} & Also print the timetags of the wmes at the first failing condition  \\
\hline
\soar{\soar{\soar{ -w, --wmes }}} & Also print the full wmes, not just the timetags, at the first failing condition.  \\
\hline
\soar{\soar{\soar{ -a, --assertions }}} & List only productions about to fire.  \\
\hline
\soar{\soar{\soar{ -r, --retractions }}} & List only productions about to retract.  \\
\hline
\end{tabular}
\subsubsection*{Description}
 The matches command prints a list of productions that have instantiations in the match set, i.e., those productions that will retract or fire in the next Propose or Apply phase. It also will print partial match information for a single, named production. 
\subsection*{Printing the match set}
 When printing the match set (i.e., no production name is specified), the default action prints only the names of the productions which are about to fire or retract. If there are multiple instantiations of a production, the total number of instantiations of that production is printed after the production name, unless \textbf{--timetags}
 or \textbf{--wmes}
 are specified, in which case each instantiation is printed on a separate line. 
 When printing the match set, the \textbf{--assertions}
 and \textbf{--retractions}
 arguments may be specified to restrict the output to print only the assertions or retractions. 
\subsection*{Printing partial matches for productions}
 In addition to printing the current match set, the \textbf{matches}
 command can be used to print information about partial matches for a named production. In this case, the conditions of the production are listed, each preceded by the number of currently active matches for that condition. If a condition is negated, it is preceded by a minus sign \textbf{-}
. The pointer \textbf{$>$$>$$>$$>$}
 before a condition indicates that this is the first condition that failed to match. 
 When printing partial matches, the default action is to print only the counts of the number of WME's that match, and is a handy tool for determining which condition failed to match for a production that you thought should have fired. At levels \textbf{--timetags}
 and \textbf{--wmes}
 the \textbf{matches}
 command displays the WME's immediately after the first condition that failed to match --- temporarily interrupting the printing of the production conditions themselves. 
\subsection*{Notes}
 When printing partial match information, some of the matches displayed by this command may have already fired, depending on when in the execution cycle this command is called. To check for the matches that are about to fire, use the \textbf{matches}
 command without a named production. 
 In Soar 8, the execution cycle (decision cycle) is input, propose, decide, apply output; it no longer stops for user input after the decision phase when running by decision cycles (\textbf{run -d 1}
). If a user wishes to print the match set immediately after the decision phase and before the apply phase, then the user must run Soar by \emph{phases}
 (\textbf{run -p 1}
). 
\subsubsection*{Examples}
matches --assertions --wmes
\end{verbatim}
matches -t my*first*production
\end{verbatim}

\subsection{\soarb{memories}}
\label{memories}
\index{memories}
Print memory usage for partial matches. 
\subsubsection*{Synopsis}
memories [-cdju] [\emph{n}
]
memories production_name
\end{verbatim}
\subsubsection*{Options}
\hline
\soar{\soar{\soar{ -c, --chunks }}} & Print memory usage of chunks.  \\
\hline
\soar{\soar{\soar{ -d, --default }}} & Print memory usage of default productions.  \\
\hline
\soar{\soar{\soar{ -j, --justifications }}} & Print memory usage of justifications.  \\
\hline
\soar{\soar{\soar{ -u, --user }}} & Print memory usage of user-defined productions.  \\
\hline
\soar{\soar{\soar{production\_name}}} & Print memory usage for a specific production.  \\
\hline
\emph{n}
 & Number of productions to print, sorted by those that use the most memory.  \\
\hline
\end{tabular}
\subsubsection*{Description}
 The \textbf{memories}
 command prints out the internal memory usage for full and partial matches of production instantiations, with the productions using the most memory printed first. With no arguments, the memories command prints memory usage for all productions. If a \emph{production\_name}
 is specified, memory usage will be printed only for that production. If a positive integer \emph{n}
 is given, only \emph{n}
 productions will be printed: the \emph{n}
 productions that use the most memory. Output may be restricted to print memory usage for particular types of productions using the command options. 
 Memory usage is recorded according to the tokens that are allocated in the rete network for the given production(s). This number is a function of the number of elements in working memory that match each production. Therefore, this command will not provide useful information at the beginning of a Soar run (when working memory is empty) and should be called in the middle (or at the end) of a Soar run. 
 The \textbf{memories}
 command is used to find the productions that are using the most memory and, therefore, may be taking the longest time to match (this is only a heuristic). By identifying these productions, you may be able to rewrite your program so that it will run more quickly. Note that memory usage is just a heuristic measure of the match time: A production might not use much memory relative to others but may still be time-consuming to match, and excising a production that uses a large number of tokens may not speed up your program, because the Rete matcher shares common structure among different productions. 
 As a rule of thumb, numbers less than 100 mean that the production is using a small amount of memory, numbers above 1000 mean that the production is using a large amount of memory, and numbers above 10,000 mean that the production is using a \emph{very}
 large amount of memory. 
\subsubsection*{See Also}
\hyperref[matches]{matches} 
\subsection{\soarb{preferences}}
\label{preferences}
\index{preferences}
Examine details about the preferences that support the specified \emph{id}
 and \emph{attribute}
. 
\subsubsection*{Synopsis}
preferences [-0123nNtw] [[id] [[^]attribute]]
\end{verbatim}
\subsubsection*{Options}
\hline
\soar{\soar{\soar{ -0, -n, --none }}} & Print just the preferences themselves  \\
\hline
\soar{\soar{\soar{ -1, -N, --names }}} & Print the preferences and the names of the productions that generated them  \\
\hline
\soar{\soar{\soar{ -2, -t, --timetags }}} & Print the information for the --names option above plus the timetags of the wmes matched by the LHS of the indicated productions  \\
\hline
\soar{\soar{\soar{ -3, -w, --wmes }}} & Print the information for the --timetags option above plus the entire wme matched on the LHS.  \\
\hline
\soar{\soar{\soar{ -o, --object }}} & Print the support for all the wmes that comprise the object (the specified ID).  \\
\hline
\soar{\soar{\soar{id}}} & Must be an existing Soar object identifier.  \\
\hline
\soar{\soar{\soar{attribute}}} & Must be an existing \emph{\^{}attribute}
 of the specified identifier.  \\
\hline
\end{tabular}
\subsubsection*{Description}
 The \textbf{preferences}
 command prints all the preferences for the given object id and attribute. If \emph{id}
 and \emph{attribute}
 are not specified, they default to the current state and the current operator. The '\^{}' is optional when specifying the attribute. The optional arguments indicates the level of detail to print about each preference. 
 This command is useful for examining which candidate operators have been proposed and what relationships, if any, exist among them. If a preference has O-support, the string, ``:O'' will also be printed. 
 When only the ID is specified on the commandline, if the ID is a state, Soar uses the default attribute \^{}operator. If the ID is not a state, Soar prints the support information for all WMEs whose $<$value$>$ is the ID. 
 When an ID and the --object flag are specified, Soar prints the preferences / wme support for all WMEs comprising the specified ID. 
\subsection*{Note}
 For the time being, \textbf{numeric-indifferent}
 preferences are listed under the heading ``binary indifferents:''. 
\subsubsection*{Examples}
soar> preferences S1 operator --names
Preferences for S1 ^operator:
acceptables:
 O2 (fill) +
   From waterjug*propose*fill
 O3 (fill) +
   From waterjug*propose*fill
unary indifferents:
 O2 (fill) =
   From waterjug*propose*fill
 O3 (fill) =
   From waterjug*propose*fill
\end{verbatim}
 preferences -n
\end{verbatim}
soar> preferences s1 jug
Preferences for S1 ^jug:
  
acceptables:
  (S1 ^jug I4) �:O 
  (S1 ^jug J1) �:O 
\end{verbatim}
soar> pref J1 -1
 Support for (31: O3 ^jug J1)
   (O3 ^jug J1) 
     From water-jug*propose*fill
 Support for (11: S1 ^jug J1)
   (S1 ^jug J1) �:O 
     From water-jug*apply*initialize-water-jug-look-ahead
\end{verbatim}
 soar> pref -o s1
 Support for S1 ^problem-space:
   (S1 ^problem-space P1) 
 Support for S1 ^name:
   (S1 ^name water-jug) �:O 
 Support for S1 ^jug:
   (S1 ^jug I4) �:O 
   (S1 ^jug J1) �:O 
 Support for S1 ^desired:
   (S1 ^desired D1) �:O 
 Support for S1 ^superstate-set:
   (S1 ^superstate-set nil) 
 Preferences for S1 ^operator:
 acceptables:
   O2 (fill) +
   O3 (fill) +
 Arch-created wmes for S1�:
 (2: S1 ^superstate nil)
 (1: S1 ^type state)
 Input (IO) wmes for S1�:
 (3: S1 ^io I1)
\end{verbatim}
\subsubsection*{See Also}

\subsection{\soarb{print}}
\label{print}
\index{print}
Print items in working memory or production memory. 
\subsubsection*{Synopsis}
print [-fFin] production_name
print -[a|c|D|j|u][fFin]
print [-i] [-d <depth>] \emph{identifier}
|\emph{timetag}
|\emph{pattern}
print -s[oS]
\end{verbatim}
\subsubsection*{Options}
\subsection*{Printing items in production memory}
\hline
\soar{\soar{\soar{ -a, --all }}} & print the names of all productions currently loaded  \\
\hline
\soar{\soar{\soar{ -c, --chunks }}} & print the names of all chunks currently loaded  \\
\hline
\soar{\soar{\soar{ -D, --defaults }}} & print the names of all default productions currently loaded  \\
\hline
\soar{\soar{\soar{ -f, --full }}} & When printing productions, print the whole production. This is the default when printing a named production.  \\
\hline
\soar{\soar{\soar{ -F, --filename }}} & also prints the name of the file that contains the production.  \\
\hline
\soar{\soar{\soar{ -i, --internal }}} & items should be printed in their internal form. For productions, this means leaving conditions in their reordered (rete net) form.  \\
\hline
\soar{\soar{\soar{ -j, --justifications }}} & print the names of all justifications currently loaded.  \\
\hline
\soar{\soar{\soar{ -n, --name }}} & When printing productions, print only the name and not the whole production. This is the default when printing any category of productions, as opposed to a named production.  \\
\hline
\soar{\soar{\soar{ -u, --user }}} & print the names of all user productions currently loaded  \\
\hline
\soar{\soar{\soar{production\_name}}} & print the production named production-name \\
\hline
\end{tabular}
\subsection*{Printing items in working memory}
\hline
 -d, --depth \emph{n}
 & This option overrides the default printing depth (see the default-wme-depth command for more detail).  \\
\hline
\soar{\soar{\soar{ -i, --internal }}} & items should be printed in their internal form. For working memory, this means printing the individual elements with their timetags, rather than the objects.  \\
\hline
\soar{\soar{\soar{ -v, --varprint }}} & Print identifiers enclosed in angle brackets.  \\
\hline
\emph{identifier}
 & print the object \emph{identifier}
. \emph{identifier}
 must be a valid Soar symbol such as \textbf{S1 }
\hline
\emph{pattern}
 & print the object whose working memory elements matching the given pattern. See Description for more information on printing objects matching a specific pattern.  \\
\hline
\emph{timetag}
 & print the object in working memory with the given \emph{timetag}
\hline
\end{tabular}
\subsection*{Printing the current subgoal stack}
\hline
\soar{\soar{\soar{ -s, --stack }}} & Specifies that the Soar goal stack should be printed. By default this includes both states and operators.  \\
\hline
\soar{\soar{\soar{ -o, --operators }}} & When printing the stack, print only \textbf{operators}
.  \\
\hline
\soar{\soar{\soar{ -S, --states }}} & When printing the stack, print only \textbf{states}
.  \\
\hline
\end{tabular}
\subsubsection*{Description}
 The \textbf{print}
(\emph{identifier}
 ^\emph{attribute value}
 [+])
\end{verbatim}
 The pattern is surrounded by parentheses. The \emph{identifier}
, \emph{attribute}
, and \emph{value}
 must be valid Soar symbols or the wildcard symbol * which matches all occurences. The optional \textbf{+}
 symbol restricts pattern matches to acceptable preferences. 
\subsubsection*{Examples}
print --internal (s1 ^* v2)
\end{verbatim}
print --stack
\end{verbatim}
print -if prodname
\end{verbatim}
print -u
\end{verbatim}
\subsubsection*{Default Aliases}
\hline
\soar{\soar{\soar{ Alias }}} & Maps to  \\
\hline
\soar{\soar{\soar{ p }}} & print  \\
\hline
\soar{\soar{\soar{ pc }}} & print --chunks  \\
\hline
\soar{\soar{\soar{ wmes }}} & print -i  \\
\hline
\end{tabular}
\subsubsection*{See Also}
\hyperref[default-wme-depth]{default-wme-depth} \hyperref[predefined-aliases]{predefined-aliases} 
\subsection{\soarb{production-find}}
\label{production-find}
\index{production-find}
\subsubsection*{Synopsis}
production-find [-lrs[n|c]] \emph{pattern}
\end{verbatim}
\subsubsection*{Options}
\hline
\soar{\soar{\soar{ -c, --chunks }}} & Look \emph{only}
 for chunks that match the pattern.  \\
\hline
\soar{\soar{\soar{ -l, --lhs }}} & Match pattern only against the conditions (left-hand side) of productions (default).  \\
\hline
\soar{\soar{\soar{ -n, --nochunks }}} &\emph{Disregard}
 chunks when looking for the pattern.  \\
\hline
\soar{\soar{\soar{ -r, --rhs }}} & Match pattern against the actions (right-hand side) of productions.  \\
\hline
\soar{\soar{\soar{ -s, --show-bindings }}} & Show the bindings associated with a wildcard pattern.  \\
\hline
\soar{\soar{\soar{ pattern }}} & Any pattern that can appear in productions.  \\
\hline
\end{tabular}
\subsubsection*{Description}
 The production-find command is used to find productions in production memory that include conditions or actions that match a given \emph{pattern}
. The pattern given specifies one or more condition elements on the left hand side of productions (or negated conditions), or one or more actions on the right-hand side of productions. Any pattern that can appear in productions can be used in this command. In addition, the asterisk symbol, *, can be used as a wildcard for an attribute or value. It is important to note that the whole pattern, including the parenthesis, must be enclosed in curly braces for it to be parsed properly. 
 The variable names used in a call to production-find do not have to match the variable names used in the productions being retrieved. 
 The production-find command can also be restricted to apply to only certain types of productions, or to look only at the conditions or only at the actions of productions by using the flags. 
\subsubsection*{Examples}
 Find productions that test that some object \emph{gumby}
 has an attribute \emph{alive}
 with value \emph{t}
. In addition, limit the rules to only those that test an operator named \emph{foo}
production-find (<state> ^gumby <gv> ^operator.name foo)(<gv> ^alive t)
\end{verbatim}
 Note that in the above command, $<$state$>$ does not have to match the exact variable name used in the production. 
 Find productions that propose the operator \emph{foo}
production-find --rhs (<x> ^operator <op> +)(<op> ^name foo)
\end{verbatim}
production-find --chunks (<x> ^pokey *)
\end{verbatim}
source demos/water-jug/water-jug.soar
production-find (<s> ^name *)(<j> ^volume *)
production-find (<s> ^name *)(<j> ^volume 3)
production-find --rhs (<j> ^* <volume>)
\end{verbatim}
\subsubsection*{See Also}
\hyperref[sp]{sp} 

% ****************************************************************************
% ----------------------------------------------------------------------------
\section{Configuring Trace Information and Debugging}
\label{DEBUG}

This section describes the commands used primarily for debugging or
to configure the trace output printed by Soar as it runs.  Users may:
specify the content of the runtime trace output; ask that
they be alerted when specific productions fire and retract; 
or request details on Soar's performance.

The specific commands described in this section are:


\paragraph{Summary}
\begin{quote}
\begin{description}
\item[chunk-name-format] - Specify format of the name to use for new chunks.
\item[firing-counts] - Print the number of times productions have fired.
%\item[format-watch] - Change the trace output that's printed as Soar runs.
%\item[interrupt] - Add \& remove pre-firing interrupts on specific productions.
%\item[monitor] - Manage attachment of Tcl scripts to Soar events.
\item[pwatch] - Trace firings and retractions of specific productions.
\item[stats] - Print information on Soar's runtime statistics.
\item[verbose] -  Control detailed information printed as Soar runs.
\item[warnings] - Toggle whether or not warnings are printed.
\item[watch] - Control the information printed as Soar runs.
\item[watch-wmes] -  Print information about wmes that match a certain pattern as they are added and removed
\end{description}
\end{quote}

Of these commands, \soar{watch} is the most often used (and the most 
complex). \soar{pwatch} is related to \soar{watch}, but applies only 
to specific, named productions. \soar{firing-counts} and \soar{stats} 
are useful for understanding how much work Soar is doing. \soar{chunk-name-format} is less-frequently
used, but allows for detailed control of Soar's chunk naming.

\input{wikicmd/tex/chunk-name-format}
\subsection{\soarb{firing-counts}}
\label{firing-counts}
\index{firing-counts}
Print the number of times each production has fired. 
\subsubsection*{Synopsis}
firing-counts [n]
firing-counts production_names
\end{verbatim}
\subsubsection*{Options}
 If given, an option can take one of two forms -- an integer or a list of production names: 
\hline
\emph{n}
 & List the top \emph{n}
 productions. If \emph{n}
 is 0, only the productions which haven't fired are listed  \\
\hline
\soar{\soar{\soar{\soar{ production\_name }}}} & For each production in production\_names, print how many times the production has fired  \\
\hline
\end{tabular}
\subsubsection*{Description}
 The \textbf{firing-counts}
 command prints the number of times each production has fired; production names are given from most requently fired to least frequently fired. With no arguments, it lists all productions. If an integer argument, \textbf{n}
, is given, only the top \emph{n}
 productions are listed. If \textbf{n}
 is zero (0), only the productions that haven't fired at all are listed. If one or more production names are given as arguments, only firing counts for these productions are printed. 
 Note that firing counts are reset by a call to \textbf{init-soar}
. 
\subsubsection*{Examples}
firing-counts 10
\end{verbatim}
firing-counts my*first*production my*second*production
\end{verbatim}
\subsubsection*{Warnings}
 Firing-counts are reset to zero after an init-soar. 
 NB: This command is slow, because the sorting takes time O(n*log n) 
\subsubsection*{Default Aliases}
\hline
\soar{\soar{\soar{\soar{ Alias }}}} & Maps to  \\
\hline
\soar{\soar{\soar{\soar{ fc }}}} & firing-counts  \\
\hline
\end{tabular}
\subsubsection*{See Also}
\hyperref[init-soar]{init-soar} 
\subsection{\soarb{pwatch}}
\label{pwatch}
\index{pwatch}
Trace firings and retractions of specific productions. 
\subsubsection*{Synopsis}
pwatch [-d|e] [production name]
\end{verbatim}
\subsubsection*{Options}
\hline
\soar{\soar{\soar{ -d, --disable, --off }}} & Turn production watching off for the specified production. If no production is specified, turn production watching off for all productions.  \\
\hline
\soar{\soar{\soar{ -e, --enable, --on }}} & Turn production watching on for the specified production. The use of this flag is optional, so this is pwatch's default behavior. If no production is specified, all productions currently being watched are listed.  \\
\hline
\soar{\soar{\soar{production name}}} & The name of the production to watch.  \\
\hline
\end{tabular}
\subsubsection*{Description}
 The \textbf{pwatch}
 command enables and disables the tracing of the firings and retractions of individual productions. This is a companion command to \textbf{watch}
, which cannot specify individual productions by name. 
 With no arguments, \textbf{pwatch}
 lists the productions currently being traced. With one production-name argument, \textbf{pwatch}
 enables tracing the production; \textbf{--enable}
 can be explicitly stated, but it is the default action. 
 If \textbf{--disable}
 is specified followed by a production-name, tracing is turned off for the production. When no production-name is specified, \textbf{pwatch --enable}
 lists all productions currently being traced, and \textbf{pwatch --disable}
 disables tracing of all productions. 
 Note that \textbf{pwatch}
 now only takes one production per command. Use multiple times to watch multiple functions. 
\subsubsection*{Default Aliases}
\hline
\soar{\soar{\soar{ Alias }}} & Maps to  \\
\hline
\soar{\soar{\soar{ pw }}} & pwatch  \\
\hline
\end{tabular}
\subsubsection*{See Also}
\hyperref[watch]{watch} 
\subsection{\soarb{stats}}
\label{stats}
\index{stats}
Print information on Soar\~A�\^a�$\neg$\^a��s runtime statistics. 
\subsubsection*{Synopsis}
stats [-s|-m|-r]
\end{verbatim}
\subsubsection*{Options}
\hline
\soar{\soar{\soar{ -m, --memory }}} & report usage for Soar's memory pools  \\
\hline
\soar{\soar{\soar{ -r, --rete }}} & report statistics about the rete structure  \\
\hline
\soar{\soar{\soar{ -s, --system }}} & report the system (agent) statistics. This is the default if no args are specified.  \\
\hline
\end{tabular}
\subsubsection*{Description}
 This command prints Soar internal statistics. The argument indicates the component of interest. 
 With the \textbf{--system}
 flag, the \textbf{stats}
\item \textbf{Version}
 --- The Soar version number, hostname, and date of the run. 
\item \textbf{Number of productions}
 --- The total number of productions loaded in the system, including all chunks built during problem solving and all default productions. 
\item \textbf{Timing Information}
 --- Might be quite detailed depending on the flags set at compile time. See note on timers below. 
\item \textbf{Decision Cycles}
 --- The total number of decision cycles in the run and the average time-per-decision-cycle in milliseconds. 
\item \textbf{Elaboration cycles}
 --- The total number of elaboration cycles that were executed during the run, the average number of elaboration cycles per decision cycle, and the average time-per-elaboration-cycle in milliseconds. This is not the total number of production firings, as productions can fire in parallel. 
\item \textbf{Production Firings}
 --- The total number of productions that were fired. 
\item \textbf{Working Memory Changes}
 --- This is the total number of changes to working memory. This includes all additions and deletions from working memory. Also prints the average match time. 
\item \textbf{Working Memory Size}
 --- This gives the current, mean and maximum number of working memory elements. 
\end{itemize}
 The optional \textbf{stats}
 argument \textbf{--memory}
 provides information about memory usage and Soar's memory pools, which are used to allocate space for the various data structures used in Soar. 
 The optional \textbf{stats}
 argument \textbf{--rete}
 provides information about node usage in the Rete net, the large data structure used for efficient matching in Soar. 
\subsubsection*{Default Aliases}
\hline
\soar{\soar{\soar{ Alias }}} & Maps to  \\
\hline
\soar{\soar{\soar{ st }}} & stats  \\
\hline
\end{tabular}
\subsubsection*{See Also}
\hyperref[timers]{timers} \subsubsection*{A Note on Timers}
total CPU time
total kernel time
phase kernel time (per phase)
phase callbacks time (per phase)
input function time
output function time
\end{verbatim}
 Total CPU time is calculated from the time a decision cycle (or number of decision cycles) is initiated until stopped. Kernel time is the time spent in core Soar functions. In this case, kernel time is defined as the all functions other than the execution of callbacks and the input and output functions. The total kernel timer is only stopped for these functions. The phase timers (for the kernel and callbacks) track the execution time for individual phases of the decision cycle (i.e., input phase, preference phase, working memory phase, output phase, and decision phase). Because there is overhead associated with turning these timers on and off, the actual kernel time will always be greater than the derived kernel time (i.e., the sum of all the phase kernel timers). Similarly, the total CPU time will always be greater than the derived total (the sum of the other timers) because the overhead of turning these timers on and off is included in the total CPU time. In general, the times reported by the single timers should always be greater than than the corresponding derived time. Additionally, as execution time increases, the difference between these two values will also increase. For those concerned about the performance cost of the timers, all the run time timing calculations can be compiled out of the code by defining NO\_TIMING\_STUFF (in soarkernel.h) before compilation. 

\subsection{\soarb{verbose}}
\label{verbose}
\index{verbose}
Control detailed information printed as Soar runs. 
\subsubsection*{Synopsis}
verbose [-ed]
\end{verbatim}
\subsubsection*{Options}
\hline
\soar{\soar{\soar{ -d, --disable, --off }}} & Turn verbosity off.  \\
\hline
\soar{\soar{\soar{ -e, --enable, --on }}} & Turn verbosity on.  \\
\hline
\end{tabular}
\subsubsection*{Description}
 Invoke with no arguments to query. (fix this) - More about what this command does? 

\subsection{\soarb{warnings}}
\label{warnings}
\index{warnings}
\subsubsection*{Synopsis}
warnings -[e|d]
\end{verbatim}
\subsubsection*{Options}
\hline
\soar{\soar{\soar{ -e, --enable, --on }}} & Default. Print all warning messages from the kernel.  \\
\hline
\soar{\soar{\soar{ -d, --disable, --off }}} & Disable all, except most critical, warning messages.  \\
\hline
\end{tabular}
\subsubsection*{Description}
 Enables and disables the printing of warning messages. If an argument is specified, then the warnings are set to that state. If no argument is given, then the current warnings status is printed. At startup, warnings are initially enabled. If warnings are disabled using this command, then some warnings may still be printed, since some are considered too important to ignore. 
 The warnings that are printed apply to the syntax of the productions, to notify the user when they are not in the correct syntax. When a lefthand side error is discovered (such as conditions that are not linked to a common state or impasse object), the production is generally loaded into production memory anyway, although this production may never match or may seriously slow down the matching process. In this case, a warning would be printed only if \textbf{warnings}
 were \textbf{--on}
. Righthand side errors, such as preferences that are not linked to the state, usually result in the production not being loaded, and a warning regardless of the \textbf{warnings}
 setting. 
\subsubsection*{Examples}
\subsubsection*{See Also}

\subsection{\soarb{watch}}
\label{watch}
\index{watch}
Control the run-time tracing of Soar. 
\subsubsection*{Synopsis}
watch
watch [--level] [0|1|2|3|4|5]
watch -N
watch -[dpPwrDujcbi] [<remove>] -[n|t|f]
watch --learning [<print|noprint|fullprint>]
\end{verbatim}
\subsubsection*{Options}
 When appropriate, a specific option may be turned off using the \textbf{remove}
 argument. This argument has a numeric alias; you can use \textbf{0}
 for \textbf{remove}
. A mix of formats is acceptable, even in the same command line. 
\paragraph*{Basic Watch Settings}
\hline
\emph{Option Flag}
 &\emph{Argument to Option}
 &\emph{Description}
\hline
\soar{\soar{\soar{ -l, --level }}} & 0 to 5 (see \textbf{Watch Levels}
 below)  & This flag is optional but recommended. Set a specific watch level using an integer 0 to 5, this is an inclusive operation  \\
\hline
\soar{\soar{\soar{ -N, --none  & No argument }}} & Turns off all printing about Soar's internals, equivalent to --level 0  \\
\hline
\soar{\soar{\soar{ -d, --decisions  & remove (optional) }}} & Controls whether state and operator decisions are printed as they are made  \\
\hline
\soar{\soar{\soar{ -p, --phases  & remove (optional) }}} & Controls whether decisions cycle phase names are printed as Soar executes  \\
\hline
\soar{\soar{\soar{ -P, --productions  & remove (optional) }}} & Controls whether the names of productions are printed as they fire and retract, equivalent to -Dujc  \\
\hline
\soar{\soar{\soar{ -w, --wmes  & remove (optional) }}} & Controls the printing of working memory elements that are added and deleted as productions are fired and retracted  \\
\hline
\soar{\soar{\soar{ -r, --preferences  & remove (optional) }}} & Controls whether the preferences generated by the traced productions are printed when those productions fire or retract  \\
\hline
\end{tabular}
\subsubsection*{ Watch Levels }
 Use of the \textbf{--level}
 (\textbf{-l}
) flag is optional but recommended. 
\hline
\soar{\soar{\soar{ 0 }}} & watch nothing; equivalent to \~A�\^a�$\neg$\^a��N  \\
\hline
\soar{\soar{\soar{ 1 }}} & watch decisions; equivalent to -d  \\
\hline
\soar{\soar{\soar{ 2 }}} & watch phases and decisions; equivalent to -dp  \\
\hline
\soar{\soar{\soar{ 3 }}} & watch productions, phases, and decisions; equivalent to -dpP  \\
\hline
\soar{\soar{\soar{ 4 }}} & watch wmes, productions, phases, and decisions; equivalent to -dpPw  \\
\hline
\soar{\soar{\soar{ 5 }}} & watch preferences, wmes, productions, phases, and decisions; equivalent to -dpPwr  \\
\hline
\end{tabular}
 It is important to note that watch level 0 turns off ALL watch options, including backtracing, indifferent selection and learning. However, the other watch levels do not change these settings. That is, if any of these settings is changed from its default, it will retain its new setting until it is either explicitly changed again or the watch level is set to 0. 
\paragraph*{Watching Productions}
 By default, the names of the productions are printed as each production fires and retracts (at \textbf{watch}
 levels \textbf{3}
 and higher). However, it may be more helpful to watch only a specific \emph{type}
 of production. The tracing of firings and retractions of productions can be limited to only certain types by the use of the following flags: 
\hline
\emph{Option Flag}
 &\emph{Argument to Option}
 &\emph{Description}
\hline
\soar{\soar{\soar{ -D, --default  & remove (optional) }}} & Control only default-productions as they fire and retract  \\
\hline
\soar{\soar{\soar{ -u, --user  & remove (optional) }}} & Control only user-productions as they fire and retract  \\
\hline
\soar{\soar{\soar{ -c, --chunks  & remove (optional) }}} & Control only chunks as they fire and retract  \\
\hline
\soar{\soar{\soar{ -j, --justifications  & remove (optional) }}} & Control only justifications as they fire and retract  \\
\hline
\end{tabular}
 \textbf{Note:}
 The pwatch command is used to watch individual productions specified by name rather than watch a type of productions, such as \textbf{--user}
. 
 Additionally, when watching productions, users may set the level of detail to be displayed for WMEs that are added or retracted as productions fire and retract. Note that detailed information about WMEs will be printed only for productions that are being watched. 
\hline
\emph{Option Flag}
 &\emph{Argument to Option}
 &\emph{Description}
\hline
\soar{\soar{\soar{ -n, --nowmes  & No argument }}} & When watching productions, do not print any information about matching wmes  \\
\hline
\soar{\soar{\soar{ -t, --timetags  & No argument }}} & When watching productions, print only the timetags for matching wmes  \\
\hline
\soar{\soar{\soar{ -f, --fullwmes  & No argument }}} & When watching productions, print the full matching wmes  \\
\hline
\end{tabular}
\paragraph*{Watching Learning}
\hline
\emph{Option Flag}
 &\emph{Argument to Option}
 &\emph{Description}
\hline
\soar{\soar{\soar{ -L, --learning  & noprint, print, or fullprint (see table below) }}} & Controls the printing of chunks/justifications as they are created  \\
\hline
\end{tabular}
 As Soar is running, it may create justifications and chunks which are added to production memory. The \textbf{watch}
 command allows users to monitor when chunks and justifications are created by specifying one of the following arguments to the \textbf{watch --learning}
 command: 
\hline
\emph{Argument}
 &\emph{Alias}
 &\emph{Effect}
\hline
\soar{\soar{\soar{ noprint  & 0 }}} & Print nothing about new chunks or justifications (default)  \\
\hline
\soar{\soar{\soar{ print  & 1 }}} & Print the names of new chunks and justifications when created  \\
\hline
\soar{\soar{\soar{ fullprint  & 2 }}} & Print entire chunks and justifications when created  \\
\hline
\end{tabular}
\paragraph*{Watching other Functions}
\hline
\emph{Option Flag}
 &\emph{Argument to Option}
 &\emph{Description}
\hline
\soar{\soar{\soar{ -b, --backtracing  & remove (optional) }}} & Controls the printing of backtracing information when a chunk or justification is created  \\
\hline
\soar{\soar{\soar{ -i, --indifferent-selection  & remove (optional) }}} & Controls the printing of the scores for tied operators in random indifferent selection mode  \\
\hline
\end{tabular}
\subsubsection*{Description}
 The \textbf{watch}
 command controls the amount of information that is printed out as Soar runs. The basic functionality of this command is to trace various \emph{levels}
 of information about Soar's internal workings. The higher the \emph{level}
, the more information is printed as Soar runs. At the lowest setting, \textbf{0 | --none}
, nothing is printed. The levels are cumulative, so that each successive level prints the information from the previous level as well as some additional information. The default setting for the \textbf{watch \emph{level}
}
 is \textbf{1}
, (or \textbf{--decisions}
). 
0 or --none
1 or --decisions
2 or --decisions --phases
3 or --decisions --phases --productions
4 or --decisions --phases --productions --wmes
5 or --decisions --phases --productions --wmes --preferences
\end{verbatim}
 The numerical arguments \emph{inclusively}
 turn on all levels up to the number specified. To use numerical arguments to turn off a level, specify a number which is less than the level to be turned off. For instance, to turn off watching of productions, specify ``watch --level 2'' (or 1 or 0). Numerical arguments are provided for shorthand convenience. For more detailed control over the watch settings, the named arguments should be used. 
 With no arguments, this command prints information about the current \textbf{watch}
 status, i.e., the values of each parameter. 
 For the named arguments, including the named argument turns on only that setting. To turn off a specific setting, follow the named argument with \emph{remove}
 or \emph{0}
. 
 The named argument \textbf{--productions}
 is shorthand for the four arguments \textbf{--default}
, \textbf{--user}
, \textbf{--justifications}
, and \textbf{--chunks}
. 
\subsubsection*{Examples}
watch --level 0
watch -l 0
watch -N
\end{verbatim}
 Although the \textbf{--level}
watch --level 5 \emph{... OK}
watch 5         \emph{... OK, but try to avoid}
\end{verbatim}
watch -r -l 2 \emph{... Incorrect: -r flag ignored, level 2 parsed after it and overrides the setting}
watch -r 2    \emph{... Syntax error: 0 or remove expected as optional argument to -r}
watch -r -l 2 \emph{... Incorrect: -r flag ignored, level 2 parsed after it and overrides the setting}
watch 2 -r    \emph{... OK, but try to avoid}
watch -l 2 -r \emph{... OK}
\end{verbatim}
watch --level 3
watch -l 3
watch --decisions --phases --productions
watch -d -p -P
\end{verbatim}
watch --level 1 --wmes
watch -l 1 -w
watch --decisions --wmes
watch -d --wmes
watch -w --decisions
watch -w -d
\end{verbatim}
watch --level 4 --phases remove
watch -l 4 -p remove
watch -l 4 -p 0
watch -d -P -w -p remove
\end{verbatim}
 To watch the firing and retraction of decisions and \emph{only}
watch -l 1 -u
watch -d -u
\end{verbatim}
 To watch decisions, phases and all productions \emph{except}
watch --decisions --phases --productions --user remove --justifications remove --fullwmes
watch -d -p -P -f -u remove -j 0 
watch -f -l 3 -u 0 -j 0
\end{verbatim}
\subsubsection*{Default Aliases}
\hline
\soar{\soar{\soar{ Alias }}} & Maps to  \\
\hline
\soar{\soar{\soar{ w }}} & watch  \\
\hline
\end{tabular}
\subsubsection*{See Also}
\hyperref[pwatch]{pwatch} \hyperref[print]{print} \hyperref[run]{run} \hyperref[watch-wmes]{watch-wmes} 
\subsection{\soarb{watch-wmes}}
\label{watch-wmes}
\index{watch-wmes}
\subsubsection*{Synopsis}
watch-wmes -[a|r]  -t <type>  pattern
watch-wmes -[l|R] [-t <type>]
\end{verbatim}
\subsubsection*{Options}
\hline
\soar{\soar{\soar{ -a, --add-filter }}} & Add a filter to print wmes that meet the type and pattern criteria.  \\
\hline
\soar{\soar{\soar{ -r, --remove-filter }}} & Delete filters for printing wmes that match the type and pattern criteria.  \\
\hline
\soar{\soar{\soar{ -l, --list-filter }}} & List the filters of this type currently in use. Does not use the pattern argument.  \\
\hline
\soar{\soar{\soar{ -R, --reset-filter }}} & Delete all filters of this type. Does not use pattern arg.  \\
\hline
\soar{\soar{\soar{ -t, --type }}} & Follow with a type of wme filter, see below.  \\
\hline
\end{tabular}
\paragraph*{Pattern}
\emph{id}
 \emph{attribute}
 \emph{value}
\end{verbatim}
 Note that \textbf{*}
 can be used in place of the id, attribute or value as a wildcard that maches any string. Note that braces are not used anymore. 
\paragraph*{Types}
 When using the -t flag, it must be followed by one of the following: 
\hline
\soar{\soar{\soar{ adds }}} & Print info when a wme is \emph{added}
.  \\
\hline
\soar{\soar{\soar{ removes }}} & Print info when a wme is \emph{retracted}
.  \\
\hline
\soar{\soar{\soar{ both }}} & Print info when a wme is added \emph{or}
 retracted.  \\
\hline
\end{tabular}
 When issuing a \textbf{-R}
 or \textbf{-l}
, the \textbf{-t}
 flag is optional. Its absence is equivalent to \textbf{-t both}
. 
\subsubsection*{Description}
 This commands allows users to improve state tracing by issuing filter-options that are applied when watching wmes. Users can selectively define which \emph{object-attribute-value}
 triplets are monitored and whether they are monitored for addition, removal or both, as they go in and out of working memory. 
 \textbf{Note:}
 The functionality of \textbf{watch-wmes}
 resided in the \textbf{watch}
 command prior to Soar 8.6. 
\subsubsection*{Examples}
 Users can \textbf{watch}
 an \emph{attribute}
soar> watch-wmes --add-filter -t both D1 speed *
\end{verbatim}
 or print WMEs that retract in a specific state (provided the \textbf{state}
soar> watch-wmes --add-filter -t removes S3 * *
\end{verbatim}
soar> watch-wmes --add-filter -t both * ontop *
\end{verbatim}


% ----------------------------------------------------------------------------
\section{Configuring Soar's Runtime Parameters}
\label{RUNTIME}

This section describes the commands that control Soar's Runtime Parameters.
Many of these commands provide options that simplify or restrict 
runtime behavior to enable easier and more localized debugging.
Others allow users to select alternative algorithms or methodologies.
Users can configure Soar's learning mechanism; examine the
backtracing information that supports chunks and justifications;
provide hints that could improve the efficiency of the Rete matcher;
limit runaway chunking and production firing;
choose an alternative algorithm for determining whether a working memory
element receives O-support;  and 
configure options for selecting between mutually indifferent operators.

The specific commands described in this section are:

\paragraph{Summary}
\begin{quote}
\begin{description}
\item[epmem] - Get/Set episodic memory parameters and statistics
\item[explain-backtraces] - Print information about chunk and justification backtraces.
\item[indifferent-selection] -  Controls indifferent preference arbitration.
\item[learn] - Set the parameters for chunking, Soar's learning mechanism.
\item[max-chunks] - Limit the number of chunks created during a decision cycle.
\item[max-dc-time] - Set a wall-clock time limit such that the agent will be interrupted when a single decision cycle exceeds this limit.
\item[max-elaborations] - Limit the maximum number of elaboration cycles in a given phase.
\item[max-goal-depth] - Limit the sub-state stack depth.
\item[max-memory-usage] - Set the number of bytes that when exceeded by an agent, will trigger the memory usage exceeded event. 
\item[max-nil-output-cycles] - Limit the maximum number of decision cycles executed without producing output. 
\item[multi-attributes] - Declare multi-attributes so as to increase Rete matching efficiency.
\item[numeric-indifferent-mode] - Select method for combining numeric preferences.
\item[o-support-mode] - Choose experimental variations of o-support.
\item[predict] - Predict the next selected operator 
\item[rl] - Get/Set RL parameters and statistics 
\item[save-backtraces] - Save trace information to explain chunks and justifications.
\item[select] - Force the next selected operator 
\item[set-stop-phase] -  Controls the phase where agents stop when running by decision.
\item[smem] - Get/Set semantic memory parameters and statistics
\item[timers] - Toggle on or off the internal timers used to profile Soar.
\item[waitsnc] - Generate a wait state rather than a state-no-change impasse.
\item[wma] - Get/Set working memory activation parameters
\end{description}
\end{quote}

% ----------------------------------------------------------------------------
\chapter{Episodic Memory}
\label{EPMEM}
\index{episodic memory}
\index{epmem}

Episodic memory is a record of an agent's stream of experience.
The episodic storage mechanism will automatically record episodes as a Soar agent executes.
The agent can later deliberately retrieve episodic knowledge to extract information and regularities that may not have been noticed during the original experience and combine them with current knowledge such as to improve performance on future tasks.

This chapter is organized as follows: episodic memory structures in working memory (\ref{EPMEM-wm}); episodic storage (\ref{EPMEM-storage}); retrieving episodes (\ref{EPMEM-retrieval}); and a discussion of performance (\ref{EPMEM-perf}).
The detailed behavior of episodic memory is determined by numerous parameters that can be controlled and configured via the \soarb{epmem} command.

Please refer to the documentation for that command in Section \ref{epmem} on page \pageref{epmem}.

\section{Working Memory Structure}
\label{EPMEM-wm}

Upon creation of a new state in working memory (see Section \ref{ARCH-impasses-types} on page \pageref{ARCH-impasses-types}; Section \ref{SYNTAX-impasses} on page \pageref{SYNTAX-impasses}), the architecture creates the following augmentations to facilitate agent interaction with episodic memory:

\begin{verbatim}
(<s> ^epmem <e>)
  (<e> ^command <e-c>)
  (<e> ^result <e-r>)
  (<e> ^present-id #)
\end{verbatim}

As rules augment the \soar{command} structure in order to retrieve episodes (\ref{EPMEM-retrieval}), episodic memory augments the \soar{result} structure in response.
Production actions should not remove augmentations of the \soar{result} structure directly, as episodic memory will maintain these WMEs.

The value of the \soar{present-id} augmentation is an integer and will update to expose to the agent the current episode number.
This information is identical to what is available via the \emph{time} statistic (see Section \ref{epmem} on page \pageref{epmem}) and the \emph{present-id} retrieval meta-data (\ref{EPMEM-meta}).

\section{Episodic Storage}
\label{EPMEM-storage}

Episodic memory records new episodes without deliberate action/consideration by the agent.
The timing and frequency of recording new episodes is controlled by the \soar{phase} and \soar{trigger} parameters.
The \soarb{phase} parameter sets the phase in the decision cycle (default: end of each decision cycle) during which episodic memory stores episodes and processes commands.
The value of the \soarb{trigger} parameter indicates to the architecture the event that concludes an episode: adding a new augmentation to the output-link (default) or each decision cycle.

For debugging purposes, the \soarb{force} parameter allows the user to manually request that an episode be recorded (or not) during the current decision cycle.
Behavior is as follows:

\vspace{-8pt}
\begin{itemize}
\item
	The value of the \soar{force} parameter is initialized to \soar{off} every decision cycle.
	\vspace{-6pt}
\item
	During the \soar{phase} of episodic storage, episodic memory tests the value of the \soar{force} parameter; if it has a value other than of off, episodic memory follows the \emph{forced} policy irrespective of the value of the \soar{trigger} parameter.
	\vspace{-6pt}
\end{itemize}

\subsection{Episode Contents}

When episodic memory stores a new episode, it captures the entire top-state of working memory.
There are currently two exceptions to this policy:

\begin{itemize}
\item
Episodic memory only supports WMEs whose attribute is a constant.
Behavior is currently undefined when attempting to store a WME that has an attribute that is an identifier.

\item
The \soarb{exclusions} parameter allows the user to specify a set of attributes for which Soar will not store WMEs.
The storage process currently walks the top-state of working memory in a breadth-first manner, and any WME that is not reachable other than via an excluded WME will not be stored.
By default, episodic memory excludes the \soar{epmem} and \soar{smem} structures, to prevent encoding of potentially large and/or frequently changing memory retrievals.

\end{itemize}

\subsection{Storage Location}
\index{epmem!storage}

Episodic memory uses SQLite to facilitate efficient and standardized storage and querying of episodes.
The episodic store can be maintained in memory or on disk (per the \soar{database} and \soar{path} parameters).
If the store is located on disk, users can use any standard SQLite programs/components to access/query its contents.
See the later discussion on performance (\ref{EPMEM-perf}) for additional parameters dealing with databases on disk.

Note that changes to storage parameters, for example \soar{database, path} and \soar{append} will not have an effect until the database is used after an initialization. This happens either shortly after launch (on first use) or after a database initialization command is issued. To switch databases or database storage types while running, set your new parameters and then perform an \soar{epmem --init} command.

The \soarb{path} parameter specifies the file system path the database is stored in. When \soar{path} is set to a valid file system path and \soar{database} mode is set to \emph{file}, then the SQLite database is written to that path.

The \soarb{append} parameter will determine whether all existing facts stored in a database on disk will be erased when episodic memory loads. Note that this affects \soar{init-soar} also.  In other words, if the \soar{append} setting is off, all episodes stored will be lost when an init-soar is performed. For episodic memory, \soar{append} mode is \soar{off} by default.

\soarit{Note}: As of version 9.3.3, Soar now uses a new schema for the episodic memory database. This means databases from 9.3.2 and below can no longer be loaded.  A conversion utility will be available in Soar 9.4 to convert from the old schema to the new one.

\section{Retrieving Episodes}
\label{EPMEM-retrieval}
\index{epmem!retrieve}

An agent retrieves episodes by creating an appropriate command (we detail the types of commands below) on the \soar{command} link of a state's \soar{epmem} structure.
At the end of the \soar{phase} of each decision, after episodic storage, episodic memory processes each state's \emph{epmem} command structure.
Results, meta-data, and errors are placed on the \soar{result} structure of that state's \soar{epmem} structure.

Only one type of retrieval command (which may include optional modifiers) can be issued per state in a single decision cycle.
Malformed commands (including attempts at multiple retrieval types) will result in an error:

\begin{verbatim}
<s> ^epmem.result.status bad-cmd
\end{verbatim}

After a command has been processed, episodic memory will ignore it until some aspect of the command structure changes (via addition/removal of WMEs).
When this occurs, the result structure is cleared and the new command (if one exists) is processed.

All retrieved episodes are recreated exactly as stored, except for any operators that have an acceptable preference, which are recreated with the attribute \soar{operator*}.
For example, if the original episode was:

\begin{verbatim}
(<s> ^operator <o1> +)
(<o1> ^name move)
\end{verbatim}

A retrieval of the episode would become:

\begin{verbatim}
(<s> ^operator* <o1>)
(<o1> ^name move)
\end{verbatim}

\subsection{Cue-Based Retrievals}
Cue-based retrieval commands are used to search for an episode in the store that best matches an agent-supplied cue, while adhering to optional modifiers.
A cue is composed of WMEs that partially describe a top-state of working memory in the retrieved episode.
All cue-based retrieval requests must contain a single \soarb{\carat query} cue and, optionally, a single \soarb{\carat neg-query} cue.

\begin{verbatim}
<s> ^epmem.command.query <required-cue>
<s> ^epmem.command.neg-query <optional-negative-cue>
\end{verbatim}

A \soar{\carat query} cue describes structures desired in the retrieved episode, whereas a \soar{\carat neg-query} cue describes non-desired structures.
For example, the following Soar production creates a \soar{\carat query} cue consisting of a particular state name and a copy of a current value on the \soar{input-link} structure:

\begin{verbatim}
sp {epmem*sample*query
    (state <s> ^epmem.command <ec>
               ^io.input-link.foo <bar>)
-->
    (<ec> ^query <q>)
    (<q> ^name my-state-name
         ^io.input-link.foo <bar>)
}
\end{verbatim}

\index{working memory activation}
As detailed below, multiple prior episodes may equally match the structure and contents of an agent's cue.
Nuxoll has produced initial evidence that in some tasks, retrieval quality improves when using \emph{activation} of cue WMEs as a form of feature weighting.
Thus, episodic memory supports integration with working memory activation (see Section \ref{wm-activation} on page \pageref{wm-activation}).
For a theoretical discussion of the Soar implementation of working memory activation, consider reading \emph{Comprehensive Working Memory Activation in Soar} (Nuxoll, A., Laird, J., James, M., ICCM 2004).

The cue-based retrieval process can be thought of conceptually as a nearest-neighbor search.
First, all candidate episodes, defined as episodes containing at least one leaf WME (a cue WME with no sub-structure) in at least one cue, are identified.
Two quantities are calculated for each candidate episode, with respect to the supplied cue(s): the cardinality of the match (defined as the number of matching leaf WMEs) and the activation of the match (defined as the sum of the activation values of each matching leaf WME).
Note that each of these values is negated when applied to a negative query.
To compute each candidate episode's match score, these quantities are combined with respect to the \soarb{balance} parameter as follows:

$$(balance)*(cardinality) + (1-balance)*(activation)$$

Performing a graph match on each candidate episode, with respect to the structure of the cue, could be very computationally expensive, so episodic memory implements a two-stage matching process.
An episode with perfect cardinality is considered a perfect \emph{surface} match and, per the \soarb{graph-match} parameter, is subjected to further \emph{structural} matching.
Whereas surface matching efficiently determines if all paths to leaf WMEs exist in a candidate episode, graph matching indicates whether or not the cue can be structurally unified with the candidate episode (paying special regard to the structural constraints imposed by shared identifiers).
Cue-based matching will return the most recent structural match, or the most recent candidate episode with the greatest match score.

A special note should be made with respect to how short- vs. long-term identifiers (see Section \ref{SMEM-kr} on page \pageref{SMEM-kr}) are interpreted in a cue.
Short-term identifiers are processed much as they are in working memory -- transient structures.
Cue matching will try to find any identifier in an episode (with respect to WME path from state) that can apply.
Long-term identifiers, however, are treated as constants.
Thus, when analyzing the cue, episodic memory will not consider long-term identifier augmentations, and will only match with the same long-term identifier (in the same context) in an episode.

The case-based retrieval process can be further controlled using optional modifiers:

\vspace{-8pt}
\begin{itemize}
\item
	The \soarb{before} command requires that the retrieved episode come relatively before a supplied time:
	\vspace{-6pt}
	\begin{verbatim}
	<s> ^epmem.command.before time
	\end{verbatim}
	\vspace{-6pt}
\item
	The \soarb{after} command requires that the retrieved episode come relatively after a supplied time:
	\vspace{-6pt}
	\begin{verbatim}
	<s> ^epmem.command.after time
	\end{verbatim}
	\vspace{-6pt}
\item
	The \soarb{prohibit} command requires that the time of the retrieved episode is not equal to a supplied time:
	\vspace{-6pt}
	\begin{verbatim}
	<s> ^epmem.command.prohibit time
	\end{verbatim}
	\vspace{-6pt}
	Multiple prohibit command WMEs may be issued as modifiers to a single CB retrieval.
	\vspace{-6pt}
\end{itemize}
\vspace{-12pt}

If no episode satisfies the cue(s) and optional modifiers an error is returned:

\begin{verbatim}
<s> ^epmem.result.failure <query> <optional-neg-query>
\end{verbatim}

If an episode is returned, there is additional meta-data supplied (\ref{EPMEM-meta}).

\subsection{Absolute Non-Cue-Based Retrieval}
At time of storage, each episode is attributed a unique \emph{time}.
This is the current value of \soarb{time} statistic and is provided as the \emph{memory-id} meta-data item of retrieved episodes (\ref{EPMEM-meta}).
An absolute non-cue-based retrieval is one that requests an episode by time.
An agent issues an absolute non-cue-based retrieval by creating a WME on the \soar{command} structure with attribute \emph{retrieve} and value equal to the desired time:

\begin{verbatim}
<s> ^epmem.command.retrieve time
\end{verbatim}

Supplying an invalid value for the \soar{retrieve} command will result in an error.

The time of the first episode in an episodic store will have value 1 and each subsequent episode's time will increase by 1.
Thus the desired time may be the mathematical result of operations performed on a known episode's time.

The current episodic memory implementation does not implement any episodic store dynamics, such as forgetting.
Thus any integer time greater than 0 and less than the current value of the \soar{time} statistic will be valid.
However, if forgetting is implemented in future versions, no such guarantee will be made.

\subsection{Relative Non-Cue-Based Retrieval}
Episodic memory supports the ability for an agent to ``play forward" episodes using relative non-cue-based retrievals.

Episodic memory stores the time of the last successful retrieval (non-cue-based or cue-based).
Agents can indirectly make use of this information by issuing \soarb{next} or \soarb{previous} commands.
Episodic memory executes these commands by attempting to retrieve the episode immediately proceeding/preceding the last successful retrieval (respectively).
To issue one of these commands, the agent must create a new WME on the \soar{command} link with the appropriate attribute (\soar{next} or \soar{previous}) and value of an arbitrary identifier:

\begin{verbatim}
<s> ^epmem.command.next <n>
<s> ^epmem.command.previous <p>
\end{verbatim}

If no such episode exists then an error is returned.

Currently, if the time of the last successfully retrieved episode is known to the agent (as could be the case by accessing result meta-data), these commands are identical to performing an absolute non-cue-based retrieval after adding/subtracting 1 to the last time (respectively).
However, if an episodic store dynamic like forgetting is implemented, these relative commands are guaranteed to return the next/previous valid episode (assuming one exists).

\subsection{Retrieval Meta-Data}
\label{EPMEM-meta}
\index{epmem!structures}

The following list details the WMEs that episodic memory creates in the \soar{result} link of the \soar{epmem} structure wherein a command was issued:

\begin{itemize}

\item \soarb{retrieved <retrieval-root>}
	If episodic memory retrieves an episode, that memory is placed here. This WME is an identifier that is treated as the root of the state that was used to create the episodic memory. If the \soar{retrieve} command was issued with an invalid time, the value of this WME will be \emph{no-memory}.
\item \soarb{success <query> <optional-neg-query>}
	If the cue-based retrieval was successful, the WME will have the status as the attribute and the value of the identifier of the query (and neg-query, if applicable).
\item \soarb{match-score}
	This WME is created whenever an episode is successfully retrieved from a cue-based retrieval command. The WME value is a decimal indicating the raw match score for that episode with respect to the cue(s).
\item \soarb{cue-size}
	This WME is created whenever an episode is successfully retrieved from a cue-based retrieval command. The WME value is an integer indicating the number of leaf WMEs in the cue(s).
\item \soarb{normalized-match-score}
	This WME is created whenever an episode is successfully retrieved from a cue-based retrieval command. The WME value is the decimal result of dividing the raw match score by the cue size. It can hypothetically be used as a measure of episodic memory's relative confidence in the retrieval.
\item \soarb{match-cardinality}
	This WME is created whenever an episode is successfully retrieved from a cue-based retrieval command. The WME value is an integer indicating the number of leaf WMEs matched in the \soar{\carat query} cue minus those matched in the \soar{\carat neg-query} cue.
\item \soarb{memory-id}
	This WME is created whenever an episode is successfully retrieved from a cue-based retrieval command. The WME value is an integer indicating the time of the retrieved episode.
\item \soarb{present-id}
	This WME is created whenever an episode is successfully retrieved from a cue-based retrieval command. The WME value is an integer indicating the current time, such as to provide a sense of ``now" in episodic memory terms. By comparing this value to the \soar{memory-id} value, the agent can gain a sense of the relative time that has passed since the retrieved episode was recorded.
\item \soarb{graph-match}
	This WME is created whenever an episode is successfully retrieved from a cue-based retrieval command and the \soar{graph-match} parameter was \soar{on}. The value is an integer with value 1 if graph matching was executed successfully and 0 otherwise.
\item \soarb{mapping <mapping-root>}
	This WME is created whenever an episode is successfully retrieved from a cue-based retrieval command, the \soar{graph-match} parameter was \soar{on}, and structural match was successful on the retrieved episode. This WME provides a mapping between identifiers in the cue and in the retrieved episode. For each identifier in the cue, there is a \soar{node} WME as an augmentation to the \soar{mapping} identifier. The node has a \soar{cue} augmentation, whose value is an identifier in the cue, and a \soar{retrieved} augmentation, whose value is an identifier in the retrieved episode. In a graph match it is possible to have multiple identifier mappings -- this map represents the ``first" unified mapping (with respect to episodic memory algorithms).
\end{itemize}

\section{Performance}
\label{EPMEM-perf}
\index{epmem!performance}

There are currently two sources of ``unbounded" computation: graph matching and cue-based queries.
Graph matching is combinatorial in the worst case.
Thus, if an episode presents a perfect surface match, but imperfect structural match (i.e. there is no way to unify the cue with the candidate episode), there is the potential for exhaustive search.
Each identifier in the cue can be assigned one of any historically consistent identifiers (with respect to the sequence of attributes that leads to the identifier from the root), termed a literal.
If the identifier is a multi-valued attribute, there will be more than one candidate literals and this situation can lead to a very expensive search process.
Currently there are no heuristics in place to attempt to combat the expensive backtracking.
Worst-case performance will be combinatorial in the total number of literals for each cue identifier (with respect to cue structure).

The cue-based query algorithm begins with the most recent candidate episode and will stop search as soon as a match is found (since this episode must be the most recent).
Given this procedure, it is trivial to create a two-WME cue that forces a linear search of the episodic store.
Episodic memory combats linear scan by only searching candidate episodes, i.e. only those that contain a change in at least one of the cue WMEs.
However, a cue that has no match and contains WMEs relevant to all episodes will force inspection of all episodes.
Thus, worst-case performance will be linear in the number of episodes.

\subsection{Performance Tweaking}
When using a database stored to disk, several parameters become crucial to performance.
The first is \soarb{commit}, which controls the number of episodes that occur between writes to disk.
If the total number of episodes (or a range) is known ahead of time, setting this value to a greater number will result in greatest performance (due to decreased I/O).

The next two parameters deal with the SQLite cache, which is a memory store used to speed operations like queries by keeping in memory structures like levels of index B+-trees.
The first parameter, \soarb{page-size}, indicates the size, in bytes, of each cache page.
The second parameter, \soarb{cache-size}, suggests to SQLite how many pages are available for the cache.
Total cache size is the product of these two parameter settings.
The cache memory is not pre-allocated, so short/small runs will not necessarily make use of this space.
Generally speaking, a greater number of cache pages will benefit query time, as SQLite can keep necessary meta-data in memory.
However, some documented situations have shown improved performance from decreasing cache pages to increase memory locality.
This is of greater concern when dealing with file-based databases, versus in-memory.
The size of each page, however, may be important whether databases are disk- or memory-based.
This setting can have far-reaching consequences, such as index B+-tree depth.
While this setting can be dependent upon a particular situation, a good heuristic is that short, simple runs should use small values of the page size (\soar{1k}, \soar{2k}, \soar{4k}), whereas longer, more complicated runs will benefit from larger values (\soar{8k}, \soar{16k}, \soar{32k}, \soar{64k}).
One known situation of concern is that as indexed tables accumulate many rows (\tild millions), insertion time of new rows can suffer an infrequent, but linearly increasing burst of computation.
In episodic memory, this situation will typically arise with many episodes and/or many working memory changes.
Increasing the page size will reduce the intensity of the spikes at the cost of increasing disk I/O and average/total time for episode storage.
Thus, the settings of page size for long, complicated runs establishes the desired balance of reactivity (i.e. max computation) and average speed.
To ground this discussion, the Figure \ref{fig:epmem-cache} depicts maximum and average episodic storage time (the value of the epmem\_storage timer, converted to milliseconds) with different page sizes after 10 million decisions (1 episode/decision) of a very basic agent (i.e. very few working memory changes per episode) running on a 2.8GHz Core i7 with Mac OS X 10.6.5.
While only a single use case, the cross-point of these data forms the basis for the decision to default the parameter at 8192 bytes.

\begin{figure}
\insertfigure{Figures/epmem-cache}{2.5in}
\insertcaption{Example episodic memory cache setting data.}
\label{fig:epmem-cache}
\end{figure}

The next parameter is \soarb{optimization}, which can be set to either \soar{safety} or \soar{performance}.
The \soar{safety} parameter setting will use SQLite default settings.
If data integrity is of importance, this setting is ideal.
The \soar{performance} setting will make use of lesser data consistency guarantees for significantly greater performance.
First, writes are no longer synchronous with the OS (synchronous pragma), thus episodic memory won't wait for writes to complete before continuing execution.
Second, transaction journaling is turned off (journal\_mode pragma), thus groups of modifications to the episodic store are not atomic (and thus interruptions due to application/os/hardware failure could lead to inconsistent database state).
Finally, upon initialization, episodic memory maintains a continuous exclusive lock to the database (locking\_mode pragma), thus other applications/agents cannot make simultaneous read/write calls to the database (thereby reducing the need for potentially expensive system calls to secure/release file locks).

Finally, maintaining accurate operation timers can be relatively expensive in Soar.
Thus, these should be enabled with caution and understanding of their limitations.
First, they will affect performance, depending on the level (set via the \soar{timers} parameter).
A level of \soar{three}, for instance, times every step in the cue-based retrieval candidate episode search.
Furthermore, because these iterations are relatively cheap (typically a single step in the linked-list of a b+-tree), timer values are typically unreliable (depending upon the system, resolution is 1 microsecond or more).

\subsection{\soarb{explain-backtraces}}
\label{explain-backtraces}
\index{explain-backtraces}
Print information about chunk and justification backtraces. 
\subsubsection*{Synopsis}
explain-backtraces -f prod_name
explain-backtraces [-c <n>] prod_name
\end{verbatim}
\subsubsection*{Options}
\hline
\soar{\soar{\soar{\soar{ (no args) }}}} & List all productions that can be ``explained''  \\
\hline
\soar{\soar{\soar{\soar{ prod\_name }}}} & List all conditions and grounds for the chunk or justification.  \\
\hline
\soar{\soar{\soar{\soar{ -c, --condition }}}} & Explain why condition number \emph{n}
 is in the chunk or justification.  \\
\hline
\soar{\soar{\soar{\soar{ -f, --full }}}} & Print the full backtrace for the named production  \\
\hline
\end{tabular}
\subsubsection*{Description}
 This command provides some interpretation of backtraces generated during chunking. 
explain-backtraces prodname 
explain-backtraces -c n prodname
\end{verbatim}
 The first variant prints a numbered list of all the conditions for the named chunk or justification, and the ground which resulted in inclusion in the chunk/justification. A \emph{ground}
 is a working memory element (WME) which was tested in the supergoal. Just knowing which WME was tested may be enough to explain why the chunk/justification exists. If not, the second variant, \textbf{explain-backtraces -c n prodname}
, where \emph{n}
 is the condition of interest, can be used to obtain a list of the productions which fired to obtain this condition in the chunk/justification (and the crucial WMEs tested along the way). 
 \textbf{save-backtraces}
 mode must be on when a chunk or justification is created or no explanation will be available. Calling \textbf{explain-backtraces}
 with no argument prints a list of all chunks and justifications for which backtracing information is available. 
\subsubsection*{Examples}
 Examining the chunk \textbf{chunk-65*d13*tie*2}
soar> explain-backtraces chunk-65*d13*tie*2
 (sp chunk-65*d13*tie*2
  (state <s2> ^name water-jug ^jug <n4> ^jug <n3>)
  (state <s1> ^name water-jug ^desired <d1> ^operator <o1> + ^jug <n1>
        ^jug <n2>)
  (<s2> ^desired <d1>)
  (<o1> ^name pour ^into <n1> ^jug <n2>)
  (<n1> ^volume 3 ^contents 0)
  (<s1> ^problem-space <p1>)
  (<p1> ^name water-jug)
  (<n4> ^volume 3 ^contents 3)
  (<n3> ^volume 5 ^contents 0)
  (<n2> ^volume 5 ^contents 3)
-->
  (<s3> ^operator <o1> -))
 1�:  (state <s2> ^name water-jug)     Ground�: (S3 ^name water-jug)
 2�:  (state <s1> ^name water-jug)     Ground�: (S5 ^name water-jug)
 3�:  (<s1> ^desired <d1>)             Ground�: (S5 ^desired D1)
 4�:  (<s2> ^desired <d1>)             Ground�: (S3 ^desired D1)
 5�:  (<s1> ^operator <o1> +)          Ground�: (S5 ^operator O18 +)
 6�:  (<o1> ^name pour)                Ground�: (O18 ^name pour)
 7�:  (<o1> ^into <n1>)                Ground�: (O18 ^into N3)
 8�:  (<n1> ^volume 3)                 Ground�: (N3 ^volume 3)
 9�:  (<n1> ^contents 0)               Ground�: (N3 ^contents 0)
10�:  (<s1> ^jug <n1>)                 Ground�: (S5 ^jug N3)
11�:  (<s1> ^problem-space <p1>)       Ground�: (S5 ^problem-space P3)
12�:  (<p1> ^name water-jug)           Ground�: (P3 ^name water-jug)
13�:  (<s2> ^jug <n4>)                 Ground�: (S3 ^jug N1)
14�:  (<n4> ^volume 3)                 Ground�: (N1 ^volume 3)
15�:  (<n4> ^contents 3)               Ground�: (N1 ^contents 3)
16�:  (<s2> ^jug <n3>)                 Ground�: (S3 ^jug N2)
17�:  (<n3> ^volume 5)                 Ground�: (N2 ^volume 5)
18�:  (<n3> ^contents 0)               Ground�: (N2 ^contents 0)
19�:  (<s1> ^jug <n2>)                 Ground�: (S5 ^jug N4)
20�:  (<n2> ^volume 5)                 Ground�: (N4 ^volume 5)
21�:  (<n2> ^contents 3)               Ground�: (N4 ^contents 3)
22�:  (<o1> ^jug <n2>)                 Ground�: (O18 ^jug N4)
\end{verbatim}
soar> explain-backtraces -c 21 chunk-65*d13*tie*2
Explanation of why condition  (N4 ^contents 3) was included in chunk-65*d13*tie*2
Production chunk-64*d13*opnochange*1 matched
    (N4 ^contents 3) which caused
production selection*select*failure-evaluation-becomes-reject-preference to match
    (E3 ^symbolic-value failure) which caused
A result to be generated.
\end{verbatim}
\subsubsection*{Default Aliases}
\hline
\soar{\soar{\soar{\soar{ Alias }}}} & Maps to  \\
\hline
\soar{\soar{\soar{\soar{ eb }}}} & explain-backtraces  \\
\hline
\end{tabular}
\subsubsection*{See Also}
\hyperref[save-backtraces]{save-backtraces} 
\subsection{\soarb{indifferent-selection}}
\label{indifferent-selection}
\index{indifferent-selection}
Controls indifferent preference arbitration. 
\subsubsection*{Synopsis}
indifferent-selection [-aflr]
\end{verbatim}
\subsubsection*{Options}
\hline
\soar{\soar{\soar{\soar{ -a, --ask }}}} & Ask the user to choose. Not implemented. \\
\hline
\soar{\soar{\soar{\soar{ -f, --first }}}} & Select the first indifferent object from Soar's internal list.  \\
\hline
\soar{\soar{\soar{\soar{ -l, --last }}}} & Select the last indifferent object from Soar's internal list.  \\
\hline
\soar{\soar{\soar{\soar{ -r, --random }}}} & Select randomly (default).  \\
\hline
\end{tabular}
\subsubsection*{Description}
 The \textbf{indifferent-selection}
 command allows the user to set which option should be used to select between operator proposals that are mutally indifferent in preference memory. 
 The default option is \textbf{--random}
 which chooses an operator at random from the set of mutually indifferent proposals, with the selection biased by any existing numeric preferences. For repeatable results, the user may choose the \textbf{--first}
 or \textbf{--last}
 option. ``First'' refers to the list of operator augmentations internal to Soar; the ordering of the augmentations is arbitrary but deterministic, so that if you run Soar repeatedly, \textbf{--first}
 will always make the same decision. Similarly, \textbf{--last}
 chooses the last of the tied objects from the internal list. For complete control over the decision process, the \textbf{--ask}
 option prompts the user to select the next operator from a list of the tied operators. 
 If no argument is provided, \textbf{indifferent-selection}
 will display the current setting. 
\subsubsection*{Default Aliases}
\hline
\soar{\soar{\soar{\soar{ Alias }}}} & Maps to  \\
\hline
\soar{\soar{\soar{\soar{ inds }}}} & indifferent-selection  \\
\hline
\end{tabular}
\subsubsection*{See Also}
\hyperref[numeric-indifferent-mode]{numeric-indifferent-mode} 
\subsection{\soarb{learn}}
\label{learn}
\index{learn}
Set the parameters for chunking, Soar\~A�\^a�$\neg$\^a��s learning mechanism. 
\subsubsection*{Synopsis}
learn [-l]
learn -[d|E|o]
learn -e [ab]
\end{verbatim}
\subsubsection*{Options}
\hline
\soar{\soar{\soar{\soar{ -e, --enable, --on }}}} & Turn chunking on. Can be modified by -a or -b.  \\
\hline
\soar{\soar{\soar{\soar{ -d, --disable, --off }}}} & Turn all chunking off. (default)  \\
\hline
\soar{\soar{\soar{\soar{ -E, --except }}}} & Learning is on, except as specified by RHS \textbf{dont-learn}
 actions.  \\
\hline
\soar{\soar{\soar{\soar{ -o, --only }}}} & Chunking is on only as specified by RHS \textbf{force-learn}
 actions.  \\
\hline
\soar{\soar{\soar{\soar{ -l, --list }}}} & Prints listings of dont-learn and force-learn states.  \\
\hline
\soar{\soar{\soar{\soar{ -a, --all-levels }}}} & Build chunks whenever a subgoal returns a result. Learning must be --enabled.  \\
\hline
\soar{\soar{\soar{\soar{ -b, --bottom-up }}}} & Build chunks only for subgoals that have not yet had any subgoals with chunks built. Learning must be --enabled.  \\
\hline
\end{tabular}
\subsubsection*{Description}
 The learn command controls the parameters for chunking (Soar's learning mechanism). With no arguments, this command prints out the current learning environment status. If arguments are provided, they will alter the learning environment as described in the options and arguments table. The watch command can be used to provide various levels of detail when productions are learned. Learning is \textbf{disabled}
 by default. 
 With the \textbf{--on}
 flag, chunking is on all the time. With the \textbf{--except}
 flag, chunking is on, but Soar will not create chunks for states that have had RHS \textbf{dont-learn}
 actions executed in them. With the \textbf{--only}
 flag, chunking is off, but Soar will create chunks for only those states that have had RHS \textbf{force-learn}
 actions executed in them. With the \textbf{--off}
 flag, chunking is off all the time. 
 The \textbf{--only}
 flag and its companion \textbf{force-learn}
 RHS action allow Soar developers to turn learning on in a particular problem space, so that they can focus on debugging the learning problems in that particular problem space without having to address the problems elsewhere in their programs at the same time. Similarly, the \textbf{--except}
 flag and its companion \textbf{dont-learn}
 RHS action allow developers to temporarily turn learning off for debugging purposes. These facilities are provided as debugging tools, and do not correspond to any theory of learning in Soar. 
 The \textbf{--all-levels}
 and \textbf{--bottom-up}
 flags are orthogonal to the \textbf{--on}
, \textbf{--except}
, \textbf{--only}
, and \textbf{--off}
 flags, and so, may be used in combination with them. With bottom-up learning, chunks are learned only in states in which no subgoal has yet generated a chunk. In this mode, chunks are learned only for the ``bottom'' of the subgoal hierarchy and not the intermediate levels. With experience, the subgoals at the bottom will be replaced by the chunks, allowing higher level subgoals to be chunked. 
 Learning can be turned on or off at any point during a run. 
\subsubsection*{Examples}
learn -e b
\end{verbatim}
 To see all the \textbf{force-learn}
 and \textbf{dont-learn}
learn -l
\end{verbatim}
\subsubsection*{Default Aliases}
\hline
\soar{\soar{\soar{\soar{ Alias }}}} & Maps to  \\
\hline
\soar{\soar{\soar{\soar{ l }}}} & learn  \\
\hline
\end{tabular}
\subsubsection*{See Also}
\hyperref[watch]{watch} \hyperref[explain-backtraces]{explain-backtraces} \hyperref[save-backtraces]{save-backtraces} 
\subsection{\soarb{max-chunks}}
\label{max-chunks}
\index{max-chunks}
Limit the number of chunks created during a decision cycle. 
\subsubsection*{Synopsis}
max-chunks [n]
\end{verbatim}
\subsubsection*{Options}
\hline
\soar{\soar{\soar{ n }}} & Maximum number of chunks allowed during a decision cycle.  \\
\hline
\end{tabular}
\subsubsection*{Description}
 The \textbf{max-chunks}
 command is used to limit the maximum number of chunks that may be created during a decision cycle. The initial value of this variable is 50; allowable settings are any integer greater than 0. 
 The chunking process will end after \textbf{max-chunks}
 chunks have been created, \emph{even if there are more results that have not been backtraced through to create chunks}
, and Soar will proceed to the next phase. A warning message is printed to notify the user that the limit has been reached. 
 This limit is included in Soar to prevent getting stuck in an infinite loop during the chunking process. This could conceivably happen because newly-built chunks may match immediately and are fired immediately when this happens; this can in turn lead to additional chunks being formed, etc. If you see this warning, something is seriously wrong; Soar is unable to guarantee consistency of its internal structures. You should not continue execution of the Soar program in this situation; stop and determine whether your program needs to build more chunks or whether you've discovered a bug (in your program or in Soar itself). 

\input{wikicmd/tex/max-dc-time}
\subsection{\soarb{max-elaborations}}
\label{max-elaborations}
\index{max-elaborations}
Limit the maximum number of elaboration cycles in a given phase. Print a warning message if the limit is reached during a run. 
\subsubsection*{Synopsis}
max-elaborations [n]
\end{verbatim}
\subsubsection*{Options}
\hline
\emph{n}
 & Maximum allowed elaboration cycles, must be a positive integer.  \\
\hline
\end{tabular}
\subsubsection*{Description}
 This command sets and prints the maximum number of elaboration cycles allowed. If \emph{n}
 is given, it must be a positive integer and is used to reset the number of allowed elaboration cycles. The default value is 100. \textbf{max-elaborations}
 with no arguments prints the current value. 
 \textbf{max-elaborations}
 controls the maximum number of elaborations allowed in a single decision cycle. The elaboration phase will end after \emph{max-elaboration}
 cycles have completed, even if there are more productions eligible to fire or retract; and Soar will proceed to the next phase after a warning message is printed to notify the user. This limits the total number of cycles of parallel production firing but does not limit the total number of productions that can fire during elaboration. 
 This limit is included in Soar to prevent getting stuck in infinite loops (such as a production that repeatedly fires in one elaboration cycle and retracts in the next); if you see the warning message, it may be a signal that you have a bug your code. However some Soar programs are designed to require a large number of elaboration cycles, so rather than a bug, you may need to increase the value of \emph{max-elaborations}
. 
 In Soar8, \emph{max-elaborations}
 is checked during both the Propose Phase and the Apply Phase. If Soar8 runs more than the max-elaborations limit in either of these phases, Soar8 proceeds to the next phase (either Decision or Output) even if quiescence has not been reached. 
\subsubsection*{Examples}
max-elaborations 
\end{verbatim}
max-elaborations 50
\end{verbatim}

\input{wikicmd/tex/max-goal-depth}
\subsection{\soarb{max-memory-usage}}
\label{max-memory-usage}
\index{max-memory-usage}
Set the amount of bytes necessary to trigger the memory usage exceeded event. 
\subsubsection*{Synopsis}
max-memory-usage [n]
\end{verbatim}
\subsubsection*{Options}
\hline
\soar{\soar{\soar{ n }}} & Size of limit in bytes.  \\
\hline
\end{tabular}
\subsubsection*{Description}
 The \textbf{max-memory-usage}
 command is used to trigger the memory usage exceeded event. The initial value of this is 100MB (100,000,000); allowable settings are any integer greater than 0. 
 Using the command with no arguments displays the current limit. 

\subsection{\soarb{max-nil-output-cycles}}
\label{max-nil-output-cycles}
\index{max-nil-output-cycles}
Limit the maximum number of decision cycles that are executed without producing output when run is invoked with run-til-output args. 
\subsubsection*{Synopsis}
max-nil-output-cycles [n]
\end{verbatim}
\subsubsection*{Options}
\hline
\emph{n}
 & Maximum number of consecutive output cycles allowed without producing output. Must be a positive integer.  \\
\hline
\end{tabular}
\subsubsection*{Description}
 This command sets and prints the maximum number of nil output cycles (output cycles that put nothing on the output link) allowed when running using run-til-output (run --output). If \emph{n}
 is not given, this command prints the current number of nil-output-cycles allowed. If \emph{n}
 is given, it must be a positive integer and is used to reset the maximum number of allowed nil output cycles. 
 \textbf{max-nil-output-cycles}
 controls the maximum number of output cycles that generate no output allowed when a \textbf{run --out}
 command is issued. After this limit has been reached, Soar stops. The default initial setting of \emph{n}
 is 15. 
\subsubsection*{Examples}
max-nil-output-cycles 
\end{verbatim}
max-nil-output-cycles 25 
\end{verbatim}
\subsubsection*{See Also}
\hyperref[run]{run} 
\subsection{\soarb{multi-attributes}}
\label{multi-attributes}
\index{multi-attributes}
Declare a symbol to be multi-attributed. 
\subsubsection*{Synopsis}
multi-attributes [symbol [\emph{n}
]]
\end{verbatim}
\subsubsection*{Options}
\hline
\soar{\soar{\soar{symbol}}} & Any Soar attribute.  \\
\hline
\emph{n}
 & Integer $>$ 1, estimate of degree of simultaneous values for attribute.  \\
\hline
\end{tabular}
\subsubsection*{Description}
 This command declares the given symbol to be an attribute which can take on multiple values. The optional \emph{n}
 is an integer ($>$1) indicating an upper limit on the number of expected values that will appear for an attribute. If \emph{n}
 is not specified, the value 10 is used for each declared multi-attribute. More informed values will tend to result in greater efficiency. This command is used only to provide hints to the production condition reorderer so it can produce better condition orderings. Better orderings enable the rete network to run faster. This command has no effect on the actual contents of working memory and most users needn't use this at all. 
 Note that multi-attributes declarations must be made before productions are loaded into soar or this command will have no effect. 
\subsubsection*{Examples}
multi-attributes thing 4
\end{verbatim}

\subsection{\soarb{numeric-indifferent-mode}}
\label{numeric-indifferent-mode}
\index{numeric-indifferent-mode}
Select method for combining numeric preferences. 
\subsubsection*{Synopsis}
numeric-indifferent-mode [-as]
\end{verbatim}
\subsubsection*{Options}
\hline
\soar{\soar{\soar{ -a, --avg, --average }}} & Use average mode (default).  \\
\hline
\soar{\soar{\soar{ -s, --sum }}} & Use sum mode.  \\
\hline
\end{tabular}
\subsubsection*{Description}
 The numeric-indifferent-mode command is used to select the method for combining numeric preferences. This command is only meaningful in indifferent-selection --random  mode. 
 The default procedure is \textbf{--avg}
\item  If the operator has at least one numeric preference, assign it the value that is the average of all of its numeric preferences. 
\item  If the operator has no numeric preferences (but has been included in the indifferent selection through some combination of non-numeric preferences), assign it the value 50. 
\end{itemize}
 The intended range of numeric-preference values for \textbf{--avg}
 mode is 0-100. 
 The other combination option \textbf{--sum}
\item  Add together any numeric preferences for the operator (defaulting to 0 if there are none). 
\item  Assign the operator the value e\^{}\{PreferenceSum / AgentTemperature\}, where AgentTemperature is a compile-time constant currently set at 25.0. 
\end{itemize}
 Any real-numbered preference may be used in \textbf{--sum}
 mode. 
 Once a value has been computed for each operator, the next operator is selected probabilistically, with each candidate operator's chance weighted by its computed value. 

\subsection{\soarb{o-support-mode}}
\label{o-support-mode}
\index{o-support-mode}
Choose experimental variations of o-support. 
\subsubsection*{Synopsis}
o-support-mode [0|1|2|3|4]
\end{verbatim}
\subsubsection*{Options}
\hline
\soar{\soar{\soar{ 0 }}} & Mode 0 is the base mode. O-support is calculated based on the structure of working memory that is tested and modified. Testing an operator or operator acceptable preference results in state or operator augmentations being o-supported. The support computation is very complex (see soar manual).  \\
\hline
\soar{\soar{\soar{ 1 }}} & Not available through gSKI.  \\
\hline
\soar{\soar{\soar{ 2 }}} & Mode 2 is the same as mode 0 except that all support is calculated the production structure, not from working memory structure. Augmentations of operators are still o-supported.  \\
\hline
\soar{\soar{\soar{ 3 }}} & Mode 3 is the same as mode 2 except that operator elaborations (adding attributes to operators) now get i-support even though you have to test the operator to elaborate an operator.  \\
\hline
\soar{\soar{\soar{ 4 }}} & Mode 4 is the default.  \\
\hline
\end{tabular}
\subsubsection*{Description}
 The \textbf{o-support-mode}
 command is used to control the way that o-support is determined for preferences. Only o-support modes 3 \& 4 can be considered current to Soar8, and o-support mode 4 should be considered an improved version of mode 3. The default o-support mode is mode 4. 
 In o-support modes 3 \& 4, support is given production by production; that is, all preferences generated by the RHS of a single instantiated production will have the same support. The difference between the two modes is in how they handle productions with both operator and non-operator augmentations on the RHS. For more information on o-support calculations, see the relevant appendix in the Soar manual. 
 Running o-support-mode with no arguments prints out the current o-support-mode. 

\input{wikicmd/tex/predict}
\chapter{Reinforcement Learning}
\label{RL}
\index{reinforcement learning}
\index{preference!numeric-indifferent}
\index{rl}

Soar has a reinforcement learning (RL) mechanism that tunes operator selection knowledge based on a given reward function.
This chapter describes the RL mechanism and how it is integrated with production memory, the decision cycle, and the state stack.
We assume that the reader is familiar with basic reinforcement learning concepts and notation. If not, we recommend first reading \emph{Reinforcement Learning: An Introduction} (1998) by Richard S. Sutton and Andrew G. Barto.
The detailed behavior of the RL mechanism is determined by numerous parameters that can be controlled and configured via the \soarb{rl} command.
Please refer to the documentation for that command in section \ref{rl} on page \pageref{rl}.

\section{RL Rules}
\label{RL-rules}

Soar's RL mechanism learns Q-values for state-operator\footnote{
In this context, the term ``state'' refers to the state of the task or environment, not a state identifier.
For the rest of this chapter, bold capital letter names such as \soarb{S1} will refer to identifiers and italic lowercase names such as $s_1$ will refer to task states.}
pairs.
Q-values are stored as numeric indifferent preferences created by specially formulated productions called \emph{RL rules}.
RL rules are identified by syntax.
A production is a RL rule if and only if its left hand side tests for a proposed operator, its right hand side creates a single numeric indifferent preference, and it is not a template rule (see \ref{RL-templates}).
These constraints ease the technical requirements of identifying/updating RL rules and makes it easy for the agent programmer to add/maintain RL capabilities within an agent.
We define an \emph{RL operator} as an operator with numeric indifferent preferences created by RL rules.

The following is an RL rule:

\begin{verbatim}
sp {rl*3*12*left
   (state <s> ^name task-name
              ^x 3
              ^y 12
	          ^operator <o> +)
   (<o> ^name move
	    ^direction left)
-->
   (<s> ^operator <o> = 1.5)
}
\end{verbatim}

Note that the LHS of the rule can test for anything as long as it contains a test for a proposed operator.
The RHS is constrained to exactly one action: creating a numeric indifferent preference for the proposed operator.

The following are not RL rules:

\begin{verbatim}
sp {multiple*preferences
   (state <s> ^operator <o> +)
-->
   (<s> ^operator <o> = 5, >)
}

sp {variable*binding
    (state <s> ^operator <o> +
               ^value <v>)
-->
    (<s> ^operator <o> = <v>)
}
\end{verbatim}

The first rule proposes multiple preferences for the proposed operator and thus does not comply with the rule format.
The second rule does not comply because it does not provide a \emph{constant} for the numeric indifferent preference value.

In the typical RL use case, the user intends for the agent to learn the best operator in each possible state of the environment.
The most straightforward way to achieve this is to give the agent a set of RL rules, each matching exactly one possible state-operator pair.
This approach is equivalent to a table-based RL algorithm, where the Q-value of each state-operator pair corresponds to the numeric indifferent preference created by exactly one RL rule.

In the more general case, multiple RL rules can match a single state-operator pair, and a single RL rule can match multiple state-operator pairs.
all numeric indifferent preferences for an operator are summed when calculating the operator's Q-value\footnote{
This is assuming the value of \soarb{numeric-indifferent-mode} is set to \soarb{sum}.
In general, the RL mechanism only works correctly when this is the case, and we assume this case in the rest of the chapter.
See page \pageref{decide-numeric-indifferent-mode} for more information about this parameter.}.
In this context, RL rules can be interpreted more generally as binary features in a linear approximator of each state-operator pair's Q-value, and their numeric indifferent preference values their weights.
In other words,
$$Q(s, a) = w_1 \phi_2 (s, a) + w_2 \phi_2 (s, a) + \ldots + w_n \phi_n (s, a)$$
where all RL rules in production memory are numbered $1 \dots n$, $Q(s, a)$ is the Q-value of the state-operator pair $(s, a)$, $w_i$ is the numeric indifferent preference value of RL rule $i$, $\phi_i (s, a) = 0$ if RL rule $i$ does not match $(s, a)$, and $\phi_i (s, a) = 1$ if it does.
This interpretation allows RL rules to simulate a number of popular function approximation schemes used in RL such as tile coding and sparse coding.

\section{Reward Representation}
\label{RL-reward}

RL updates are driven by reward signals.
In Soar, these reward signals are given to the RL mechanism through a working memory link called the \soarb{reward-link}.
Each state in Soar's state stack is automatically populated with a \soarb{reward-link} structure upon creation.
Soar will check this structure for a numeric reward signal for the last operator executed in the associated state at the beginning of every decision phase.
Reward is also collected when the agent is halted or a state is retracted.
% What happens when an agent with multiple states is halted? Do the rewards in the substates get collected?

In order to be recognized, the reward signal must follow this pattern:

\begin{verbatim}
(<r1> ^reward <r2>)
(<r2> ^value [val])
\end{verbatim}

where \verb=<r1>= is the \soarb{reward-link} identifier, \verb=<r2>= is some intermediate identifier, and \verb=[val]= is any constant numeric value.
Any structure that does not match this pattern is ignored.
If there are multiple valid reward signals, their values are summed into a single reward signal.
As an example, consider the following state:

\begin{verbatim}
(S1 ^reward-link R1)
  (R1 ^reward R2)
    (R2 ^value 1.0)
  (R1 ^reward R3)
    (R3 ^value -0.2)
\end{verbatim}  

In this state, there are two reward signals with values 1.0 and -0.2.
They will be summed together for a total reward of 0.8 and this will be the value given to the RL update algorithm.

There are two reasons for requiring the intermediate identifier.
The first is so that multiple reward signals with the same value can exist simultaneously.
Since working memory is a set, multiple WMEs with identical values in all three positions (identifier, attribute, value) cannot exist simultaneously.
Without an intermediate identifier, specifying two rewards with the same value would require a WME structure such as

\begin{verbatim}
(S1 ^reward-link R1)
  (R1 ^reward 1.0)
  (R1 ^reward 1.0)
\end{verbatim}

which is invalid. With the intermediate identifier, the rewards would be specified as

\begin{verbatim}
(S1 ^reward-link R1)
  (R1 ^reward R2)
    (R2 ^value 1.0)
  (R1 ^reward R3)
    (R3 ^value 1.0)
\end{verbatim}

which is valid.
The second reason for requiring an intermediate identifier in the reward signal is so that the rewards can be augmented with additional information, such as their source or how long they have existed.
Although this information will be ignored by the RL mechanism, it can be useful to the agent or programmer.
For example:

\begin{verbatim}
(S1 ^reward-link R1)
  (R1 ^reward R2)
    (R2 ^value 1.0)
    (R2 ^source environment)
  (R1 ^reward R3)
    (R3 ^value -0.2)
    (R3 ^source intrinsic)
    (R3 ^duration 5)
\end{verbatim}  

The \verb=(R2 ^source environment)=, \verb=(R3 ^source intrinsic)=, and \verb=(R3 ^duration 5)= \\
WMEs are arbitrary and ignored by RL, but were added by the agent to keep 
track of where the rewards came from and for how long.

Note that the \soarb{reward-link} is not part of the \soarb{io} structure and is not modified directly by the environment.
Reward information from the environment should be copied, via rules, from the \soarb{input-link} to the \soarb{reward-link}.
Also note that when collecting rewards, Soar simply scans the \soarb{reward-link} and sums the values of all valid reward WMEs.
The WMEs are not modified and no bookkeeping is done to keep track of previously seen WMEs.
This means that reward WMEs that exist for multiple decision cycles will be collected multiple times.

\section{Updating RL Rule Values}
\label{RL-algo}

Soar's RL mechanism is integrated naturally with the decision cycle and performs online updates of RL rules.
Whenever an RL operator is selected, the values of the corresponding RL rules will be updated.
The update can be on-policy (Sarsa) or off-policy (Q-Learning), as controlled by the \soarb{learning-policy} parameter of the \soarb{rl} command.
Let $\delta_t$ be the amount the Q-value of an RL operator changes in an update.
For Sarsa, we have
$$ \delta_t = \alpha \left[ r_{t+1} + \gamma Q(s_{t+1}, a_{t+1}) - Q(s_t, a_t) \right] $$
where 
\begin{itemize}
\item $Q(s_t, a_t)$ is the Q-value of the state and chosen operator in decision cycle $t$.
\item $Q(s_{t+1}, a_{t+1})$ is the Q-value of the state and chosen RL operator in the next decision cycle.
\item $r_{t+1}$ is the total reward collected in the next decision cycle.
\item $\alpha$ and $\gamma$ are the settings of the \soarb{learning-rate} and \soarb{discount-rate} parameters of the \soarb{rl} command, respectively.
\end{itemize}

Note that since $\delta_t$ depends on $Q(s_{t+1}, a_{t+1})$, the update for the operator selected in decision cycle $t$ is not applied until the next RL operator is chosen.
For Q-Learning, we have
$$ \delta_t = \alpha \left[ r_{t+1} + \gamma \underset{a \in A_{t+1}}{\max} Q(s_{t+1}, a) - Q(s_t, a_t) \right] $$
where $A_{t+1}$ is the set of RL operators proposed in the next decision cycle.

Finally, $\delta_t$ is divided by the number of RL rules comprising the Q-value for the operator and the numeric indifferent values for each RL rule is updated by that amount.

An example walkthrough of a Sarsa update with $\alpha = 0.3$ and $\gamma = 0.9$ (the default settings in Soar) follows.

\begin{enumerate}

\item In decision cycle $t$, an operator \soarb{O1} is proposed, and RL rules \soarb{rl-1} and \soarb{rl-2} create the following numeric indifferent preferences for it:
\begin{verbatim}
   rl-1: (S1 ^operator O1 = 2.3)
   rl-2: (S1 ^operator O1 =  -1)
\end{verbatim}  
	The Q-value for \soarb{O1} is $Q(s_t, \soarb{O1}) = 2.3 - 1 = 1.3$.
	 
\item \soarb{O1} is selected and executed, so $Q(s_t, a_t) = Q(s_t, \soarb{O1}) = 1.3$.

\item In decision cycle $t+1$, a total reward of 1.0 is collected on the \soarb{reward-link}, an operator \soarb{O2} is proposed, and another RL rule \soarb{rl-3} creates the following numeric indifferent preference for it:
\begin{verbatim}
	rl-3: (S1 ^operator O2 = 0.5)
\end{verbatim}
	So $Q(s_{t+1}, \soarb{O2}) = 0.5$.

\item \soarb{O2} is selected, so $Q(s_{t+1}, a_{t+1}) = Q(s_{t+1}, \soarb{O2}) = 0.5$
	Therefore, 
	$$\delta_t = \alpha \left[r_{t+1} + \gamma Q(s_{t+1}, a_{t+1}) - Q(s_t, a_t) \right] = 0.3 \times [ 1.0 + 0.9 \times 0.5 - 1.3 ] = 0.045$$
	Since \soarb{rl-1} and \soarb{rl-2} both contributed to the Q-value of \soarb{O1}, $\delta_t$ is evenly divided amongst them, resulting in updated values of
\begin{verbatim}
   rl-1: (<s> ^operator <o> = 2.3225)
   rl-2: (<s> ^operator <o> = -0.9775)
\end{verbatim}

\item \soarb{rl-3} will be updated when the next RL operator is selected.
\end{enumerate}

\subsection{Gaps in Rule Coverage}
\label{RL-gaps}

Call an operator with numeric indifferent preferences an RL operator.
The previous description had assumed that RL operators were selected in both decision cycles $t$ and $t+1$.
If the operator selected in $t+1$ is not an RL operator, then $Q(s_{t+1}, a_{t+1})$ would not be defined, and an update for the RL operator selected at time $t$ will be undefined.
We will call a sequence of one or more decision cycles in which RL operators are not selected between two decision cycles in which RL operators are selected a \emph{gap}.
Conceptually, it is desirable to use the temporal difference information from the RL operator after the gap to update the Q-value of the RL operator before the gap.
There are no intermediate storage locations for these updates.
Requiring that RL rules support operators at every decision can be difficult for agent programmers, particularly for operators that do not represent steps in a task, but instead perform generic maintenance functions, such as cleaning processed output-link structures.

To address this issue, Soar's RL mechanism supports automatic propagation of updates over gaps.
For a gap of length $n$, the Sarsa update is
$$\delta_t = \alpha \left[ \sum_{i=t}^{t+n}{\gamma^{i-t} r_i} + \gamma^{n+1} Q(s_{t+n+1}, a_{t+n+1}) - Q(s_t, a_t) \right]$$
and the Q-Learning update is
$$\delta_t = \alpha \left[ \sum_{i=t}^{t+n}{\gamma^{i-t} r_i} + \gamma^{n+1} \underset{a \in A_{t+n+1}}{\max} Q(s_{t+n+1}, a) - Q(s_t, a_t) \right]$$

Note that rewards will still be collected during the gap, but they are discounted based on the number of decisions they are removed from the initial RL operator.

Gap propagation can be disabled by setting the \soarb{temporal-extension} parameter of the \soarb{rl} command to \soarb{off}.
When gap propagation is disabled, the RL rules preceding a gap are updated using $Q(s_{t+1}, a_{t+1}) = 0$.
The \soarb{rl} setting of the \soarb{watch} command (see Section \ref{trace} on page \pageref{trace}) is useful in identifying gaps.


\subsection{RL and Substates}
\label{RL-substates}

When an agent has multiple states in its state stack, the RL mechanism will treat each substate independently.
As mentioned previously, each state has its own \soarb{reward-link}.
When an RL operator is selected in a state \soarb{S}, the RL updates for that operator are only affected by the rewards collected on the \soarb{reward-link} for \soarb{S} and the Q-values of subsequent RL operators selected in \soarb{S}.

The only exception to this independence is when a selected RL operator forces an operator-no-change impasse.
When this occurs, the number of decision cycles the RL operator at the superstate remains selected is dependent upon the processing in the impasse state.
Consider the operator trace in Figure \ref{fig:rl-optrace}.

\begin{itemize}
\item At decision cycle 1, RL operator \soarb{O1} is selected in \soarb{S1} and causes an operator-no-change impass for three decision cycles.
\item In the substate \soarb{S2}, operators \soarb{O2}, \soarb{O3}, and \soarb{O4} are selected and applied sequentially.
\item Meanwhile in \soarb{S1}, reward values $r_2$, $r_3$, and $r_4$ are put on the \soarb{reward-link} sequentially.
\item Finally, the impasse is resolved by \soarb{O4}, the proposal for \soarb{O1} is retracted, and RL operator \soarb{O5} is selected in \soarb{S1}.
\end{itemize}

\begin{figure}
\insertfigure{Figures/rl-optrace}{1.5in}
\insertcaption{Example Soar substate operator trace.}
\label{fig:rl-optrace}
\end{figure}

In this scenario, only the RL update for $Q(s_1, \soarb{O1})$ will be different from the ordinary case.
Its value depends on the setting of the \soarb{hrl-discount} parameter of the \soarb{rl} command.
When this parameter is set to the default value \soarb{on}, the rewards on \soarb{S1} and the Q-value of \soarb{O5} are discounted by the number of decision cycles they are removed from the selection of \soarb{O1}.
In this case the update for $Q(s_1, \soarb{O1})$ is
$$\delta_1 = \alpha \left[ r_2 + \gamma r_3 + \gamma^2 r_4 + \gamma^3 Q(s_5, \soarb{O5}) - Q(s_1, \soarb{O1}) \right]$$
which is equivalent to having a three decision gap separating \soarb{O1} and \soarb{O5}.

When \soarb{hrl-discount} is set to \soarb{off}, the number of cycles \soarb{O1} has been impassed will be ignored.
Thus the update would be
$$\delta_1 = \alpha \left[ r_2 + r_3 + r_4 + \gamma Q(s_5, \soarb{O5}) - Q(s_1, \soarb{O1}) \right]$$

For impasses other than operator no-change, RL acts as if the impasse hadn't occurred.
If \soarb{O1} is the last RL operator selected before the impasse, $r_2$ the reward received in the decision cycle immediately following, and \soarb{O}$_\soarb{n}$, the first operator selected after the impasse, then \soarb{O1} is updated with 
$$\delta_1 = \alpha \left[ r_2 + \gamma Q(s_n, \soarb{O}_\soarb{n}) - Q(s_1, \soarb{O1}) \right]$$

If an RL operator is selected in a substate immediately prior to the state's retraction, the RL rules will be updated based only on the reward signals present and not on the Q-values of future operators.
This point is not covered in traditional RL theory.
The retraction of a substate corresponds to a suspension of the RL task in that state rather than its termination, so the last update assumes the lack of information about future rewards rather than the discontinuation of future rewards.
To handle this case, the numeric indifferent preference value of each RL rule is stored as two separate values, the expected current reward (ECR) and expected future reward (EFR).
The ECR is an estimate of the expected immediate reward signal for executing the corresponding RL operator.
The EFR is an estimate of the time discounted Q-value of the next RL operator.
Normal updates correspond to traditional RL theory (showing the Sarsa case for simplicity):
\begin{align*}
\delta_{ECR} &= \alpha \left[ r_t - ECR(s_t, a_t) \right] \\
\delta_{EFR} &= \alpha \left[ \gamma Q(s_{t+1}, a_{t+1}) - EFR(s_t, a_t) \right] \\
\delta_t &= \delta_{ECR} + \delta_{EFR} \\
&= \alpha \left[ r_t + \gamma Q(s_{t+1}, a_{t+1}) - \left( ECR(s_t, a_t) + EFR(s_t, a_t) \right) \right] \\
&= \alpha \left[ r_t + \gamma Q(s_{t+1}, a_{t+1}) - Q(s_t, a_t) \right]
\end{align*}
During substate retraction, only the ECR is updated based on the reward signals present at the time of retraction, and the EFR is unchanged.

Soar's automatic subgoaling and RL mechanisms can be combined to naturally implement hierarchical reinforcement learning algorithms such as MAXQ and options.

\subsection{Eligibility Traces}
\label{RL-et}
The RL mechanism supports eligibility traces, which can improve the speed of learning by 
updating RL rules across multiple sequential steps. \\
The \soarb{eligibility-trace-decay-rate} and \soarb{eligibility-trace-tolerance} parameters control this mechanism.
By setting \soarb{eligibility-trace-decay-rate} to \soarb{0} (default), eligibility traces are in effect disabled.
When eligibility traces are enabled, the particular algorithm used is dependent upon the learning policy.
For Sarsa, the eligibility trace implementation is \emph{Sarsa($\lambda$)}. 
For Q-Learning, the eligibility trace implementation is \emph{Watkin's Q($\lambda$)}.

\subsubsection{Exploration}

The \soarb{indifferent-selection} command (page \pageref{decide-indifferent-selection}) determines how operators are selected based on their numeric indifferent preferences.
Although all the indifferent selection settings are valid regardless of how the numeric indifferent preferences were arrived at, the \soarb{epsilon-greedy} and \soarb{boltzmann} settings are specifically designed for use with RL and correspond to the two most common exploration strategies.
In an effort to maintain backwards compatibility, the default exploration policy is \soarb{softmax}.
As a result, one should change to \soarb{epsilon-greedy} or \soarb{boltzmann} when the reinforcement learning mechanism is enabled.

\subsection{GQ($\lambda$)}

\emph{Sarsa($\lambda$)} and \emph{Watkin's Q($\lambda$)} help agents to solve the temporal credit assignment problem more quickly.
However, if you wish to implement something akin to CMACs to generalize from experience, convergence is not guaranteed by these algorithms.
\emph(GQ($\lambda$)} is a gradient descent algorithm designed to ensure convergence when learning off-policy.
Soar provides both \soarb{on-policy-gq-lambda} and \soarb{off-policy-gq-lambda} to increase the likelihood of convergence when learning under these conditions.
If you should choose to use one of these algorithms, we recommend setting \soarb{step-size-parameter} to something small, such as $0.01$
in order to ensure that the secondary set of weights used by \emph(GQ($\lambda$)} change slowly enough for efficient convergence.

\section{Automatic Generation of RL Rules}

The number of RL rules required for an agent to accurately approximate operator Q-values is usually infeasibly large to write by hand, even for small domains.
Therefore, several methods exist to automate this.

\subsection{The gp Command}
The \soar{gp} command can be used to generate productions based on simple patterns.
This is useful if the states and operators of the environment can be distinguished by a fixed number of dimensions with finite domains.
An example is a grid world where the states are described by integer row/column coordinates, and the available operators are to move north, south, east, or west.
In this case, a single \soar{gp} command will generate all necessary RL rules:
	
\begin{verbatim}
gp {gen*rl*rules
   (state <s> ^name gridworld
              ^operator <o> +
              ^row [ 1 2 3 4 ]
              ^col [ 1 2 3 4 ])
   (<o> ^name move
        ^direction [ north south east west ])
-->
   (<s> ^operator <o> = 0.0)
}
\end{verbatim}
	
For more information see the documentation for this command on page \pageref{gp}.

\subsection{Rule Templates}
\label{RL-templates}

Rule templates allow Soar to dynamically generate new RL rules based on a predefined pattern as the agent encounters novel states.
This is useful when either the domains of environment dimensions are not known ahead of time, or when the enumerable state space of the environment is too large to capture in its entirety using \soar{gp}, but the agent will only encounter a small fraction of that space during its execution.
For example, consider the grid world example with 1000 rows and columns.
Attempting to generate RL rules for each grid cell and action a priori will result in $1000 \times 1000 \times 4 = 4 \times 10^6$ productions.
However, if most of those cells are unreachable due to walls, then the agent will never fire or update most of those productions.
Templates give the programmer the convenience of the \soar{gp} command without filling production memory with unnecessary rules.

Rule templates have variables that are filled in to generate RL rules as the agent encounters novel combinations of variable values.
A rule template is valid if and only if it is marked with the \soarb{:template} flag and, in all other respects, adheres to the format of an RL rule.
However, whereas an RL rule may only use constants as the numeric indifference preference value, a rule template may use a variable.
Consider the following rule template:

\begin{verbatim}
sp {sample*rule*template
    :template
    (state <s> ^operator <o> +
               ^value <v>)
-->
    (<s> ^operator <o> = <v>)
}
\end{verbatim}

During agent execution, this rule template will match working memory and create new productions by substituting all variables in the rule template that matched against constant values with the values themselves.
Suppose that the LHS of the rule template matched against the state

\begin{verbatim}
(S1 ^value 3.2)
(S1 ^operator O1 +)
\end{verbatim}

Then the following production will be added to production memory:

\begin{verbatim}
sp {rl*sample*rule*template*1
    (state <s> ^operator <o> +
               ^value 3.2)
-->
    (<s> ^operator <o> = 3.2)
}
\end{verbatim}

The variable \soar{<v>} is replaced by \soar{3.2} on both the LHS and the RHS, but \soar{<s>} and \soar{<o>} are not replaced because they matches against identifiers (\soar{S1} and \soar{O1}).
As with other RL rules, the value of \soar{3.2} on the RHS of this rule may be updated later by reinforcement learning, whereas the value of \soar{3.2} on the LHS will remain unchanged.
If \soar{<v>} had matched against a non-numeric constant, it will be replaced by that constant on the LHS, but the RHS numeric indifference preference value will be set to zero to make the new rule valid.

The new production's name adheres to the following pattern:
\soarb{rl*template-name*id}, where \soarb{template-name} is the name of the originating rule template and \soarb{id} is monotonically increasing integer that guarantees the uniqueness of the name.

If an identical production already exists in production memory, then the newly generate production is discarded.
It should be noted that the current process of identifying unique template match instances can become quite expensive in long agent runs.
Therefore, it is recommended to generate all necessary RL rules using the \soar{gp} command or via custom scripting when possible.

\subsection{Chunking}
Since RL rules are regular productions, they can be learned by chunking just like any other production.
This method is more general than using the \soar{gp} command or rule templates, and is useful if the environment state consists of arbitrarily complex relational structures that cannot be enumerated.

\subsection{\soarb{save-backtraces}}
\label{save-backtraces}
\index{save-backtraces}
Save trace information to explain chunks and justifications. 
\subsubsection*{Synopsis}
save-backtraces [-ed]
\end{verbatim}
\subsubsection*{Options}
\hline
\soar{\soar{\soar{ -e, --enable, --on }}} & Turn explain sysparam on.  \\
\hline
\soar{\soar{\soar{ -d, --disable, --off }}} & Turn explain sysparam off.  \\
\hline
\end{tabular}
\subsubsection*{Description}
 The \textbf{save-backtraces}
 variable is a toggle that controls whether or not backtracing information (from chunks and justifications) is saved. 
 When \textbf{save-backtraces}
 is set to \textbf{off}
, backtracing information is not saved and explanations of the chunks and justifications that are formed can not be retrieved. When \textbf{save-backtraces}
 is set to \textbf{on}
, backtracing information can be retrieved by using the explain-backtraces command. Saving backtracing information may slow down the execution of your Soar program, but it can be a very useful tool in understanding how chunks are formed. 
\subsubsection*{See Also}
\hyperref[explain-backtraces]{explain-backtraces} 
\input{wikicmd/tex/select}
\subsection{\soarb{set-stop-phase}}
\label{set-stop-phase}
\index{set-stop-phase}
Controls the phase where agents stop when running by decision. 
\subsubsection*{Synopsis}
set-stop-phase -[ABadiop] 
\end{verbatim}
\subsubsection*{Options}
 Options -A and -B are optional and mutually exclusive. If not specified, the default is -B. 
 Only one of -a, -d, -i, -o, -p must be selected. 
 With no options, reports the current stop phase. 
\hline
\soar{\soar{\soar{ -A, --after }}} & Stop after specified phase.  \\
\hline
\soar{\soar{\soar{ -B, --before }}} & Stop before specified phase (the default).  \\
\hline
\soar{\soar{\soar{ -a, --apply }}} & Select the apply phase.  \\
\hline
\soar{\soar{\soar{ -d, --decision }}} & Select the decision phase.  \\
\hline
\soar{\soar{\soar{ -i, --input }}} & Select the input phase.  \\
\hline
\soar{\soar{\soar{ -o, --output }}} & Select the output phase.  \\
\hline
\soar{\soar{\soar{ -p, --proposal }}} & Select the proposal phase.  \\
\hline
\end{tabular}
\subsubsection*{Description}
 When running by decision cycle it can be helpful to have agents stop at a particular point in its execution cycle. This command allows the user to control which phase Soar stops in. The precise definition is that \emph{running for $<$n$>$ decisions and stopping before phase $<$ph$>$}
 means to run until the decision cycle counter has increased by $<$n$>$ and then stop when the next phase is $<$ph$>$. The phase sequence (as of this writing) is: input, proposal, decision, apply, output. Stopping after one phase is exactly equivalent to stopping before the next phase. 
 On initialization Soar defaults to stopping before the input phase (or after the output phase, however you like to think of it). 
 Setting the stop phase applies to all agents. 
\subsubsection*{Examples}
set-stop-phase -Bi                 // stop before input phase
set-stop-phase -Ad                 // stop after decision phase (before apply phase)
set-stop-phase -d                  // stop before decision phase
set-stop-phase --after --output    // stop after output phase
set-stop-phase                     // reports the current stop phase
\end{verbatim}
\subsubsection*{See Also}

\chapter{Semantic Memory}
\label{SMEM}
\index{semantic memory}
\index{smem}

Soar's semantic memory is a repository for long-term declarative knowledge, supplementing what is contained in short-term working memory (and production memory). 
Episodic memory, which contains memories of the agent's experiences, is described in Chapter \ref{EPMEM}. 
The knowledge encoded in episodic memory is organized temporally, and specific information is embedded within the context of when it was experienced, whereas knowledge in semantic memory is independent of any specific context, representing more general facts about the world.

This chapter is organized as follows: semantic memory structures in working memory (\ref{SMEM-wm}); representation of knowledge in semantic memory (\ref{SMEM-kr}); storing semantic knowledge (\ref{SMEM-store}); retrieving semantic knowledge (\ref{SMEM-retrieve}); and a discussion of performance (\ref{SMEM-perf}). 
The detailed behavior of semantic memory is determined by numerous parameters that can be controlled and configured via the \soarb{smem} command. 
Please refer to the documentation for that command in Section \ref{smem} on page \pageref{smem}.


\section{Working Memory Structure}
\label{SMEM-wm}

Upon creation of a new state in working memory (see Section \ref{ARCH-impasses-types} on page \pageref{ARCH-impasses-types}; Section \ref{SYNTAX-impasses} on page \pageref{SYNTAX-impasses}), the architecture creates the following augmentations to facilitate agent interaction with semantic memory:

\begin{verbatim}
(<s> ^smem <smem>)
  (<smem> ^command <smem-c>)
  (<smem> ^result <smem-r>)
\end{verbatim}

As rules augment the \emph{command} structure in order to access/change semantic knowledge (\ref{SMEM-store}, \ref{SMEM-retrieve}), semantic memory augments the \emph{result} structure in response.
Production actions should not remove augmentations of the \emph{result} structure directly, as semantic memory will maintain these WMEs.



\section{Knowledge Representation}
\label{SMEM-kr}

The representation of knowledge in semantic memory is similar to that in working memory (see Section \ref{ARCH-wm} on page \pageref{ARCH-wm}) -- both include graph structures that are composed of symbolic elements consisting of an identifier, an attribute, and a value. 
It is important to note, however, key differences:

\begin{itemize}

\item 
Currently semantic memory only supports attributes that are symbolic constants (string, integer, or decimal), but \emph{not} attributes that are identifiers

\item 
Whereas working memory is a single, connected, directed graph, semantic memory can be disconnected, consisting of multiple directed, connected sub-graphs

\end{itemize}

\emph{Long-term} identifiers (LTIs) are defined as identifiers that exist in semantic memory.
The specific letter-number combination that labels an LTI (e.g. S5 or C7) is permanently associated with that long-term identifier: any retrievals of the long-term identifier are guaranteed to return the associated letter-number pair.  
For clarity, when printed, a long-term identifier is prefaced with the {@} symbol (e.g. {@}S5 or {@}C7). 
Also, when presented in a figure, long-term identifiers will be indicated by a double-circle. 
For instance, Figure \ref{fig:smem-concept} depicts the long-term identifier {@}A68, with four augmentations, representing the addition fact of ${6+7=13}$ (or, rather, 3, carry 1, in context of multi-column arithmetic).

\begin{figure}
\insertfigure{Figures/smem-concept}{1.5in}
\insertcaption{Example long-term identifier with four augmentations.}
\label{fig:smem-concept}
\end{figure}

\subsection{Integrating Long-Term Identifiers with Soar}
Integrating long-term identifiers in Soar presents a number of theoretical and implementation challenges.  
This section discusses the state of integration with each of Soar's memories/learning mechanisms.

\subsubsection{Working Memory}
Long-term identifiers exist as peers with short-term identifiers in Working Memory.

\subsubsection{Procedural Memory}
Soar's production parser (i.e. the \soarb{sp} command) has been modified to allow specification of long-term identifiers (prefaced with an {@} symbol) in any context where a variable is valid.
If a rule contains a long-term identifier that is not currently in semantic memory, a fatal error will be raised and Soar will quit.  
Once added to the rete, the long-term identifier is treated as a constant for matching purposes.  
If specified as the value of a WME in an action, a long-term identifier will be added to working memory if it does not already exist.  
There is also preliminary support for chunking over long-term identifiers.

It is currently possible to create production actions wherein the identifier of a new WME is a long-term identifier that exists neither in the production conditions, nor as the attribute or value of a prior action.  
Such rules will wreak havoc within Soar and are not supported.  
They will be detected and disallowed in future versions of semantic memory.

\subsubsection{Episodic Memory}
Episodic memory (see Section \ref{EPMEM} on page \pageref{EPMEM}) faithfully captures short- vs. long-term identifiers, including the episode of transition.  
Cues are handled in much the same way as cue-based retrievals, with respect to the differences in semantics of a short- vs. long-term identifier.

\section{Storing Semantic Knowledge}
\label{SMEM-store}

An agent stores a long-term identifier to semantic memory by creating a \emph{store} command: this is a WME whose identifier is the \emph{command} link of a state's \emph{smem} structure, the attribute is \emph{store}, and the value is an identifier (short or long).

\begin{verbatim}
<s> ^smem.command.store <identifier>
\end{verbatim}

Semantic memory will encode and store all WMEs whose identifier is the value of the store command.  
Storing deeper levels of working memory is achieved through multiple store commands.

Multiple store commands can be issued in parallel.  
Storage commands are processed on every state at the end of every phase of every decision cycle.  
Storage is guaranteed to succeed and a status WME will be created, where the identifier is the \emph{result} link of the \emph{smem} structure of that state, the attribute is \emph{success}, and the value is the value of the store command above.

\begin{verbatim}
<s> ^smem.result.success <identifier>
\end{verbatim}

Any short-term identifiers that compose the stored WMEs will be converted to long-term identifiers. 
If a long-term identifier is the value of a store command, the stored WMEs replace those associated with the LTI in semantic memory. 
It should be noted that between issuing store commands, it is possible that the augmentations of a long-term identifier in working memory are inconsistent with those in semantic memory.

\subsection{User-Initiated Storage}
Semantic memory provides agent designers the ability to store semantic knowledge via the \soarb{add} switch of the \soarb{smem} command (see Section \ref{smem} on page \pageref{smem}).  
The format of the command is nearly identical to the working memory manipulation components of the RHS of a production (i.e. no RHS-functions; see Section \ref{SYNTAX-pm-action} on page \pageref{SYNTAX-pm-action}).  
For instance:

\begin{verbatim}
smem --add {
   (<arithmetic> ^add10-facts <a01> <a02> <a03>)
   (<a01> ^digit1 1 ^digit-10 11)
   (<a02> ^digit1 2 ^digit-10 12)
   (<a03> ^digit1 3 ^digit-10 13)
}
\end{verbatim}

Unlike agent storage, declarative storage is automatically recursive.  
Thus, this command instance will add a new long-term identifier (represented by the temporary 'arithmetic' variable) with three augmentations.  
The value of each augmentation will each become an LTI with two constant attribute/value pairs.  
Manual storage can be arbitrarily complex and use standard dot-notation.

\subsection{Storage Location}
Semantic memory uses SQLite to facilitate efficient and standardized storage and querying of knowledge.  
The semantic store can be maintained in memory or on disk (per the \soarb{database} and \soarb{path} parameters). 
If the store is located on disk, users can use any standard SQLite programs/components to access/query its contents.
However, using a disk-based semantic store is very costly (performance is discussed in greater detail in Section \ref{SMEM-perf} on page \pageref{SMEM-perf}), and running in memory is recommended for most runs.

The \soarb{lazy-commit} parameter is a performance optimization. 
If set to \soarb{on} (default), disk databases will not reflect semantic memory changes until the Soar kernel shuts down. 
This improves performance by avoiding disk writes. 
The \soarb{optimization} parameter (see Section \ref{SMEM-perf} on page \pageref{SMEM-perf}) will have an affect on whether databases on disk can be opened while the Soar kernel is running.


\section{Retrieving Semantic Knowledge}
\label{SMEM-retrieve}

An agent retrieves knowledge from semantic memory by creating an appropriate command (we detail the types of commands below) on the \emph{command} link of a state's \emph{smem} structure. 
At the end of the output of each decision, semantic memory processes each state's \emph{smem} command structure.  
Results, meta-data, and errors are added to the \emph{result} structure of that state's \emph{smem} structure.

Only one type of retrieval command (which may include optional modifiers) can be issued per state in a single decision cycle.  
Malformed commands (including attempts at multiple retrieval types) will result in an error:

\begin{verbatim}
<s> ^smem.result.bad-cmd <smem-c>
\end{verbatim}

Where the \soarb{smem-c} variable refers to the \emph{command} structure of the state.

After a command has been processed, semantic memory will ignore it until some aspect of the command structure changes (via addition/removal of WMEs).  
When this occurs, the result structure is cleared and the new command (if one exists) is processed.

\subsection{Non-Cue-Based Retrievals}
A non-cue-based retrieval is a request by the agent to reflect in working memory the current augmentations of a long-term identifier in semantic memory. 
The command WME has a \emph{retrieve} attribute and a long-term identifier value:

\begin{verbatim}
<s> ^smem.command.retrieve <lti>
\end{verbatim}

If the value of the command is not a long-term identifier, an error will result: 

\begin{verbatim}
<s> ^smem.result.failure <lti>
\end{verbatim}

Otherwise, two new WMEs will be placed on the result structure:

\begin{verbatim}
<s> ^smem.result.success <lti>
<s> ^smem.result.retrieved <lti>
\end{verbatim}

All augmentations of the long-term identifier in semantic memory will be created as new WMEs in working memory.

\subsection{Cue-Based Retrievals}
A cue-based retrieval performs a search for a long-term identifier in semantic memory whose augmentations exactly match an agent-supplied cue, as well as optional cue modifiers.

A cue is composed of WMEs that describe the augmentations of a long-term identifier.  
A cue WME with a constant value denotes an exact match of both attribute and value.  
A cue WME with a long-term identifier as its value denotes an exact match as well.  
A cue WME with a short-term identifier as its value denotes an exact match of attribute, but with any value (constant or identifier).  

A cue-based retrieval command has a \emph{query} attribute and an identifier value, the cue:

\begin{verbatim}
<s> ^smem.command.query <cue>
\end{verbatim}

For instance, consider the following rule that creates a cue-based retrieval command:

\begin{verbatim}
sp {smem*sample*query
    (state <s> ^smem.command <sc>
               ^lti <lti>
               ^input-link.foo <bar>)
-->
    (<sc> ^query <q>)
    (<q> ^name <any-name>
         ^foo <bar>
         ^associate <lti>
         ^age 25)
}
\end{verbatim}

In this example, assume that the \soar{<lti>} variable will match a long-term identifier and the \soar{<bar>} variable will match a constant.  
Thus, the query requests retrieval of a long-term identifier from semantic memory with augmentations that satisfy ALL of the following requirements:

\begin{itemize}

\item 
Attribute \soar{name} and ANY value

\item 
Attribute \soar{foo} and value equal to the value of variable \soar{<bar>} at the time this rule fires

\item 
Attribute \soar{associate} and value equal to the long-term identifier \soar{<lti>} at the time this rule fires

\item 
Attribute \soar{age} and integer value \soar{25}

\end{itemize}

If no long-term identifier satisfies ALL of these requirements, an error is returned:

\begin{verbatim}
<s> ^smem.result.failure <cue>
\end{verbatim}

Otherwise, two WMEs are added:

\begin{verbatim}
<s> ^smem.result.success <cue>
<s> ^smem.result.retrieved <retrieved-lti>
\end{verbatim}

During a cue-based retrieval it is possible that the retrieved long-term identifier is not in working memory.  
If this is the case, semantic memory will add the long-term identifier to working memory with letter-number pair as was originally stored.

As with non-cue-based retrievals all of the augmentations of the long-term identifier in semantic memory are added as new WMEs to working memory.

It is possible that multiple long-term identifiers match the cue equally well. 
In this case, semantic memory will retrieve the long-term identifier that was most recently stored/retrieved.

The cue-based retrieval process can be further tempered using optional modifiers:

\begin{itemize}

\item 
The \emph{prohibit} command requires that the retrieved long-term identifier is not equal to a supplied long-term identifier:
\begin{verbatim}
<s> ^smem.command.prohibit <bad-lti>
\end{verbatim}
Multiple prohibit command WMEs may be issued as modifiers to a single cue-based retrieval.  
This method can be used to iterate over all matching long-term identifiers.

\item 
The \emph{neg-query} command requires that the retrieved long-term identifier does NOT contain a set of attributes/attribute-value pairs:
\begin{verbatim}
<s> ^smem.command.neg-query <cue>
\end{verbatim}
The syntax of this command is identical to that of regular/positive \emph{query} command.

\item
The \emph{math-query} command requires that the retrieved long term identifier contains an attribute value pair that meets a specified mathematical condition. 
This condition can either be a conditional query or a superlative query. 
Conditional queries are of the format:
\begin{verbatim}
<s> ^smem.command.math-query.<cue-attribute>.<condition-name> <cue-value>
\end{verbatim}
Superlative queries do not use a value argument and are of the format:
\begin{verbatim}
<s> ^smem.command.math-query.<cue-attribute>.<condition-name>
\end{verbatim}
Values used in math queries must be integer or float type values.
Currently supported condition names are:
\begin{description}
  \item[less] A value less than the given argument
  \item[greater] A value greater than the given argument
  \item[less-or-equal] A value less than or equal to the given argument
  \item[greater-or-equal] A value greater than or equal to the given argument
  \item[max] The maximum value for the attribute
  \item[min] The minimum value for the attribute
\end{description}
\end{itemize}

\section{Performance}
\label{SMEM-perf}

Initial empirical results with toy agents show that semantic memory queries carry up to a 40\% overhead as compared to comparable rete matching.  
However, the retrieval mechanism implements some basic query optimization: statistics are maintained about all stored knowledge.  
When a query is issued, semantic memory re-orders the cue such as to minimize expected query time.  
Because only perfect matches are acceptable, and there is no symbol variablization, semantic memory retrievals do not contend with the same combinatorial search space as the rete.  
Preliminary empirical study shows that semantic memory maintains sub-millisecond retrieval time for a large class of queries, even in very large stores (millions of nodes/edges).

Once the number of long-term identifiers overcomes initial overhead (about 1000 WMEs), initial empirical study shows that semantic storage requires far less than 1KB per stored WME.

\subsection{Math queries}
There are some additional performance considerations when using math queries during retrieval.
Initial testing indicates that conditional queries show the same time growth with respect to the number of memories as similar non-math restricted queries, however the actual time for retrieval may be slightly longer.
Superelative queries will often show a worse result than similar non-superelative queries, because the current implementation of semantic memory requires them to iterate over any memory that matches all other involved cues.

\subsection{Performance Tweaking}

When using a database stored to disk, several parameters become crucial to performance.  
The first is \soarb{lazy-commit}, which controls when database changes are written to disk.   
The default setting (\soarb{on}) will keep all writes in memory and only commit to disk upon re-initialization (quitting the agent or issuing the \soarb{init} command).  
The \soarb{off} setting will write each change to disk and thus incurs massive I/O delay.

The next parameter is \soarb{thresh}. 
This has to do with the locality of storing/updating activation information with semantic augmentations. 
By default, all WME augmentations are incrementally sorted by activation, such that cue-based retrievals need not sort large number of candidate long-term identifiers on demand, and thus retrieval time is independent of cue selectivity. 
However, each activation update (such as after a retrieval) incurs an update cost linear in the number of augmentations. 
If the number of augmentations for a long-term identifier is large, this cost can dominate. 
Thus, the \soarb{thresh} parameter sets the upper bound of augmentations, after which activation is stored with the long-term identifier. 
This allows the user to establish a balance between cost of updating augmentation activation and the number of long-term identifiers that must be pre-sorted during a cue-based retrieval. 
As long as the threshold is greater than the number of augmentations of most long-term identifiers, performance should be fine (as it will bound the effects of selectivity).

The next two parameters deal with the SQLite cache, which is a memory store used to speed operations like queries by keeping in memory structures like levels of index B+-trees. 
The first parameter, \soarb{page-size}, indicates the size, in bytes, of each cache page. 
The second parameter, \soarb{cache-size}, suggests to SQLite how many pages are available for the cache. 
Total cache size is the product of these two parameter settings. 
The cache memory is not pre-allocated, so short/small runs will not necessarily make use of this space. 
Generally speaking, a greater number of cache pages will benefit query time, as SQLite can keep necessary meta-data in memory. 
However, some documented situations have shown improved performance from decreasing cache pages to increase memory locality. 
This is of greater concern when dealing with file-based databases, versus in-memory. 
The size of each page, however, may be important whether databases are disk- or memory-based. 
This setting can have far-reaching consequences, such as index B+-tree depth. 
While this setting can be dependent upon a particular situation, a good heuristic is that short, simple runs should use small values of the page size (\soarb{1k}, \soarb{2k}, \soarb{4k}), whereas longer, more complicated runs will benefit from larger values (\soarb{8k}, \soarb{16k}, \soarb{32k}, \soarb{64k}). 
The episodic memory chapter (see Section \ref{EPMEM-perf} on page \pageref{EPMEM-perf}) has some further empirical evidence to assist in setting these parameters for very large stores.

The next parameter is \soarb{optimization}.  
The \soarb{safety} parameter setting will use SQLite default settings.  
If data integrity is of importance, this setting is ideal.  
The \soarb{performance} setting will make use of lesser data consistency guarantees for significantly greater performance.  
First, writes are no longer synchronous with the OS (synchronous pragma), thus semantic memory won't wait for writes to complete before continuing execution.  
Second, transaction journaling is turned off (journal\_mode pragma), thus groups of modifications to the semantic store are not atomic (and thus interruptions due to application/os/hardware failure could lead to inconsistent database state).  
Finally, upon initialization, semantic memory maintains a continuous exclusive lock to the database (locking\_mode pragma), thus other applications/agents cannot make simultaneous read/write calls to the database (thereby reducing the need for potentially expensive system calls to secure/release file locks).

Finally, maintaining accurate operation timers can be relatively expensive in Soar.  
Thus, these should be enabled with caution and understanding of their limitations.  
First, they will affect performance, depending on the level (set via the \soarb{timers} parameter).  
A level of \soarb{three}, for instance, times every modification to long-term identifier recency statistics.  
Furthermore, because these iterations are relatively cheap (typically a single step in the linked-list of a b+-tree), timer values are typically unreliable (depending upon the system, resolution is 1 microsecond or more).


\subsection{\soarb{timers}}
\label{timers}
\index{timers}
Toggle on or off the internal timers used to profile Soar. 
\subsubsection*{Synopsis}
timers [-ed]
\end{verbatim}
\subsubsection*{Options}
\hline
\soar{\soar{\soar{ -d, --disable, --off }}} & Disable all timers.  \\
\hline
\soar{\soar{\soar{ -e, --enable, --on }}} & Enable timers as compiled.  \\
\hline
\end{tabular}
\subsubsection*{Description}
 This command is used to control the timers that collect internal profiling information while Soar is running. With no arguments, this command prints out the current timer status. Timers are ENABLED by default. The default compilation flags for soar enable the basic timers and disable the detailed timers. The \textbf{timers}
 command can only enable or disable timers that have already been enabled with compiler directives. See the stats command for more info on the Soar timing system. 
\subsubsection*{See Also}
\hyperref[stats]{stats} 
\subsection{\soarb{waitsnc}}
\label{waitsnc}
\index{waitsnc}
\subsubsection*{Synopsis}
wait -[e|d]
\end{verbatim}
\subsubsection*{Options}
\hline
\soar{\soar{\soar{ -e, --enable, --on }}} & Turns a state-no-change into a \emph{wait}
 state.  \\
\hline
\soar{\soar{\soar{ -d, --disable, --off }}} & Default. A state-no-change generates an impasse.  \\
\hline
\end{tabular}
\subsubsection*{Description}
 In some systems, espcially those that model expert (fully chunked) knowledge, a state-no-change may represent a \emph{wait state}
 rather than an impasse. The waitsnc command allows the user to switch to a mode where a state-no-change that would normally generate an impasse (and subgoaling), instead generates a \emph{wait}
 state. At a \emph{wait}
 state, the decision cycle will repeat (and the decision cycle count is incremented) but no state-no-change impasse (and therefore no substate) will be generated. 
 When issued with no arguments, waitsnc returns its current setting. 

\input{wikicmd/tex/wma}

% ----------------------------------------------------------------------------

\section{File System I/O Commands}
\label{FILE-IO}

This section describes commands which interact in one way or another
with operating system input and output, or file I/O.  Users can
save/retrieve information to/from files, redirect the information
printed by Soar as it runs, and save and load the binary representation
of productions.
The specific commands described in this section are:

\paragraph{Summary}
\begin{quote}
\begin{description}
%\item[command-to-file] - Evaluate a command and print its results to a file.
%\item[\emph{directory functions}] - \soar{cd, dirs, popd, pushd, pwd}
\item[cd] - Change directory.
\item[clog] - Record all user-interface input and output to a file. \emph{(was \soar{log})}
\item[command-to-file] - Dump the printed output and results of a command to a file. 
\item[dirs] - List the directory stack.
\item[echo] -  Print a string to the current output device.
\item[ls] - List the contents of the current working directory.
\item[popd] - Pop the current working directory off the stack and change to the next directory on the stack.
\item[pushd] - Push a directory onto the directory stack, changing to it.
\item[pwd] - Print the current working directory.
\item[rete-net] - Save the current Rete net, or restore a previous one.
\item[set-library-location] - Set the top level directory containing demos/help/etc.
%\item[output-strings-destination] - Redirect the Soar output stream.
\item[source] - Load and evaluate the contents of a file.
\end{description}
\end{quote}

The \textbf{source} command is used for nearly every Soar program.  The
directory functions are important to understand so that users can
navigate directories/folders to load/save the files of interest.  
Soar applications that include a graphical interface or other
simulation environment will often require the use of \textbf{echo}  .


\input{wikicmd/tex/cd}
\input{wikicmd/tex/clog}
\input{wikicmd/tex/command-to-file}
\input{wikicmd/tex/dirs}
\input{wikicmd/tex/echo}
\subsection{\soarb{ls}}
\label{ls}
\index{ls}
List the contents of the current working directory. 
\subsubsection*{Synopsis}
ls
\end{verbatim}
\subsubsection*{Options}
 No options. 
\subsubsection*{Description}
 List the contents of the working directory. 
\subsubsection*{Default Aliases}
\hline
\soar{\soar{\soar{ Alias }}} & Maps to  \\
\hline
\soar{\soar{\soar{ dir }}} & ls  \\
\hline
\end{tabular}
\subsubsection*{See Also}
\hyperref[cd]{cd} \hyperref[dirs]{dirs} \hyperref[home]{home} \hyperref[pushd]{pushd} \hyperref[popd]{popd} \hyperref[source]{source} \hyperref[topd]{topd} 
\subsection{\soarb{popd}}
\label{popd}
\index{popd}
Pop the current working directory off the stack and change to the next directory on the stack. Can be relative pathname or fully specified path. 
\subsubsection*{Synopsis}
popd
\end{verbatim}
\subsubsection*{Options}
 No options. 
\subsubsection*{Description}
 This command pops a directory off of the directory stack and cd's to it. See the dirs command for an explanation of the directory stack. 
\subsubsection*{See Also}
\hyperref[cd]{cd} \hyperref[dirs]{dirs} \hyperref[home]{home} \hyperref[ls]{ls} \hyperref[pushd]{pushd} \hyperref[source]{source} \hyperref[topd]{topd} 
\subsection{\soarb{pushd}}
\label{pushd}
\index{pushd}
Push a directory onto the directory stack, changing to it. 
\subsubsection*{Synopsis}
pushd directory
\end{verbatim}
\subsubsection*{Options}
\hline
\soar{\soar{\soar{ directory }}} & Directory to change to, saving the current directory on to the stack.  \\
\hline
\end{tabular}
\subsubsection*{Description}
 Maintain a stack of working directories and push the directory on to the stack. Can be relative path name or fully specified. 
\subsubsection*{See Also}
\hyperref[cd]{cd} \hyperref[dirs]{dirs} \hyperref[home]{home} \hyperref[ls]{ls} \hyperref[popd]{popd} \hyperref[source]{source} \hyperref[topd]{topd} 
\subsection{\soarb{pwd}}
\label{pwd}
\index{pwd}
Print the current working directory. 
\subsubsection*{Synopsis}
pwd
\end{verbatim}
\subsubsection*{Options}
 No options. 
\subsubsection*{Description}
 Prints the current working directory of Soar. 
\subsubsection*{Default Aliases}
\hline
\soar{\soar{\soar{ Alias }}} & Maps to  \\
\hline
\soar{\soar{\soar{ topd }}} & pwd  \\
\hline
\end{tabular}

\subsection{\soarb{rete-net}}
\label{rete-net}
\index{rete-net}
Save the current Rete net, or restore a previous one. 
\subsubsection*{Synopsis}
rete-net -s|l filename
\end{verbatim}
\subsubsection*{Options}
\hline
\soar{\soar{\soar{ -s, --save }}} & Save the Rete net in the named file. Cannot be saved when there are justifications present. Use excise -j \\
\hline
\soar{\soar{\soar{ -l, -r, --load, --restore }}} & Load the named file into the Rete network. working memory and production memory must both be empty. Use excise -a \\
\hline
\soar{\soar{\soar{filename}}} & The name of the file to save or load.  \\
\hline
\end{tabular}
\subsubsection*{Description}
 The rete-net command saves the current Rete net to a file or restores a Rete net previously saved. The Rete net is Soar's internal representation of production and working memory; the conditions of productions are reordered and common substructures are shared across different productions. This command provides a fast method of saving and loading productions since a special format is used and no parsing is necessary. Rete-net files are portable across platforms that support Soar. 
 Normally users wish to save only production memory. Note that \emph{justifications}
 cannot be present when saving the Rete net. Issuing an init-soar before saving a Rete net will remove all justifications and working memory elements. \\ 
 If the filename contains a suffix of ``.Z'', then the file is compressed automatically when it is saved and uncompressed when it is loaded. Compressed files may not be portable to another platform if that platform does not support the same uncompress utility. 
\subsubsection*{Default Aliases}
\hline
\soar{\soar{\soar{ Alias }}} & Maps to  \\
\hline
\soar{\soar{\soar{ rn }}} & rete-net  \\
\hline
\end{tabular}
\subsubsection*{See Also}
\hyperref[excise]{excise} \hyperref[init-soar]{init-soar} 
\subsection{\soarb{set-library-location}}
\label{set-library-location}
\index{set-library-location}
Set the top level directory containing demos/help/etc. 
\subsubsection*{Synopsis}
set-library-location [directory] 
\end{verbatim}
\subsubsection*{Options}
\hline
\soar{\soar{\soar{ directory }}} & The new desired library location.  \\
\hline
\end{tabular}
\subsubsection*{Description}
 Invoke with no arguments to query what the current library location is. The library location should contain at least the help/ subdirectory and the command-names file for help to work. 
\subsubsection*{See Also}
\hyperref[help]{help} 
\subsection{\soarb{source}}
\label{source}
\index{source}
Load and evaluate the contents of a file. 
\subsubsection*{Synopsis}
source -[adv] filename
\end{verbatim}
\subsubsection*{Options}
\hline
\soar{\soar{\soar{filename}}} & The file of Soar productions and commands to load.  \\
\hline
\soar{\soar{\soar{ -a, --all }}} & Enable a summary for each file sourced  \\
\hline
\soar{\soar{\soar{ -d, --disable }}} & Disable all summaries  \\
\hline
\soar{\soar{\soar{ -v, --verbose }}} & Print excised production names  \\
\hline
\end{tabular}
\subsubsection*{Description}
 Load and evaluate the contents of a file. The \emph{filename}
 can be a relative path or a fully qualified path. \textbf{source}
 will generate an implicit push to the new directory, execute the command, and then pop back to the current working directory from which the command was issued. 
agent> source demos/mac/mac.soar
******************
Total: 18 productions sourced.
Source finished.
agent> source demos/mac/mac.soar
#*#*#*#*#*#*#*#*#*#*#*#*#*#*#*#*#*#*
Total: 18 productions sourced. 18 productions excised.
Source finished.
\end{verbatim}
agent> source demos/mac/mac.soar -d
******************
Source finished.
agent> source demos/mac/mac.soar -d
#*#*#*#*#*#*#*#*#*#*#*#*#*#*#*#*#*#*
Source finished.
\end{verbatim}
agent> source demos/mac/mac.soar -v
#*#*#*#*#*#*#*#*#*#*#*#*#*#*#*#*#*#*
Total: 18 productions sourced. 18 productions excised.
Excised productions:
        mac*detect*state*success
        mac*evaluate*state*failure*more*cannibals
        monitor*move-boat
        monitor*state*left
...
\end{verbatim}
agent> source demos/mac/mac.soar -a
_firstload.soar: 0 productions sourced.
all_source.soar: 0 productions sourced.
**
goal-test.soar: 2 productions sourced.
***
monitor.soar: 3 productions sourced.
****
search-control.soar: 4 productions sourced.
top-state.soar: 0 productions sourced.
elaborations_source.soar: 0 productions sourced.
_readme.soar: 0 productions sourced.
**
initialize-mac.soar: 2 productions sourced.
*******
move-boat.soar: 7 productions sourced.
mac_source.soar: 0 productions sourced.
mac.soar: 0 productions sourced.
Total: 18 productions sourced.
Source finished.
\end{verbatim}
 Combining the -a and -v flag adds excised production names to the output for each file. 
\subsubsection*{See Also}
\hyperref[cd]{cd} \hyperref[dirs]{dirs} \hyperref[home]{home} \hyperref[ls]{ls} \hyperref[pushd]{pushd} \hyperref[popd]{popd} \hyperref[topd]{topd} 

% ***************************************************************************
% ----------------------------------------------------------------------------
\section{Soar I/O Commands}
\label{SOAR-IO}

This section describes the commands used to manage Soar's Input/Output
(I/O) system, which provides a mechanism for allowing Soar to interact 
with external systems, such as a computer game environment or a robot.  

Soar I/O functions make calls to \soar{add-wme} and \soar{remove-wme}
to add and remove elements to the \textbf{io} structure of Soar's working
memory. 
 
The specific commands described in this section are:

\paragraph{Summary}
\begin{quote}
\begin{description}
\item[add-wme] - Manually add an element to working memory.
\item[capture-input] - Store the input wmes in a file for reloading later.
\item[remove-wme] - Manually remove an element from working memory.
\item[replay-input] - Load input wmes for each decision cycle from a file.
\end{description}
\end{quote}

These commands are used mainly  when Soar needs to interact with an
external environment.  Users might take advantage of these commands when
debugging agents, but care should be used in adding and removing wmes this
way as they do not fall under Soar's truth maintenance system.

\input{wikicmd/tex/add-wme}
\input{wikicmd/tex/capture-input}
\subsection{\soarb{remove-wme}}
\label{remove-wme}
\index{remove-wme}
Manually remove an element from working memory. 
\subsubsection*{Synopsis}
remove-wme \emph{timetag}
\end{verbatim}
\subsubsection*{Options}
\hline
\soar{\soar{\soar{ timetag }}} & A positive integer matching the timetag of an existing working memory element.  \\
\hline
\end{tabular}
\subsubsection*{Description}
 The remove-wme command removes the working memory element with the given timetag. This command is provided primarily for use in Soar input functions; although there is no programming enforcement, remove-wme should only be called from registered input functions to delete working memory elements on Soar's input link. 
 Beware of weird side effects, including system crashes. 
\subsubsection*{Default Aliases}
\hline
\soar{\soar{\soar{ Alias }}} & Maps to  \\
\hline
\soar{\soar{\soar{ rw }}} & remove-wme  \\
\hline
\end{tabular}
\subsubsection*{See Also}
\hyperref[add-wme]{add-wme} \subsubsection*{Warnings}
 remove-wme should never be called from the RHS: if you try to match a wme on the LHS of a production, and then remove the matched wme on the RHS, Soar will crash. 
 If used other than by input and output functions interfaced with Soar, this command may have weird side effects (possibly even including system crashes). Removing input wmes or context/impasse wmes may have unexpected side effects. You've been warned. 

\input{wikicmd/tex/replay-input}

% ***************************************************************************
% ----------------------------------------------------------------------------
\section{Miscellaneous}
\label{MISC}


\comment{this section still needs to be rewritten...}

\nocomment{This section describes the commands used to inspect production memory,
working memory, and preference memory; to see what productions will 
match and fire in the next Propose or Apply phase;  and to examine the 
goal dependency set.  These commands are particularly useful when
running or debugging Soar, as they let users see what Soar is ``thinking.''}
The specific commands described in this section are:


\paragraph{Summary}
\begin{quote}
\begin{description}
\item[alias] - Define a new alias, or command, using existing commands and arguments.
\item[allocate] - Allocate additional 32 kilobyte blocks of memory for a specified memory pool without running Soar.
\item[echo-commands] - Set whether or not commands are echoed to other connected debuggers. 
\item[edit-production] - Fire event to Move focus in an open editor to this production.
\item[load-library] - Load a shared library into the local client
\item[port] - Returns the port the kernel instance is listening on.
\item[rand] - Generate a random number.
\item[soarnews] - Prints information about the current release.
\item[srand] -  Seed the random number generator.
\item[time] - Uses a default system clock timer to record the wall time required while executing a command.
\item[unalias] - Remove an existing alias.
\item[version] - Returns version number of Soar kernel.
\end{description}
\end{quote}

\input{wikicmd/tex/alias}
\input{wikicmd/tex/allocate}
\input{wikicmd/tex/echo-commands}
\subsection{\soarb{edit-production}}
\label{edit-production}
\index{edit-production}
Move focus in an editor to this production. 
\subsubsection*{Synopsis}
edit-production production_name
\end{verbatim}
\subsubsection*{Options}
 production\_name The name of the production to edit. 
\subsubsection*{Description}
 If an editor (currently limited to Visual Soar) is open and connected to Soar, this command causes the editor to open the file containing this production and move the cursor to the start of the production. If there is no editor connected to Soar, the command does nothing. In order to connect Visual Soar to Soar, launch Visual Soar and choose Connect from the Soar Runtime menu. Then open the Visual Soar project that you're working on. At that point, you're set up and edit-production will start to work. 
\subsubsection*{Examples}
edit-production my*production*name
\end{verbatim}
\subsubsection*{See Also}
\hyperref[sp]{sp} 
\input{wikicmd/tex/load-library}
\input{wikicmd/tex/port}
\input{wikicmd/tex/rand}
\subsection{\soarb{srand}}
\label{srand}
\index{srand}
Seed the random number generator. 
\subsubsection*{Synopsis}
srand [seed]
\end{verbatim}
\subsubsection*{Options}
\hline
\soar{\soar{\soar{ seed }}} & Random number generator seed.  \\
\hline
\end{tabular}
\subsubsection*{Description}
 Seeds the random number generator with the passed seed. Calling srand without providing a seed will seed the generator based on the contents of /dev/urandom (if available) or else based on time() and clock() values. 
\subsubsection*{Examples}
srand 0
\end{verbatim}
\subsubsection*{See Also}

\subsection{\soarb{soarnews}}
\label{soarnews}
\index{soarnews}
Prints information about the current release. 
\subsubsection*{Synopsis}
soarnews
\end{verbatim}
\subsubsection*{Default Aliases}
\hline
\soar{\soar{\soar{ Alias }}} & Maps to  \\
\hline
\soar{\soar{\soar{ sn }}} & soarnews  \\
\hline
\end{tabular}

\subsection{\soarb{time}}
\label{time}
\index{time}
Use a default system clock timer to record the wall time required while executing a command. 
\subsubsection*{Synopsis}
time command [arguments]
\end{verbatim}
\subsubsection*{Options}
\hline
\soar{\soar{\soar{ command }}} & The command to execute.  \\
\hline
\soar{\soar{\soar{ arguments }}} & Optional command arguments.  \\
\hline
\end{tabular}
\subsubsection*{Description}

\subsection{\soarb{unalias}}
\label{unalias}
\index{unalias}
Undefine an existing alias 
\subsubsection*{Synopsis}
unalias name
\end{verbatim}
\subsubsection*{Options}
 No options. 
\subsubsection*{Description}
 This command undefines a previously created alias. This command takes exactly one argument: the name of the alias to remove. Use the alias command by itself to list all defined aliases. 
\subsubsection*{Examples}
unalias varprint
\end{verbatim}
\subsubsection*{Default Aliases}
\hline
\soar{\soar{\soar{ Alias }}} & Maps to  \\
\hline
\soar{\soar{\soar{un}}} &\textbf{unalias}
\hline
\end{tabular}
\subsubsection*{See Also}
\hyperref[alias]{alias} 
\subsection{\soarb{version}}
\label{version}
\index{version}
\subsubsection*{Synopsis}
 version
\end{verbatim}
\subsubsection*{Options}
 No options 
\subsubsection*{Description}
 This command gives version information about the current Soar kernel. It returns the version number and build date which can then be stored by the agent or the application. 

