% ------------------------------------------------------------------
\typeout{--------------- The Soar User INTERFACE --------------------}
\chapter{The Soar User Interface}
\label{INTERFACE}
\index{interface}
%\index{user interface}
%\index{function definitions}

\nocomment{for each command, use the 'funsum' command with a brief
	description. This writes to the manual.glo file which can be edited
	into the function summary and index (see that file for more
	instructions). This is a bit tedious, but the reason I've set it up
	this way is that the command set is in flux right now -- this lessens
	the chance that a command will be inadvertently omitted from the
	function summary (or that a defunct command will be inadvertently
	included). 
	}

\nocomment{\begin{figure}[h]
\psfig{figure=dilbert-living.ps,height=2.2in} \vspace{12pt}
\end{figure}
}
% ----------------------------------------------------------------------------


This chapter describes the set of user interface commands for Soar. All commands and examples are presented as 
if they are being entered at the Soar command prompt.

This chapter is organized into 7 sections:
\begin{enumerate}
\item Basic Commands for Running Soar
\item Examining Memory
\item Configuring Trace Information and Debugging
\item Configuring Soar's Run-Time Parameters
\item File System I/O Commands
\item Soar I/O commands
\item Miscellaneous Commands
\end{enumerate}

Each section begins with a summary description of the commands covered
in that section, including the role of the command and its importance
to the user.  Command syntax and usage are then described fully, in
alphabetical order.

The following pages were automatically generated from the git repository
at

\hspace{2em}\soar{\htmladdnormallink{https://github.com/SoarGroup/Soar/wiki}{https://github.com/SoarGroup/Soar/wiki}}

on the date listed on the title page of this manual.  Please consult
the repository directly for the most accurate and up-to-date information.

For a concise overview of the Soar interface functions, see the Function
Summary and Index on page \pageref{func-sum}. This index is intended to be a
quick reference into the commands described in this chapter.

\subsubsection*{Notation}

\nocomment{check for all commands that I've got the notation current}

The notation used to denote the syntax for each user-interface command follows
some general conventions:\vspace{-12pt}
\begin{itemize}
\item The command name itself is given in a \soarb{bold} font.\vspace{-8pt}
\item Optional command arguments are enclosed within square brackets,
	\soar{[} and \soar{]}.\vspace{-8pt}
\item A vertical bar, \soar{|}, separates alternatives.\vspace{-8pt}
\item Curly braces, \soar{\{\}}, are used to group arguments when at least
one argument from the set is required.
\item The commandline prompt that is printed by Soar, is normally
the agent name, followed by '\soar{>}'.  In the examples in this manual, 
we use ``\soar{soar>}''.
\item Comments in the examples are preceded by
a '\soar{\#}', and in-line comments are preceded by '\soar{;\#}'.
\end{itemize}

For many commands, there is some flexibility in the order in which the
arguments may be given. (See the online help for each command for more
information.)  We have not incorporated this flexible ordering into the syntax
specified for each command because doing so complicates the specification of
the command.  When the order of arguments will affect the output
produced by a command, the reader will be alerted.

Note that the command list was revamped and simplified in Soar 9.6.0.  While 
we encourage people to learn the new syntax, aliases and some special mechanism 
have been added to maintain backwards compatibility with old Soar commands.  As a 
result, many of the sub-commands of the newer commands may use different styles of 
arguments.

% ----------------------------------------------------------------------------
\section{Basic Commands for Running Soar}
\label{BASIC}

This section describes the commands used to start, run and stop a Soar 
program; to invoke on-line help information; and to create and 
delete Soar productions.  The specific commands described in this
section are:

\paragraph{Summary}
\begin{quote}
\begin{description}
\item[soar] - Commands and settings related to running Soar.  Use \textbf{soar ?} for a summary of sub-commands listed below.
\item[soar init] - Reinitialize Soar so a program can be rerun from scratch.
\item[soar stop] - Interrupt a running Soar program.
\item[soar max-chunks] - Limit the number of chunks created during a decision cycle.
\item[soar max-dc-time] - Set a wall-clock time limit such that the agent will be interrupted when a single decision cycle exceeds this limit.
\item[soar max-elaborations] - Limit the maximum number of elaboration cycles in a given phase.
\item[soar max-goal-depth] - Limit the sub-state stack depth.
\item[soar max-memory-usage] - Set the number of bytes that when exceeded by an agent, will trigger the memory usage exceeded event. 
\item[soar max-nil-output-cycles] - Limit the maximum number of decision cycles executed without producing output. 
\item[soar max-gp] - Set the upper-limit to the number of productions generated by the gp command.
\item[soar stop-phase] -  Controls the phase where agents stop when running by decision.
\item[soar tcl] -  Controls whether Soar Tcl mode is enabled.
\item[soar timers] - Toggle on or off the internal timers used to profile Soar.
\item[soar version] - Returns version number of Soar kernel.
\item[soar waitsnc] - Generate a wait state rather than a state-no-change impasse.
\item[gp] - Define a pattern used to generate and source a set of Soar productions.
\item[run] - Begin Soar's execution cycle.
\item[sp] - Create a production and add it to production memory.
\item[help] - Provide formatted, on-line information about Soar commands.
\end{description}
\end{quote}
These commands are all frequently used anytime Soar is run.

\divider 
\input{wikicmd/tex/soar} 

\divider
\input{wikicmd/tex/gp} 

\divider
\subsection{\soarb{help}}
\label{help}
\index{help}
Provide formatted usage information about Soar commands. 
\subsubsection*{Synopsis}
help [command_name]
\end{verbatim}
\subsubsection*{Options}
\hline
\soar{\soar{\soar{\soar{ command\_name }}}} & Print usage syntax for the command.  \\
\hline
\end{tabular}
\subsubsection*{Description}
 This command prints formatted help for the given command name. 
\subsubsection*{Examples}
 To see the syntax for the \emph{excise}
help excise
\end{verbatim}
help
\end{verbatim}
\subsubsection*{Default Aliases}
\hline
\soar{\soar{\soar{\soar{ Alias }}}} & Maps to  \\
\hline
\soar{\soar{\soar{\soar{�? }}}} & help  \\
\hline
\soar{\soar{\soar{\soar{ h }}}} & help  \\
\hline
\soar{\soar{\soar{\soar{ man }}}} & help  \\
\hline
\end{tabular}


\divider 
\subsection{\soarb{run}}
\label{run}
\index{run}
Begin Soar\~A�\^a�$\neg$\^a��s execution cycle. 
\subsubsection*{Synopsis}
run  [f|\emph{count}
]
run -[d|e|o|p][s][un] [f|\emph{count}
]
run -[d|e|o|p][un] \emph{count}
 [-i <e|p|d|o>]
\end{verbatim}
\subsubsection*{Options}
\hline
\soar{\soar{\soar{ -d, --decision }}} & Run Soar for count decision cycles.  \\
\hline
\soar{\soar{\soar{ -e, --elaboration }}} & Run Soar for count elaboration cycles.  \\
\hline
\soar{\soar{\soar{ -o, --output }}} & Run Soar until the nth time output is generated by the agent. Limited by the value of max-nil-output-cycles.  \\
\hline
\soar{\soar{\soar{ -p, --phase }}} & Run Soar by phases. A phase is either an input phase, proposal phase, decision phase, apply phase, or output phase.  \\
\hline
\soar{\soar{\soar{ -s, --self }}} & If other agents exist within the kernel, do not run them at this time.  \\
\hline
\soar{\soar{\soar{ -u, --update }}} & Sets a flag in the update event callback requesting that an environment updates. This is the default if --self is not specified.  \\
\hline
\soar{\soar{\soar{ -n, --noupdate }}} & Sets a flag in the update event callback requesting that an environment does not update. This is the default if --self is specified.  \\
\hline
\soar{\soar{\soar{ f, forever }}} & Run until halted by problem-solving completion or until stopped by an interrupt.  \\
\hline
\soar{\soar{\soar{ count }}} & A single integer which specifies the number of cycles to run Soar.  \\
\hline
\soar{\soar{\soar{ -i, --interleave }}} & Support round robin execution across agents at a finer grain than the run-size parameter. e = elaboration, p = phase, d = decision, o = output  \\
\hline
\end{tabular}
\paragraph*{Deprecated Options}
 These may be reimplemented in the future. 
\hline
\soar{\soar{\soar{ --operator }}} & Run Soar until the nth time an operator is selected.  \\
\hline
\soar{\soar{\soar{ --state }}} & Run Soar until the nth time a state is selected.  \\
\hline
\end{tabular}
\subsubsection*{Description}
 The \textbf{run}
 command starts the Soar execution cycle or continues any execution that was temporarily stopped. The default behavior of \textbf{run}
, with no arguments, is to cause Soar to execute until it is halted or interrupted by an action of a production, or until an external interrupt is issued by the user. The \textbf{run}
 command can also specify that Soar should run only for a specific number of Soar cycles or phases (which may also be prematurely stopped by a production action or the stop-soar command). This is helpful for debugging sessions, where users may want to pay careful attention to the specific productions that are firing and retracting. 
 The \textbf{run}
 command takes optional arguments: an integer, \emph{count}
, which specifies how many units to run; and a \emph{units}
 flag indicating what steps or increments to use. If \emph{count}
 is specified, but no \emph{units}
 are specified, then Soar is run by decision cycles. If \emph{units}
 are specified, but \emph{count}
 is unpecified, then \emph{count}
 defaults to '1'. The argument \textbf{forever}
 (can be shortened to \textbf{f}
) is a keyword used instead of an integer \emph{count}
 and indicates Soar should be run indefinitely, until halted by problem-solving completion, or stopped by an interrupt. 
 If there are multiple Soar agents that exist in the same Soar process, then issuing a \textbf{run}
 command in any agent will cause all agents to run with the same set of parameters, unless the flag \textbf{--self}
 is specified, in which case only that agent will execute. 
 If an environment is registered for the kernel's update event, then when the event it triggered, the environment will get information about how the \textbf{run}
 was executed. If a \textbf{run}
 was executed with the --update option, then then event sends a flag requesting that the environment actually update itself. If a \textbf{run}
 was executed with the --noupdate option, then the event sends a flag requesting that the environment not update itself. The --update option is the default when run is specified without the --self option is not specified. If the --self option is specified, then the --noupdate option is on by default. It is up to the environment to check for these flags and honor them. 
 Some use cases include: 
\hline
\soar{\soar{\soar{ run --self }}} & runs one agent but not the environment  \\
\hline
\soar{\soar{\soar{ run --self --update }}} & runs one agent and the environment  \\
\hline
\soar{\soar{\soar{ run }}} & runs all agents and the environment  \\
\hline
\soar{\soar{\soar{ run --noupdate }}} & runs all agents but not the environment  \\
\hline
\end{tabular}
\paragraph*{Setting an interleave size}
 When there are multiple agents running within the same process, it may be useful to keep agents more closely aligned in their execution cycle than the run increment (--elaboration, --phases, --decisions, --output) specifies. For instance, it may be necessary to keep agents in ``lock step'' at the phase level, eventhough the \textbf{run}
 command issued is for 5 decisions. Some use cases include: 
\hline
\soar{\soar{\soar{ run -d 5 -inteleave p }}} & run the agent one phase and then move to the next agent, \\ 
 looping over agents until they have run for 5 decision cycles  \\
\hline
\soar{\soar{\soar{ run -o 3 -interleave d }}} & run the agent one decision cycle and then move to the next agent. When an agent \\ 
generates output for the 3rd time, it no longer runs even if other agents continue.  \\
\hline
\end{tabular}
 The \textbf{interleave}
 parameter must always be equal to or smaller than the specified \textbf{run}
 parameter. This option is not currently compatible with the \textbf{forever}
 option. 
\paragraph*{Note}
 If Soar has been stopped due to a \textbf{halt}
 action, an \textbf{init-soar}
 command must be issued before Soar can be restarted with the \textbf{run}
 command. 
\subsubsection*{Default Aliases}
\hline
\soar{\soar{\soar{ Alias }}} & Maps to  \\
\hline
\soar{\soar{\soar{ d }}} & run -d 1  \\
\hline
\soar{\soar{\soar{ e }}} & run -e 1  \\
\hline
\soar{\soar{\soar{ step }}} & run 1  \\
\hline
\end{tabular}


\divider 
\subsection{exit}\label{exit}
Terminates Soar and exits the kernel.
\subsubsection{Default Aliases}
\verb|stop          exit|

\divider
\subsection{\soarb{sp}}
\label{sp}
\index{sp}
Define a Soar production. 
\subsubsection*{Synopsis}
sp {production_body}
\end{verbatim}
\subsubsection*{Options}
\hline
\soar{\soar{\soar{ production\_body }}} & A Soar production.  \\
\hline
\end{tabular}
\subsubsection*{Description}
 The \textbf{sp}
 command creates a new production and loads it into production memory. \emph{production\_body}
  name 
      ["documentation-string"] 
      [FLAG*]
      LHS
      -->
      RHS
\end{verbatim}
 The first element of a rule is its name. Conventions for names are given in the Soar Users Manual. If given, the documentation-string must be enclosed in double quotes. Optional flags define the type of rule and the form of support its right-hand side assertions will receive. The specific flags are listed in a separate section below. The LHS defines the left-hand side of the production and specifies the conditions under which the rule can be fired. Its syntax is given in detail in a subsequent section. The --$>$ symbol serves to separate the LHS and RHS portions. The RHS defines the right-hand side of the production and specifies the assertions to be made and the actions to be performed when the rule fires. The syntax of the allowable right-hand side actions are given in a later section. The Soar Users Manual gives an elaborate discussion of the design and coding of productions. Please see that reference for tutorial information about productions. 
 If the name of the new production is the same as an existing one, the old production will be overwritten (excised). 
 \textbf{RULE FLAGS}
\\ 
:o-support      specifies that all the RHS actions are to be given
                o-support when the production fires 
:no-support     specifies that all the RHS actions are only to be given
                i-support when the production fires 
:default        specifies that this production is a default production 
                (this matters for excise -task and watch task) 
:chunk          specifies that this production is a chunk 
                (this matters for learn trace)
:interrupt      specifies that Soar should stop running when this 
                production matches but before it fires
                (this is a useful debugging tool)
\end{verbatim}
 Multiple flags may be used, but not both of \textbf{o-support}
 and \textbf{no-support}
. 
 Although you could force your productions to provide O-support or I-support by using these commands --- regardless of the structure of the conditions and actions of the production --- this is not proper coding style. The \textbf{o-support}
 and \textbf{no-support}
 flags are included to help with debugging, but should not be used in a standard Soar program. 
\subsubsection*{Examples}
sp {blocks*create-problem-space   
     "This creates the top-level space"
     (state <s1> ^superstate nil)
     -->
     (<s1> ^name solve-blocks-world ^problem-space <p1>)
     (<p1> ^name blocks-world)
}
\end{verbatim}
\subsubsection*{See Also}
\hyperref[excise]{excise} \hyperref[learn]{learn} \hyperref[watch]{watch} 

\divider 


\section{Examining Memory Systems}
\label{MEMORY}

This section describes the commands used to inspect production memory,
working memory, and preference memory; to see what productions will 
match and fire in the next Propose or Apply phase;  and to examine the 
goal dependency set.  These commands are particularly useful when
running or debugging Soar, as they let users see what Soar is ``thinking.''
The specific commands described in this section are:

\paragraph{Summary}
\begin{quote}
\begin{description}
\item[preferences] - Examine items in preference memory.
\item[production] - Commands to manipulate Soar rules and analyze their usage
\item[production break] - Set interrupt flag on specific productions.
\item[production excise] - This command removes productions from Soar's memory.
\item[production find] - Find productions that contain a given pattern.
\item[production firing-counts] - Print the number of times productions have fired.
\item[production matches] - Print information about the match set and partial matches.
\item[production memory-usage] - Print memory usage for production matches.
\item[production optimize-attribute] - Declare an attribute as multi-attributes so as to increase Rete production matching efficiency.
\item[production watch] - Trace firings and retractions of specific productions.
\item[print] - Print items in working, semantic and production memory.  Can also print the print the WMEs in the goal dependency set for each goal.
\item[wm] - Commands and settings related to working memory and working memory activation.
\item[wm activation] - Get/Set working memory activation parameters
\item[wm add] - Manually add an element to working memory.
\item[wm remove] - Manually remove an element from working memory.
\item[wm watch] - Print information about wmes that match a certain pattern as they are added and removed

\end{description}
\end{quote}

Of these commands, \textbf{print} is the most often used (and the most
complex) followed by \textbf{soar matches} and \textbf{soar memory-usage}. \textbf{print gds}
is useful for examining the goal dependecy set when subgoals seem to
be disappearing unexpectedly. \textbf{preferences}
is used to examine which candidate operators have been proposed.
\textbf{production find} is especially useful when the number of
productions loaded is high.  \soar{production watch} is related to \soar{watch}, but applies only 
to specific, named productions. \soar{production firing-counts} is used to see if how many times
certain rules fire.  

\divider 
\subsection{\soarb{preferences}}
\label{preferences}
\index{preferences}
Examine details about the preferences that support the specified \emph{id}
 and \emph{attribute}
. 
\subsubsection*{Synopsis}
preferences [-0123nNtw] [[id] [[^]attribute]]
\end{verbatim}
\subsubsection*{Options}
\hline
\soar{\soar{\soar{ -0, -n, --none }}} & Print just the preferences themselves  \\
\hline
\soar{\soar{\soar{ -1, -N, --names }}} & Print the preferences and the names of the productions that generated them  \\
\hline
\soar{\soar{\soar{ -2, -t, --timetags }}} & Print the information for the --names option above plus the timetags of the wmes matched by the LHS of the indicated productions  \\
\hline
\soar{\soar{\soar{ -3, -w, --wmes }}} & Print the information for the --timetags option above plus the entire wme matched on the LHS.  \\
\hline
\soar{\soar{\soar{ -o, --object }}} & Print the support for all the wmes that comprise the object (the specified ID).  \\
\hline
\soar{\soar{\soar{id}}} & Must be an existing Soar object identifier.  \\
\hline
\soar{\soar{\soar{attribute}}} & Must be an existing \emph{\^{}attribute}
 of the specified identifier.  \\
\hline
\end{tabular}
\subsubsection*{Description}
 The \textbf{preferences}
 command prints all the preferences for the given object id and attribute. If \emph{id}
 and \emph{attribute}
 are not specified, they default to the current state and the current operator. The '\^{}' is optional when specifying the attribute. The optional arguments indicates the level of detail to print about each preference. 
 This command is useful for examining which candidate operators have been proposed and what relationships, if any, exist among them. If a preference has O-support, the string, ``:O'' will also be printed. 
 When only the ID is specified on the commandline, if the ID is a state, Soar uses the default attribute \^{}operator. If the ID is not a state, Soar prints the support information for all WMEs whose $<$value$>$ is the ID. 
 When an ID and the --object flag are specified, Soar prints the preferences / wme support for all WMEs comprising the specified ID. 
\subsection*{Note}
 For the time being, \textbf{numeric-indifferent}
 preferences are listed under the heading ``binary indifferents:''. 
\subsubsection*{Examples}
soar> preferences S1 operator --names
Preferences for S1 ^operator:
acceptables:
 O2 (fill) +
   From waterjug*propose*fill
 O3 (fill) +
   From waterjug*propose*fill
unary indifferents:
 O2 (fill) =
   From waterjug*propose*fill
 O3 (fill) =
   From waterjug*propose*fill
\end{verbatim}
 preferences -n
\end{verbatim}
soar> preferences s1 jug
Preferences for S1 ^jug:
  
acceptables:
  (S1 ^jug I4) �:O 
  (S1 ^jug J1) �:O 
\end{verbatim}
soar> pref J1 -1
 Support for (31: O3 ^jug J1)
   (O3 ^jug J1) 
     From water-jug*propose*fill
 Support for (11: S1 ^jug J1)
   (S1 ^jug J1) �:O 
     From water-jug*apply*initialize-water-jug-look-ahead
\end{verbatim}
 soar> pref -o s1
 Support for S1 ^problem-space:
   (S1 ^problem-space P1) 
 Support for S1 ^name:
   (S1 ^name water-jug) �:O 
 Support for S1 ^jug:
   (S1 ^jug I4) �:O 
   (S1 ^jug J1) �:O 
 Support for S1 ^desired:
   (S1 ^desired D1) �:O 
 Support for S1 ^superstate-set:
   (S1 ^superstate-set nil) 
 Preferences for S1 ^operator:
 acceptables:
   O2 (fill) +
   O3 (fill) +
 Arch-created wmes for S1�:
 (2: S1 ^superstate nil)
 (1: S1 ^type state)
 Input (IO) wmes for S1�:
 (3: S1 ^io I1)
\end{verbatim}
\subsubsection*{See Also}


\divider 
\input{wikicmd/tex/production}

\divider 
\subsection{\soarb{print}}
\label{print}
\index{print}
Print items in working memory or production memory. 
\subsubsection*{Synopsis}
print [-fFin] production_name
print -[a|c|D|j|u][fFin]
print [-i] [-d <depth>] \emph{identifier}
|\emph{timetag}
|\emph{pattern}
print -s[oS]
\end{verbatim}
\subsubsection*{Options}
\subsection*{Printing items in production memory}
\hline
\soar{\soar{\soar{ -a, --all }}} & print the names of all productions currently loaded  \\
\hline
\soar{\soar{\soar{ -c, --chunks }}} & print the names of all chunks currently loaded  \\
\hline
\soar{\soar{\soar{ -D, --defaults }}} & print the names of all default productions currently loaded  \\
\hline
\soar{\soar{\soar{ -f, --full }}} & When printing productions, print the whole production. This is the default when printing a named production.  \\
\hline
\soar{\soar{\soar{ -F, --filename }}} & also prints the name of the file that contains the production.  \\
\hline
\soar{\soar{\soar{ -i, --internal }}} & items should be printed in their internal form. For productions, this means leaving conditions in their reordered (rete net) form.  \\
\hline
\soar{\soar{\soar{ -j, --justifications }}} & print the names of all justifications currently loaded.  \\
\hline
\soar{\soar{\soar{ -n, --name }}} & When printing productions, print only the name and not the whole production. This is the default when printing any category of productions, as opposed to a named production.  \\
\hline
\soar{\soar{\soar{ -u, --user }}} & print the names of all user productions currently loaded  \\
\hline
\soar{\soar{\soar{production\_name}}} & print the production named production-name \\
\hline
\end{tabular}
\subsection*{Printing items in working memory}
\hline
 -d, --depth \emph{n}
 & This option overrides the default printing depth (see the default-wme-depth command for more detail).  \\
\hline
\soar{\soar{\soar{ -i, --internal }}} & items should be printed in their internal form. For working memory, this means printing the individual elements with their timetags, rather than the objects.  \\
\hline
\soar{\soar{\soar{ -v, --varprint }}} & Print identifiers enclosed in angle brackets.  \\
\hline
\emph{identifier}
 & print the object \emph{identifier}
. \emph{identifier}
 must be a valid Soar symbol such as \textbf{S1 }
\hline
\emph{pattern}
 & print the object whose working memory elements matching the given pattern. See Description for more information on printing objects matching a specific pattern.  \\
\hline
\emph{timetag}
 & print the object in working memory with the given \emph{timetag}
\hline
\end{tabular}
\subsection*{Printing the current subgoal stack}
\hline
\soar{\soar{\soar{ -s, --stack }}} & Specifies that the Soar goal stack should be printed. By default this includes both states and operators.  \\
\hline
\soar{\soar{\soar{ -o, --operators }}} & When printing the stack, print only \textbf{operators}
.  \\
\hline
\soar{\soar{\soar{ -S, --states }}} & When printing the stack, print only \textbf{states}
.  \\
\hline
\end{tabular}
\subsubsection*{Description}
 The \textbf{print}
(\emph{identifier}
 ^\emph{attribute value}
 [+])
\end{verbatim}
 The pattern is surrounded by parentheses. The \emph{identifier}
, \emph{attribute}
, and \emph{value}
 must be valid Soar symbols or the wildcard symbol * which matches all occurences. The optional \textbf{+}
 symbol restricts pattern matches to acceptable preferences. 
\subsubsection*{Examples}
print --internal (s1 ^* v2)
\end{verbatim}
print --stack
\end{verbatim}
print -if prodname
\end{verbatim}
print -u
\end{verbatim}
\subsubsection*{Default Aliases}
\hline
\soar{\soar{\soar{ Alias }}} & Maps to  \\
\hline
\soar{\soar{\soar{ p }}} & print  \\
\hline
\soar{\soar{\soar{ pc }}} & print --chunks  \\
\hline
\soar{\soar{\soar{ wmes }}} & print -i  \\
\hline
\end{tabular}
\subsubsection*{See Also}
\hyperref[default-wme-depth]{default-wme-depth} \hyperref[predefined-aliases]{predefined-aliases} 

\divider 
\input{wikicmd/tex/wm}

\divider 

% ****************************************************************************
% ----------------------------------------------------------------------------
\section{Configuring Trace Information and Output}
\label{DEBUG}

This section describes the commands used primarily for debugging or
to configure the trace output printed by Soar as it runs.  Users may:
specify the content of the runtime trace output; ask that
they be alerted when specific productions fire and retract; 
or request details on Soar's performance.

The specific commands described in this section are:


\paragraph{Summary}
\begin{quote}
\begin{description}
\item[echo] - Prints a string to the current output device.
\item[output] - Controls sub-commands and settings related to Soar's output.
\item[output enabled] - Toggles printing at the lowest level.
\item[output console] - Redirects printing to the the terminal.  Most users will not change this.
\item[output callbacks] - Toggles standard Soar agent callback-based printing.
\item[output log] - Record all user-interface input and output to a file. 
\item[output command-to-file] - Dump the printed output and results of a command to a file.
\item[output print-depth] - Set how many generations of an identifier's children that Soar will print
\item[output warnings] - Toggle whether or not warnings are printed.
\item[output verbose] - Control detailed information printed as Soar runs.
\item[output echo-commands] - Set whether or not commands are echoed to other connected debuggers. 
\item[stats] - Print information on Soar's runtime statistics.
\item[trace] - Control the information printed as Soar runs. \emph{(was \soar{watch})}
\item[visualize] - Creates graph visualizations of Soar's memory systems or processing.
\end{description}
\end{quote}

Of these commands, \soar{trace} is the most often used (and the most 
complex).  \textbf{output print-depth} is related to the \textbf{print} command. \soar{stats} 
is useful for understanding how much work Soar is doing. Soar applications that include a graphical interface or other
simulation environment will often require the use of \textbf{echo} 

\divider 
\input{wikicmd/tex/echo}

\divider 
\input{wikicmd/tex/output}

\divider 
\subsection{\soarb{stats}}
\label{stats}
\index{stats}
Print information on Soar\~A�\^a�$\neg$\^a��s runtime statistics. 
\subsubsection*{Synopsis}
stats [-s|-m|-r]
\end{verbatim}
\subsubsection*{Options}
\hline
\soar{\soar{\soar{ -m, --memory }}} & report usage for Soar's memory pools  \\
\hline
\soar{\soar{\soar{ -r, --rete }}} & report statistics about the rete structure  \\
\hline
\soar{\soar{\soar{ -s, --system }}} & report the system (agent) statistics. This is the default if no args are specified.  \\
\hline
\end{tabular}
\subsubsection*{Description}
 This command prints Soar internal statistics. The argument indicates the component of interest. 
 With the \textbf{--system}
 flag, the \textbf{stats}
\item \textbf{Version}
 --- The Soar version number, hostname, and date of the run. 
\item \textbf{Number of productions}
 --- The total number of productions loaded in the system, including all chunks built during problem solving and all default productions. 
\item \textbf{Timing Information}
 --- Might be quite detailed depending on the flags set at compile time. See note on timers below. 
\item \textbf{Decision Cycles}
 --- The total number of decision cycles in the run and the average time-per-decision-cycle in milliseconds. 
\item \textbf{Elaboration cycles}
 --- The total number of elaboration cycles that were executed during the run, the average number of elaboration cycles per decision cycle, and the average time-per-elaboration-cycle in milliseconds. This is not the total number of production firings, as productions can fire in parallel. 
\item \textbf{Production Firings}
 --- The total number of productions that were fired. 
\item \textbf{Working Memory Changes}
 --- This is the total number of changes to working memory. This includes all additions and deletions from working memory. Also prints the average match time. 
\item \textbf{Working Memory Size}
 --- This gives the current, mean and maximum number of working memory elements. 
\end{itemize}
 The optional \textbf{stats}
 argument \textbf{--memory}
 provides information about memory usage and Soar's memory pools, which are used to allocate space for the various data structures used in Soar. 
 The optional \textbf{stats}
 argument \textbf{--rete}
 provides information about node usage in the Rete net, the large data structure used for efficient matching in Soar. 
\subsubsection*{Default Aliases}
\hline
\soar{\soar{\soar{ Alias }}} & Maps to  \\
\hline
\soar{\soar{\soar{ st }}} & stats  \\
\hline
\end{tabular}
\subsubsection*{See Also}
\hyperref[timers]{timers} \subsubsection*{A Note on Timers}
total CPU time
total kernel time
phase kernel time (per phase)
phase callbacks time (per phase)
input function time
output function time
\end{verbatim}
 Total CPU time is calculated from the time a decision cycle (or number of decision cycles) is initiated until stopped. Kernel time is the time spent in core Soar functions. In this case, kernel time is defined as the all functions other than the execution of callbacks and the input and output functions. The total kernel timer is only stopped for these functions. The phase timers (for the kernel and callbacks) track the execution time for individual phases of the decision cycle (i.e., input phase, preference phase, working memory phase, output phase, and decision phase). Because there is overhead associated with turning these timers on and off, the actual kernel time will always be greater than the derived kernel time (i.e., the sum of all the phase kernel timers). Similarly, the total CPU time will always be greater than the derived total (the sum of the other timers) because the overhead of turning these timers on and off is included in the total CPU time. In general, the times reported by the single timers should always be greater than than the corresponding derived time. Additionally, as execution time increases, the difference between these two values will also increase. For those concerned about the performance cost of the timers, all the run time timing calculations can be compiled out of the code by defining NO\_TIMING\_STUFF (in soarkernel.h) before compilation. 


\divider 
\input{wikicmd/tex/trace}

\divider 
\input{wikicmd/tex/visualize}

\divider 

% ----------------------------------------------------------------------------
\section{Configuring Soar's Runtime Parameters}
\label{RUNTIME}

This section describes the commands that control Soar's Runtime Parameters.
Many of these commands provide options that simplify or restrict 
runtime behavior to enable easier and more localized debugging.
Others allow users to select alternative algorithms or methodologies.
Users can configure Soar's learning mechanism; examine the
backtracing information that supports chunks and justifications;
and configure options for selecting between mutually indifferent operators.

The specific commands described in this section are:

\paragraph{Summary}
\begin{quote}
\begin{description}
\item[chunk] - Set the parameters for explanation-based chunking, Soar's learning mechanism.
\item[explain] - Provides interactive exploration of why a rule was learned.
\item[decide ] - Commands and settings related to the selection of operators during the Soar decision process
\item[decide indifferent-selection] -  Controls indifferent preference arbitration.
\item[decide numeric-indifferent-mode] - Select method for combining numeric preferences.
\item[decide predict] - Predict the next selected operator 
\item[decide select] - Force the next selected operator 
\item[decide set-random-seed] - Seed the random number generator.
\item[epmem] - Get/Set episodic memory parameters and statistics
\item[rl] - Get/Set RL parameters and statistics 
\item[smem] - Get/Set semantic memory parameters and statistics
\item[svs] - Perform spatial visual system commands
\end{description}
\end{quote}

% ----------------------------------------------------------------------------
\divider 
\input{wikicmd/tex/chunk}

\divider 
\input{wikicmd/tex/explain}

\divider 
\input{wikicmd/tex/decide}

\divider 
\chapter{Episodic Memory}
\label{EPMEM}
\index{episodic memory}
\index{epmem}

Episodic memory is a record of an agent's stream of experience.
The episodic storage mechanism will automatically record episodes as a Soar agent executes.
The agent can later deliberately retrieve episodic knowledge to extract information and regularities that may not have been noticed during the original experience and combine them with current knowledge such as to improve performance on future tasks.

This chapter is organized as follows: episodic memory structures in working memory (\ref{EPMEM-wm}); episodic storage (\ref{EPMEM-storage}); retrieving episodes (\ref{EPMEM-retrieval}); and a discussion of performance (\ref{EPMEM-perf}).
The detailed behavior of episodic memory is determined by numerous parameters that can be controlled and configured via the \soarb{epmem} command.

Please refer to the documentation for that command in Section \ref{epmem} on page \pageref{epmem}.

\section{Working Memory Structure}
\label{EPMEM-wm}

Upon creation of a new state in working memory (see Section \ref{ARCH-impasses-types} on page \pageref{ARCH-impasses-types}; Section \ref{SYNTAX-impasses} on page \pageref{SYNTAX-impasses}), the architecture creates the following augmentations to facilitate agent interaction with episodic memory:

\begin{verbatim}
(<s> ^epmem <e>)
  (<e> ^command <e-c>)
  (<e> ^result <e-r>)
  (<e> ^present-id #)
\end{verbatim}

As rules augment the \soar{command} structure in order to retrieve episodes (\ref{EPMEM-retrieval}), episodic memory augments the \soar{result} structure in response.
Production actions should not remove augmentations of the \soar{result} structure directly, as episodic memory will maintain these WMEs.

The value of the \soar{present-id} augmentation is an integer and will update to expose to the agent the current episode number.
This information is identical to what is available via the \emph{time} statistic (see Section \ref{epmem} on page \pageref{epmem}) and the \emph{present-id} retrieval meta-data (\ref{EPMEM-meta}).

\section{Episodic Storage}
\label{EPMEM-storage}

Episodic memory records new episodes without deliberate action/consideration by the agent.
The timing and frequency of recording new episodes is controlled by the \soar{phase} and \soar{trigger} parameters.
The \soarb{phase} parameter sets the phase in the decision cycle (default: end of each decision cycle) during which episodic memory stores episodes and processes commands.
The value of the \soarb{trigger} parameter indicates to the architecture the event that concludes an episode: adding a new augmentation to the output-link (default) or each decision cycle.

For debugging purposes, the \soarb{force} parameter allows the user to manually request that an episode be recorded (or not) during the current decision cycle.
Behavior is as follows:

\vspace{-8pt}
\begin{itemize}
\item
	The value of the \soar{force} parameter is initialized to \soar{off} every decision cycle.
	\vspace{-6pt}
\item
	During the \soar{phase} of episodic storage, episodic memory tests the value of the \soar{force} parameter; if it has a value other than of off, episodic memory follows the \emph{forced} policy irrespective of the value of the \soar{trigger} parameter.
	\vspace{-6pt}
\end{itemize}

\subsection{Episode Contents}

When episodic memory stores a new episode, it captures the entire top-state of working memory.
There are currently two exceptions to this policy:

\begin{itemize}
\item
Episodic memory only supports WMEs whose attribute is a constant.
Behavior is currently undefined when attempting to store a WME that has an attribute that is an identifier.

\item
The \soarb{exclusions} parameter allows the user to specify a set of attributes for which Soar will not store WMEs.
The storage process currently walks the top-state of working memory in a breadth-first manner, and any WME that is not reachable other than via an excluded WME will not be stored.
By default, episodic memory excludes the \soar{epmem} and \soar{smem} structures, to prevent encoding of potentially large and/or frequently changing memory retrievals.

\end{itemize}

\subsection{Storage Location}
\index{epmem!storage}

Episodic memory uses SQLite to facilitate efficient and standardized storage and querying of episodes.
The episodic store can be maintained in memory or on disk (per the \soar{database} and \soar{path} parameters).
If the store is located on disk, users can use any standard SQLite programs/components to access/query its contents.
See the later discussion on performance (\ref{EPMEM-perf}) for additional parameters dealing with databases on disk.

Note that changes to storage parameters, for example \soar{database, path} and \soar{append} will not have an effect until the database is used after an initialization. This happens either shortly after launch (on first use) or after a database initialization command is issued. To switch databases or database storage types while running, set your new parameters and then perform an \soar{epmem --init} command.

The \soarb{path} parameter specifies the file system path the database is stored in. When \soar{path} is set to a valid file system path and \soar{database} mode is set to \emph{file}, then the SQLite database is written to that path.

The \soarb{append} parameter will determine whether all existing facts stored in a database on disk will be erased when episodic memory loads. Note that this affects \soar{init-soar} also.  In other words, if the \soar{append} setting is off, all episodes stored will be lost when an init-soar is performed. For episodic memory, \soar{append} mode is \soar{off} by default.

\soarit{Note}: As of version 9.3.3, Soar now uses a new schema for the episodic memory database. This means databases from 9.3.2 and below can no longer be loaded.  A conversion utility will be available in Soar 9.4 to convert from the old schema to the new one.

\section{Retrieving Episodes}
\label{EPMEM-retrieval}
\index{epmem!retrieve}

An agent retrieves episodes by creating an appropriate command (we detail the types of commands below) on the \soar{command} link of a state's \soar{epmem} structure.
At the end of the \soar{phase} of each decision, after episodic storage, episodic memory processes each state's \emph{epmem} command structure.
Results, meta-data, and errors are placed on the \soar{result} structure of that state's \soar{epmem} structure.

Only one type of retrieval command (which may include optional modifiers) can be issued per state in a single decision cycle.
Malformed commands (including attempts at multiple retrieval types) will result in an error:

\begin{verbatim}
<s> ^epmem.result.status bad-cmd
\end{verbatim}

After a command has been processed, episodic memory will ignore it until some aspect of the command structure changes (via addition/removal of WMEs).
When this occurs, the result structure is cleared and the new command (if one exists) is processed.

All retrieved episodes are recreated exactly as stored, except for any operators that have an acceptable preference, which are recreated with the attribute \soar{operator*}.
For example, if the original episode was:

\begin{verbatim}
(<s> ^operator <o1> +)
(<o1> ^name move)
\end{verbatim}

A retrieval of the episode would become:

\begin{verbatim}
(<s> ^operator* <o1>)
(<o1> ^name move)
\end{verbatim}

\subsection{Cue-Based Retrievals}
Cue-based retrieval commands are used to search for an episode in the store that best matches an agent-supplied cue, while adhering to optional modifiers.
A cue is composed of WMEs that partially describe a top-state of working memory in the retrieved episode.
All cue-based retrieval requests must contain a single \soarb{\carat query} cue and, optionally, a single \soarb{\carat neg-query} cue.

\begin{verbatim}
<s> ^epmem.command.query <required-cue>
<s> ^epmem.command.neg-query <optional-negative-cue>
\end{verbatim}

A \soar{\carat query} cue describes structures desired in the retrieved episode, whereas a \soar{\carat neg-query} cue describes non-desired structures.
For example, the following Soar production creates a \soar{\carat query} cue consisting of a particular state name and a copy of a current value on the \soar{input-link} structure:

\begin{verbatim}
sp {epmem*sample*query
    (state <s> ^epmem.command <ec>
               ^io.input-link.foo <bar>)
-->
    (<ec> ^query <q>)
    (<q> ^name my-state-name
         ^io.input-link.foo <bar>)
}
\end{verbatim}

\index{working memory activation}
As detailed below, multiple prior episodes may equally match the structure and contents of an agent's cue.
Nuxoll has produced initial evidence that in some tasks, retrieval quality improves when using \emph{activation} of cue WMEs as a form of feature weighting.
Thus, episodic memory supports integration with working memory activation (see Section \ref{wm-activation} on page \pageref{wm-activation}).
For a theoretical discussion of the Soar implementation of working memory activation, consider reading \emph{Comprehensive Working Memory Activation in Soar} (Nuxoll, A., Laird, J., James, M., ICCM 2004).

The cue-based retrieval process can be thought of conceptually as a nearest-neighbor search.
First, all candidate episodes, defined as episodes containing at least one leaf WME (a cue WME with no sub-structure) in at least one cue, are identified.
Two quantities are calculated for each candidate episode, with respect to the supplied cue(s): the cardinality of the match (defined as the number of matching leaf WMEs) and the activation of the match (defined as the sum of the activation values of each matching leaf WME).
Note that each of these values is negated when applied to a negative query.
To compute each candidate episode's match score, these quantities are combined with respect to the \soarb{balance} parameter as follows:

$$(balance)*(cardinality) + (1-balance)*(activation)$$

Performing a graph match on each candidate episode, with respect to the structure of the cue, could be very computationally expensive, so episodic memory implements a two-stage matching process.
An episode with perfect cardinality is considered a perfect \emph{surface} match and, per the \soarb{graph-match} parameter, is subjected to further \emph{structural} matching.
Whereas surface matching efficiently determines if all paths to leaf WMEs exist in a candidate episode, graph matching indicates whether or not the cue can be structurally unified with the candidate episode (paying special regard to the structural constraints imposed by shared identifiers).
Cue-based matching will return the most recent structural match, or the most recent candidate episode with the greatest match score.

A special note should be made with respect to how short- vs. long-term identifiers (see Section \ref{SMEM-kr} on page \pageref{SMEM-kr}) are interpreted in a cue.
Short-term identifiers are processed much as they are in working memory -- transient structures.
Cue matching will try to find any identifier in an episode (with respect to WME path from state) that can apply.
Long-term identifiers, however, are treated as constants.
Thus, when analyzing the cue, episodic memory will not consider long-term identifier augmentations, and will only match with the same long-term identifier (in the same context) in an episode.

The case-based retrieval process can be further controlled using optional modifiers:

\vspace{-8pt}
\begin{itemize}
\item
	The \soarb{before} command requires that the retrieved episode come relatively before a supplied time:
	\vspace{-6pt}
	\begin{verbatim}
	<s> ^epmem.command.before time
	\end{verbatim}
	\vspace{-6pt}
\item
	The \soarb{after} command requires that the retrieved episode come relatively after a supplied time:
	\vspace{-6pt}
	\begin{verbatim}
	<s> ^epmem.command.after time
	\end{verbatim}
	\vspace{-6pt}
\item
	The \soarb{prohibit} command requires that the time of the retrieved episode is not equal to a supplied time:
	\vspace{-6pt}
	\begin{verbatim}
	<s> ^epmem.command.prohibit time
	\end{verbatim}
	\vspace{-6pt}
	Multiple prohibit command WMEs may be issued as modifiers to a single CB retrieval.
	\vspace{-6pt}
\end{itemize}
\vspace{-12pt}

If no episode satisfies the cue(s) and optional modifiers an error is returned:

\begin{verbatim}
<s> ^epmem.result.failure <query> <optional-neg-query>
\end{verbatim}

If an episode is returned, there is additional meta-data supplied (\ref{EPMEM-meta}).

\subsection{Absolute Non-Cue-Based Retrieval}
At time of storage, each episode is attributed a unique \emph{time}.
This is the current value of \soarb{time} statistic and is provided as the \emph{memory-id} meta-data item of retrieved episodes (\ref{EPMEM-meta}).
An absolute non-cue-based retrieval is one that requests an episode by time.
An agent issues an absolute non-cue-based retrieval by creating a WME on the \soar{command} structure with attribute \emph{retrieve} and value equal to the desired time:

\begin{verbatim}
<s> ^epmem.command.retrieve time
\end{verbatim}

Supplying an invalid value for the \soar{retrieve} command will result in an error.

The time of the first episode in an episodic store will have value 1 and each subsequent episode's time will increase by 1.
Thus the desired time may be the mathematical result of operations performed on a known episode's time.

The current episodic memory implementation does not implement any episodic store dynamics, such as forgetting.
Thus any integer time greater than 0 and less than the current value of the \soar{time} statistic will be valid.
However, if forgetting is implemented in future versions, no such guarantee will be made.

\subsection{Relative Non-Cue-Based Retrieval}
Episodic memory supports the ability for an agent to ``play forward" episodes using relative non-cue-based retrievals.

Episodic memory stores the time of the last successful retrieval (non-cue-based or cue-based).
Agents can indirectly make use of this information by issuing \soarb{next} or \soarb{previous} commands.
Episodic memory executes these commands by attempting to retrieve the episode immediately proceeding/preceding the last successful retrieval (respectively).
To issue one of these commands, the agent must create a new WME on the \soar{command} link with the appropriate attribute (\soar{next} or \soar{previous}) and value of an arbitrary identifier:

\begin{verbatim}
<s> ^epmem.command.next <n>
<s> ^epmem.command.previous <p>
\end{verbatim}

If no such episode exists then an error is returned.

Currently, if the time of the last successfully retrieved episode is known to the agent (as could be the case by accessing result meta-data), these commands are identical to performing an absolute non-cue-based retrieval after adding/subtracting 1 to the last time (respectively).
However, if an episodic store dynamic like forgetting is implemented, these relative commands are guaranteed to return the next/previous valid episode (assuming one exists).

\subsection{Retrieval Meta-Data}
\label{EPMEM-meta}
\index{epmem!structures}

The following list details the WMEs that episodic memory creates in the \soar{result} link of the \soar{epmem} structure wherein a command was issued:

\begin{itemize}

\item \soarb{retrieved <retrieval-root>}
	If episodic memory retrieves an episode, that memory is placed here. This WME is an identifier that is treated as the root of the state that was used to create the episodic memory. If the \soar{retrieve} command was issued with an invalid time, the value of this WME will be \emph{no-memory}.
\item \soarb{success <query> <optional-neg-query>}
	If the cue-based retrieval was successful, the WME will have the status as the attribute and the value of the identifier of the query (and neg-query, if applicable).
\item \soarb{match-score}
	This WME is created whenever an episode is successfully retrieved from a cue-based retrieval command. The WME value is a decimal indicating the raw match score for that episode with respect to the cue(s).
\item \soarb{cue-size}
	This WME is created whenever an episode is successfully retrieved from a cue-based retrieval command. The WME value is an integer indicating the number of leaf WMEs in the cue(s).
\item \soarb{normalized-match-score}
	This WME is created whenever an episode is successfully retrieved from a cue-based retrieval command. The WME value is the decimal result of dividing the raw match score by the cue size. It can hypothetically be used as a measure of episodic memory's relative confidence in the retrieval.
\item \soarb{match-cardinality}
	This WME is created whenever an episode is successfully retrieved from a cue-based retrieval command. The WME value is an integer indicating the number of leaf WMEs matched in the \soar{\carat query} cue minus those matched in the \soar{\carat neg-query} cue.
\item \soarb{memory-id}
	This WME is created whenever an episode is successfully retrieved from a cue-based retrieval command. The WME value is an integer indicating the time of the retrieved episode.
\item \soarb{present-id}
	This WME is created whenever an episode is successfully retrieved from a cue-based retrieval command. The WME value is an integer indicating the current time, such as to provide a sense of ``now" in episodic memory terms. By comparing this value to the \soar{memory-id} value, the agent can gain a sense of the relative time that has passed since the retrieved episode was recorded.
\item \soarb{graph-match}
	This WME is created whenever an episode is successfully retrieved from a cue-based retrieval command and the \soar{graph-match} parameter was \soar{on}. The value is an integer with value 1 if graph matching was executed successfully and 0 otherwise.
\item \soarb{mapping <mapping-root>}
	This WME is created whenever an episode is successfully retrieved from a cue-based retrieval command, the \soar{graph-match} parameter was \soar{on}, and structural match was successful on the retrieved episode. This WME provides a mapping between identifiers in the cue and in the retrieved episode. For each identifier in the cue, there is a \soar{node} WME as an augmentation to the \soar{mapping} identifier. The node has a \soar{cue} augmentation, whose value is an identifier in the cue, and a \soar{retrieved} augmentation, whose value is an identifier in the retrieved episode. In a graph match it is possible to have multiple identifier mappings -- this map represents the ``first" unified mapping (with respect to episodic memory algorithms).
\end{itemize}

\section{Performance}
\label{EPMEM-perf}
\index{epmem!performance}

There are currently two sources of ``unbounded" computation: graph matching and cue-based queries.
Graph matching is combinatorial in the worst case.
Thus, if an episode presents a perfect surface match, but imperfect structural match (i.e. there is no way to unify the cue with the candidate episode), there is the potential for exhaustive search.
Each identifier in the cue can be assigned one of any historically consistent identifiers (with respect to the sequence of attributes that leads to the identifier from the root), termed a literal.
If the identifier is a multi-valued attribute, there will be more than one candidate literals and this situation can lead to a very expensive search process.
Currently there are no heuristics in place to attempt to combat the expensive backtracking.
Worst-case performance will be combinatorial in the total number of literals for each cue identifier (with respect to cue structure).

The cue-based query algorithm begins with the most recent candidate episode and will stop search as soon as a match is found (since this episode must be the most recent).
Given this procedure, it is trivial to create a two-WME cue that forces a linear search of the episodic store.
Episodic memory combats linear scan by only searching candidate episodes, i.e. only those that contain a change in at least one of the cue WMEs.
However, a cue that has no match and contains WMEs relevant to all episodes will force inspection of all episodes.
Thus, worst-case performance will be linear in the number of episodes.

\subsection{Performance Tweaking}
When using a database stored to disk, several parameters become crucial to performance.
The first is \soarb{commit}, which controls the number of episodes that occur between writes to disk.
If the total number of episodes (or a range) is known ahead of time, setting this value to a greater number will result in greatest performance (due to decreased I/O).

The next two parameters deal with the SQLite cache, which is a memory store used to speed operations like queries by keeping in memory structures like levels of index B+-trees.
The first parameter, \soarb{page-size}, indicates the size, in bytes, of each cache page.
The second parameter, \soarb{cache-size}, suggests to SQLite how many pages are available for the cache.
Total cache size is the product of these two parameter settings.
The cache memory is not pre-allocated, so short/small runs will not necessarily make use of this space.
Generally speaking, a greater number of cache pages will benefit query time, as SQLite can keep necessary meta-data in memory.
However, some documented situations have shown improved performance from decreasing cache pages to increase memory locality.
This is of greater concern when dealing with file-based databases, versus in-memory.
The size of each page, however, may be important whether databases are disk- or memory-based.
This setting can have far-reaching consequences, such as index B+-tree depth.
While this setting can be dependent upon a particular situation, a good heuristic is that short, simple runs should use small values of the page size (\soar{1k}, \soar{2k}, \soar{4k}), whereas longer, more complicated runs will benefit from larger values (\soar{8k}, \soar{16k}, \soar{32k}, \soar{64k}).
One known situation of concern is that as indexed tables accumulate many rows (\tild millions), insertion time of new rows can suffer an infrequent, but linearly increasing burst of computation.
In episodic memory, this situation will typically arise with many episodes and/or many working memory changes.
Increasing the page size will reduce the intensity of the spikes at the cost of increasing disk I/O and average/total time for episode storage.
Thus, the settings of page size for long, complicated runs establishes the desired balance of reactivity (i.e. max computation) and average speed.
To ground this discussion, the Figure \ref{fig:epmem-cache} depicts maximum and average episodic storage time (the value of the epmem\_storage timer, converted to milliseconds) with different page sizes after 10 million decisions (1 episode/decision) of a very basic agent (i.e. very few working memory changes per episode) running on a 2.8GHz Core i7 with Mac OS X 10.6.5.
While only a single use case, the cross-point of these data forms the basis for the decision to default the parameter at 8192 bytes.

\begin{figure}
\insertfigure{Figures/epmem-cache}{2.5in}
\insertcaption{Example episodic memory cache setting data.}
\label{fig:epmem-cache}
\end{figure}

The next parameter is \soarb{optimization}, which can be set to either \soar{safety} or \soar{performance}.
The \soar{safety} parameter setting will use SQLite default settings.
If data integrity is of importance, this setting is ideal.
The \soar{performance} setting will make use of lesser data consistency guarantees for significantly greater performance.
First, writes are no longer synchronous with the OS (synchronous pragma), thus episodic memory won't wait for writes to complete before continuing execution.
Second, transaction journaling is turned off (journal\_mode pragma), thus groups of modifications to the episodic store are not atomic (and thus interruptions due to application/os/hardware failure could lead to inconsistent database state).
Finally, upon initialization, episodic memory maintains a continuous exclusive lock to the database (locking\_mode pragma), thus other applications/agents cannot make simultaneous read/write calls to the database (thereby reducing the need for potentially expensive system calls to secure/release file locks).

Finally, maintaining accurate operation timers can be relatively expensive in Soar.
Thus, these should be enabled with caution and understanding of their limitations.
First, they will affect performance, depending on the level (set via the \soar{timers} parameter).
A level of \soar{three}, for instance, times every step in the cue-based retrieval candidate episode search.
Furthermore, because these iterations are relatively cheap (typically a single step in the linked-list of a b+-tree), timer values are typically unreliable (depending upon the system, resolution is 1 microsecond or more).


\divider 
\chapter{Reinforcement Learning}
\label{RL}
\index{reinforcement learning}
\index{preference!numeric-indifferent}
\index{rl}

Soar has a reinforcement learning (RL) mechanism that tunes operator selection knowledge based on a given reward function.
This chapter describes the RL mechanism and how it is integrated with production memory, the decision cycle, and the state stack.
We assume that the reader is familiar with basic reinforcement learning concepts and notation. If not, we recommend first reading \emph{Reinforcement Learning: An Introduction} (1998) by Richard S. Sutton and Andrew G. Barto.
The detailed behavior of the RL mechanism is determined by numerous parameters that can be controlled and configured via the \soarb{rl} command.
Please refer to the documentation for that command in section \ref{rl} on page \pageref{rl}.

\section{RL Rules}
\label{RL-rules}

Soar's RL mechanism learns Q-values for state-operator\footnote{
In this context, the term ``state'' refers to the state of the task or environment, not a state identifier.
For the rest of this chapter, bold capital letter names such as \soarb{S1} will refer to identifiers and italic lowercase names such as $s_1$ will refer to task states.}
pairs.
Q-values are stored as numeric indifferent preferences created by specially formulated productions called \emph{RL rules}.
RL rules are identified by syntax.
A production is a RL rule if and only if its left hand side tests for a proposed operator, its right hand side creates a single numeric indifferent preference, and it is not a template rule (see \ref{RL-templates}).
These constraints ease the technical requirements of identifying/updating RL rules and makes it easy for the agent programmer to add/maintain RL capabilities within an agent.
We define an \emph{RL operator} as an operator with numeric indifferent preferences created by RL rules.

The following is an RL rule:

\begin{verbatim}
sp {rl*3*12*left
   (state <s> ^name task-name
              ^x 3
              ^y 12
	          ^operator <o> +)
   (<o> ^name move
	    ^direction left)
-->
   (<s> ^operator <o> = 1.5)
}
\end{verbatim}

Note that the LHS of the rule can test for anything as long as it contains a test for a proposed operator.
The RHS is constrained to exactly one action: creating a numeric indifferent preference for the proposed operator.

The following are not RL rules:

\begin{verbatim}
sp {multiple*preferences
   (state <s> ^operator <o> +)
-->
   (<s> ^operator <o> = 5, >)
}

sp {variable*binding
    (state <s> ^operator <o> +
               ^value <v>)
-->
    (<s> ^operator <o> = <v>)
}
\end{verbatim}

The first rule proposes multiple preferences for the proposed operator and thus does not comply with the rule format.
The second rule does not comply because it does not provide a \emph{constant} for the numeric indifferent preference value.

In the typical RL use case, the user intends for the agent to learn the best operator in each possible state of the environment.
The most straightforward way to achieve this is to give the agent a set of RL rules, each matching exactly one possible state-operator pair.
This approach is equivalent to a table-based RL algorithm, where the Q-value of each state-operator pair corresponds to the numeric indifferent preference created by exactly one RL rule.

In the more general case, multiple RL rules can match a single state-operator pair, and a single RL rule can match multiple state-operator pairs.
all numeric indifferent preferences for an operator are summed when calculating the operator's Q-value\footnote{
This is assuming the value of \soarb{numeric-indifferent-mode} is set to \soarb{sum}.
In general, the RL mechanism only works correctly when this is the case, and we assume this case in the rest of the chapter.
See page \pageref{decide-numeric-indifferent-mode} for more information about this parameter.}.
In this context, RL rules can be interpreted more generally as binary features in a linear approximator of each state-operator pair's Q-value, and their numeric indifferent preference values their weights.
In other words,
$$Q(s, a) = w_1 \phi_2 (s, a) + w_2 \phi_2 (s, a) + \ldots + w_n \phi_n (s, a)$$
where all RL rules in production memory are numbered $1 \dots n$, $Q(s, a)$ is the Q-value of the state-operator pair $(s, a)$, $w_i$ is the numeric indifferent preference value of RL rule $i$, $\phi_i (s, a) = 0$ if RL rule $i$ does not match $(s, a)$, and $\phi_i (s, a) = 1$ if it does.
This interpretation allows RL rules to simulate a number of popular function approximation schemes used in RL such as tile coding and sparse coding.

\section{Reward Representation}
\label{RL-reward}

RL updates are driven by reward signals.
In Soar, these reward signals are given to the RL mechanism through a working memory link called the \soarb{reward-link}.
Each state in Soar's state stack is automatically populated with a \soarb{reward-link} structure upon creation.
Soar will check this structure for a numeric reward signal for the last operator executed in the associated state at the beginning of every decision phase.
Reward is also collected when the agent is halted or a state is retracted.
% What happens when an agent with multiple states is halted? Do the rewards in the substates get collected?

In order to be recognized, the reward signal must follow this pattern:

\begin{verbatim}
(<r1> ^reward <r2>)
(<r2> ^value [val])
\end{verbatim}

where \verb=<r1>= is the \soarb{reward-link} identifier, \verb=<r2>= is some intermediate identifier, and \verb=[val]= is any constant numeric value.
Any structure that does not match this pattern is ignored.
If there are multiple valid reward signals, their values are summed into a single reward signal.
As an example, consider the following state:

\begin{verbatim}
(S1 ^reward-link R1)
  (R1 ^reward R2)
    (R2 ^value 1.0)
  (R1 ^reward R3)
    (R3 ^value -0.2)
\end{verbatim}  

In this state, there are two reward signals with values 1.0 and -0.2.
They will be summed together for a total reward of 0.8 and this will be the value given to the RL update algorithm.

There are two reasons for requiring the intermediate identifier.
The first is so that multiple reward signals with the same value can exist simultaneously.
Since working memory is a set, multiple WMEs with identical values in all three positions (identifier, attribute, value) cannot exist simultaneously.
Without an intermediate identifier, specifying two rewards with the same value would require a WME structure such as

\begin{verbatim}
(S1 ^reward-link R1)
  (R1 ^reward 1.0)
  (R1 ^reward 1.0)
\end{verbatim}

which is invalid. With the intermediate identifier, the rewards would be specified as

\begin{verbatim}
(S1 ^reward-link R1)
  (R1 ^reward R2)
    (R2 ^value 1.0)
  (R1 ^reward R3)
    (R3 ^value 1.0)
\end{verbatim}

which is valid.
The second reason for requiring an intermediate identifier in the reward signal is so that the rewards can be augmented with additional information, such as their source or how long they have existed.
Although this information will be ignored by the RL mechanism, it can be useful to the agent or programmer.
For example:

\begin{verbatim}
(S1 ^reward-link R1)
  (R1 ^reward R2)
    (R2 ^value 1.0)
    (R2 ^source environment)
  (R1 ^reward R3)
    (R3 ^value -0.2)
    (R3 ^source intrinsic)
    (R3 ^duration 5)
\end{verbatim}  

The \verb=(R2 ^source environment)=, \verb=(R3 ^source intrinsic)=, and \verb=(R3 ^duration 5)= \\
WMEs are arbitrary and ignored by RL, but were added by the agent to keep 
track of where the rewards came from and for how long.

Note that the \soarb{reward-link} is not part of the \soarb{io} structure and is not modified directly by the environment.
Reward information from the environment should be copied, via rules, from the \soarb{input-link} to the \soarb{reward-link}.
Also note that when collecting rewards, Soar simply scans the \soarb{reward-link} and sums the values of all valid reward WMEs.
The WMEs are not modified and no bookkeeping is done to keep track of previously seen WMEs.
This means that reward WMEs that exist for multiple decision cycles will be collected multiple times.

\section{Updating RL Rule Values}
\label{RL-algo}

Soar's RL mechanism is integrated naturally with the decision cycle and performs online updates of RL rules.
Whenever an RL operator is selected, the values of the corresponding RL rules will be updated.
The update can be on-policy (Sarsa) or off-policy (Q-Learning), as controlled by the \soarb{learning-policy} parameter of the \soarb{rl} command.
Let $\delta_t$ be the amount the Q-value of an RL operator changes in an update.
For Sarsa, we have
$$ \delta_t = \alpha \left[ r_{t+1} + \gamma Q(s_{t+1}, a_{t+1}) - Q(s_t, a_t) \right] $$
where 
\begin{itemize}
\item $Q(s_t, a_t)$ is the Q-value of the state and chosen operator in decision cycle $t$.
\item $Q(s_{t+1}, a_{t+1})$ is the Q-value of the state and chosen RL operator in the next decision cycle.
\item $r_{t+1}$ is the total reward collected in the next decision cycle.
\item $\alpha$ and $\gamma$ are the settings of the \soarb{learning-rate} and \soarb{discount-rate} parameters of the \soarb{rl} command, respectively.
\end{itemize}

Note that since $\delta_t$ depends on $Q(s_{t+1}, a_{t+1})$, the update for the operator selected in decision cycle $t$ is not applied until the next RL operator is chosen.
For Q-Learning, we have
$$ \delta_t = \alpha \left[ r_{t+1} + \gamma \underset{a \in A_{t+1}}{\max} Q(s_{t+1}, a) - Q(s_t, a_t) \right] $$
where $A_{t+1}$ is the set of RL operators proposed in the next decision cycle.

Finally, $\delta_t$ is divided by the number of RL rules comprising the Q-value for the operator and the numeric indifferent values for each RL rule is updated by that amount.

An example walkthrough of a Sarsa update with $\alpha = 0.3$ and $\gamma = 0.9$ (the default settings in Soar) follows.

\begin{enumerate}

\item In decision cycle $t$, an operator \soarb{O1} is proposed, and RL rules \soarb{rl-1} and \soarb{rl-2} create the following numeric indifferent preferences for it:
\begin{verbatim}
   rl-1: (S1 ^operator O1 = 2.3)
   rl-2: (S1 ^operator O1 =  -1)
\end{verbatim}  
	The Q-value for \soarb{O1} is $Q(s_t, \soarb{O1}) = 2.3 - 1 = 1.3$.
	 
\item \soarb{O1} is selected and executed, so $Q(s_t, a_t) = Q(s_t, \soarb{O1}) = 1.3$.

\item In decision cycle $t+1$, a total reward of 1.0 is collected on the \soarb{reward-link}, an operator \soarb{O2} is proposed, and another RL rule \soarb{rl-3} creates the following numeric indifferent preference for it:
\begin{verbatim}
	rl-3: (S1 ^operator O2 = 0.5)
\end{verbatim}
	So $Q(s_{t+1}, \soarb{O2}) = 0.5$.

\item \soarb{O2} is selected, so $Q(s_{t+1}, a_{t+1}) = Q(s_{t+1}, \soarb{O2}) = 0.5$
	Therefore, 
	$$\delta_t = \alpha \left[r_{t+1} + \gamma Q(s_{t+1}, a_{t+1}) - Q(s_t, a_t) \right] = 0.3 \times [ 1.0 + 0.9 \times 0.5 - 1.3 ] = 0.045$$
	Since \soarb{rl-1} and \soarb{rl-2} both contributed to the Q-value of \soarb{O1}, $\delta_t$ is evenly divided amongst them, resulting in updated values of
\begin{verbatim}
   rl-1: (<s> ^operator <o> = 2.3225)
   rl-2: (<s> ^operator <o> = -0.9775)
\end{verbatim}

\item \soarb{rl-3} will be updated when the next RL operator is selected.
\end{enumerate}

\subsection{Gaps in Rule Coverage}
\label{RL-gaps}

Call an operator with numeric indifferent preferences an RL operator.
The previous description had assumed that RL operators were selected in both decision cycles $t$ and $t+1$.
If the operator selected in $t+1$ is not an RL operator, then $Q(s_{t+1}, a_{t+1})$ would not be defined, and an update for the RL operator selected at time $t$ will be undefined.
We will call a sequence of one or more decision cycles in which RL operators are not selected between two decision cycles in which RL operators are selected a \emph{gap}.
Conceptually, it is desirable to use the temporal difference information from the RL operator after the gap to update the Q-value of the RL operator before the gap.
There are no intermediate storage locations for these updates.
Requiring that RL rules support operators at every decision can be difficult for agent programmers, particularly for operators that do not represent steps in a task, but instead perform generic maintenance functions, such as cleaning processed output-link structures.

To address this issue, Soar's RL mechanism supports automatic propagation of updates over gaps.
For a gap of length $n$, the Sarsa update is
$$\delta_t = \alpha \left[ \sum_{i=t}^{t+n}{\gamma^{i-t} r_i} + \gamma^{n+1} Q(s_{t+n+1}, a_{t+n+1}) - Q(s_t, a_t) \right]$$
and the Q-Learning update is
$$\delta_t = \alpha \left[ \sum_{i=t}^{t+n}{\gamma^{i-t} r_i} + \gamma^{n+1} \underset{a \in A_{t+n+1}}{\max} Q(s_{t+n+1}, a) - Q(s_t, a_t) \right]$$

Note that rewards will still be collected during the gap, but they are discounted based on the number of decisions they are removed from the initial RL operator.

Gap propagation can be disabled by setting the \soarb{temporal-extension} parameter of the \soarb{rl} command to \soarb{off}.
When gap propagation is disabled, the RL rules preceding a gap are updated using $Q(s_{t+1}, a_{t+1}) = 0$.
The \soarb{rl} setting of the \soarb{watch} command (see Section \ref{trace} on page \pageref{trace}) is useful in identifying gaps.


\subsection{RL and Substates}
\label{RL-substates}

When an agent has multiple states in its state stack, the RL mechanism will treat each substate independently.
As mentioned previously, each state has its own \soarb{reward-link}.
When an RL operator is selected in a state \soarb{S}, the RL updates for that operator are only affected by the rewards collected on the \soarb{reward-link} for \soarb{S} and the Q-values of subsequent RL operators selected in \soarb{S}.

The only exception to this independence is when a selected RL operator forces an operator-no-change impasse.
When this occurs, the number of decision cycles the RL operator at the superstate remains selected is dependent upon the processing in the impasse state.
Consider the operator trace in Figure \ref{fig:rl-optrace}.

\begin{itemize}
\item At decision cycle 1, RL operator \soarb{O1} is selected in \soarb{S1} and causes an operator-no-change impass for three decision cycles.
\item In the substate \soarb{S2}, operators \soarb{O2}, \soarb{O3}, and \soarb{O4} are selected and applied sequentially.
\item Meanwhile in \soarb{S1}, reward values $r_2$, $r_3$, and $r_4$ are put on the \soarb{reward-link} sequentially.
\item Finally, the impasse is resolved by \soarb{O4}, the proposal for \soarb{O1} is retracted, and RL operator \soarb{O5} is selected in \soarb{S1}.
\end{itemize}

\begin{figure}
\insertfigure{Figures/rl-optrace}{1.5in}
\insertcaption{Example Soar substate operator trace.}
\label{fig:rl-optrace}
\end{figure}

In this scenario, only the RL update for $Q(s_1, \soarb{O1})$ will be different from the ordinary case.
Its value depends on the setting of the \soarb{hrl-discount} parameter of the \soarb{rl} command.
When this parameter is set to the default value \soarb{on}, the rewards on \soarb{S1} and the Q-value of \soarb{O5} are discounted by the number of decision cycles they are removed from the selection of \soarb{O1}.
In this case the update for $Q(s_1, \soarb{O1})$ is
$$\delta_1 = \alpha \left[ r_2 + \gamma r_3 + \gamma^2 r_4 + \gamma^3 Q(s_5, \soarb{O5}) - Q(s_1, \soarb{O1}) \right]$$
which is equivalent to having a three decision gap separating \soarb{O1} and \soarb{O5}.

When \soarb{hrl-discount} is set to \soarb{off}, the number of cycles \soarb{O1} has been impassed will be ignored.
Thus the update would be
$$\delta_1 = \alpha \left[ r_2 + r_3 + r_4 + \gamma Q(s_5, \soarb{O5}) - Q(s_1, \soarb{O1}) \right]$$

For impasses other than operator no-change, RL acts as if the impasse hadn't occurred.
If \soarb{O1} is the last RL operator selected before the impasse, $r_2$ the reward received in the decision cycle immediately following, and \soarb{O}$_\soarb{n}$, the first operator selected after the impasse, then \soarb{O1} is updated with 
$$\delta_1 = \alpha \left[ r_2 + \gamma Q(s_n, \soarb{O}_\soarb{n}) - Q(s_1, \soarb{O1}) \right]$$

If an RL operator is selected in a substate immediately prior to the state's retraction, the RL rules will be updated based only on the reward signals present and not on the Q-values of future operators.
This point is not covered in traditional RL theory.
The retraction of a substate corresponds to a suspension of the RL task in that state rather than its termination, so the last update assumes the lack of information about future rewards rather than the discontinuation of future rewards.
To handle this case, the numeric indifferent preference value of each RL rule is stored as two separate values, the expected current reward (ECR) and expected future reward (EFR).
The ECR is an estimate of the expected immediate reward signal for executing the corresponding RL operator.
The EFR is an estimate of the time discounted Q-value of the next RL operator.
Normal updates correspond to traditional RL theory (showing the Sarsa case for simplicity):
\begin{align*}
\delta_{ECR} &= \alpha \left[ r_t - ECR(s_t, a_t) \right] \\
\delta_{EFR} &= \alpha \left[ \gamma Q(s_{t+1}, a_{t+1}) - EFR(s_t, a_t) \right] \\
\delta_t &= \delta_{ECR} + \delta_{EFR} \\
&= \alpha \left[ r_t + \gamma Q(s_{t+1}, a_{t+1}) - \left( ECR(s_t, a_t) + EFR(s_t, a_t) \right) \right] \\
&= \alpha \left[ r_t + \gamma Q(s_{t+1}, a_{t+1}) - Q(s_t, a_t) \right]
\end{align*}
During substate retraction, only the ECR is updated based on the reward signals present at the time of retraction, and the EFR is unchanged.

Soar's automatic subgoaling and RL mechanisms can be combined to naturally implement hierarchical reinforcement learning algorithms such as MAXQ and options.

\subsection{Eligibility Traces}
\label{RL-et}
The RL mechanism supports eligibility traces, which can improve the speed of learning by 
updating RL rules across multiple sequential steps. \\
The \soarb{eligibility-trace-decay-rate} and \soarb{eligibility-trace-tolerance} parameters control this mechanism.
By setting \soarb{eligibility-trace-decay-rate} to \soarb{0} (default), eligibility traces are in effect disabled.
When eligibility traces are enabled, the particular algorithm used is dependent upon the learning policy.
For Sarsa, the eligibility trace implementation is \emph{Sarsa($\lambda$)}. 
For Q-Learning, the eligibility trace implementation is \emph{Watkin's Q($\lambda$)}.

\subsubsection{Exploration}

The \soarb{indifferent-selection} command (page \pageref{decide-indifferent-selection}) determines how operators are selected based on their numeric indifferent preferences.
Although all the indifferent selection settings are valid regardless of how the numeric indifferent preferences were arrived at, the \soarb{epsilon-greedy} and \soarb{boltzmann} settings are specifically designed for use with RL and correspond to the two most common exploration strategies.
In an effort to maintain backwards compatibility, the default exploration policy is \soarb{softmax}.
As a result, one should change to \soarb{epsilon-greedy} or \soarb{boltzmann} when the reinforcement learning mechanism is enabled.

\subsection{GQ($\lambda$)}

\emph{Sarsa($\lambda$)} and \emph{Watkin's Q($\lambda$)} help agents to solve the temporal credit assignment problem more quickly.
However, if you wish to implement something akin to CMACs to generalize from experience, convergence is not guaranteed by these algorithms.
\emph(GQ($\lambda$)} is a gradient descent algorithm designed to ensure convergence when learning off-policy.
Soar provides both \soarb{on-policy-gq-lambda} and \soarb{off-policy-gq-lambda} to increase the likelihood of convergence when learning under these conditions.
If you should choose to use one of these algorithms, we recommend setting \soarb{step-size-parameter} to something small, such as $0.01$
in order to ensure that the secondary set of weights used by \emph(GQ($\lambda$)} change slowly enough for efficient convergence.

\section{Automatic Generation of RL Rules}

The number of RL rules required for an agent to accurately approximate operator Q-values is usually infeasibly large to write by hand, even for small domains.
Therefore, several methods exist to automate this.

\subsection{The gp Command}
The \soar{gp} command can be used to generate productions based on simple patterns.
This is useful if the states and operators of the environment can be distinguished by a fixed number of dimensions with finite domains.
An example is a grid world where the states are described by integer row/column coordinates, and the available operators are to move north, south, east, or west.
In this case, a single \soar{gp} command will generate all necessary RL rules:
	
\begin{verbatim}
gp {gen*rl*rules
   (state <s> ^name gridworld
              ^operator <o> +
              ^row [ 1 2 3 4 ]
              ^col [ 1 2 3 4 ])
   (<o> ^name move
        ^direction [ north south east west ])
-->
   (<s> ^operator <o> = 0.0)
}
\end{verbatim}
	
For more information see the documentation for this command on page \pageref{gp}.

\subsection{Rule Templates}
\label{RL-templates}

Rule templates allow Soar to dynamically generate new RL rules based on a predefined pattern as the agent encounters novel states.
This is useful when either the domains of environment dimensions are not known ahead of time, or when the enumerable state space of the environment is too large to capture in its entirety using \soar{gp}, but the agent will only encounter a small fraction of that space during its execution.
For example, consider the grid world example with 1000 rows and columns.
Attempting to generate RL rules for each grid cell and action a priori will result in $1000 \times 1000 \times 4 = 4 \times 10^6$ productions.
However, if most of those cells are unreachable due to walls, then the agent will never fire or update most of those productions.
Templates give the programmer the convenience of the \soar{gp} command without filling production memory with unnecessary rules.

Rule templates have variables that are filled in to generate RL rules as the agent encounters novel combinations of variable values.
A rule template is valid if and only if it is marked with the \soarb{:template} flag and, in all other respects, adheres to the format of an RL rule.
However, whereas an RL rule may only use constants as the numeric indifference preference value, a rule template may use a variable.
Consider the following rule template:

\begin{verbatim}
sp {sample*rule*template
    :template
    (state <s> ^operator <o> +
               ^value <v>)
-->
    (<s> ^operator <o> = <v>)
}
\end{verbatim}

During agent execution, this rule template will match working memory and create new productions by substituting all variables in the rule template that matched against constant values with the values themselves.
Suppose that the LHS of the rule template matched against the state

\begin{verbatim}
(S1 ^value 3.2)
(S1 ^operator O1 +)
\end{verbatim}

Then the following production will be added to production memory:

\begin{verbatim}
sp {rl*sample*rule*template*1
    (state <s> ^operator <o> +
               ^value 3.2)
-->
    (<s> ^operator <o> = 3.2)
}
\end{verbatim}

The variable \soar{<v>} is replaced by \soar{3.2} on both the LHS and the RHS, but \soar{<s>} and \soar{<o>} are not replaced because they matches against identifiers (\soar{S1} and \soar{O1}).
As with other RL rules, the value of \soar{3.2} on the RHS of this rule may be updated later by reinforcement learning, whereas the value of \soar{3.2} on the LHS will remain unchanged.
If \soar{<v>} had matched against a non-numeric constant, it will be replaced by that constant on the LHS, but the RHS numeric indifference preference value will be set to zero to make the new rule valid.

The new production's name adheres to the following pattern:
\soarb{rl*template-name*id}, where \soarb{template-name} is the name of the originating rule template and \soarb{id} is monotonically increasing integer that guarantees the uniqueness of the name.

If an identical production already exists in production memory, then the newly generate production is discarded.
It should be noted that the current process of identifying unique template match instances can become quite expensive in long agent runs.
Therefore, it is recommended to generate all necessary RL rules using the \soar{gp} command or via custom scripting when possible.

\subsection{Chunking}
Since RL rules are regular productions, they can be learned by chunking just like any other production.
This method is more general than using the \soar{gp} command or rule templates, and is useful if the environment state consists of arbitrarily complex relational structures that cannot be enumerated.


\divider 
\chapter{Semantic Memory}
\label{SMEM}
\index{semantic memory}
\index{smem}

Soar's semantic memory is a repository for long-term declarative knowledge, supplementing what is contained in short-term working memory (and production memory). 
Episodic memory, which contains memories of the agent's experiences, is described in Chapter \ref{EPMEM}. 
The knowledge encoded in episodic memory is organized temporally, and specific information is embedded within the context of when it was experienced, whereas knowledge in semantic memory is independent of any specific context, representing more general facts about the world.

This chapter is organized as follows: semantic memory structures in working memory (\ref{SMEM-wm}); representation of knowledge in semantic memory (\ref{SMEM-kr}); storing semantic knowledge (\ref{SMEM-store}); retrieving semantic knowledge (\ref{SMEM-retrieve}); and a discussion of performance (\ref{SMEM-perf}). 
The detailed behavior of semantic memory is determined by numerous parameters that can be controlled and configured via the \soarb{smem} command. 
Please refer to the documentation for that command in Section \ref{smem} on page \pageref{smem}.


\section{Working Memory Structure}
\label{SMEM-wm}

Upon creation of a new state in working memory (see Section \ref{ARCH-impasses-types} on page \pageref{ARCH-impasses-types}; Section \ref{SYNTAX-impasses} on page \pageref{SYNTAX-impasses}), the architecture creates the following augmentations to facilitate agent interaction with semantic memory:

\begin{verbatim}
(<s> ^smem <smem>)
  (<smem> ^command <smem-c>)
  (<smem> ^result <smem-r>)
\end{verbatim}

As rules augment the \emph{command} structure in order to access/change semantic knowledge (\ref{SMEM-store}, \ref{SMEM-retrieve}), semantic memory augments the \emph{result} structure in response.
Production actions should not remove augmentations of the \emph{result} structure directly, as semantic memory will maintain these WMEs.



\section{Knowledge Representation}
\label{SMEM-kr}

The representation of knowledge in semantic memory is similar to that in working memory (see Section \ref{ARCH-wm} on page \pageref{ARCH-wm}) -- both include graph structures that are composed of symbolic elements consisting of an identifier, an attribute, and a value. 
It is important to note, however, key differences:

\begin{itemize}

\item 
Currently semantic memory only supports attributes that are symbolic constants (string, integer, or decimal), but \emph{not} attributes that are identifiers

\item 
Whereas working memory is a single, connected, directed graph, semantic memory can be disconnected, consisting of multiple directed, connected sub-graphs

\end{itemize}

\emph{Long-term} identifiers (LTIs) are defined as identifiers that exist in semantic memory.
The specific letter-number combination that labels an LTI (e.g. S5 or C7) is permanently associated with that long-term identifier: any retrievals of the long-term identifier are guaranteed to return the associated letter-number pair.  
For clarity, when printed, a long-term identifier is prefaced with the {@} symbol (e.g. {@}S5 or {@}C7). 
Also, when presented in a figure, long-term identifiers will be indicated by a double-circle. 
For instance, Figure \ref{fig:smem-concept} depicts the long-term identifier {@}A68, with four augmentations, representing the addition fact of ${6+7=13}$ (or, rather, 3, carry 1, in context of multi-column arithmetic).

\begin{figure}
\insertfigure{Figures/smem-concept}{1.5in}
\insertcaption{Example long-term identifier with four augmentations.}
\label{fig:smem-concept}
\end{figure}

\subsection{Integrating Long-Term Identifiers with Soar}
Integrating long-term identifiers in Soar presents a number of theoretical and implementation challenges.  
This section discusses the state of integration with each of Soar's memories/learning mechanisms.

\subsubsection{Working Memory}
Long-term identifiers exist as peers with short-term identifiers in Working Memory.

\subsubsection{Procedural Memory}
Soar's production parser (i.e. the \soarb{sp} command) has been modified to allow specification of long-term identifiers (prefaced with an {@} symbol) in any context where a variable is valid.
If a rule contains a long-term identifier that is not currently in semantic memory, a fatal error will be raised and Soar will quit.  
Once added to the rete, the long-term identifier is treated as a constant for matching purposes.  
If specified as the value of a WME in an action, a long-term identifier will be added to working memory if it does not already exist.  
There is also preliminary support for chunking over long-term identifiers.

It is currently possible to create production actions wherein the identifier of a new WME is a long-term identifier that exists neither in the production conditions, nor as the attribute or value of a prior action.  
Such rules will wreak havoc within Soar and are not supported.  
They will be detected and disallowed in future versions of semantic memory.

\subsubsection{Episodic Memory}
Episodic memory (see Section \ref{EPMEM} on page \pageref{EPMEM}) faithfully captures short- vs. long-term identifiers, including the episode of transition.  
Cues are handled in much the same way as cue-based retrievals, with respect to the differences in semantics of a short- vs. long-term identifier.

\section{Storing Semantic Knowledge}
\label{SMEM-store}

An agent stores a long-term identifier to semantic memory by creating a \emph{store} command: this is a WME whose identifier is the \emph{command} link of a state's \emph{smem} structure, the attribute is \emph{store}, and the value is an identifier (short or long).

\begin{verbatim}
<s> ^smem.command.store <identifier>
\end{verbatim}

Semantic memory will encode and store all WMEs whose identifier is the value of the store command.  
Storing deeper levels of working memory is achieved through multiple store commands.

Multiple store commands can be issued in parallel.  
Storage commands are processed on every state at the end of every phase of every decision cycle.  
Storage is guaranteed to succeed and a status WME will be created, where the identifier is the \emph{result} link of the \emph{smem} structure of that state, the attribute is \emph{success}, and the value is the value of the store command above.

\begin{verbatim}
<s> ^smem.result.success <identifier>
\end{verbatim}

Any short-term identifiers that compose the stored WMEs will be converted to long-term identifiers. 
If a long-term identifier is the value of a store command, the stored WMEs replace those associated with the LTI in semantic memory. 
It should be noted that between issuing store commands, it is possible that the augmentations of a long-term identifier in working memory are inconsistent with those in semantic memory.

\subsection{User-Initiated Storage}
Semantic memory provides agent designers the ability to store semantic knowledge via the \soarb{add} switch of the \soarb{smem} command (see Section \ref{smem} on page \pageref{smem}).  
The format of the command is nearly identical to the working memory manipulation components of the RHS of a production (i.e. no RHS-functions; see Section \ref{SYNTAX-pm-action} on page \pageref{SYNTAX-pm-action}).  
For instance:

\begin{verbatim}
smem --add {
   (<arithmetic> ^add10-facts <a01> <a02> <a03>)
   (<a01> ^digit1 1 ^digit-10 11)
   (<a02> ^digit1 2 ^digit-10 12)
   (<a03> ^digit1 3 ^digit-10 13)
}
\end{verbatim}

Unlike agent storage, declarative storage is automatically recursive.  
Thus, this command instance will add a new long-term identifier (represented by the temporary 'arithmetic' variable) with three augmentations.  
The value of each augmentation will each become an LTI with two constant attribute/value pairs.  
Manual storage can be arbitrarily complex and use standard dot-notation.

\subsection{Storage Location}
Semantic memory uses SQLite to facilitate efficient and standardized storage and querying of knowledge.  
The semantic store can be maintained in memory or on disk (per the \soarb{database} and \soarb{path} parameters). 
If the store is located on disk, users can use any standard SQLite programs/components to access/query its contents.
However, using a disk-based semantic store is very costly (performance is discussed in greater detail in Section \ref{SMEM-perf} on page \pageref{SMEM-perf}), and running in memory is recommended for most runs.

The \soarb{lazy-commit} parameter is a performance optimization. 
If set to \soarb{on} (default), disk databases will not reflect semantic memory changes until the Soar kernel shuts down. 
This improves performance by avoiding disk writes. 
The \soarb{optimization} parameter (see Section \ref{SMEM-perf} on page \pageref{SMEM-perf}) will have an affect on whether databases on disk can be opened while the Soar kernel is running.


\section{Retrieving Semantic Knowledge}
\label{SMEM-retrieve}

An agent retrieves knowledge from semantic memory by creating an appropriate command (we detail the types of commands below) on the \emph{command} link of a state's \emph{smem} structure. 
At the end of the output of each decision, semantic memory processes each state's \emph{smem} command structure.  
Results, meta-data, and errors are added to the \emph{result} structure of that state's \emph{smem} structure.

Only one type of retrieval command (which may include optional modifiers) can be issued per state in a single decision cycle.  
Malformed commands (including attempts at multiple retrieval types) will result in an error:

\begin{verbatim}
<s> ^smem.result.bad-cmd <smem-c>
\end{verbatim}

Where the \soarb{smem-c} variable refers to the \emph{command} structure of the state.

After a command has been processed, semantic memory will ignore it until some aspect of the command structure changes (via addition/removal of WMEs).  
When this occurs, the result structure is cleared and the new command (if one exists) is processed.

\subsection{Non-Cue-Based Retrievals}
A non-cue-based retrieval is a request by the agent to reflect in working memory the current augmentations of a long-term identifier in semantic memory. 
The command WME has a \emph{retrieve} attribute and a long-term identifier value:

\begin{verbatim}
<s> ^smem.command.retrieve <lti>
\end{verbatim}

If the value of the command is not a long-term identifier, an error will result: 

\begin{verbatim}
<s> ^smem.result.failure <lti>
\end{verbatim}

Otherwise, two new WMEs will be placed on the result structure:

\begin{verbatim}
<s> ^smem.result.success <lti>
<s> ^smem.result.retrieved <lti>
\end{verbatim}

All augmentations of the long-term identifier in semantic memory will be created as new WMEs in working memory.

\subsection{Cue-Based Retrievals}
A cue-based retrieval performs a search for a long-term identifier in semantic memory whose augmentations exactly match an agent-supplied cue, as well as optional cue modifiers.

A cue is composed of WMEs that describe the augmentations of a long-term identifier.  
A cue WME with a constant value denotes an exact match of both attribute and value.  
A cue WME with a long-term identifier as its value denotes an exact match as well.  
A cue WME with a short-term identifier as its value denotes an exact match of attribute, but with any value (constant or identifier).  

A cue-based retrieval command has a \emph{query} attribute and an identifier value, the cue:

\begin{verbatim}
<s> ^smem.command.query <cue>
\end{verbatim}

For instance, consider the following rule that creates a cue-based retrieval command:

\begin{verbatim}
sp {smem*sample*query
    (state <s> ^smem.command <sc>
               ^lti <lti>
               ^input-link.foo <bar>)
-->
    (<sc> ^query <q>)
    (<q> ^name <any-name>
         ^foo <bar>
         ^associate <lti>
         ^age 25)
}
\end{verbatim}

In this example, assume that the \soar{<lti>} variable will match a long-term identifier and the \soar{<bar>} variable will match a constant.  
Thus, the query requests retrieval of a long-term identifier from semantic memory with augmentations that satisfy ALL of the following requirements:

\begin{itemize}

\item 
Attribute \soar{name} and ANY value

\item 
Attribute \soar{foo} and value equal to the value of variable \soar{<bar>} at the time this rule fires

\item 
Attribute \soar{associate} and value equal to the long-term identifier \soar{<lti>} at the time this rule fires

\item 
Attribute \soar{age} and integer value \soar{25}

\end{itemize}

If no long-term identifier satisfies ALL of these requirements, an error is returned:

\begin{verbatim}
<s> ^smem.result.failure <cue>
\end{verbatim}

Otherwise, two WMEs are added:

\begin{verbatim}
<s> ^smem.result.success <cue>
<s> ^smem.result.retrieved <retrieved-lti>
\end{verbatim}

During a cue-based retrieval it is possible that the retrieved long-term identifier is not in working memory.  
If this is the case, semantic memory will add the long-term identifier to working memory with letter-number pair as was originally stored.

As with non-cue-based retrievals all of the augmentations of the long-term identifier in semantic memory are added as new WMEs to working memory.

It is possible that multiple long-term identifiers match the cue equally well. 
In this case, semantic memory will retrieve the long-term identifier that was most recently stored/retrieved.

The cue-based retrieval process can be further tempered using optional modifiers:

\begin{itemize}

\item 
The \emph{prohibit} command requires that the retrieved long-term identifier is not equal to a supplied long-term identifier:
\begin{verbatim}
<s> ^smem.command.prohibit <bad-lti>
\end{verbatim}
Multiple prohibit command WMEs may be issued as modifiers to a single cue-based retrieval.  
This method can be used to iterate over all matching long-term identifiers.

\item 
The \emph{neg-query} command requires that the retrieved long-term identifier does NOT contain a set of attributes/attribute-value pairs:
\begin{verbatim}
<s> ^smem.command.neg-query <cue>
\end{verbatim}
The syntax of this command is identical to that of regular/positive \emph{query} command.

\item
The \emph{math-query} command requires that the retrieved long term identifier contains an attribute value pair that meets a specified mathematical condition. 
This condition can either be a conditional query or a superlative query. 
Conditional queries are of the format:
\begin{verbatim}
<s> ^smem.command.math-query.<cue-attribute>.<condition-name> <cue-value>
\end{verbatim}
Superlative queries do not use a value argument and are of the format:
\begin{verbatim}
<s> ^smem.command.math-query.<cue-attribute>.<condition-name>
\end{verbatim}
Values used in math queries must be integer or float type values.
Currently supported condition names are:
\begin{description}
  \item[less] A value less than the given argument
  \item[greater] A value greater than the given argument
  \item[less-or-equal] A value less than or equal to the given argument
  \item[greater-or-equal] A value greater than or equal to the given argument
  \item[max] The maximum value for the attribute
  \item[min] The minimum value for the attribute
\end{description}
\end{itemize}

\section{Performance}
\label{SMEM-perf}

Initial empirical results with toy agents show that semantic memory queries carry up to a 40\% overhead as compared to comparable rete matching.  
However, the retrieval mechanism implements some basic query optimization: statistics are maintained about all stored knowledge.  
When a query is issued, semantic memory re-orders the cue such as to minimize expected query time.  
Because only perfect matches are acceptable, and there is no symbol variablization, semantic memory retrievals do not contend with the same combinatorial search space as the rete.  
Preliminary empirical study shows that semantic memory maintains sub-millisecond retrieval time for a large class of queries, even in very large stores (millions of nodes/edges).

Once the number of long-term identifiers overcomes initial overhead (about 1000 WMEs), initial empirical study shows that semantic storage requires far less than 1KB per stored WME.

\subsection{Math queries}
There are some additional performance considerations when using math queries during retrieval.
Initial testing indicates that conditional queries show the same time growth with respect to the number of memories as similar non-math restricted queries, however the actual time for retrieval may be slightly longer.
Superelative queries will often show a worse result than similar non-superelative queries, because the current implementation of semantic memory requires them to iterate over any memory that matches all other involved cues.

\subsection{Performance Tweaking}

When using a database stored to disk, several parameters become crucial to performance.  
The first is \soarb{lazy-commit}, which controls when database changes are written to disk.   
The default setting (\soarb{on}) will keep all writes in memory and only commit to disk upon re-initialization (quitting the agent or issuing the \soarb{init} command).  
The \soarb{off} setting will write each change to disk and thus incurs massive I/O delay.

The next parameter is \soarb{thresh}. 
This has to do with the locality of storing/updating activation information with semantic augmentations. 
By default, all WME augmentations are incrementally sorted by activation, such that cue-based retrievals need not sort large number of candidate long-term identifiers on demand, and thus retrieval time is independent of cue selectivity. 
However, each activation update (such as after a retrieval) incurs an update cost linear in the number of augmentations. 
If the number of augmentations for a long-term identifier is large, this cost can dominate. 
Thus, the \soarb{thresh} parameter sets the upper bound of augmentations, after which activation is stored with the long-term identifier. 
This allows the user to establish a balance between cost of updating augmentation activation and the number of long-term identifiers that must be pre-sorted during a cue-based retrieval. 
As long as the threshold is greater than the number of augmentations of most long-term identifiers, performance should be fine (as it will bound the effects of selectivity).

The next two parameters deal with the SQLite cache, which is a memory store used to speed operations like queries by keeping in memory structures like levels of index B+-trees. 
The first parameter, \soarb{page-size}, indicates the size, in bytes, of each cache page. 
The second parameter, \soarb{cache-size}, suggests to SQLite how many pages are available for the cache. 
Total cache size is the product of these two parameter settings. 
The cache memory is not pre-allocated, so short/small runs will not necessarily make use of this space. 
Generally speaking, a greater number of cache pages will benefit query time, as SQLite can keep necessary meta-data in memory. 
However, some documented situations have shown improved performance from decreasing cache pages to increase memory locality. 
This is of greater concern when dealing with file-based databases, versus in-memory. 
The size of each page, however, may be important whether databases are disk- or memory-based. 
This setting can have far-reaching consequences, such as index B+-tree depth. 
While this setting can be dependent upon a particular situation, a good heuristic is that short, simple runs should use small values of the page size (\soarb{1k}, \soarb{2k}, \soarb{4k}), whereas longer, more complicated runs will benefit from larger values (\soarb{8k}, \soarb{16k}, \soarb{32k}, \soarb{64k}). 
The episodic memory chapter (see Section \ref{EPMEM-perf} on page \pageref{EPMEM-perf}) has some further empirical evidence to assist in setting these parameters for very large stores.

The next parameter is \soarb{optimization}.  
The \soarb{safety} parameter setting will use SQLite default settings.  
If data integrity is of importance, this setting is ideal.  
The \soarb{performance} setting will make use of lesser data consistency guarantees for significantly greater performance.  
First, writes are no longer synchronous with the OS (synchronous pragma), thus semantic memory won't wait for writes to complete before continuing execution.  
Second, transaction journaling is turned off (journal\_mode pragma), thus groups of modifications to the semantic store are not atomic (and thus interruptions due to application/os/hardware failure could lead to inconsistent database state).  
Finally, upon initialization, semantic memory maintains a continuous exclusive lock to the database (locking\_mode pragma), thus other applications/agents cannot make simultaneous read/write calls to the database (thereby reducing the need for potentially expensive system calls to secure/release file locks).

Finally, maintaining accurate operation timers can be relatively expensive in Soar.  
Thus, these should be enabled with caution and understanding of their limitations.  
First, they will affect performance, depending on the level (set via the \soarb{timers} parameter).  
A level of \soarb{three}, for instance, times every modification to long-term identifier recency statistics.  
Furthermore, because these iterations are relatively cheap (typically a single step in the linked-list of a b+-tree), timer values are typically unreliable (depending upon the system, resolution is 1 microsecond or more).



\divider 
\chapter{Spatial Visual System}
\label{SVS}
\index{Spatial Visual System}
\index{SVS}
\index{svs}

\begin{figure}
\insertfigure{Figures/svs-setup}{4in}
\insertcaption{(a) Typical environment setup without using SVS. (b) Same environment using SVS.}
\label{fig:svs-setup}
\end{figure}

The Spatial Visual System (SVS) allows Soar to effectively represent and reason about continuous, three dimensional environments.
SVS maintains an internal representation of the environment as a collection of discrete objects with simple geometric shapes, called the scene graph.
The Soar agent can query for spatial relationships between the objects in the scene graph through a working memory interface similar to that of episodic and semantic memory.
Figure \ref{fig:svs-setup} illustrates the typical use case for SVS by contrasting it with an agent that does not utilize it.
The agent that does not use SVS (a. in the figure) relies on the environment to provide a symblic representation of the continuous state.
On the other hand, the agent that uses SVS (b) accepts a continuous representation of the environment state directly, and then performs queries on the scene graph to extract a symbolic representation internally.
This allows the agent to build more flexible symbolic representations without requiring modifications to the environment code.
Furthermore, it allows the agent to manipulate internal copies of the scene graph and then extract spatial relationships from the modified states, which is useful for look-ahead search and action modeling.
This type of imagery operation naturally captures and propogates the relationships implicit in spatial environments, and doesn't suffer from the frame problem that relational representations have.

\section{The scene graph}

The primary data structure of SVS is the \emph{scene graph}.
The scene graph is a tree in which the nodes represent objects in the scene and the edges represent ``part-of'' relationships between objects.
An example scene graph consisting of a car and a pole is shown in Figure \ref{fig:scene-graph}.
The scene graph's leaves are \emph{geometry nodes} and its interior nodes are \emph{group nodes}.
Geometry nodes represent atomic objects that have intrinsic shape, such as the wheels and chassis in the example.
Currently, the shapes supported by SVS are points, lines, convex polyhedrons, and spheres.
Group nodes represent objects that are the aggregates of their child nodes, such as the car object in the example.
The shape of a group node is the union of the shapes of its children.
Structuring complex objects in this way allows Soar to reason about them naturally at different levels of abstraction.
The agent can query SVS for relationships between the car as a whole with other objects (e.g. does it intersect the pole?), or the relationships between its parts (e.g. are the wheels pointing left or right with respect to the chassis?).
The scene graph always contains at least a root node: the \emph{world node}.

\begin{figure}
\insertfigure{Figures/scene_graph}{5in}
\insertcaption{(a) A 3D scene. (b) The scene graph representation.}
\label{fig:scene-graph}
\end{figure}

Each node other than the world node has a transform with respect to its parent.
A transform consists of three components:

\begin{description}
\item[position $(x,y,z)$]
Specifies the $x$, $y$, and $z$ offsets of the node's origin with respect to its parent's origin.

\item[rotation $(x,y,z)$]
Specifies the rotation of the node relative to its origin in Euler angles.
This means that the node is rotated the specified number of radians along each axis in the order $x-y-z$.
For more information, see \url{http://en.wikipedia.org/wiki/Euler_angles}.

\item[scaling $(x,y,z)$]
Specifies the factors by which the node is scaled along each axis.

\end{description}

The component transforms are applied in the order scaling, then rotation, then position.
Each node's transform is applied with respect to its parent's coordinate system, so the transforms accumulate down the tree.
A node's transform with respect to the world node, or its world transform, is the aggregate of all its ancestor transforms.
For example, if the car has a position transform of $(1,0,0)$ and a wheel on the car has a position transform of $(0,1,0)$, then the world position transform of the wheel is $(1,1,0)$.

SVS represents the scene graph structure in working memory under the \soarb{\^{}spatial-scene} link.
The working memory representation of the car and pole scene graph is

\begin{verbatim}
(S1 ^svs S3)
  (S3 ^command C3 ^spatial-scene S4)
    (S4 ^child C10 ^child C4 ^id world)
      (C10 ^id pole)
      (C4 ^child C9 ^child C8 ^child C7 ^child C6 ^child C5 ^id car)
        (C9 ^id chassis)
        (C8 ^id wheel3)
        (C7 ^id wheel2)
        (C6 ^id wheel1)
        (C5 ^id wheel0)
\end{verbatim}

Each state in working memory has its own scene graph.
When a new state is created, it will receive an independent copy of its parent's scene graph.
This is useful for performing look-ahead search, as it allows the agent to destructively modify the scene graph in a search state using mental imagery operations.

\subsection{svs\_viewer}

A viewer has been provided to show the scene graph visually. 
Run the program \texttt{svs\_viewer -s PORT} from the soar/out folder 
to launch the viewer listening on the given port. Once the viewer is running, 
from within soar use the command \texttt{svs connect\_viewer PORT} to connect 
to the viewer and begin drawing the scene graph. Any changes to the scene graph
will be reflected in the viewer. The viewer by default draws the topstate scene graph, 
to draw that on a substate first stop drawing the topstate with 
\texttt{svs S1.scene.draw off} and then \texttt{svs S7.scene.draw on}. 

\section{Scene Graph Edit Language}

The Scene Graph Edit Language (SGEL) is a simple, plain text, line oriented language that is used by SVS to modify the contents of the scene graph.
Typically, the scene graph is used to represent the state of the external environment, and the programmer sends SGEL commands reflecting changes in the environment to SVS via the Agent::SendSVSInput function in the SML API.
These commands are buffered by the agent and processed at the beginning of each input phase.
Therefore, it is common to send scene changes through SendSVSInput \emph{before} the input phase.
If you send SGEL commands at the end of the input phase, 
the results won't be processed until the following decison cycle.

Each SGEL command begins with a single word command type and ends with a newline.
The four command types are

\begin{description}
\item[\texttt{add ID PARENT\_ID [GEOMETRY] [TRANSFORM]}] \hfill \\
Add a node to the scene graph with the given \texttt{ID}, as a child of \texttt{PARENT\_ID}, 
and with type \texttt{TYPE} (usually object).
The \texttt{GEOMETRY} and \texttt{TRANSFORM} arguments are optional and described below.

\item[\texttt{change ID [GEOMETRY] [TRANSFORM]}] \hfill \\
Change the transform and/or geometry of the node with the given \texttt{ID}.

\item[\texttt{delete ID}] \hfill \\
  Delete the node with the given \texttt{ID}.

\item[\texttt{tag [add|change|delete] ID TAG\_NAME TAG\_VALUE}] \hfill \\
  Adds, changes, or deletes a tag from an object.
  A tag consists of a \texttt{TAG\_NAME}  
  and \texttt{TAG\_VALUE} pair and is added to the node with the given \texttt{ID}.
  Both \texttt{TAG\_NAME} and \texttt{TAG\_VALUE} must be strings.
  Tags can differentiate nodes (e.g. as a type field) and can be used in conjunction with 
  the \texttt{tag\_select} filter to choose a subset of the nodes. 

\end{description}

The \texttt{TRANSFORM} argument has the form \texttt{[p X Y Z] [r X Y Z] [s X Y Z]}, corresponding to the position, rotation, and scaling components of the transform, respectively.
All the components are optional; any combination of them can be excluded, and the included components can appear in any order.

The \texttt{GEOMETRY} argument has two forms:

\begin{description}

\item[\texttt{b RADIUS}] \hfill \\
Make the node a geometry node with sphere shape with radius \texttt{RADIUS}.

\item[\texttt{v X1 Y1 Z1 X2 Y2 Z2 ...}] \hfill \\
Make the node a geometry node with a convex polyhedron shape with the specified vertices.
Any number of vertices can be listed.

\end{description}

\subsection{Examples}

Creating a sphere called ball4 with radius 5 at location (4, 4, 0). \\
\texttt{add ball4 world b 5 p 4 4 0}

Creating a triangle in the xy plane, then rotate it vertically, double its size, and move it to (1, 1, 1).  \\
\texttt{add tri9 world v -1 -1 0 1 -1 0 0 0.5 0 p 1 1 1 r 1.507 0 0 s 2 2 2}

Creating a snowman shape of 3 spheres stacked on each other and located at (2, 2, 0). \\
\texttt{add snowman world p 2 2 0} \\
\texttt{add bottomball snowman b 3 p 0 0 3} \\
\texttt{add middleball snowman b 2 p 0 0 8} \\
\texttt{add topball snowman b 1 p 0 0 11} 

Set the rotation transform on box11 to 180 degrees around the z axis. \\
\texttt{change box11 r 0 0 3.14159}

Changing the color tag on box7 to green. \\
\texttt{tag change box7 color green}


\section{Commands}

The Soar agent initiates commands in SVS via the \soarb{\^{}command} link, 
similar to semantic and episodic memory. These commands allow the agent to 
modify the scene graph and extract filters. 
Commands are processed during the output phase and the results are added to 
working memory during the input phase. 
SVS supports the following commands:

\begin{description}
  \item{\textbf{add\_node}}
  Creates a new node and adds it to the scene graph
\item{\textbf{copy\_node}}
  Creates a copy of an existing node
\item{\textbf{delete\_node}}
  Removes a node from the scene graph and deletes it
\item{\textbf{set\_transform}}
  Changes the position, rotation, and/or scale of a node
\item{\textbf{set\_tag}}
  Adds or changes a tag on a node
\item{\textbf{delete\_tag}}
  Deletes a tag from a node
\item{\textbf{extract}}
	Compute the truth value of spatial relationships in the current scene graph.
\item{\textbf{extract\_once}}
  Same as extract, except it is only computed once and doesn't update when the scene changes.
\end{description}

\subsection{add\_node}

This commands adds a new node to the scene graph. 
\begin{description}
  \item{\soarb{\^{}id [string]}} The id of the node to create
  \item{\soarb{\^{}parent [string]}} The id of the node to attach the new node to (default is world)
  \item{\soarb{\^{}geometry << group point ball box >> }} The geometry the node should have 
  \item{\soarb{\^{}position.\{\^{}x \^{}y \^{}z\} }} Position of the node (optional)
  \item{\soarb{\^{}rotation.\{\^{}x \^{}y \^{}z\} }} Rotation of the node (optional)
  \item{\soarb{\^{}scale.\{\^{}x \^{}y \^{}z\} }} Scale of the node (optional)
\end{description}

The following example creates a node called \texttt{box5} and adds it to the world. 
The node has a box shape of side length 0.1 and is placed at position (1, 1, 0). 
\begin{verbatim}
(S1 ^svs S3)
  (S3 ^command C3 ^spatial-scene S4)
    (C3 ^add_node A1)
      (A1 ^id box5 ^parent world ^geometry box ^position P1 ^scale S6)
        (P1 ^x 1.0 ^y 1.0 ^z 0.0)
        (S6 ^x 0.1 ^y 0.1 ^z 0.1)
\end{verbatim}

\subsection{copy\_node}
This command creates a copy of an existing node and adds it to the scene graph. 
This copy is not recursive, it only copies the node itself, not its children. 
The position, rotation, and scale transforms are also copied from the source node
but they can be changed if desired. 
\begin{description}
  \item{\soarb{\^{}id [string]}} The id of the node to create
  \item{\soarb{\^{}source [string]}} The id of the node to copy
  \item{\soarb{\^{}parent [string]}} The id of the node to attach the new node to (default is world)
  \item{\soarb{\^{}position.\{\^{}x \^{}y \^{}z\} }} Position of the node (optional)
  \item{\soarb{\^{}rotation.\{\^{}x \^{}y \^{}z\} }} Rotation of the node (optional)
  \item{\soarb{\^{}scale.\{\^{}x \^{}y \^{}z\} }} Scale of the node (optional)
\end{description}

The following example copies a node called \texttt{box5} as new node \texttt{box6}
and moves it to position (2, 0, 2).
\begin{verbatim}
(S1 ^svs S3)
  (S3 ^command C3 ^spatial-scene S4)
    (C3 ^copy_node A1)
      (A1 ^id box6 ^source box5 ^position P1)
        (P1 ^x 2.0 ^y 0.0 ^z 2.0)
\end{verbatim}

\subsection{delete\_node}
This command deletes a node from the scene graph. Any children will also be deleted. 
\begin{description}
  \item{\soarb{\^{}id [string]}} The id of the node to delete
\end{description}

The following example deletes a node called \texttt{box5}
\begin{verbatim}
(S1 ^svs S3)
  (S3 ^command C3 ^spatial-scene S4)
    (C3 ^delete_node D1)
      (D1 ^id box5)
\end{verbatim}

\subsection{set\_transform}
This command allows you to change the position, rotation, and/or scale of an 
exisiting node. You can specify any combination of the three transforms. 
\begin{description}
  \item{\soarb{\^{}id [string]}} The id of the node to change
  \item{\soarb{\^{}position.\{\^{}x \^{}y \^{}z\} }} Position of the node (optional)
  \item{\soarb{\^{}rotation.\{\^{}x \^{}y \^{}z\} }} Rotation of the node (optional)
  \item{\soarb{\^{}scale.\{\^{}x \^{}y \^{}z\} }} Scale of the node (optional)
\end{description}

The following example moves and rotates a node called \texttt{box5}.
\begin{verbatim}
(S1 ^svs S3)
  (S3 ^command C3 ^spatial-scene S4)
    (C3 ^set_transform S6)
      (S6 ^id box5 ^position P1 ^rotation R1)
        (P1 ^x 2.0 ^y 2.0 ^z 0.0)
        (R1 ^x 0.0 ^y 0.0 ^z 1.57)
\end{verbatim}

\subsection{set\_tag}
This command allows you to add or change a tag on a node.
If a tag with the same id already exists, 
the existing value will be replaced with the new value.
\begin{description}
  \item{\soarb{\^{}id [string]}} The id of the node to set the tag on
  \item{\soarb{\^{}tag\_name [string]}} The name of the tag to add
  \item{\soarb{\^{}tag\_value [string]}} The value of the tag to add
\end{description}

The following example adds a shape tag to the node \texttt{box5}.
\begin{verbatim}
(S1 ^svs S3)
  (S3 ^command C3 ^spatial-scene S4)
    (C3 ^set_tag S6)
      (S6 ^id box5 ^tag_name shape ^tag_value cube)
\end{verbatim}

\subsection{delete\_tag}
This command allows you to delete a tag from a node.
\begin{description}
  \item{\soarb{\^{}id [string]}} The id of the node to delete the tag from
  \item{\soarb{\^{}tag\_name [string]}} The name of the tag to delete
\end{description}

The following example deletes the shape tag from the node \texttt{box5}.
\begin{verbatim}
(S1 ^svs S3)
  (S3 ^command C3 ^spatial-scene S4)
    (C3 ^delete_tag D1)
      (D1 ^name box5 ^tag_name shape)
\end{verbatim}

\subsection{extract and extract\_once}
This command is commonly used to compute spatial relationships in the scene graph.
More generally, it puts the result of a filter pipeline (described in section \ref{sec:svs-filters}) in working memory.
Its syntax is the same as filter pipeline syntax.
During the input phase, SVS will evaluate the filter and 
put a \soarb{\^{}result} attribute on the command's identifier.
Under the \soarb{\^{}result} attribute is a multi-valued \soarb{\^{}record} attribute.
Each record corresponds to an output value from the head of the filter pipeline, along with the parameters that produced the value.
With the regular \texttt{extract} command, these records will be updated as the scene graph
changes. With the \texttt{extract\_once} command, the records will be created once
and will not change. 
Note that you should not change the structure of a filter once it is created 
(SVS only processes a command once). 
Instead to extract something different you must create a new command. 
The following is an example of an extract command which tests whether the 
car and pole objects are intersecting. The \soar{\^{}status} and \soar{\^{}result} wmes are 
added by SVS when the command is finished. 

\begin{verbatim}
(S1 ^svs S3)
  (S3 ^command C3 ^spatial-scene S4)
    (C3 ^extract E2)
      (E2 ^a A1 ^b B1 ^result R7 ^status success ^type intersect)
        (A1 ^id car ^status success ^type node)
        (B1 ^id pole ^status success ^type node)
        (R7 ^record R17)
          (R17 ^params P1 ^value false)
            (P1 ^a car ^b pole)
\end{verbatim}

\section{Filters}
\label{sec:svs-filters}

Filters are the basic unit of computation in SVS.
They transform the continuous information in the scene graph into symbolic information that can be used by the rest of Soar.
Each filter accepts a number of labeled parameters as input, and produces a single output.
Filters can be arranged into pipelines in which the outputs of some filters are fed into the inputs of other filters.
The Soar agent creates filter pipelines by building an analogous structure in working memory as an argument to an "extract" command.
For example, the following structure defines a set of filters that reports whether the car intersects the pole:

\begin{verbatim}
(S1 ^svs S3)
  (S3 ^command C3 ^spatial-scene S4)
    (C3 ^extract E2)
      (E2 ^a A1 ^b B1 ^type intersect)
        (A1 ^id car ^type node)
        (B1 ^id pole ^type node)
\end{verbatim}

The \soarb{\^{}type} attribute specifies the type of filter to instantiate, and the other attributes specify parameters.
This command will create three filters: an \soarb{intersect} filter and two \soarb{node} filters.
A \soarb{node} filter take an \soarb{id} parameter and returns the scene graph node with that ID as its result.
Here, the outputs of the \soarb{car} and \soarb{pole} node filters are fed into the \soarb{\^{}a} and \soarb{\^{}b} parameters of the \soarb{intersect} filter.
SVS will update each filter's output once every decision cycle, at the end of the input phase.
The output of the \soarb{intersect} filter is a boolean value indicating whether the two objects are intersecting.
This is placed into working memory as the result of the extract command:

\begin{verbatim}
(S1 ^svs S3)
  (S3 ^command C3 ^spatial-scene S4)
    (C3 ^extract E2)
      (E2 ^a A1 ^b B1 ^result R7 ^status success ^type intersect)
        (A1 ^id car ^status success ^type node)
        (B1 ^id pole ^status success ^type node)
        (R7 ^record R17)
          (R17 ^params P1 ^value false)
            (P1 ^a car ^b pole)
\end{verbatim}

Notice that a \soarb{\^{}status} success is placed on each identifier corresponding to a filter.
A \soarb{\^{}result} WME is placed on the extract command with a single record with value \soarb{false}.

\subsection{Result lists}

Spatial queries often involve a large number of objects.
For example, the agent may want to compute whether an object intersects any others in the scene graph.
It would be inconvenient to build the extract command to process this query if the agent had to specify each object involved explicitly.
Too many WMEs would be required, which would slow down the production matcher as well as SVS because it must spend more time interpreting the command structure.
To handle these cases, all filter parameters and results can be lists of values.
For example, the query for whether one object intersects all others can be expressed as

\begin{verbatim}
(S1 ^svs S3)
  (S3 ^command C3)
    (C3 ^extract E2)
      (E2 ^a A1 ^b B1 ^result R7 ^status success ^type intersect)
        (A1 ^id car ^status success ^type node)
        (B1 ^status success ^type all_nodes)
        (R7 ^record R9 ^record R8)
          (R9 ^params P2 ^value false)
            (P2 ^a car ^b pole)
          (R8 ^params P1 ^value true)
            (P1 ^a car ^b car)
\end{verbatim}

The \soarb{all\_nodes} filter outputs a list of all nodes in the scene graph, and the \soarb{intersect} filter outputs a list of boolean values indicating whether the car intersects each node, represented by the multi-valued attribute \soarb{record}.
Notice that each \soarb{record} contains both the result of the query as well as the parameters that produced that result.
Not only is this approach more convenient than creating a separate command for each pair of nodes, but it also allows the \soarb{intersect} filter to answer the query more efficiently using special algorithms that can quickly rule out non-intersecting objects.

\subsection{Filter List}
The following is a list of all filters that are included in SVS. 
You can also get this list by using the cli command \texttt{svs filters} and 
get information about a specific filter using the command \texttt{svs filters.FILTER\_NAME}.
Many filters have a \texttt{\_select} version. The select version returns a subset
of the input nodes which pass a test. For example, the \texttt{intersect} filter returns
boolean values for each input (a, b) pair, while the \texttt{intersect\_select} filter
returns the nodes in set b which intersect the input node a. This is useful for passing
the results of one filter into another (e.g. take the nodes that intersect node a and find
the largest of them). 

\begin{description}
  \item{\soarb{node}} \\
    Given an \soarb{\^{}id}, outputs the node with that id.
  \item{\soarb{all\_nodes}} \\
    Outputs all the nodes in the scene
  \item{\soarb{combine\_nodes}} \\
    Given multiple node inputs as \soarb{\^{}a}, concates them into a single output set.
  \item{\soarb{remove\_node}} \\
    Removes node \soarb{\^{}id} from the input set \soarb{\^{}a} and outputs the rest. 
  \item{\soarb{node\_position}} \\
    Outputs the position of each node in input \soarb{\^{}a}.
  \item{\soarb{node\_rotation}} \\
    Outputs the rotation of each node in input \soarb{\^{}a}.
  \item{\soarb{node\_scale}} \\
    Outputs the scale of each node in input \soarb{\^{}a}.
  \item{\soarb{node\_bbox}} \\
    Outputs the bounding box of each node in input \soarb{\^{}a}.
\item{\soarb{distance} and \soarb{distance\_select}} \\
  Outputs the distance between input nodes \soarb{\^{}a} and \soarb{\^{}b}. 
  Distance can be specified by \soarb{\^{}distance\_type << centroid hull >>}, where
  \texttt{centroid} is the euclidean distance between the centers, and the \texttt{hull}
  is the minimum distance between the node surfaces. \texttt{distance\_select} chooses
  nodes in set b in which the distance to node a falls within the range \soarb{\^{}min} and \soarb{\^{}max}.
\item{\soarb{closest} and \soarb{farthest}} \\
  Outputs the node in set \soarb{\^{}b} closest to or farthest from \soarb{\^{}a}
  (also uses \soarb{distance\_type}).
\item{\soarb{axis\_distance} and \soarb{axis\_distance\_select}} \\
  Outputs the distance from input node \soarb{\^{}a} to \soarb{\^{}b} along
  a particular axis (\soarb{\^{}axis << x y z >>}). This distance is based on 
  bounding boxes. A value of 0 indicates the nodes overlap on the given axis, otherwise 
  the result is a signed value indicating whether node b is greater or less than 
  node a on the given axis.  
  The \texttt{axis\_distance\_select} filter also uses \soarb{\^{}min} and \soarb{\^{}max}
  to select nodes in set b. 
\item{\soarb{volume} and \soarb{volume\_select}} \\
  Outputs the bounding box volume of each node in set \soarb{\^{}a}. 
  For \texttt{volume\_select}, it outputs a subset of the nodes whose volumes
  fall within the range \soarb{\^{}min} and \soarb{\^{}max}.
\item{\soarb{largest} and \soarb{smallest}} \\
  Outputs the node in set \soarb{\^{}a} with the largest or smallest volume.
\item{\soarb{larger} and \soarb{larger\_select}}\\
  Outputs whether input node \soarb{\^{}a} is larger than each input node \soarb{\^{}b}, 
  or selects all nodes in b for which a is larger. 
\item{\soarb{smaller} and \soarb{smaller\_select}}\\
  Outputs whether input node \soarb{\^{}a} is smaller than each input node \soarb{\^{}b}, 
  or selects all nodes in b for which a is smaller. 
\item{\soarb{contain} and \soarb{contain\_select}} \\
  Outputs whether the bounding box of each input node \soarb{\^{}a} contains
  the bounding box of each input node \soarb{\^{}b}, or selects those nodes
  in b which are contained by node a. 
\item{\soarb{intersect} and \soarb{intersect\_select}} \\
  Outputs whether each input node \soarb{\^{}a} intersects each input node \soarb{\^{}b}, 
  or selects those nodes in b which intersect node a. Intersection is specified
  by \soarb{\^{}intersect\_type << hull box >>}; either the convex hull of the node
  or the axis-aligned bounding box. 
  \item{\soarb{tag\_select}} \\
    Outputs all the nodes in input set \soarb{\^{}a} which have the tag specified by 
    \soarb{\^{}tag\_name} and \soarb{\^{}tag\_value}. 
\end{description}

\subsection{Examples}

Select all the objects with a volume between 1 and 2. 
\begin{verbatim}
(S1 ^svs S3)
  (S3 ^command C3)
    (C3 ^extract E1)
      (E1 ^type volume_select ^a A1 ^min 1 ^max 2)
        (A1 ^type all_nodes)
\end{verbatim} 

Find the distance between the centroid of ball3 and all other objects. 
\begin{verbatim}
(S1 ^svs S3)
  (S3 ^command C3)
    (C3 ^extract E1)
      (E1 ^type distance ^a A1 ^b B1 ^distance_type centroid)
        (A1 ^type node ^id ball3)
        (B1 ^type all_nodes)
\end{verbatim} 

Test where ball2 intersects any red objects. 
\begin{verbatim}
(S1 ^svs S3)
  (S3 ^command C3)
    (C3 ^extract E1)
      (E1 ^type intersect ^a A1 ^b B1 ^intersect_type hull)
        (A1 ^type node ^id ball2)
        (B1 ^type tag_select ^a A2 ^tag_name color ^tag_value red)
          (A2 ^type all_nodes)
\end{verbatim}

Find all the objects on the table. This is done by selecting nodes 
where the distance between them and the table along the z axis is a small positive number. 
\begin{verbatim}
(S1 ^svs S3)
  (S3 ^command C3)
    (C3 ^extract E1)
      (E1 ^type axis_distance_select ^a A1 ^b B1 ^axis z ^min 0.0001 ^max 0.1)
        (A1 ^type node ^id table)
        (B1 ^type all_nodes)
\end{verbatim}

Find the smallest object that intersects the table (excluding itself). 
\begin{verbatim}
(S1 ^svs S3)
  (S3 ^command C3)
    (C3 ^extract E1)
      (E1 ^type smallest ^a A1)
        (A1 ^type intersect_select ^a A2 ^b B2 ^intersect_type hull)
          (A2 ^type node ^id table)
          (B1 ^type remove_node ^id table ^a A3)
            (A3 ^type all_nodes)
\end{verbatim}




\section{Writing new filters}

SVS contains a small set of generally useful filters, but many users will need additional specialized filters for their application.
Writing new filters for SVS is conceptually simple.

\begin{enumerate}
\item Write a C++ class that inherits from the appropriate filter subclass.
\item Register the new class in a global table of all filters (\texttt{filter\_table.cpp}).
\item Recompile the kernel. 
\end{enumerate}

\subsection{Filter subclasses}

The fact that filter inputs and outputs are lists rather than single values introduces some complexity to how filters are implemented.
Depending on the functionality of the filter, the multiple inputs into multiple parameters must be combined in different ways, and sets of inputs will map in different ways onto the output values.
Furthermore, the outputs of filters are cached so that the filter does not repeat computations on sets of inputs that do not change.
To shield the user from this complexity, a set of generally useful filter paradigms were implemented as subclasses of the basic \texttt{filter} class.
When writing custom filters, try to inherit from one of these classes instead of from \texttt{filter} directly.

\subsubsection{Map filter}
This is the most straightforward and useful class of filters.
A filter of this class takes the Cartesian product of all input values in all parameters,
and performs the same computation on each combination, generating one output.
In other words, this class implements a one-to-one mapping from input combinations to output values.

To write a new filter of this class, inherit from the \texttt{map\_filter} class, 
and define the \texttt{compute} function. Below is an example template:

\begin{verbatim}
class new_map_filter : public map_filter<double> // templated with output type
{
  public:
    new_map_filter(Symbol *root, soar_interface *si, filter_input *input, scene *scn)
    : map_filter<double>(root, si, input)   // call superclass constructor
    {}

    /* Compute
       Do the proper computation based on the input filter_params 
       and set the out parameter to the result 
       Return true if successful, false if an error occured */
    bool compute(const filter_params* p, double& out){
      sgnode* a;
      if(!get_filter_param(this, p, "a", a)){
        set_status("Need input node a");
        return false;
      }
      out = // Your computation here
    }
};
\end{verbatim}

\subsubsection{Select filter}
This is very similar to a map filter, except for each input combination from the 
Cartesian product the output is optional. This is useful for selecting and returning
a subset of the outputs. 

To write a new filter of this class, inherit from the \texttt{select\_filter} class, 
and define the \texttt{compute} function. Below is an example template:

\begin{verbatim}
class new_select_filter : public select_filter<double> // templated with output type
{
  public:
    new_select_filter(Symbol *root, soar_interface *si, filter_input *input, scene *scn)
    : select_filter<double>(root, si, input)   // call superclass constructor
    {}

    /* Compute
       Do the proper computation based on the input filter_params 
       and set the out parameter to the result (if desired)
       Also set the select bit to true if you want to the result to be output. 
       Return true if successful, false if an error occured */
    bool compute(const filter_params* p, double& out, bool& select){
      sgnode* a;
      if(!get_filter_param(this, p, "a", a)){
        set_status("Need input node a");
        return false;
      }
      out = // Your computation here
      select = // test for when to output the result of the computation
    }
};
\end{verbatim}

\subsubsection{Rank filter}
A filter where a ranking is computed for each combination from the Cartesian
product of the input and only the combination which results in the highest 
(or lowest) value is output. The default behavior is to select the highest, 
to select the lowest you can call \texttt{set\_select\_highest(false)} on the filter.

To write a new filter of this class, inherit from the \texttt{rank\_filter} class, 
and define the \texttt{rank} function. Below is an example template:

\begin{verbatim}
class new_rank_filter : public rank_filter
{
  public:
    new_rank_filter(Symbol *root, soar_interface *si, filter_input *input, scene *scn)
    : rank_filter(root, si, input)   // call superclass constructor
    {}

    /* Compute
       Do the proper computation based on the input filter_params 
       And set r to the ranking result. 
       Return true if successful, false if an error occured */
    bool compute(const filter_params* p, double& r){
      sgnode* a;
      if(!get_filter_param(this, p, "a", a)){
        set_status("Need input node a");
        return false;
      }
      r = // Ranking computation
    }
};
\end{verbatim}

\subsection{Generic Node Filters}
There are also a set of generic filters specialized for computations involving nodes. 
With these you only need to specify a predicate function involving nodes. 
Also see \texttt{filters/base\_node\_filters.h}.
There are three types of these filters. 

\subsubsection{Node Test Filters}
These filters involve a binary test between two nodes (e.g. intersection or larger). 
You must specify a test function of the following form:
\begin{verbatim}
bool node_test(sgnode* a, sgnode* b, const filter_params* p)
\end{verbatim}
For an example of how the following base filters are used, see \texttt{filters/intersect.cpp}.

\textbf{node\_test\_filter} \\
For each input pair (a, b) this outputs the boolean result of \texttt{node\_test(a, b)}.

\textbf{node\_test\_select\_filter} \\
For each input pair (a, b) this outputs node b if \texttt{node\_test(a, b) == true}. 
(Can choose to select b if the test is false by calling \texttt{set\_select\_true(false)}).

\subsubsection{Node Comparison Filters}
These filters involve a numerical comparison between two nodes (e.g. distance). 
You must specify a comparison function of the following form:
\begin{verbatim}
double node_comparison(sgnode* a, sgnode* b, const filter_params* p)
\end{verbatim}

For an example of how the following base filters are used, see \texttt{filters/distance.cpp}.

\textbf{node\_comparison\_filter} \\
For each input pair (a, b) this outputs the numerical result of \texttt{node\_comparison(a, b)}. 

\textbf{node\_comparison\_select\_filter} \\
For each input pair (a, b), this outputs node b if 
\texttt{min <= node\_comparison(a, b) <= max}. 
Min and max can be set through calling \texttt{set\_min(double)} 
and \texttt{set\_max(double)}, or as specified by the user through the filter\_params. 

\textbf{node\_comparison\_rank\_filter} \\
This outputs the input pair (a, b) for which \texttt{node\_comparison(a, b)} 
produces the highest value. To instead have the lowest value output call \texttt{set\_select\_highest(true)}.


\subsubsection{Node Evaluation Filters}
These filters involve a numerical evaluation of a single node (e.g. volume). 
You must specify an evaluation function of the following form:
\begin{verbatim}
double node_evaluation(sgnode* a, const filter_params* p)
\end{verbatim}

For an example of how the following base filters are used, see \texttt{filters/volume.cpp}.

\textbf{node\_evaluation\_filter} \\
For each input node a, this outputs the numerical result of \texttt{node\_evaluation(a)}. 

\textbf{node\_evaluation\_select\_filter} \\
For each input node a, this outputs the node if 
\texttt{min <= node\_evaluation(a) <= max}. 
Min and max can be set through calling \texttt{set\_min(double)} 
and \texttt{set\_max(double)}, or as specified by the user through the filter\_params. 

\textbf{node\_evaluation\_rank\_filter} \\
This outputs the input node a for which \texttt{node\_evaluation(a)} 
produces the highest value. To instead have the lowest value output call \texttt{set\_select\_highest(true)}.


\section{Command line interface}

The user can query and modify the runtime behavior of SVS using the \soarb{svs} command.
The syntax of this command differs from other Soar commands due to the complexity and object-oriented nature of the SVS implementation.
The basic idea is to allow the user to access each object in the SVS implementation (not to be confused with objects in the scene graph) at runtime.
Therefore, the command has the form \texttt{svs PATH [ARGUMENTS]}, where \texttt{PATH} uniquely identifies an object or the method of an object.
\texttt{ARGUMENTS} is a space separated list of strings that each object or function interprets in its own way.
For example, \texttt{svs S1.scene.world.car} identifies the car object 
in the scene graph of the top state.
As another example, \verb|svs connect_viewer 5999| calls the method to connect to the SVS visualizer with 5999 being the TCP port to connect on.
Every path has two special arguments.

\begin{itemize}
\item{\texttt{svs PATH dir}} prints all the children of the object at \texttt{PATH}.
\item{\texttt{svs PATH help}} prints text about how to use the object, if available.
\end{itemize}



\divider 

% ----------------------------------------------------------------------------

\section{File System I/O Commands}
\label{FILE-IO}

This section describes commands which interact in one way or another
with operating system input and output, or file I/O.  Users can
save/retrieve information to/from files, redirect the information
printed by Soar as it runs, and save and load the binary representation
of productions.
The specific commands described in this section are:

\paragraph{Summary}
\begin{quote}
\begin{description}
\item[cd] - Change directory.
\item[dirs] - List the directory stack.
\item[load] - Loads soar files, rete networks, saved percept streams and external libraries.
\item[load file] - Sources a file containing soar commands and productions.  May also contain Tcl code if Tcl mode is enabled.
\item[load library] - Loads an external library that extends functionality of Soar.
\item[load rete-network] - Loads a rete network that represents rules loaded in production memory.
\item[load library] - Loads soar files, rete networks, saved percept streams and external libraries.
\item[ls] - List the contents of the current working directory.
\item[popd] - Pop the current working directory off the stack and change to the next directory on the stack.
\item[pushd] - Push a directory onto the directory stack, changing to it.
\item[pwd] - Print the current working directory.
\item[save] - Saves chunks, rete networks and percept streams.
\item[save chunks] - Saves chunks into a file.
\item[save percepts] - Saves future input link structures into a file.
\item[save rete-network] - Saves the current rete networks that represents rules loaded in production memory.
\end{description}
\end{quote}

(See also the \href{output}{output} command.)

The \textbf{load file} command, previously known as \textbf{source} is used for nearly every Soar program.  The
directory functions are important to understand so that users can
navigate directories/folders to load/save the files of interest.  
Saving and loading percept streams are used mainly  when Soar needs to interact with an
external environment.  Users might take advantage of these commands when
debugging agents, but care should be used in adding and removing wmes this
way as they do not fall under Soar's truth maintenance system.

\divider 
\input{wikicmd/tex/file-system}

\divider 
\input{wikicmd/tex/load}

\divider 
\input{wikicmd/tex/save}

\divider 

% ***************************************************************************
% ----------------------------------------------------------------------------
\section{Miscellaneous}
\label{MISC}


\comment{this section still needs to be rewritten...}

\nocomment{This section describes the commands used to inspect production memory,
working memory, and preference memory; to see what productions will 
match and fire in the next Propose or Apply phase;  and to examine the 
goal dependency set.  These commands are particularly useful when
running or debugging Soar, as they let users see what Soar is ``thinking.''}
The specific commands described in this section are:


\paragraph{Summary}
\begin{quote}
\begin{description}
\item[alias] - Define a new alias, or command, using existing commands and arguments.
\item[debug] - Contains commands that provide access to Soar's internals. Most users will not need to access these commands
\item[debug allocate] - Allocate additional 32 kilobyte blocks of memory for a specified memory pool without running Soar.
\item[debug port] - Returns the port the kernel instance is listening on.
\item[debug time] - Uses a default system clock timer to record the wall time required while executing a command.
\item[debug internal-symbols] - Print information about the Soar symbol table.
\end{description}
\end{quote}

\divider 
\input{wikicmd/tex/alias}

\divider 
\input{wikicmd/tex/debug}
