\chapter{Semantic Memory}
\label{SMEM}
\index{semantic memory}
\index{smem}

Soar's semantic memory is a repository for long-term declarative knowledge, supplementing what is contained in short-term working memory (and production memory). 
Episodic memory, which contains memories of the agent's experiences, is described in Chapter \ref{EPMEM}. 
The knowledge encoded in episodic memory is organized temporally, and specific information is embedded within the context of when it was experienced, whereas knowledge in semantic memory is independent of any specific context, representing more general facts about the world.

This chapter is organized as follows: semantic memory structures in working memory (\ref{SMEM-wm}); representation of knowledge in semantic memory (\ref{SMEM-kr}); storing semantic knowledge (\ref{SMEM-store}); retrieving semantic knowledge (\ref{SMEM-retrieve}); and a discussion of performance (\ref{SMEM-perf}). 
The detailed behavior of semantic memory is determined by numerous parameters that can be controlled and configured via the \soarb{smem} command. 
Please refer to the documentation for that command in Section \ref{smem} on page \pageref{smem}.


\section{Working Memory Structure}
\label{SMEM-wm}

Upon creation of a new state in working memory (see Section \ref{ARCH-impasses-types} on page \pageref{ARCH-impasses-types}; Section \ref{SYNTAX-impasses} on page \pageref{SYNTAX-impasses}), the architecture creates the following augmentations to facilitate agent interaction with semantic memory:

\begin{verbatim}
(<s> ^smem <smem>)
  (<smem> ^command <smem-c>)
  (<smem> ^result <smem-r>)
\end{verbatim}

As rules augment the \soar{command} structure in order to access/change semantic knowledge (\ref{SMEM-store}, \ref{SMEM-retrieve}), semantic memory augments the \soar{result} structure in response.
Production actions should not remove augmentations of the \soar{result} structure directly, as semantic memory will maintain these WMEs.



\section{Knowledge Representation}
\label{SMEM-kr}
\index{LTI!definition}
\index{@!LTI label}

The representation of knowledge in semantic memory is similar to that in working memory (see Section \ref{ARCH-wm} on page \pageref{ARCH-wm}) -- both include graph structures that are composed of symbolic elements consisting of an identifier, an attribute, and a value. 
It is important to note, however, key differences:

\begin{itemize}

\item 
Currently semantic memory only supports attributes that are symbolic constants \\
(string, integer, or decimal), but \emph{not} attributes that are identifiers

\item 
Whereas working memory is a single, connected, directed graph, semantic memory can be disconnected, consisting of multiple directed, connected sub-graphs

\end{itemize}

From Soar 9.6 onward, \soarbit{Long-term} \soarb{identifiers} (LTIs) are defined as identifiers that exist in semantic memory \emph{only}.
Each LTI is permanently associated with a specific number that labels it (e.g. {@}5 or {@}7).
Instances of an LTI can be loaded into working memory as regular \emph{short-term} identifiers (STIs) linked with that specific LTI.
For clarity, when printed, a short-term identifier associated with an LTI is followed with the label of that LTI.
For example, if the working memory ID L7 is associated with the LTI named {@}29, printing that STI would appear as \soar{L7 ({@}29)}. 

When presented in a figure, long-term identifiers will be indicated by a double-circle. 
For instance, Figure \ref{fig:smem-concept} depicts the long-term identifier {@}68, with four augmentations, representing the addition fact of ${6+7=13}$ (or, rather, 3, carry 1, in context of multi-column arithmetic).

\begin{figure}
\insertfigure{Figures/smem-concept}{2in}
\insertcaption{Example long-term identifier with four augmentations.}
\label{fig:smem-concept}
\end{figure}

\subsection{Integrating Long-Term Identifiers with Soar}
Integrating long-term identifiers in Soar presents a number of theoretical and implementation challenges.  
This section discusses the state of integration with each of Soar's memories/learning mechanisms.

\subsubsection{Working Memory}
Long-term identifiers themselves never exist in working memory. Rather, instances of long term memories are loaded into working memory through queries or retrievals, and manipulated just like any other WMEs. Changes to any STI augmentations do not directly have any effect upon linked LTIs in semantic memory. Changes to LTIs themselves only occur though \soar{store} commands on the command link  or through command-line directives such as \soar{smem --add} (see below).

Each time an agent loads an instance of a certain LTI from semantic memory into working memory using queries or retrievals, the instance created will always be a new unique LTI. This means that if same long-term memory is retrieved multiple times in succession, each retrieval will result in a different STI instance, each linked to the same LTI. A benefit of this is that a retrieved long-term memory can be modified without compromising the ability to recall what the actual stored memory is.\footnote{
	Before Soar 9.6, LTIs were themselves retrieved into working memory. This meant all augmentations to such IDs, whether from the original retrieval or added after retrieval, would always be merged under the same ID, unless \soar{deep-copy} was used to make a duplicate short-term memory.}

\subsubsection{Procedural Memory}
Soar productions can use various conditions to test whether an STI is associated with an LTI or whether two STIs are linked to the same LTI (see Section \ref{SYNTAX-pm-lhs-predicates} on page \pageref{SYNTAX-pm-lhs-predicates}).
LTI names (e.g. {@}6) may \emph{not} appear in the action side of productions.

\subsubsection{Episodic Memory}
Episodic memory (see Section \ref{EPMEM} on page \pageref{EPMEM}) faithfully captures LTI-linked STIs, including the episode of transition. 
Retrieved episodes contain STIs as they existed during the episode, regardless of any changes to linked LTIs that transpired since the episode occurred.

\section{Storing Semantic Knowledge}
\label{SMEM-store}
\index{smem!store}

\subsection{Store command}

An agent stores a long-term identifier in semantic memory by creating a \soarb{\carat store} command: this is a WME whose identifier is the \soar{command} link of a state's \soar{smem} structure, the attribute is \soar{store}, and the value is a short-term identifier.

\begin{verbatim}
<s> ^smem.command.store <identifier>
\end{verbatim}

Semantic memory will encode and store all WMEs whose identifier is the value of the store command.  
Storing deeper levels of working memory is achieved through multiple store commands.

Multiple store commands can be issued in parallel.  
Storage commands are processed on every state at the end of every phase of every decision cycle.  
Storage is guaranteed to succeed and a status WME will be created, where the identifier is the \soarb{\carat result} link of the \soar{smem} structure of that state, the attribute is \soar{success}, and the value is the value of the store command above.

\begin{verbatim}
<s> ^smem.result.success <identifier>
\end{verbatim}

If the identifier used in the store command is not linked to any existing LTIs, a new LTI will be created in smem and the stored STI will be linked to it. If the identifier used in the store command is already linked to an LTI, the store will overwrite that long-term memory.
For example, if an existing LTI \soar{{@}5} had augmentations \soar{\carat A do \carat B re \carat C mi}, and a \soar{store} command stored short-term identifier \soar{L35} which was linked to \soar{{@}5} but had only the augmentation \soar{\carat D fa}, the LTI \soar{{@}5} would be changed to only have \soar{\carat D fa}.

\subsection{Store-new command}
\index{smem!store-new}

The \soarb{\carat store-new} command structure is just like the \soar{\carat store} command, except that smem will always store the given memory as an entirely new structure, regardless of whether the given STI was linked to an existing LTI or not. Any STIs that don't already have links will get linked to the newly created LTIs. But if a stored STI was already linked to some LTI, Soar will not re-link it to the newly created LTI. 

If this behavior is not desired, the agent can add a \soarb{\carat link-to-new-LTM yes} augmentation to override this behavior. One use for this setting is to allow chunking to backtrace through a stored memory in a manner that will be consistent with a later state of memory when the newly stored LTI is retrieved again.

\subsection{User-Initiated Storage}
\index{smem (command)!add}

Semantic memory provides agent designers the ability to store semantic knowledge via the \soarb{add} switch of the \soarb{smem} command (see Section \ref{smem} on page \pageref{smem}).  
The format of the command is nearly identical to the working memory manipulation components of the RHS of a production (i.e. no RHS-functions; see Section \ref{SYNTAX-pm-rhs} on page \pageref{SYNTAX-pm-rhs}).  
For instance:

\begin{verbatim}
smem --add {
   (<arithmetic> ^add10-facts <a01> <a02> <a03>)
   (<a01> ^digit1 1 ^digit-10 11)
   (<a02> ^digit1 2 ^digit-10 12)
   (<a03> ^digit1 3 ^digit-10 13)
}
\end{verbatim}

Unlike agent storage, declarative storage is automatically recursive.  
Thus, this command instance will add a new long-term identifier (represented by the temporary 'arithmetic' variable) with three augmentations.  
The value of each augmentation will each become an LTI with two constant attribute/value pairs.  
Manual storage can be arbitrarily complex and use standard dot-notation.
The add command also supports hardcoded LTI ids such as \soar{{@}1} in place of variables.

\subsection{Storage Location}
\index{smem!storage}

Semantic memory uses SQLite to facilitate efficient and standardized storage and querying of knowledge.  
The semantic store can be maintained in memory or on disk (per the \soar{database} and \soar{path} parameters; see Section \ref{smem}). 
If the store is located on disk, users can use any standard SQLite programs/components to access/query its contents.
However, using a disk-based semantic store is very costly (performance is discussed in greater detail in Section \ref{SMEM-perf} on page \pageref{SMEM-perf}), and running in memory is recommended for most runs.

Note that changes to storage parameters, for example \soar{database, path} and \soar{append} will not have an effect until the database is used after an initialization. This happens either shortly after launch (on first use) or after a database initialization command is issued. To switch databases or database storage types while running, set your new parameters and then perform an --init command.

The \soarb{path} parameter specifies the file system path the database is stored in. When \soar{path} is set to a valid file system path and \soar{database} mode is set to \emph{file}, then the SQLite database is written to that path.

\index{soar (command)!init}
The \soarb{append} parameter will determine whether all existing facts stored in a database on disk will be erased when semantic memory loads. Note that this affects \soar{soar init} also.  In other words, if the \soar{append} setting is off, all semantic facts stored to disk will be lost when a \soar{soar init} is performed. For semantic memory, \soar{append} mode is \soar{on} by default.

\soarit{Note}: As of version 9.3.3, Soar used a new schema for the semantic memory database. This means databases from 9.3.2 and below can no longer be loaded.  A conversion utility is available in Soar 9.4 to convert from the old schema to the new one.
%TODO Mazin, could you talk about the import/export command, here?

The \soarb{lazy-commit} parameter is a performance optimization. 
If set to \soar{on} (default), disk databases will not reflect semantic memory changes until the Soar kernel shuts down. 
This improves performance by avoiding disk writes. 
The \soar{optimization} parameter (see Section \ref{SMEM-perf} on page \pageref{SMEM-perf}) will have an affect on whether databases on disk can be opened while the Soar kernel is running.


\section{Retrieving Semantic Knowledge}
\label{SMEM-retrieve}
\index{smem!retrieve}

An agent retrieves knowledge from semantic memory by creating an appropriate command (we detail the types of commands below) on the \soar{command} link of a state's \soar{smem} structure. 
At the end of the output of each decision, semantic memory processes each state's \soar{smem \carat command} structure.  
Results, meta-data, and errors are added to the \soar{result} structure of that state's \soar{smem} structure.

Only one type of retrieval command (which may include optional modifiers) can be issued per state in a single decision cycle.  
Malformed commands (including attempts at multiple retrieval types) will result in an error:

\begin{verbatim}
<s> ^smem.result.bad-cmd <smem-c>
\end{verbatim}

Where the \soar{<smem-c>} variable refers to the \soar{command} structure of the state.

After a command has been processed, semantic memory will ignore it until some aspect of the command structure changes (via addition/removal of WMEs).  
When this occurs, the result structure is cleared and the new command (if one exists) is processed.

\subsection{Non-Cue-Based Retrievals}
A non-cue-based retrieval is a request by the agent to reflect in working memory the current augmentations of an LTI in semantic memory. 
The command WME has a \soarb{retrieve} attribute and an LTI-linked identifier value:

\begin{verbatim}
<s> ^smem.command.retrieve <lti>
\end{verbatim}

If the value of the command is not an LTI-linked identifier, an error will result: 

\begin{verbatim}
<s> ^smem.result.failure <lti>
\end{verbatim}

Otherwise, two new WMEs will be placed on the result structure:

\begin{verbatim}
<s> ^smem.result.success <lti>
<s> ^smem.result.retrieved <lti>
\end{verbatim}

All augmentations of the long-term identifier in semantic memory will be created as new WMEs in working memory.

\subsection{Cue-Based Retrievals}
\index{smem!query}

A cue-based retrieval performs a search for a long-term identifier in semantic memory whose augmentations exactly match an agent-supplied cue, as well as optional cue modifiers.

A cue is composed of WMEs that describe the augmentations of a long-term identifier.  
A cue WME with a constant value denotes an exact match of both attribute and value.  
A cue WME with an LTI-linked identifier as its value denotes an exact match of attribute and linked LTI.  
A cue WME with a short-term identifier as its value denotes an exact match of attribute, but with any value (constant or identifier).  

A cue-based retrieval command has a \soarb{query} attribute and an identifier value, the cue:

\begin{verbatim}
<s> ^smem.command.query <cue>
\end{verbatim}

For instance, consider the following rule that creates a cue-based retrieval command:

\begin{verbatim}
sp {smem*sample*query
    (state <s> ^smem.command <scmd>
               ^lti <lti>
               ^input-link.foo <bar>)
-->
    (<scmd> ^query <q>)
    (<q> ^name <any-name>
         ^foo <bar>
         ^associate <lti>
         ^age 25)
}
\end{verbatim}

In this example, assume that the \soar{<lti>} variable will match a short-term identifier which is linked to a long-term identifier and that the \soar{<bar>} variable will match a constant.  
Thus, the query requests retrieval of a long-term memory with augmentations that satisfy ALL of the following requirements:

\vspace{-12pt}
\begin{itemize}
	\item 
		Attribute \soar{name} with ANY value
		\vspace{-6pt}
	\item 
		Attribute \soar{foo} with value equal to that of variable \soar{<bar>} at the time this rule fires
		\vspace{-6pt}
	\item 
		Attribute \soar{associate} with value that is the same long-term identifier as that linked to by the \soar{<lti>} STI at the time this rule fires
		\vspace{-6pt}
	\item 
		Attribute \soar{age} with integer value \soar{25}
		\vspace{-6pt}
\end{itemize}

If no long-term identifier satisfies ALL of these requirements, an error is returned:

\begin{verbatim}
<s> ^smem.result.failure <cue>
\end{verbatim}

Otherwise, two WMEs are added:

\begin{verbatim}
<s> ^smem.result.success <cue>
<s> ^smem.result.retrieved <retrieved-lti>
\end{verbatim}

The result \soar{<retrieved-lti>} will be a new short-term identifier linked to the result LTI. 

As with non-cue-based retrievals, all of the augmentations of the long-term identifier in semantic memory are added as new WMEs to working memory. If these augmentations include other LTIs in smem, they too are instantiated into new short-term identifiers in working memory.

It is possible that multiple long-term identifiers match the cue equally well. 
In this case, semantic memory will retrieve the long-term identifier that was most recently stored/retrieved.
(More accurately, it will retrieve the LTI with the greatest \emph{activation} value. See below.)

The cue-based retrieval process can be further tempered using optional modifiers:

\begin{itemize}
\item 
	\index{smem!prohibit}
	The \soarb{prohibit} command requires that the retrieved long-term identifier is not equal to that linked with the supplied long-term identifier:
	
	\vspace{-6pt}
	\begin{verbatim}
	<s> ^smem.command.prohibit <bad-lti>
	\end{verbatim}
	\vspace{-6pt}
	
	Multiple prohibit command WMEs may be issued as modifiers to a single cue-based retrieval.  
	This method can be used to iterate over all matching long-term identifiers.
\item 
	\index{smem!neg-query}
	The \soarb{neg-query} command requires that the retrieved long-term identifier does NOT contain a set of attributes/attribute-value pairs:

	\vspace{-6pt}
	\begin{verbatim}
	<s> ^smem.command.neg-query <cue>
	\end{verbatim}
	\vspace{-6pt}
	
	The syntax of this command is identical to that of regular/positive \soar{query} command.
\item
	\index{math-query}
	The \soarb{math-query} command requires that the retrieved long term identifier contains an attribute value pair that meets a specified mathematical condition. 
	This condition can either be a \textit{\textbf{conditional}} query or a \textit{\textbf{superlative}} query. 
	
	Conditional queries are of the format:
	
	\vspace{-6pt}
	\begin{verbatim}
	<s> ^smem.command.math-query.<cue-attribute>.<condition-name> <value>
	\end{verbatim}
	\vspace{-6pt}
	
	Superlative queries do not use a value argument and are of the format:
	
	\vspace{-6pt}
	\begin{verbatim}
	<s> ^smem.command.math-query.<cue-attribute>.<condition-name>
	\end{verbatim}
	\vspace{-6pt}
	
	Values used in math queries must be integer or float type values.
	Currently supported condition names are:
	
	\begin{description}
		\item[less] A value less than the given argument
		\item[greater] A value greater than the given argument
		\item[less-or-equal] A value less than or equal to the given argument
		\item[greater-or-equal] A value greater than or equal to the given argument
		\item[max] The maximum value for the attribute
		\item[min] The minimum value for the attribute
	\end{description}
	\vspace{-6pt}
\end{itemize}

\subsubsection{Activation}
\index{smem!activation}

When an agent issues a cue-based retrieval and multiple LTIs match the cue, the LTI which semantic memory provides to working memory as the result is the LTI which not only matches the cue, but also has the highest \soarb{activation} value. Semantic memory has several activation methods available for this purpose.

The simplest activation methods are \soarb{recency} and \soarb{frequency} activation. Recency activation attaches a time-stamp to each LTI and records the time of last retrieval. Using recency activation, the LTI which matches the cue and was also most-recently retrieved is the one which is returned as the result for a query. Frequency activation attaches a counter to each LTI and records the number of retrievals for that LTI. Using frequency activation, the LTI which matches the cue and also was most frequently used is returned as the result of the query. By default, Soar uses recency activation.

\soarb{Base-level activation} can be thought of as a mixture of both recency and frequency. Soar makes use of the following equation (known as the Petrov approximation\footnote{Petrov, Alexander A. ``Computationally efficient approximation of the base-level learning equation in ACT-R.'' \textit{Proceedings of the seventh international conference on cognitive modeling.} 2006.}) for calculating base-level activation:

$$BLA = \log \left[ \sum\limits_{i=1}^{k} t_i^{-d} + \dfrac{(n-k)(t_n^{1-d} - t_k^{1-d})}{(1-d)(t_n-t_k)} \right]$$

where n is the number of activation boosts, $t_n$ is the time since the first boost, $t_k$ is the time of the $k$th boost, $d$ is the decay factor, and $k$ is the number of recent activation boosts which are stored. (In Soar, $k$ is hard-coded to 10.) To use base-level activation, use the following CLI command when sourcing an agent:

\begin{verbatim}
smem --set activation-mode base-level
\end{verbatim}

\soarb{Spreading activation} is new to Soar 9.6.0 and provides a secondary type of activation beyond the previous methods. First, spreading activation requires that base-level activation is also being used. They are considered additive. This value does not represent recency or frequency of use, but rather context-relatedness. Spreading activation increases the activation of LTIs which are linked to by identifiers currently present in working memory.\footnote{
	Specifically, linked to by STIs that have augmentations.} 
Such LTIs may be thought of as \textit{spreading sources}.

Spreading activation values spread. That is, spreading sources will add to the spreading activation values of any of their child LTIs, according to the directed graph structure within \soar{smem} (not working memory).
The amount of spread is controlled by the \\
\soarb{spreading-continue-probability} parameter. By default this value is set to \soar{0.9}. This would mean that \soar{90\%} of an LTI's spreading activation value would be divided among its direct children (without subtracting from its own value). This value is multiplicative with depth. A ``grandchild'' LTI, connected at a distance of two from a source LTI, would receive spreading according to $0.9\times 0.9 = 0.81$ of the source spreading activation value.

Spreading activation values are updated each decision cycle only as needed for specific \soar{smem} retrievals. For efficiency, two limits exist for the amount of spread calculated. The \soarb{spreading-limit} parameter limits how many LTIs can receive spread from a given spreading source LTI. By default, this value is (\soar{300}). Spread is distributed in a breadth-first manner to all descendants of a source. Once the number of LTIs that have been given spread from a given source reaches the max value indicated by \soar{spreading-limit}, no more is calculated for that source that update cycle, and the next spreading source's contributions are calculated. The maximum depth of descendants that can receive spread contributions from a source is similarly given by the \soarb{spreading-depth-limit} parameter. By default, this value is (\soar{10}).

In order to use spreading activation, use the following command:

\begin{verbatim}
smem --set spreading on
\end{verbatim}

Also, spreading activation can make use of working memory activation for adjusting edge weights and for providing nonuniform initial magnitude of spreading for sources of spread. This functionality is optional. To enable the updating of edge-weights, use the command: 

\begin{verbatim}
smem --set spreading-edge-updating on
\end{verbatim} 

and to enable working memory activation to modulate the magnitude of spread from sources, use the command: 

\begin{verbatim}
smem --set spreading-wma-source on
\end{verbatim}

For most use-cases, base-level activation is sufficient to provide an agent with relevant knowledge in response to a query. However, to provide an agent with more context-relevant results as opposed to results based only on historical usage, one must use spreading activation.

\subsection{Retrieval with Depth}

For either cue-based or non-cue-based retrieval, it is possible to retrieve a long-term identifier with additional depth. Using the \soarb{depth} parameter allows the agent to retrieve a greater amount of the memory structure than it would have by retrieving not only the long-term identifier's attributes and values, but also by recursively adding to working memory the attributes and values of that long-term identifier's children.

Depth is an additional \soar{command} attribute, like query:

\begin{verbatim}
<s> ^smem.command.query <cue>
    ^smem.command.depth <integer>
\end{verbatim}

For instance, the following rule uses depth with a cue-based retrieval:

\begin{verbatim}
sp {smem*sample*query
    (state <s> ^smem.command <sc>
               ^input-link.foo <bar>)
-->
    (<sc> ^query <q>
          ^depth 2)
    (<q> ^name <any-name>
         ^foo <bar>
         ^associate <lti>
         ^age 25)
}
\end{verbatim}

In the example above and without using depth, the long-term identifier referenced by
\begin{verbatim}
^associate <lti>
\end{verbatim}
would not also have its attributes and values be retrieved. With a depth of 2 or more, that long-term identifier also has its attributes and values added to working memory.

Depth can incur a large cost depending on the specified depth and the structures stored in semantic memory.

\section{Performance}
\label{SMEM-perf}

Initial empirical results with toy agents show that semantic memory queries carry up to a 40\% overhead as compared to comparable rete matching.  
However, the retrieval mechanism implements some basic query optimization: statistics are maintained about all stored knowledge.  
When a query is issued, semantic memory re-orders the cue such as to minimize expected query time.  
Because only perfect matches are acceptable, and there is no symbol variablization, semantic memory retrievals do not contend with the same combinatorial search space as the rete.  
Preliminary empirical study shows that semantic memory maintains sub-millisecond retrieval time for a large class of queries, even in very large stores (millions of nodes/edges).

Once the number of long-term identifiers overcomes initial overhead (about 1000 WMEs), initial empirical study shows that semantic storage requires far less than 1KB per stored WME.

\subsection{Math queries}
There are some additional performance considerations when using math queries during retrieval.
Initial testing indicates that conditional queries show the same time growth with respect to the number of memories in comparison to non-math queries, however the actual time for retrieval may be slightly longer.
Superlative queries will often show a worse result than similar non-superlative queries, because the current implementation of semantic memory requires them to iterate over any memory that matches all other involved cues.

\subsection{Performance Tweaking}

When using a database stored to disk, several parameters become crucial to performance.  
The first is \soarb{lazy-commit}, which controls when database changes are written to disk.   
The default setting (\soar{on}) will keep all writes in memory and only commit to disk upon re-initialization (quitting the agent or issuing the \soar{init} command).  
The \soar{off} setting will write each change to disk and thus incurs massive I/O delay.

The next parameter is \soarb{thresh}. 
This has to do with the locality of storing/updating activation information with semantic augmentations. 
By default, all WME augmentations are incrementally sorted by activation, such that cue-based retrievals need not sort large number of candidate long-term identifiers on demand, and thus retrieval time is independent of cue selectivity. 
However, each activation update (such as after a retrieval) incurs an update cost linear in the number of augmentations. 
If the number of augmentations for a long-term identifier is large, this cost can dominate. 
Thus, the \soar{thresh} parameter sets the upper bound of augmentations, after which activation is stored with the long-term identifier. 
This allows the user to establish a balance between cost of updating augmentation activation and the number of long-term identifiers that must be pre-sorted during a cue-based retrieval. 
As long as the threshold is greater than the number of augmentations of most long-term identifiers, performance should be fine (as it will bound the effects of selectivity).

The next two parameters deal with the SQLite cache, which is a memory store used to speed operations like queries by keeping in memory structures like levels of index B+-trees. 
The first parameter, \soarb{page-size}, indicates the size, in bytes, of each cache page. 
The second parameter, \soarb{cache-size}, suggests to SQLite how many pages are available for the cache. 
Total cache size is the product of these two parameter settings. 
The cache memory is not pre-allocated, so short/small runs will not necessarily make use of this space. 
Generally speaking, a greater number of cache pages will benefit query time, as SQLite can keep necessary meta-data in memory. 
However, some documented situations have shown improved performance from decreasing cache pages to increase memory locality. 
This is of greater concern when dealing with file-based databases, versus in-memory. 
The size of each page, however, may be important whether databases are disk- or memory-based. 
This setting can have far-reaching consequences, such as index B+-tree depth. 
While this setting can be dependent upon a particular situation, a good heuristic is that short, simple runs should use small values of the page size (\soar{1k}, \soar{2k}, \soar{4k}), whereas longer, more complicated runs will benefit from larger values (\soar{8k}, \soar{16k}, \soar{32k}, \soar{64k}). 
The episodic memory chapter (see Section \ref{EPMEM-perf} on page \pageref{EPMEM-perf}) has some further empirical evidence to assist in setting these parameters for very large stores.

The next parameter is \soarb{optimization}.  
The \soar{safety} parameter setting will use SQLite default settings.  
If data integrity is of importance, this setting is ideal.  
The \soar{performance} setting will make use of lesser data consistency guarantees for significantly greater performance.  
First, writes are no longer synchronous with the OS (synchronous pragma), thus semantic memory won't wait for writes to complete before continuing execution.  
Second, transaction journaling is turned off (journal\_mode pragma), thus groups of modifications to the semantic store are not atomic (and thus interruptions due to application/os/hardware failure could lead to inconsistent database state).  
Finally, upon initialization, semantic memory maintains a continuous exclusive lock to the database (locking\_mode pragma), thus other applications/agents cannot make simultaneous read/write calls to the database (thereby reducing the need for potentially expensive system calls to secure/release file locks).

Finally, maintaining accurate operation timers can be relatively expensive in Soar.  
Thus, these should be enabled with caution and understanding of their limitations.  
First, they will affect performance, depending on the level (set via the \soarb{timers} parameter).  
A level of \soar{three}, for instance, times every modification to long-term identifier recency statistics.  
Furthermore, because these iterations are relatively cheap (typically a single step in the linked-list of a b+-tree), timer values are typically unreliable (depending upon the system, resolution is 1 microsecond or more).

