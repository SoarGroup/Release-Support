\chapter{Spatial Visual System}
\label{SVS}
\index{Spatial Visual System}
\index{SVS|see{Spatial Visual System}}

\begin{figure}
	\insertfigure{Figures/svs-setup}{4in}
	\insertcaption{(a) Typical environment setup without using SVS. (b) Same environment using SVS.}
	\label{fig:svs-setup}
\end{figure}

\index{scene graph}
The Spatial Visual System (SVS) allows Soar to effectively represent and reason about continuous, three dimensional environments.
SVS maintains an internal representation of the environment as a collection of discrete objects with simple geometric shapes, called the \textbf{scene graph}.
The Soar agent can query for spatial relationships between the objects in the scene graph through a working memory interface similar to that of episodic and semantic memory.
Figure \ref{fig:svs-setup} illustrates the typical use case for SVS by contrasting it with an agent that does not utilize it.
The agent that does not use SVS (a. in the figure) relies on the environment to provide a symblic representation of the continuous state.
On the other hand, the agent that uses SVS (b) accepts a continuous representation of the environment state directly, and then performs queries on the scene graph to extract a symbolic representation internally.
This allows the agent to build more flexible symbolic representations without requiring modifications to the environment code.
Furthermore, it allows the agent to manipulate internal copies of the scene graph and then extract spatial relationships from the modified states, which is useful for look-ahead search and action modeling.
This type of imagery operation naturally captures and propogates the relationships implicit in spatial environments, and doesn't suffer from the frame problem that relational representations have.


% ----------------------------------------------------------------------------
% ----------------------------------------------------------------------------
\section{The scene graph}
\index{scene graph}

The primary data structure of SVS is the scene graph.
The scene graph is a tree in which the nodes represent objects in the scene and the edges represent ``part-of'' relationships between objects.
An example scene graph consisting of a car and a pole is shown in Figure \ref{fig:scene-graph}.
The scene graph's leaves are \emph{geometry nodes} and its interior nodes are \emph{group nodes}.
Geometry nodes represent atomic objects that have intrinsic shape, such as the wheels and chassis in the example.
Currently, the shapes supported by SVS are points, lines, convex polyhedrons, and spheres.
Group nodes represent objects that are the aggregates of their child nodes, such as the car object in the example.
The shape of a group node is the union of the shapes of its children.
Structuring complex objects in this way allows Soar to reason about them naturally at different levels of abstraction.
The agent can query SVS for relationships between the car as a whole with other objects (e.g. does it intersect the pole?), or the relationships between its parts (e.g. are the wheels pointing left or right with respect to the chassis?).
The scene graph always contains at least a root node: the \emph{world node}.

\begin{figure}
	\insertfigure{Figures/scene_graph}{5in}
	\insertcaption{(a) A 3D scene. (b) The scene graph representation.}
	\label{fig:scene-graph}
\end{figure}

Each node other than the world node has a transform with respect to its parent.
A transform consists of three components:

\vspace{-12pt}
\begin{description}
	\item[position $(x,y,z)$]
		Specifies the $x$, $y$, and $z$ offsets of the node's origin with respect to its parent's origin.
		\vspace{-6pt}
	\item[rotation $(x,y,z)$]
		Specifies the rotation of the node relative to its origin in Euler angles. This means that the node is rotated the specified number of radians along each axis in the order $x-y-z$. For more information, see \url{http://en.wikipedia.org/wiki/Euler_angles}.
		\vspace{-6pt}
	\item[scaling $(x,y,z)$]
		Specifies the factors by which the node is scaled along each axis.
		\vspace{-6pt}
\end{description}

The component transforms are applied in the order scaling, then rotation, then position.
Each node's transform is applied with respect to its parent's coordinate system, so the transforms accumulate down the tree.
A node's transform with respect to the world node, or its world transform, is the aggregate of all its ancestor transforms.
For example, if the car has a position transform of $(1,0,0)$ and a wheel on the car has a position transform of $(0,1,0)$, then the world position transform of the wheel is $(1,1,0)$.

SVS represents the scene graph structure in working memory under the \soar{\carat spatial-scene} link.
The working memory representation of the car and pole scene graph is

\begin{verbatim}
(S1 ^svs S3)
  (S3 ^command C3 ^spatial-scene S4)
    (S4 ^child C10 ^child C4 ^id world)
      (C10 ^id pole)
      (C4 ^child C9 ^child C8 ^child C7 ^child C6 ^child C5 ^id car)
        (C9 ^id chassis)
        (C8 ^id wheel3)
        (C7 ^id wheel2)
        (C6 ^id wheel1)
        (C5 ^id wheel0)
\end{verbatim}

Each state in working memory has its own scene graph.
When a new state is created, it will receive an independent copy of its parent's scene graph.
This is useful for performing look-ahead search, as it allows the agent to destructively modify the scene graph in a search state using mental imagery operations.


% ----------------------------------------------------------------------------
\subsection{svs\_viewer}

A viewer has been provided to show the scene graph visually. 
Run the program \soar{svs\_viewer -s PORT} from the soar/out folder 
to launch the viewer listening on the given port. Once the viewer is running, 
from within soar use the command \soar{svs connect\_viewer PORT} to connect 
to the viewer and begin drawing the scene graph. Any changes to the scene graph
will be reflected in the viewer. The viewer by default draws the topstate scene graph, 
to draw that on a substate first stop drawing the topstate with 
\soar{svs S1.scene.draw off} and then \soar{svs S7.scene.draw on}. 

\section{Scene Graph Edit Language}
\index{scene graph!Scene Graph Edit Language}
\index{SML}

The \soarb{Scene Graph Edit Language} (SGEL) is a simple, plain text, line oriented language that is used by SVS to modify the contents of the scene graph.
Typically, the scene graph is used to represent the state of the external environment, and the programmer sends SGEL commands reflecting changes in the environment to SVS via the \soar{Agent::SendSVSInput} function in the SML API.
These commands are buffered by the agent and processed at the beginning of each input phase.
Therefore, it is common to send scene changes through SendSVSInput \emph{before} the input phase.
If you send SGEL commands at the end of the input phase, 
the results won't be processed until the following decision cycle.

Each SGEL command begins with a single word command type and ends with a newline.
The four command types are

\vspace{-12pt}
\begin{description}
	\item[\soar{add ID PARENT\_ID [GEOMETRY] [TRANSFORM]}] \hfill \\
		Add a node to the scene graph with the given \soar{ID}, as a child of \soar{PARENT\_ID}, and with type \soar{TYPE} (usually object).The \soar{GEOMETRY} and \soar{TRANSFORM} arguments are optional and described below.
	
	\item[\soar{change ID [GEOMETRY] [TRANSFORM]}] \hfill \\
		Change the transform and/or geometry of the node with the given \soar{ID}.
	
	\item[\soar{delete ID}] \hfill \\
		Delete the node with the given \soar{ID}.
	
	\item[\soar{tag [add|change|delete] ID TAG\_NAME TAG\_VALUE}] \hfill \\
	  Adds, changes, or deletes a tag from an object. A tag consists of a \soar{TAG\_NAME} and \soar{TAG\_VALUE} pair and is added to the node with the given \soar{ID}. Both \soar{TAG\_NAME} and \soar{TAG\_VALUE} must be strings. Tags can differentiate nodes (e.g. as a type field) and can be used in conjunction with the \soar{tag\_select} filter to choose a subset of the nodes. 
\end{description}
\vspace{-6pt}

The \soar{TRANSFORM} argument has the form \soar{[p X Y Z] [r X Y Z] [s X Y Z]}, corresponding to the position, rotation, and scaling components of the transform, respectively.
All the components are optional; any combination of them can be excluded, and the included components can appear in any order.

The \soar{GEOMETRY} argument has two forms:

\begin{description}
	\item[\soar{b RADIUS}] \hfill \\
		Make the node a geometry node with sphere shape with radius \soar{RADIUS}.
	\item[\soar{v X1 Y1 Z1 X2 Y2 Z2 ...}] \hfill \\
		Make the node a geometry node with a convex polyhedron shape with the specified vertices.
		Any number of vertices can be listed.
\end{description}


% ----------------------------------------------------------------------------
\subsection{Examples}

Creating a sphere called ball4 with radius 5 at location (4, 4, 0). \\
\soar{add ball4 world b 5 p 4 4 0}

Creating a triangle in the xy plane, then rotate it vertically, double its size, and move it to (1, 1, 1).  \\
\soar{add tri9 world v -1 -1 0 1 -1 0 0 0.5 0 p 1 1 1 r 1.507 0 0 s 2 2 2}

Creating a snowman shape of 3 spheres stacked on each other and located at (2, 2, 0). \\
\soar{add snowman world p 2 2 0} \\
\soar{add bottomball snowman b 3 p 0 0 3} \\
\soar{add middleball snowman b 2 p 0 0 8} \\
\soar{add topball snowman b 1 p 0 0 11} 

Set the rotation transform on box11 to 180 degrees around the z axis. \\
\soar{change box11 r 0 0 3.14159}

Changing the color tag on box7 to green. \\
\soar{tag change box7 color green}


\section{Commands}

The Soar agent initiates commands in SVS via the \soarb{\carat command} link, 
similar to semantic and episodic memory. These commands allow the agent to 
modify the scene graph and extract filters. 
Commands are processed during the output phase and the results are added to 
working memory during the input phase. 
SVS supports the following commands:

\begin{description}
	\item{\textbf{add\_node}}
		Creates a new node and adds it to the scene graph
	\item{\textbf{copy\_node}}
		Creates a copy of an existing node
	\item{\textbf{delete\_node}}
		Removes a node from the scene graph and deletes it
	\item{\textbf{set\_transform}}
		Changes the position, rotation, and/or scale of a node
	\item{\textbf{set\_tag}}
		Adds or changes a tag on a node
	\item{\textbf{delete\_tag}}
		Deletes a tag from a node
	\item{\textbf{extract}}
		Compute the truth value of spatial relationships in the current scene graph.
	\item{\textbf{extract\_once}}
		Same as extract, except it is only computed once and doesn't update when the scene changes.
\end{description}


% ----------------------------------------------------------------------------
\subsection{add\_node}

This commands adds a new node to the scene graph. 

\begin{description}
	\item{\soarb{\carat id [string]}} The id of the node to create
	\item{\soarb{\carat parent [string]}} The id of the node to attach the new node to (default is world)
	\item{\soarb{\carat geometry << group point ball box >> }} The geometry the node should have 
	\item{\soarb{\carat position.\{\carat x \carat y \carat z\} }} Position of the node (optional)
	\item{\soarb{\carat rotation.\{\carat x \carat y \carat z\} }} Rotation of the node (optional)
	\item{\soarb{\carat scale.\{\carat x \carat y \carat z\} }} Scale of the node (optional)
\end{description}

The following example creates a node called \soar{box5} and adds it to the world. 
The node has a box shape of side length 0.1 and is placed at position (1, 1, 0). 

\begin{verbatim}
(S1 ^svs S3)
  (S3 ^command C3 ^spatial-scene S4)
    (C3 ^add_node A1)
      (A1 ^id box5 ^parent world ^geometry box ^position P1 ^scale S6)
        (P1 ^x 1.0 ^y 1.0 ^z 0.0)
        (S6 ^x 0.1 ^y 0.1 ^z 0.1)
\end{verbatim}


% ----------------------------------------------------------------------------
\subsection{copy\_node}

This command creates a copy of an existing node and adds it to the scene graph. 
This copy is not recursive, it only copies the node itself, not its children. 
The position, rotation, and scale transforms are also copied from the source node
but they can be changed if desired. 

\begin{description}
	\item{\soarb{\carat id [string]}} The id of the node to create
	\item{\soarb{\carat source [string]}} The id of the node to copy
	\item{\soarb{\carat parent [string]}} The id of the node to attach the new node to (default is world)
	\item{\soarb{\carat position.\{\carat x \carat y \carat z\} }} Position of the node (optional)
	\item{\soarb{\carat rotation.\{\carat x \carat y \carat z\} }} Rotation of the node (optional)
	\item{\soarb{\carat scale.\{\carat x \carat y \carat z\} }} Scale of the node (optional)
\end{description}

The following example copies a node called \soar{box5} as new node \soar{box6}
and moves it to position (2, 0, 2).

\begin{verbatim}
(S1 ^svs S3)
  (S3 ^command C3 ^spatial-scene S4)
    (C3 ^copy_node A1)
      (A1 ^id box6 ^source box5 ^position P1)
        (P1 ^x 2.0 ^y 0.0 ^z 2.0)
\end{verbatim}


% ----------------------------------------------------------------------------
\subsection{delete\_node}

This command deletes a node from the scene graph. Any children will also be deleted. 

\begin{description}
	\item{\soarb{\carat id [string]}} The id of the node to delete
\end{description}

The following example deletes a node called \soar{box5}

\begin{verbatim}
(S1 ^svs S3)
  (S3 ^command C3 ^spatial-scene S4)
    (C3 ^delete_node D1)
      (D1 ^id box5)
\end{verbatim}


% ----------------------------------------------------------------------------
\subsection{set\_transform}

This command allows you to change the position, rotation, and/or scale of an 
exisiting node. You can specify any combination of the three transforms. 

\begin{description}
	\item{\soarb{\carat id [string]}} The id of the node to change
	\item{\soarb{\carat position.\{\carat x \carat y \carat z\} }} Position of the node (optional)
	\item{\soarb{\carat rotation.\{\carat x \carat y \carat z\} }} Rotation of the node (optional)
	\item{\soarb{\carat scale.\{\carat x \carat y \carat z\} }} Scale of the node (optional)
\end{description}

The following example moves and rotates a node called \soar{box5}.

\begin{verbatim}
(S1 ^svs S3)
  (S3 ^command C3 ^spatial-scene S4)
    (C3 ^set_transform S6)
      (S6 ^id box5 ^position P1 ^rotation R1)
        (P1 ^x 2.0 ^y 2.0 ^z 0.0)
        (R1 ^x 0.0 ^y 0.0 ^z 1.57)
\end{verbatim}


% ----------------------------------------------------------------------------
\subsection{set\_tag}

This command allows you to add or change a tag on a node.
If a tag with the same id already exists, 
the existing value will be replaced with the new value.

\begin{description}
	\item{\soarb{\carat id [string]}} The id of the node to set the tag on
	\item{\soarb{\carat tag\_name [string]}} The name of the tag to add
	\item{\soarb{\carat tag\_value [string]}} The value of the tag to add
\end{description}

The following example adds a shape tag to the node \soar{box5}.

\begin{verbatim}
(S1 ^svs S3)
  (S3 ^command C3 ^spatial-scene S4)
    (C3 ^set_tag S6)
      (S6 ^id box5 ^tag_name shape ^tag_value cube)
\end{verbatim}


% ----------------------------------------------------------------------------
\subsection{delete\_tag}

This command allows you to delete a tag from a node.

\begin{description}
	\item{\soarb{\carat id [string]}} The id of the node to delete the tag from
	\item{\soarb{\carat tag\_name [string]}} The name of the tag to delete
\end{description}

The following example deletes the shape tag from the node \soar{box5}.

\begin{verbatim}
(S1 ^svs S3)
  (S3 ^command C3 ^spatial-scene S4)
    (C3 ^delete_tag D1)
      (D1 ^name box5 ^tag_name shape)
\end{verbatim}


% ----------------------------------------------------------------------------
\subsection{extract and extract\_once}

This command is commonly used to compute spatial relationships in the scene graph.
More generally, it puts the \soarb{result} of a filter pipeline (described in section \ref{sec:svs-filters}) in working memory.
Its syntax is the same as filter pipeline syntax.
During the input phase, SVS will evaluate the filter and 
put a \soar{\carat result} attribute on the command's identifier.
Under the \soar{\carat result} attribute is a multi-valued \soar{\carat record} attribute.
Each record corresponds to an output value from the head of the filter pipeline, along with the parameters that produced the value.
With the regular \soar{extract} command, these records will be updated as the scene graph
changes. With the \soar{extract\_once} command, the records will be created once
and will not change. 
Note that you should not change the structure of a filter once it is created 
(SVS only processes a command once). 
Instead to extract something different you must create a new command. 
The following is an example of an extract command which tests whether the 
car and pole objects are intersecting. The \soar{\carat status} and \soar{\carat result} WMEs are 
added by SVS when the command is finished. 

\begin{verbatim}
(S1 ^svs S3)
  (S3 ^command C3 ^spatial-scene S4)
    (C3 ^extract E2)
      (E2 ^a A1 ^b B1 ^result R7 ^status success ^type intersect)
        (A1 ^id car ^status success ^type node)
        (B1 ^id pole ^status success ^type node)
        (R7 ^record R17)
          (R17 ^params P1 ^value false)
            (P1 ^a car ^b pole)
\end{verbatim}


% ----------------------------------------------------------------------------
% ----------------------------------------------------------------------------
\section{Filters}
\label{sec:svs-filters}
\index{Spatial Visual System!filters}

\textbf{Filters} are the basic unit of computation in SVS.
They transform the continuous information in the scene graph into symbolic information that can be used by the rest of Soar.
Each filter accepts a number of labeled parameters as input, and produces a single output.
Filters can be arranged into pipelines in which the outputs of some filters are fed into the inputs of other filters.
The Soar agent creates filter pipelines by building an analogous structure in working memory as an argument to an "extract" command.
For example, the following structure defines a set of filters that reports whether the car intersects the pole:

\begin{verbatim}
(S1 ^svs S3)
  (S3 ^command C3 ^spatial-scene S4)
    (C3 ^extract E2)
      (E2 ^a A1 ^b B1 ^type intersect)
        (A1 ^id car ^type node)
        (B1 ^id pole ^type node)
\end{verbatim}

The \soarb{\carat type} attribute specifies the type of filter to instantiate, and the other attributes specify parameters.
This command will create three filters: an \soar{intersect} filter and two \soar{node} filters.
A \soarb{node} filter takes an \soar{id} parameter and returns the scene graph node with that ID as its result.
Here, the outputs of the \soar{car} and \soar{pole} node filters are fed into the \soar{\carat a} and \soar{\carat b} parameters of the \soar{intersect} filter.
SVS will update each filter's output once every decision cycle, at the end of the input phase.
The output of the \soarb{intersect} filter is a boolean value indicating whether the two objects are intersecting.
This is placed into working memory as the result of the extract command:

\begin{verbatim}
(S1 ^svs S3)
  (S3 ^command C3 ^spatial-scene S4)
    (C3 ^extract E2)
      (E2 ^a A1 ^b B1 ^result R7 ^status success ^type intersect)
        (A1 ^id car ^status success ^type node)
        (B1 ^id pole ^status success ^type node)
        (R7 ^record R17)
          (R17 ^params P1 ^value false)
            (P1 ^a car ^b pole)
\end{verbatim}

Notice that a \soar{\carat status} success is placed on each identifier corresponding to a filter.
A \soar{\carat result} WME is placed on the extract command with a single record with value \soarb{false}.


% ----------------------------------------------------------------------------
\subsection{Result lists}

Spatial queries often involve a large number of objects.
For example, the agent may want to compute whether an object intersects any others in the scene graph.
It would be inconvenient to build the extract command to process this query if the agent had to specify each object involved explicitly.
Too many WMEs would be required, which would slow down the production matcher as well as SVS because it must spend more time interpreting the command structure.
To handle these cases, all filter parameters and results can be lists of values.
For example, the query for whether one object intersects all others can be expressed as

\begin{verbatim}
(S1 ^svs S3)
  (S3 ^command C3)
    (C3 ^extract E2)
      (E2 ^a A1 ^b B1 ^result R7 ^status success ^type intersect)
        (A1 ^id car ^status success ^type node)
        (B1 ^status success ^type all_nodes)
        (R7 ^record R9 ^record R8)
          (R9 ^params P2 ^value false)
            (P2 ^a car ^b pole)
          (R8 ^params P1 ^value true)
            (P1 ^a car ^b car)
\end{verbatim}

The \soarb{all\_nodes} filter outputs a list of all nodes in the scene graph, and the \soar{intersect} filter outputs a list of boolean values indicating whether the car intersects each node, represented by the multi-valued attribute \soarb{record}.
Notice that each \soar{record} contains both the result of the query as well as the parameters that produced that result.
Not only is this approach more convenient than creating a separate command for each pair of nodes, but it also allows the \soar{intersect} filter to answer the query more efficiently using special algorithms that can quickly rule out non-intersecting objects.


% ----------------------------------------------------------------------------
\subsection{Filter List}

The following is a list of all filters that are included in SVS. 
You can also get this list by using the cli command \soar{svs filters} and 
get information about a specific filter using the command \soar{svs filters.FILTER\_NAME}.
Many filters have a \soar{\_select} version. The select version returns a subset
of the input nodes which pass a test. For example, the \soar{intersect} filter returns
boolean values for each input (a, b) pair, while the \soar{intersect\_select} filter
returns the nodes in set b which intersect the input node a. This is useful for passing
the results of one filter into another (e.g. take the nodes that intersect node a and find
the largest of them). 

\begin{description}
	\item{\soarb{node}} \\
		Given an \soarb{\carat id}, outputs the node with that id.
	\item{\soarb{all\_nodes}} \\
		Outputs all the nodes in the scene
	\item{\soarb{combine\_nodes}} \\
		Given multiple node inputs as \soarb{\carat a}, concates them into a single output set.
	\item{\soarb{remove\_node}} \\
		Removes node \soarb{\carat id} from the input set \soarb{\carat a} and outputs the rest. 
	\item{\soarb{node\_position}} \\
		Outputs the position of each node in input \soarb{\carat a}.
	\item{\soarb{node\_rotation}} \\
		Outputs the rotation of each node in input \soarb{\carat a}.
	\item{\soarb{node\_scale}} \\
		Outputs the scale of each node in input \soarb{\carat a}.
	\item{\soarb{node\_bbox}} \\
		Outputs the bounding box of each node in input \soarb{\carat a}.
	\item{\soarb{distance} and \soarb{distance\_select}} \\
		Outputs the distance between input nodes \soarb{\carat a} and \soarb{\carat b}. Distance can be specified by \soarb{\carat distance\_type << centroid hull >>}, where \soar{centroid} is the euclidean distance between the centers, and the \soar{hull} is the minimum distance between the node surfaces. \soar{distance\_select} chooses nodes in set b in which the distance to node a falls within the range \soarb{\carat min} and \soarb{\carat max}.
	\item{\soarb{closest} and \soarb{farthest}} \\
		Outputs the node in set \soarb{\carat b} closest to or farthest from \soarb{\carat a} (also uses \soarb{distance\_type}).
	\item{\soarb{axis\_distance} and \soarb{axis\_distance\_select}} \\
		Outputs the distance from input node \soarb{\carat a} to \soarb{\carat b} along a particular axis (\soarb{\carat axis << x y z >>}). This distance is based on bounding boxes. A value of 0 indicates the nodes overlap on the given axis, otherwise the result is a signed value indicating whether node b is greater or less than node a on the given axis. The \soar{axis\_distance\_select} filter also uses \soarb{\carat min} and \soarb{\carat max} to select nodes in set b. 
	\item{\soarb{volume} and \soarb{volume\_select}} \\
		Outputs the bounding box volume of each node in set \soarb{\carat a}. For \soar{volume\_select}, it outputs a subset of the nodes whose volumes fall within the range \soarb{\carat min} and \soarb{\carat max}.
	\item{\soarb{largest} and \soarb{smallest}} \\
		Outputs the node in set \soarb{\carat a} with the largest or smallest volume.
	\item{\soarb{larger} and \soarb{larger\_select}}\\
		Outputs whether input node \soarb{\carat a} is larger than each input node \soarb{\carat b}, or selects all nodes in b for which a is larger. 
	\item{\soarb{smaller} and \soarb{smaller\_select}}\\
		Outputs whether input node \soarb{\carat a} is smaller than each input node \soarb{\carat b}, or selects all nodes in b for which a is smaller. 
	\item{\soarb{contain} and \soarb{contain\_select}} \\
		Outputs whether the bounding box of each input node \soarb{\carat a} contains the bounding box of each input node \soarb{\carat b}, or selects those nodes in b which are contained by node a. 
	\item{\soarb{intersect} and \soarb{intersect\_select}} \\
		Outputs whether each input node \soarb{\carat a} intersects each input node \soarb{\carat b}, or selects those nodes in b which intersect node a. Intersection is specified by \soarb{\carat intersect\_type << hull box >>}; either the convex hull of the node or the axis-aligned bounding box. 
	\item{\soarb{tag\_select}} \\
		Outputs all the nodes in input set \soarb{\carat a} which have the tag specified by \soarb{\carat tag\_name} and \soarb{\carat tag\_value}. 
\end{description}


% ----------------------------------------------------------------------------
\subsection{Examples}

Select all the objects with a volume between 1 and 2. 

\begin{verbatim}
(S1 ^svs S3)
  (S3 ^command C3)
    (C3 ^extract E1)
      (E1 ^type volume_select ^a A1 ^min 1 ^max 2)
        (A1 ^type all_nodes)
\end{verbatim} 

Find the distance between the centroid of ball3 and all other objects. 

\begin{verbatim}
(S1 ^svs S3)
  (S3 ^command C3)
    (C3 ^extract E1)
      (E1 ^type distance ^a A1 ^b B1 ^distance_type centroid)
        (A1 ^type node ^id ball3)
        (B1 ^type all_nodes)
\end{verbatim} 

Test where ball2 intersects any red objects. 

\begin{verbatim}
(S1 ^svs S3)
  (S3 ^command C3)
    (C3 ^extract E1)
      (E1 ^type intersect ^a A1 ^b B1 ^intersect_type hull)
        (A1 ^type node ^id ball2)
        (B1 ^type tag_select ^a A2 ^tag_name color ^tag_value red)
          (A2 ^type all_nodes)
\end{verbatim}

Find all the objects on the table. This is done by selecting nodes 
where the distance between them and the table along the z axis is a small positive number. 

\begin{verbatim}
(S1 ^svs S3)
  (S3 ^command C3)
    (C3 ^extract E1)
      (E1 ^type axis_distance_select ^a A1 ^b B1 ^axis z ^min .0001 ^max .1)
        (A1 ^type node ^id table)
        (B1 ^type all_nodes)
\end{verbatim}

Find the smallest object that intersects the table (excluding itself). 

\begin{verbatim}
(S1 ^svs S3)
  (S3 ^command C3)
    (C3 ^extract E1)
      (E1 ^type smallest ^a A1)
        (A1 ^type intersect_select ^a A2 ^b B2 ^intersect_type hull)
          (A2 ^type node ^id table)
          (B1 ^type remove_node ^id table ^a A3)
            (A3 ^type all_nodes)
\end{verbatim}


% ----------------------------------------------------------------------------
% ----------------------------------------------------------------------------
\section{Writing new filters}

SVS contains a small set of generally useful filters, but many users will need additional specialized filters for their application.
Writing new filters for SVS is conceptually simple.

\vspace{-12pt}
\begin{enumerate}
	\item Write a C++ class that inherits from the appropriate filter subclass.
	\vspace{-6pt}
	\item Register the new class in a global table of all filters (\soar{filter\_table.cpp}).
	\vspace{-6pt}
	\item Recompile the kernel. 
	\vspace{-6pt}
\end{enumerate}


% ----------------------------------------------------------------------------
\subsection{Filter subclasses}

The fact that filter inputs and outputs are lists rather than single values introduces some complexity to how filters are implemented.
Depending on the functionality of the filter, the multiple inputs into multiple parameters must be combined in different ways, and sets of inputs will map in different ways onto the output values.
Furthermore, the outputs of filters are cached so that the filter does not repeat computations on sets of inputs that do not change.
To shield the user from this complexity, a set of generally useful filter paradigms were implemented as subclasses of the basic \soar{filter} class.
When writing custom filters, try to inherit from one of these classes instead of from \soar{filter} directly.


\subsubsection{Map filter}

This is the most straightforward and useful class of filters.
A filter of this class takes the Cartesian product of all input values in all parameters,
and performs the same computation on each combination, generating one output.
In other words, this class implements a one-to-one mapping from input combinations to output values.

To write a new filter of this class, inherit from the \soar{map\_filter} class, 
and define the \soar{compute} function. Below is an example template:

{\footnotesize
\begin{verbatim}
class new_map_filter : public map_filter<double> // templated with output type
{
  public:
    new_map_filter(Symbol *root, soar_interface *si, filter_input *input, scene *scn)
    : map_filter<double>(root, si, input)   // call superclass constructor
    {}

    /* Compute
       Do the proper computation based on the input filter_params 
       and set the out parameter to the result 
       Return true if successful, false if an error occured */
    bool compute(const filter_params* p, double& out){
      sgnode* a;
      if(!get_filter_param(this, p, "a", a)){
        set_status("Need input node a");
        return false;
      }
      out = // Your computation here
    }
};
\end{verbatim}
}


\subsubsection{Select filter}

This is very similar to a map filter, except for each input combination from the 
Cartesian product the output is optional. This is useful for selecting and returning
a subset of the outputs. 

To write a new filter of this class, inherit from the \soar{select\_filter} class, 
and define the \soar{compute} function. Below is an example template:

{\footnotesize
\begin{verbatim}
class new_select_filter : public select_filter<double> // templated with output type
{
  public:
    new_select_filter(Symbol *root, soar_interface *si, filter_input *input, scene *scn)
    : select_filter<double>(root, si, input)   // call superclass constructor
    {}

    /* Compute
       Do the proper computation based on the input filter_params 
       and set the out parameter to the result (if desired)
       Also set the select bit to true if you want to the result to be output. 
       Return true if successful, false if an error occured */
    bool compute(const filter_params* p, double& out, bool& select){
      sgnode* a;
      if(!get_filter_param(this, p, "a", a)){
        set_status("Need input node a");
        return false;
      }
      out = // Your computation here
      select = // test for when to output the result of the computation
    }
};
\end{verbatim}
}


\subsubsection{Rank filter}

A filter where a ranking is computed for each combination from the Cartesian
product of the input and only the combination which results in the highest 
(or lowest) value is output. The default behavior is to select the highest, 
to select the lowest you can call \soar{set\_select\_highest(false)} on the filter.

To write a new filter of this class, inherit from the \soar{rank\_filter} class, 
and define the \soar{rank} function. Below is an example template:

{\footnotesize
\begin{verbatim}
class new_rank_filter : public rank_filter
{
  public:
    new_rank_filter(Symbol *root, soar_interface *si, filter_input *input, scene *scn)
    : rank_filter(root, si, input)   // call superclass constructor
    {}

    /* Compute
       Do the proper computation based on the input filter_params 
       And set r to the ranking result. 
       Return true if successful, false if an error occured */
    bool compute(const filter_params* p, double& r){
      sgnode* a;
      if(!get_filter_param(this, p, "a", a)){
        set_status("Need input node a");
        return false;
      }
      r = // Ranking computation
    }
};
\end{verbatim}
}


\subsection{Generic Node Filters}

There are also a set of generic filters specialized for computations involving nodes. 
With these you only need to specify a predicate function involving nodes. (Also see \\
\soar{filters/base\_node\_filters.h}).

There are three types of these filters:

\subsubsection{Node Test Filters}

These filters involve a binary test between two nodes (e.g. intersection or larger). 
You must specify a test function of the following form:

\begin{verbatim}
bool node_test(sgnode* a, sgnode* b, const filter_params* p)
\end{verbatim}

For an example of how the following base filters are used, see \soar{filters/intersect.cpp}.

\begin{description}
	\item[node\_test\_filter] \hfill \\
		For each input pair (a, b) this outputs the boolean result of \soar{node\_test(a, b)}.
	\item[node\_test\_select\_filter] \hfill \\
		For each input pair (a, b) this outputs node b if \soar{node\_test(a, b) == true}. \\
		(Can choose to select b if the test is false by calling \soar{set\_select\_true(false)}).
\end{description}

\subsubsection{Node Comparison Filters}
These filters involve a numerical comparison between two nodes (e.g. distance). 
You must specify a comparison function of the following form:
\begin{verbatim}
double node_comparison(sgnode* a, sgnode* b, const filter_params* p)
\end{verbatim}

For an example of how the following base filters are used, see \soar{filters/distance.cpp}.

\begin{description}
	\item[node\_comparison\_filter] \hfill \\
		For each input pair (a, b), outputs the numerical result of \soar{node\_comparison(a, b)}. 
	\item[node\_comparison\_select\_filter] \hfill \\
		For each input pair (a, b), outputs node b if \soar{min <= node\_comparison(a, b) <= max}. Min and max can be set through calling \soar{set\_min(double)} and \soar{set\_max(double)}, or as specified by the user through the filter\_params. 
	\item[node\_comparison\_rank\_filter] \hfill \\
		This outputs the input pair (a, b) for which \soar{node\_comparison(a, b)} produces the highest value. To instead have the lowest value output call \soar{set\_select\_highest(true)}.
\end{description}


\subsubsection{Node Evaluation Filters}

These filters involve a numerical evaluation of a single node (e.g. volume). 
You must specify an evaluation function of the following form:

\begin{verbatim}
double node_evaluation(sgnode* a, const filter_params* p)
\end{verbatim}

For an example of how the following base filters are used, see \soar{filters/volume.cpp}.

\begin{description}
	\item[node\_evaluation\_filter] \hfill \\
		For each input node a, this outputs the numerical result of \soar{node\_evaluation(a)}. 
	\item[node\_evaluation\_select\_filter] \hfill \\
		For each input node a, this outputs the node if \soar{min <= node\_evaluation(a) <= max}. Min and max can be set through calling \soar{set\_min(double)} and \soar{set\_max(double)}, or as specified by the user through the filter\_params. 
	\item[node\_evaluation\_rank\_filter] \hfill \\
		This outputs the input node a for which \soar{node\_evaluation(a)} produces the highest value. To instead have the lowest value output call \soar{set\_select\_highest(true)}.
\end{description}


% ----------------------------------------------------------------------------
% ----------------------------------------------------------------------------
\section{Command line interface}

The user can query and modify the runtime behavior of SVS using the \soarb{svs} command.
The syntax of this command differs from other Soar commands due to the complexity and object-oriented nature of the SVS implementation.
The basic idea is to allow the user to access each object in the SVS implementation (not to be confused with objects in the scene graph) at runtime.
Therefore, the command has the form \soar{svs PATH [ARGUMENTS]}, where \soar{PATH} uniquely identifies an object or the method of an object.
\soar{ARGUMENTS} is a space separated list of strings that each object or function interprets in its own way.
For example, \soar{svs S1.scene.world.car} identifies the car object 
in the scene graph of the top state.
As another example, \verb|svs connect_viewer 5999| calls the method to connect to the SVS visualizer with 5999 being the TCP port to connect on.
Every path has two special arguments.

\begin{itemize}
	\item{\soar{svs PATH dir}} prints all the children of the object at \soar{PATH}.
	\item{\soar{svs PATH help}} prints text about how to use the object, if available.
\end{itemize}

See Section \ref{svs} on page \pageref{svs} for more details.